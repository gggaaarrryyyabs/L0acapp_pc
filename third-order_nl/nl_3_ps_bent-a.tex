\documentclass{article}
\usepackage{fullpage,enumitem,amsthm,amsmath,amssymb,graphicx,empheq,bm}
\usepackage{physics}
\usepackage{tcolorbox}
\usepackage[citecolor=blue]{hyperref}
\usepackage{tikz}
\usepackage[noadjust]{cite}
\usepackage{multirow}
\usepackage{threeparttable}


\newcommand{\Z}{\mathbf{Z}}
\newcommand{\F}{\mathbb{F}}
\newcommand{\Com}{\mathbf{C}}
\newcommand{\ord}{\operatorname{ord}}
\newcommand{\Q}{\mathbf{Q}}
\newcommand{\R}{\mathbf{R}}
\newcommand{\E}{\mathbb{E}}
\newcommand{\0}{\textbf{0}}
\newcommand{\1}{\textbf{1}}
\newcommand{\wt}{\operatorname{wt}}
\newcommand{\B}{\mathcal{B}}
\newcommand{\nl}{\mathrm{nl}}
\newcommand{\TRACE}{\operatorname{Tr}_1^k}
\newcommand{\TrN}{\operatorname{Tr}_1^n}
\theoremstyle{plain}

\newtheorem{lemma}{Lemma}
\newtheorem{theorem}{Theorem}
\newtheorem{remark}{Remark}
\newtheorem{corollary}{Corollary}
\newtheorem{definition}{Definition}
% \theoremstyle{nonumberplain}



\newtheorem{construction}{Construction}
% \newcommand{\Tr}{\mathrm{Tr}_1^n}
% \newcommand{\tr}{\mathrm{Tr}_1^k}

\title{ 2022 second-order of simplest \mathcal{PS} bent function}




\begin{document}
%   \maketitle
  \noindent
  \rule{\linewidth}{0.4pt}


\section{Introduction}
    Boolean functions play an important role in the design of the symmetric cryptography and coding theory, see \cite{Carlet2007book,BhattacharyyaKSSZ2010gowers,CohenHLL1997RMcodecover}. 
    The $ r $-th order nonlinearities of Boolean functions are of great interest, for being the most important cryptographic criteria for the symmetric cryptography. 
    This cryptographic criterion, denoted by $ nl_r(f) $, measures the minimum Hamming distance of the Boolean function $ f $ in $ n $ variables to the set of all functions in $ n $ variables of algebraic degree at most $ r\le n $, where $ r $ is a positive number. 
    The Boolean functions used in symmetric cryptography must have high $ r $-th order nonlinearities for being against attacks illustrated in several papers \cite{Golic1996lower_order_approximation,IwataK1999highorderbentfunction,KnudsenR1996nonlinear_approximation,Courtois2002XL_algorithm_and_NL_r}. 
    In coding theory, $ nl_r(f) $ equals the distance from $ f $ to the Reed-Muller code $ \mathcal{RM}(r,n) $ of length $ 2^n $ and of order $ r $. 
    Thus, the maximum $ r $-th order nonlinearity of all Boolean functions in $ n $ variables equals the covering radius of $ \mathcal{RM}(r,n) $ \cite{CohenHLL1997RMcodecover}. 
    This parameter is also related to the Gowers norm in theoretical computer science, since the correlation between a Boolean function $ f $ to the closest degree $ d $ polynomial is at most its $ (d+1) $-th Gowers norm \cite{BhattacharyyaKSSZ2010gowers}.
    % , coincides with the definition of the $ r $-th order nonlinearity of $ f $. 

    However, computing the $ r $-th order nonlinearity of a given Boolean function with algebraic degree strictly greater than $ r $ is a hard task for $ r>1 $, even the second-order nonlinearity is known only for a few peculiar functions and functions with small number of variables. 
    Fortunately, in the case of $ r=1 $ (we call the first order nonlinearity of $ f $ by the nonlinearity of $ f $ and denote it by $ nl(f) $ instead of $ nl_1(f) $), the nonlinearity is related with Walsh transform, which can be computed by the algorithm. 
    For the second-order nonlinearity, Kabatiansky and Tavernier \cite{KabatianskyT2005listdecoding_RM_2_n} proposed an algorithm using list decoding of second-order Reed-Muller codes. 
    Soon, Fourquet and Tavernier improved and implemented this algorithm to quadratic Boolean functions up to $ n=11 $ and some special quadratic functions up to $ n=13 $ in \cite{FourquetT2008improved_listdecoding_RM_2_n}.

    It is also a difficult task that proving lower bounds on the $ r $-th order nonlinearity of functions, even for $ r=2 $. 
    In \cite{Carlet2008lowbound_NL_profile}, Carlet gives two lemmas about lower bounds on the nonlinearity profile of a Boolean function by a recursive approach. Meanwhile, he derives some lower bounds on the nonlinearity profiles of Maiorana-McFarland, Welch, Kasami and inverse functions. 
    Thanks to Carlet's recursive approach, numerous authors have obtained lower bounds on $ r $-th order nonlinearities of special functions, mostly for $ r=2,3 $ \cite{YanT2020NL_2,Liu2023NL_2,TangYZZ2020NL_2bent,SihemKJ2020NL_2cubic,SunW2009NL_2,SarkarG2009NL_2MM,GangopadhyayST2010NL_2,GodeG2010NL_3Kasami,SunW2011NL_2,TangCT2013NL_2bent,Singh2014NL_3_biquadratic,GaoT2017NL_2_MM}. 

    In this article, we derive a lower bound on the third-order nonlinearity of the simplest $ \mathcal{PS} $ bent functions $ f(x,y) $.  
    Using the Carlet's lemma twice, we only need to determine the nonlinearities of the second-order derivatives of $ f(x,y) $ for all possible pairs $ (\alpha,\beta) $. 
    To obtain the nonlinearities of $ D_{\beta}D_{\alpha}f(x,y) $, the Walsh transform makes it equivalent to determine the values of the special character sums with all cases of $ (\alpha,\beta)\in\F_{2^k}^2 $. 
    By simple algebraic geometry lemmas and the calculation, an upper bound on the nonlinearities of $ D_{\beta}D_{\alpha}f(x,y) $ for all trivial cases is derived. 
    While for general cases, we obtain the nonlinearities of $ D_{\beta}D_{\alpha}f(x,y) $ by determining the number of solutions of the system of trace functions. 
    We derive then, explictly and straightforward, a lower bound on the third-order nonlinearity of the simplest $ \mathcal{PS} $ bent functions $ f(x,y) $. 

    In the present paper, we derive a lower bound on the third-order nonlinearity of the simplest $ \mathcal{PS} $ bent functions $ f(x,y)=\TRACE\left( \frac{\lambda x}{y} \right) $. We improve the previous lower bounds in \cite{TangCT2013NL_2bent,Carlet2011NL_Profile_Dillon}, and the comparsion of those three lower bounds can be found in Table \ref{table:MyTableLabel} for small concrete values. 
    Specifically, our lower bound is asymptotically equivalent to $ 2^{n-1}-2^{\frac{7n}{8}-\frac{1}{2}} $, whose 
    % comparing with the best known lower bound, Carlet bound in \cite{Carlet2011NL_Profile_Dillon}, 
    improvement is approximate $ (\sqrt{2}-1)2^{\frac{7n}{8}-\frac{1}{2}} $, compared with Carlet bound. 
    Additionally, the lower bound we obtained is much more efficient than the previous lower bounds when $ n $ is small as we have showed.  

    The remainder of this paper is organized as follows.  
    Section 2 introduces some basic notions which will be used in the manuscript. 
    In Section 3, we present some lemmas, which are useful in Section 4.  
    Based on known lemmas, Section 4 determines the lower bounds on the third-order nonlinearities of the simplest $ \mathcal{PS} $ bent functions. 
    Finally, Section 5 is a conclusion. 
    Throughout this work, for any integer $ n>0 $, let $ \F_{2^n} $ denote the finite field with $ 2^n $ elements. For any integers $ m\mid n $, denote $ \operatorname{Tr}_m^n $ the trace function from $ \F_{2^n} $ to $ \F_{2^m} $. For any set $ S $, $ \#S $ denotes the cardinality of $ S $. And for the integer $ n $, denote $ |n| $ the absolute value of $ n $. 

\section{Preliminaries}

    Let $ \F_2 $ be the field with two elements $ \{0,1\} $ and $ \F_2^n $ be the vector space of $ n $-tuples over $ \F_2 $. 
    Let $ \F_{2^n} $ be the finite field of order $ 2^n $, which can be viewed as an $ n $-dimensional vector space over $ \F_2 $.
    The Hamming weight of the vector $ \bm{a}=(a_1,\dots,a_n)\in\F_2^n $ is $ \wt(\bm{a})=\#\left\{ 1\le i\le n\mid a_i\ne 0 \right\} $. Meanwhile, the Hamming weight of an integer $ a $ is the Hamming weight of the binary expansion of $ a $, that is, $ \wt(\bar{a}) $, where $ \bar{a}=(a_1,\dots,a_n)\in\F_2^n $ and $ a=\sum_{i=0}^{n-1}a_i2^i $. 
    The Hamming distance between two vectors $ \bm{a},\bm{b}\in\F_2^n $ is defined as $ d_H(\bm{a},\bm{b})=\wt(\bm{a}\oplus\bm{b}) $, where $ \oplus $ is the addition in $ \F_2^n $.  
    We call $ n $-variable Boolean functions the functions from $ \F_2^n $ to $ \F_2 $. 
    The set of all $ n $-variables Boolean functions from $ \F_2^n $ to $ \F_2 $ will be denoted by $ \mathcal{B}_n $. 

    Any $ f\in\mathcal{B}_n $ with variables $ \bm{x}=(x_1,\dots,x_n)\in\F_2^n $ can be  represented by its algebraic normal form (ANF)
    \[f(x_1,\dots,x_n) = \bigoplus_{u\in\F_2^n}a_{u}\bm{x}^{u},\]
    where $ a_u\in\F_2 $ and the term $ \bm{x}^{u}=\prod_{i=1}^nx_j^{u_j} $ is called a monomial.
    The algebraic degree of the Boolean function $ f $ is denoted by $ \deg(f)=\max\left\{ W_H(u)\mid a_u\ne 0 \right\} $. 
    Note that a Boolean function is affine if and only if it has algebraic degree at most $ 1 $. 
    
    %  and defined as the maximum $ 2 $-weight of the exponents in the univariate polynomial, where the $ 2 $-weight is the Hamming weight of the binary expansion (the Hamming weight of the binary expansion of $ i $ is the number of $ 1 $'s in the binary expansion of $ i $).  
    % We denote by $ \mathcal{B}_n $ the set of all the Boolean functions in $ n $ variables. 
    % A Boolean function in $ n $ variables is a function from $ \F_2^n $ into $ \F_2 $. 
    % the set of Boolean functions

    Due to the isomorphism between the finite field $ \F_{2^n} $ and the vector space $ \F_2^n $, 
    any $ n $-variable Boolean function can also be defined over $ \F_{2^n} $ and represented uniquely as a univariate polynomial over $ \F_{2^n} $, that is with variable $ x\in\F_{2^n} $,  
    \[f(x) = \sum_{i=0}^{2^n-1}\delta_ix^i,\] 
    where $ \delta_0,\delta_{2^n-1}\in\F_2 $ and for every $ i=1,2,\dots,2^n-2 $, $ \delta_{2i\pmod{2^n-1}}=\delta_i^2 $. 
    In this case, the algebraic degree of $ f $ is $ \max\left\{ \wt(\bar{i})\mid \delta_i\ne 0, 1\le i\le n \right\} $. 
    % One of the most important representations of an
    % is the algebraic normal form (ANF): \[f(x_1,x_2,...,x_n)=\sum_{I\subseteq\left\{ 1,2,...,n \right\}}a_I\prod_{i\in I}x_i,\] where all $ a_I $ belong to $ \F_2 $. 

    
    % \[\deg(f)=\max\left\{ |I|:a_I\ne 0 \right\},\]
    % where $ |I| $ denotes the size of $ I $.     

    % and functions with algebraic degree no greater than $ 2 $ are always called quadratic functions.  
    % Due to the identification between the finite field $ \F_{2^n} $ and the vector space $ \F_2^n $, we can represent any Boolean function as a polynomial in one variable $ x\in\F_{2^n} $ of the form $ f(x)=\sum_{i=0}^{2^n-1}f_ix^i $, where $ f_i $'s are elements of the field. 
    Specifically, when $ n $ is even, the third representation of an $ n $-variable Boolean function $ f $ is possible, which is a bivariate polynomial of the form $ f(x,y)=\sum_{0\le i,j\le 2^{n/2}-1}f_{i,j}x^iy^j $, where $ f_{i,j} $'s are elements of the field $ \F_{2^{n/2}} $ and $ (x,y)\in\F_{2^{n/2}}^2 $.  
    The algebraic degree under bivariate polynomial representation is defined as $ \max\left\{ \wt(\bar{i})+\wt(\bar{j})\mid f_{i,j}\ne 0, 1\le i,j\le n \right\} $. 
    Note that all of three definition of the algebraic degree are the same. 
    Besides, the $ r $-th order Reed-Muller code of length $ 2^n $ is denoted by $ \mathcal{RM}(r, n) $. 
    It can be presented by the set of $ n $-variable Boolean functions of algebraic degree not greater than $ r $. 
    % The Hamming weight $ w_H(f) $ of a Boolean function is the size of its support, $ \left\{ x\in\F_{2^n}\mid f(x)=1 \right\} $. 
    % And the Hamming distance between two $ n $-variable Boolean functions is the Hamming weight of $ f\oplus g $. 
    With above knowledgment, we introduce the definition of $ r $-th order nonlinearity. 

    \begin{definition}
        Let $ f $ be an $ n $-variable Boolean function. Let $ r $ be a positive integer such that $ r<n $. 
        The $ r $-th order nonlinearity of $ f $ is the minimum Hamming distance from $ f $ to all elements of $ \mathcal{RM}(r,n) $. 
        % whether introduce the definition of RM(r,n)?
        We denote the $ r $-th order nonlinearity of $ f $ by $ nl_r(f) $.
    \end{definition}

    % Note that the lower bound of the third-order nonlinearity of a Boolean function relates to the nonlinearities of the second derivative of the function.
    To bound the third-order nonlinearity of a Boolean function, we must consider the nonlinearities of its second-order derivatives.

    \begin{definition}
        The first-order derivative of a Boolean function $ f $ over $ \F_{2^n} $ at the point of $ \alpha\in\F_{2^n} $ is defined as $ D_{\alpha}f(x)=f(x)+f(x+\alpha) $. 
        And the second-order derivative of a Boolean function $ f $ in the pair of $ (\alpha,\beta)\in\F_{2^n}^2 $ is defined as $ D_{\beta}D_{\alpha}f(x)=f(x)+f(x+a)+f(x+\beta)+f(x+\alpha+\beta) $.
    \end{definition}

    As mentioned in the introduction, one of the essential tools to determine $ nl(f) $ for any Boolean function $ f $ is called Walsh transform. 
    \begin{definition}
        Let $ f $ be a Boolean function over $ \F_2^n $. The Walsh transform of $ f $ at the point $ \bm{a}\in\F_2^n $ is defined as 
        \[W_f(\bm{a})=\sum_{\bm{x}\in\F_2^n}(-1)^{f(\bm{x})+\bm{a}\cdot\bm{x}},\]
        where $ \bm{a}\cdot \bm{x} $ is the usual inner product in $ \F_2^n $. 

        If $ f $ is over $ \F_{2^n} $, the Walsh transform of $ f $ at the point $ \alpha\in\F_{2^n} $ is defined as 
        \[W_f(\alpha)=\sum_{x\in\F_{2^n}}(-1)^{f(x)+\TrN(\alpha x)}.\]
        In addition, if $ n=2k $ is even and $ f(x,y) $ is the bivariate polynomial form of $ f $, the Walsh transform of $ f $ at $ (\alpha,\beta)\in\F_{2^k}^2 $ is defined as 
        \[W_f(\alpha,\beta)=\sum_{(\alpha,\beta)\in\F_{2^k}^2}(-1)^{f(x,y)+\Tr_1^k(\alpha x+\beta y)}.\]
    \end{definition}
    Therefore, the nonlinearity of a Boolean function $ f\in\mathcal{B}_n $ can be computed as 
    \begin{align*}
        nl(f) &= 2^{n-1} - \frac{1}{2}\max_{a\in\F_2^n}|W_f(a)|\\
              &= 2^{n-1} - \frac{1}{2}\max_{\alpha\in\F_{2^n}}|W_f(\alpha)|\\
              &= 2^{n-1} - \frac{1}{2}\max_{(\alpha,\beta)\in\F_{2^k}^2}|W_f(\alpha,\beta)|,\text{~if~}n=2k\text{~is~even}. 
    \end{align*}
    The above equation points out the importance of Walsh transform for computing the nonlinearity of a Boolean function.
    
    To achieve the goal of efficiently bounding the $ r $-order nonlinearity of a given function, a lemma derived in \cite{Carlet2008lowbound_NL_profile} are introduced, when lower bounds exist for the $ (r-1) $-th order nonlinearities of the derivatives of $ f $: 
    \begin{lemma}\label{thm:High_order_nl_bound1}
        Let $ f $ be an $ n $-variable Boolean function, and let $ 0<r<n $ be an integer. We have 
        \[nl_r(f)\ge 2^{n-1}-\frac{1}{2}\sqrt{2^{2n}-2\sum_{a\in\F_2^n}nl_{r-1}(D_af)}.\] 
    \end{lemma}
    % Indeed, Mesnager et al. in  construct a family of Boolean functions, whose second-order nonlinearity possesses the property that the latter bound is sharp.  
    We ignore another lemma in \cite{Carlet2008lowbound_NL_profile}, since the bound deduced from it, in general, is not tighter than the bound derived from Lemma \ref{thm:High_order_nl_bound1}.
    
    There is a class of character sums with polynomial arguments needed to be treated in the proof of the main theorem.  
    To solve this, we need the following lemma about algebraic geometry. 
    % the genus of the algebraic function field. 
    \begin{lemma}[\cite{Stichtenoth2008book_algebraicfunctionfieldsandcodes}]\label{L:genus_K_F}
        Let $ K=F(X,Y) $, where $ X,Y $ are transcendentals over $ F $. 
        Then the genus $ g $ of the function field $ K/F $ satisfies: 
        \[g\le ([K : F(X)] - 1)([K : F(Y)] - 1).\]
    \end{lemma}
        % The Walsh spectrum of $ f $ is the multiplicity set of the values of the Walsh transform of $ f $. 
    % We have the relation 
    % \[nl(f)=2^{n-1}-\frac{1}{2}\max_{\alpha\in\F_{2^n}}\left\lvert W_f(\alpha)\right\rvert. \]  
    % \begin{lemma}\label{L:Parsevalrelation}
    % For any $ n $-variable Boolean function $ f $, we have \[\sum_{\alpha\in\F_{2^n}}W_f(a)^2=2^{2n}.\]
    % \end{lemma}


% \begin{lemma}\label{L:autoshift}
% Let $f$ be an arbitrary $n$-variable Boolean function. For any $\alpha,\beta\in\F_{2^n}*$, we have
% $$\sum_{x\in\F_{2^n}}(-1)^{f(\alpha x)+f(\beta x)}=2^{-n}\sum_{u\in\F_{2^n}}\widehat{f}(\alpha^{-1}u)\widehat{f}(\beta^{-1}u).$$
% \end{lemma}

\section{The multiplicative inverse function}
    In this section, we are going to provide a prelude about the multiplicative inverse function, which is useful for proving our main theorem. 
%     For any integer $n>0$, let us define $I_\nu(x)=\TrN(\nu x^{-1})$ over $\B_n$.
%     The Kloosterman sums over $\F_{2^n}$ are defined as
%     $\mathcal{K}(a)=\widehat{I_1}(\alpha)=\sum_{x\in\F_{2^n}}(-1)^{\TrN(x^{-1}+\alpha x)}$, where $\alpha\in\F_{2^n}$.
%     In fact, the Kloosterman sums are generally defined on the multiplicative
%     group $\F_{2^n}^*$. We extend them to $0$ by assuming $(-1)^0=1$.

%    \begin{proof} 
%     For any $\mu,\nu,\tau\in\F_{2^n}^*$,
%     we have (still using the convention $\frac 10=0$)
%     \begin{eqnarray*}
%     &&C_{{\mu,\nu}}(\tau)\\
%     &=&\sum_{x\in\F_{2^n}}(-1)^{\TrN(\frac{\mu}{x}+\frac{\nu}{x+\tau})}\\
%     &=&\sum_{x\in\F_{2^n}\setminus\{0,\tau\}}(-1)^{\TrN(\frac{\mu}{x}+\frac{\nu}{x+\tau})}+(-1)^{\TrN(\frac{\mu}{\tau})}+(-1)^{\TrN(\frac{\nu}{\tau})}\\
%     &=&\sum_{x\in\F_{2^n}\setminus\{0,\tau^{-1}\}}(-1)^{\TrN(\mu x+\frac{\nu x}{1+\tau x})}+(-1)^{\TrN(\frac{\mu}{\tau})}+(-1)^{\TrN(\frac{\nu}{\tau})}\\
%     &=&\sum_{x\in\F_{2^n}\setminus\{0,\tau^{-1}\}}(-1)^{\TrN(\mu x+\frac{1}{1+\tau x}\cdot\frac{\nu}{\tau}+\frac{\nu}{\tau})}+(-1)^{\TrN(\frac{\mu}{\tau})}+(-1)^{\TrN(\frac{\nu}{\tau})}\\
%     &=&\sum_{x\in\F_{2^n}\setminus\{0,1\}}(-1)^{\TrN(\frac{\mu x}{\tau}+\frac{\nu}{\tau x}+\frac{\mu}{\tau}+\frac{\nu}{\tau})}+(-1)^{\TrN(\frac{\mu}{\tau})}+(-1)^{\TrN(\frac{\nu}{\tau})}\\
%     &=&\sum_{x\in\F_{2^n}\setminus\{0,\frac{\tau}{\nu}\}}(-1)^{\TrN(\frac{1}{x}+\frac{\mu \nu}{\tau^2}x)+\TrN(\frac{\mu}{\tau}+\frac{\nu}{\tau})}+(-1)^{\TrN(\frac{\mu}{\tau})}+(-1)^{\TrN(\frac{\nu}{\tau})}\\
%     &=&\sum_{x\in\F_{2^n}}(-1)^{\TrN(\frac{1}{x}+\frac{\mu \nu}{\tau^2}x)+\TrN(\frac{\mu}{\tau}+\frac{\nu}{\tau})}-
%     (-1)^{\TrN(\frac{\mu}{\tau}+\frac{\nu}{\tau})}-(-1)^{\TrN(0)}+(-1)^{\TrN(\frac{\mu}{\tau})}+(-1)^{\TrN(\frac{\nu}{\tau})}
%    \end{eqnarray*}
%    where the third, fifth, and sixth identities hold by changing $x$ to ${1\over x}$, ${x+1\over \tau}$, and ${\nu x\over \tau}$ respectively.
%    Note that $-(-1)^{\TrN(\frac{\mu}{\tau}+\frac{\nu}{\tau})}-(-1)^{\TrN(0)}+(-1)^{\TrN(\frac{\mu}{\tau})}+(-1)^{\TrN(\frac{\nu}{\tau})}$
%    equals $0$ or $-4$. According to Lemma~\ref{inverse-nl}, we can see that $C_{{\mu,\nu}}(\tau)$ belongs to $[-2^{{n/2}+1}-3, 2^{{n/2}+1}+1]$ and is divisible by $4$.
%    This finishes the proof.
%    \end{proof}




% \subsection{The multiplicative inverse function}
    For any finite field $\F_{2^n}$, the multiplicative inverse function of $\F_{2^n}$, denoted by $I$, is defined as $I(x)=x^{2^n-2}$. In the sequel, we will use $x^{-1}$ or $\frac{1}{x}$ to
    denote $x^{2^n-2}$ with the convention that $x^{-1}=\frac{1}{x}=0$ when $x=0$. We recall that, for any $v \neq 0$, $I_v(x) = \mathrm{Tr}_1^n(vx^{-1})$ is a component function of $I$.
    The Walsh transform of $I_1$ at any point $\alpha$ is commonly known as Kloosterman sum over $\F_{2^n}$ at $\alpha$, which is usually denoted by $\mathcal{K}(\alpha)$,
    \emph{i.e.}, $\mathcal{K}(\alpha)=\widehat{I_1}(\alpha)=\sum_{x\in\F_{2^n}}(-1)^{\mathrm{Tr}_1^n(x^{-1}+\alpha x)}$.
    The original Kloosterman sums are generally defined on the multiplicative group $\F_{2^n}^*$. We extend them to $0$ by assuming $(0)^{-1}=1$. Regarding the Kloosterman sums,
    the following results are well known and we will use them in the sequel.
% \begin{lemma}\cite{CarlitzKloo1969}
% \label{L:Kloostermansumsone}
% For any integer $n>0$, $\widehat{I_1}(1)=1-\sum_{t=0}^{\lfloor n/2\rfloor}(-1)^{n-t}\frac{n}{n-t}{{n-t}\choose {t}}2^t$.
% \end{lemma}
% \begin{lemma}\cite{LW90}
% \label{inverse-nl}
% For any positive integer $n$ and arbitrary $a\in\F_{2^n}^*$, the Walsh--Hadamard spectrum of $I_1(x)$ defined on $\F_{2^n}$ can take any value divisible by $4$ in the range
% $[-2^{{n/2}+1}+1,2^{{n/2}+1}+1]$.
% \end{lemma}
% Let $n=2t+1$ be an odd integer and $P$ be the largest positive integer such that $P \equiv 0 \pmod 4$ and $P\leq 2^{t+1}\sqrt{2}+1$.
% \begin{remark}
% \label{rem-max-min}
% The possible maximum absolute value of Walsh--Hadamard spectrum of $I_1$ over $\mathbb F_{2^n}$ is
% \begin{eqnarray*}
% \max_{\alpha\in\mathbb F_{2^n}^*}|\widehat{I_1}(\alpha)|=\left\{
% \begin{array}{llll}
% 2^{\frac{n}{2}+1}, &\mbox{ if } n \mbox{ is even}\\
% P, &\mbox{ if } n \mbox{ is odd}
% \end{array}
% \right.,
% \end{eqnarray*}
% where $P$ is as defined above.
% \end{remark}
% \subsection{Basic Lemmas}
% \begin{lemma}[\cite{MS1977}]\label{L:solutiondegree2}
% For any $(\alpha,\beta)\in\F_{2^n}^*\times\F_{2^n}$, we define a polynomial
% $\mu(x)=\alpha x^2+\beta x+\gamma\in\F_{2^n}[x]$.
% Then the equation $\mu(x)=0$ has 2 solutions if and only if $\mathrm{Tr}_1^n(\beta^{-2}\alpha\gamma)=0$.
% \end{lemma}
% \begin{lemma}\label{L:tr0sum}
% Let $n$ be a positive integer and $T_0=\{\upsilon^2+\upsilon : \upsilon\in\F_{2^n}\}$.
% Then we have
% \begin{eqnarray*}
% \sum_{x\in T_0}(-1)^{\Tr\left(\frac{1}{x+1}\right)}=\frac12(-1)^{n}\widehat{I_1}(1).
% \end{eqnarray*}
% \end{lemma}


% \begin{lemma}\label{L:rootssum}
% Let $n$ be a positive integer. We have
% \begin{eqnarray*}
% \sum_{\upsilon\in\F_{2^n}\setminus\F_2}(-1)^{\mathrm{Tr}_1^n
% \left(\frac{{\upsilon}^2}{\upsilon^2+\upsilon+1}\right)}=\sum_{\upsilon\in\F_{2^n}\setminus\F_2}(-1)^{\mathrm{Tr}_1^n\left(\frac{{\upsilon}^2+1}{\upsilon^2+\upsilon+1}\right)}=(-1)^n\left(\widehat{I}_1(1)-2\right).
% \end{eqnarray*}
% \end{lemma}

    \begin{lemma}[\cite{tang2022invfunc}]\label{L:SumInv00}
        Let $n\geq 3$ be an arbitrary integer. We define
        $$L=\#\left\{c\in\F_{2^n} : \mathrm{Tr}_1^n\left(\frac{1}{c^2+c+1}\right)=\mathrm{Tr}_1^n\left(\frac{c^2}{c^2+c+1}\right)=0\right\}.$$
        Then we have $L=2^{n-2}+\frac{3}{4}(-1)^n\widehat{I_1}(1)+\frac{1}{2}\left(1-(-1)^n\right)$, where $ \widehat{I_1}(1)=1-\sum_{t=0}^{\lfloor n/2\rfloor}(-1)^{n-t}\frac{n}{n-t}{{n-t}\choose {t}}2^t $.
        % where $\widehat{I_1}(1)$ can be computed using Lemma~\ref{L:Kloostermansumsone}.
    \end{lemma}

    Let $ f $ be a function from $ \F_{2^n} $ to $ \F_{2^m} $. For any $ \gamma,\eta\in\F_{2^n} $ and $ \omega\in\F_{2^m} $, let us define
    \[\mathcal{S}_f(\gamma,\eta,\omega)=\left\{x\in\F_{2^n} : f(x)+f(x+\gamma)+f(x+\eta)+f(x+\eta+\gamma)=\omega\right\},\] 
    associated with $ \mathcal{N}_f(\gamma,\eta,\omega)=\#\mathcal{S}_f(\gamma,\eta,\omega) $. 
    % It is clear that for $\gamma=0$ or $\eta=0$ or $\gamma=\eta$, we have $\mathcal{N}_F(\gamma,\eta,0)=2^n$, and when $\omega\neq 0$, $\mathcal{N}_F(\gamma,\eta,\omega)=0$. 
    % If $F$ is the multiplicative inverse function over $\mathbb F_{2^n}$, we denote $\mathcal{N}_I(\gamma,\eta,\omega)$ by $\mathcal{N}(\gamma,\eta,\omega)$.

    % \begin{lemma}\cite{tang2022invfunc}\label{Secondderivativesolution}
    % Let $n\geq 3$ be a positive integer and $\mathcal{N}(\gamma,\eta,\omega)$ be defined as in \eqref{2nddrivative}.
    % Let $\gamma,\eta$ be two elements of $\F_{2^n}^*$ such that $\gamma\neq \eta$. Then for any $\omega\in\F_{2^n}$,  we have $\mathcal{N}(\gamma,\eta,\omega)\in \{0,4,8\}$.
    % Moreover, the number of $(\gamma,\eta,\omega)\in\F_{2^n}^3$ such that $\mathcal{N}(\gamma,\eta,\omega)=8$ is
    % $$\left(2^{n-2}+\frac{3}{4}(-1)^{n}\widehat{I_1}(1)-\frac{5}{2}(-1)^{n}-\frac{3}{2}\right)\left(2^n-1\right).$$
    % \end{lemma}

    \begin{lemma}\label{lemma:num_sol_second_dev}
        Let $ n\ge 3 $ be an integer. For $ \alpha,\beta,\mu\in\F_{2^n} $ with $ \lambda\in\F_{2^n}^* $, define $ f(x)=\lambda x^{2^n-2} $, then 
        \[S_f(\alpha,\beta,\mu)=\left\{ x\in\F_{2^n}:\frac{\lambda}{x}+\frac{\lambda}{x+\alpha}+\frac{\lambda}{x+\beta}+\frac{\lambda}{x+\alpha+\beta}=\mu \right\},\]
        $ \mathcal{N}_f(\alpha,\beta,\mu) $ can be determined: 
        \begin{enumerate}[label=(\arabic{*})]
            \item If $ \alpha=\beta\in\F_{2^n}^* $ or $ \alpha=0 $ or $ \beta=0 $, 
            then $ \mathcal{N}_f(\alpha,\beta,\mu)=2^n $ when $ \mu = 0 $ and $ \mathcal{N}_f(\alpha,\beta,\mu)=0 $ when $ \mu\in\F_{2^n}^* $.
            \item If $ \alpha,\beta\in\F_{2^n}^* $ such that $ \alpha\ne\beta $, we have: 
            \begin{enumerate}[ref=(\alph{*})]
                \item If $ \lambda(\alpha^2+\beta^2+\alpha\beta)+\mu(\alpha^2\beta+\alpha\beta^2)=0 $, we have 
                $ \{0,\alpha,\beta,\alpha+\beta\}\subseteq\mathcal{S}_f(\alpha,\beta,\mu) $.\label{item_a}
                \item If $ \mu\ne 0 $, $ \TrN\left(\frac{\lambda\alpha}{\mu \beta(\alpha+\beta)}\right)=0 $ and 
                $ \TrN\left(\frac{\lambda \beta}{\mu \alpha(\alpha+\beta)}\right)=0 $, we have 
                $ \{y_0,y_0+\alpha,y_0+\beta,y_0+\alpha+\beta\}\subseteq\mathcal{S}_f(\alpha,\beta,\mu) $, where $ y_0\notin\{0,\alpha,\beta,\alpha+\beta\} $.\label{item_b}
                \item If both conditions \ref{item_a} and \ref{item_b} cannot hold, $ \mathcal{S}_f(\alpha,\beta,\mu)=\emptyset $ and then $ \mathcal{N}_f(\alpha,\beta,\mu)=0  $.
            \end{enumerate}
        \end{enumerate}
    \end{lemma}
    \begin{proof}
        The proof is analogue to the proof of Lemma 13 in \cite{tang2022invfunc} and we omit it.
    \end{proof}
    \begin{remark}
        For any $ \alpha\in\F_{2^n}^* $, there exist $ L $ different $ \beta $ 
        such that $ \mathcal{N}_f(\alpha,\beta,\mu)=8 $ for some $ \lambda\in\F_{2^n}^* $ and $ \mu\in\F_{2^n} $. 
        Indeed, the conditions $ \alpha,\beta\in\F_{2^n}^* $, $ \alpha\ne\beta $ and $ \mu\ne 0 $ 
        can tell us $ \mu(\alpha^2\beta+\alpha\beta^2)\ne 0 $, 
        resulting in $ \lambda(\alpha^2+\beta^2+\alpha\beta)\ne 0 $, 
        which implies $ \frac{\beta}{\alpha}\notin\F_4 $.
        So take $ \mu=\frac{\lambda(\alpha^2+\beta^2+\alpha\beta)}{\alpha^2\beta+\alpha\beta^2} $ 
        into $ \TrN\left(\frac{\lambda \alpha}{\mu \beta(\alpha+\beta)}\right)=0 $ 
        and $ \TrN\left(\frac{\lambda \beta}{\mu \alpha(\alpha+\beta)}\right)=0 $ respectively, 
        we have $ \TrN\left(\frac{1}{\gamma^2+\gamma+1}\right)=0 $ and $ \TrN\left(\frac{\gamma^2}{\gamma^2+\gamma+1}\right)=0 $, 
        where $ \gamma=\frac{\beta}{\alpha}\in\F_{2^n}\setminus\F_{4} $. 
        Therefore, according to Lemma \ref{L:SumInv00}, 
        the number of $ \gamma=\frac{\beta}{\alpha}\in\F_{2^n}\setminus\F_{4} $ satisfying 
        $ \TrN\left(\frac{1}{\gamma^2+\gamma+1}\right)=0 $ and $ \TrN\left(\frac{\gamma^2}{\gamma^2+\gamma+1}\right)=0 $ 
        is $ L $.
    \end{remark}
   


\section{The third-order nonlinearity of the simplest $ \mathcal{PS} $ bent function}
    First, we begin with the definition of Dillon $ \mathcal{PS} $ bent fucntions in this section. 
    Then, the estimation for a class of character sums is given with the use of algebraic geometry. 
    The lemmas, about the number of solutions for systems of trace functions, will be proved in the sequel. 
    Therefore, we can determine the nonlinearities of the second-order derivatives of the given function. 
    As a result, the lower bound on the third-order nonlinearity of the simplest $ \mathcal{PS} $ bent function can be derived, which is tigher than the known lower bounds as illustrated in the table.
    \subsection{Dillon $ \mathcal{PS} $ bent functions}
    Dillon presented a $ \mathcal{PS} $ bent function class $ f(x,y) $ from $ \F_{2^n}=\F_{2^k}^2 $ 
    to $ \F_2 $ as 
    \begin{equation*}\label{Eqn_PS_bent}
        \mathcal{D}(x,y)=g\left({x\over y}\right),
    \end{equation*}
    where $ g $ is a balanced Boolean function on $ \F_{2^{k}} $ with $ g(0)=0 $, 
    and ${ x\over y }$ is defined to be $ 0 $ if $ y=0 $ 
    (we shall always assume this kind of convention in the sequel).

    In this paper, our goal is to give a lower bound on the third-order nonlinearity of the simplest 
    $ \mathcal{PS} $ bent function, \emph{i.e.}
    \begin{equation*}\label{sub-bent}
        f(x,y)=\TRACE\left(\frac{\lambda x}{y}\right),
    \end{equation*}
    where $ (x,y)\in\F_{2^k}^2 $, $ \lambda\in\F_{2^k}^{*} $ 
    and $ \TRACE(x)=\sum\limits_{i=0}^{n-1}x^{2^i} $ is the trace function from
    $ \F_{2^k} $ to $ \F_2 $.

    \subsection{The nonlinearities of the second-order derivatives of the simplest $ \mathcal{PS} $ bent function}

    Before dealing with the nonlinearities of the second-order derivatives of the simplest $ \mathcal{PS} $ bent function, 
    We first introduce some useful lemmas about a class of character sums and the number of solutions for systems of trace functions that are needed in the sequel.

    % We explictly give the number of solutions of systems of two or three trace functions since  
    % trace functions are balanced and $ \F_2 $-linear in finite field $ \F_{2^n} $.

    \begin{lemma}
        Let $ k\ge 3 $ be a positive integer and assume 
        \[S(\alpha,\beta,v)=\sum_{x\in\F_{2^k}}(-1)^{\TRACE\left( \frac{\alpha}{x+\beta}+\frac{\alpha}{x}+vx \right)},\]
        where $ \alpha,\beta,v\in\F_{2^k}^* $. We have 
        \[\left\lvert S(\alpha,\beta,v)\right\rvert\le 2\left\lfloor 2^{\frac{k}{2}+1}\right\rfloor+4 .\]
    \end{lemma}
    \begin{proof}
        Note that $ S(1,\beta/\alpha,v\alpha)=S(\alpha,\beta,v) $.
        Determine the value of $ S(\alpha,\beta,v) $ is equivalent to determining the number of $ x\in\F_{2^k} $ 
        for which $ \TRACE\left( \frac{1}{x+\beta}+\frac{1}{x}+vx \right)=0 $. 
        By Hilbert's Theorem 90, this is equivalent to determining the number of solutions $ (x,y) $ 
        in $ \F_{2^k} $ of $ y^2+y=\frac{1}{x+\beta}+\frac{1}{x}+vx $.
        
        Let us define 
        \[\mathcal{S}_{\beta,v}=\left\{(x,y)\in\F_{2^k}\times\F_{2^k}:y^2+y=\frac{1}{x+\beta}+\frac{1}{x}+vx\right\}.\]
        Note that $ y\mapsto y^2+y $ is $ 2 $-to-$ 1 $, then we have 
        \begin{equation}\label{eq:tracesum_S}
            S(\alpha,\beta,v)=\frac{\#\mathcal{S}_{\beta,v}}{2}-\left(2^k-\frac{\#\mathcal{S}_{\beta,v}}{2}\right)=\#\mathcal{S}_{\beta,v}-2^k.
        \end{equation}
        Note that $ \#\mathcal{S}_{\beta,v} $ is even and then $ S(\alpha,\beta,v) $ must be even, too. 
        Consider the function field $ K=\F_{2^k}(x,y) $ with defining equation 
        \begin{equation}\label{eq:trace_curve}
            y^2+y=\frac{1}{x+\beta}+\frac{1}{x}+vx.
        \end{equation}
        By Lemma \ref{L:genus_K_F}, we can easily obtain that the genus of $ K $ is not greater than $ 2-\delta_v $, 
        where $ \delta_v=1 $ if $ v=0 $ and $ \delta_v=0 $ otherwise. 
        Denote by $ \mathcal{N} $ the number of the places with degree one of $ K/\F_{2^k} $. 
        Then by Serre bound\cite{Serre1982serrebound}, we have 
        \begin{equation}\label{eq:N_genus_inequality}
            \left\lvert \mathcal{N}-(2^k+1)\right\rvert\le g\left\lfloor 2^{\frac{k}{2}+1}\right\rfloor,
        \end{equation}
        where $ g $ is the genus of the function field $ K/\F_{2^k} $. 
        And we also need to the equality 
        \begin{equation}\label{eq:N_S_M_equality}
            \mathcal{N}=\#\mathcal{S}_{\beta,v}-\mathcal{M}_{\beta,v},
        \end{equation}
        where $ \mathcal{M}_{\beta,v} $ is the number of the points at infinity of equation \eqref{eq:trace_curve}. 
        So we homogenize \eqref{eq:trace_curve} to 
        \begin{equation}\label{eq:homogenize}
            \left( \frac{Y}{Z} \right)^2+\frac{Y}{Z}=\frac{Z}{X+\beta Z}+\frac{Z}{X}+\frac{vX}{Z}.
        \end{equation}
        Multiplying both sides of \eqref{eq:homogenize} by $ Z^2X\left( X+\beta Z \right) $ and then let $ Z=0 $, 
        we have 
        \[X^2Y^2=0,\]
        hence the points at infinite are $ (1:0:0) $ and $ (0:1:0) $. 
        We now consider the multiplicity of roots of $ (0 : 1 : 0) $ and $ (1 : 0 : 0) $, respectively. 
        For the point $ (0 : 1 : 0 ) $, \emph{i.e.}, $ Y = 1 $, we have
        \begin{equation}\label{eq:points_010}
            \left( \frac{1}{z} \right)^2+\frac{1}{z}=\frac{z}{x+\beta z}+\frac{z}{x}+\frac{vx}{z}.
        \end{equation}
        And multiplying \eqref{eq:points_010} by $ z^2x(x+\beta z) $ gives 
        \[x^2+\beta xz+R_{\beta,v}(x,z)=0,\]
        where $ R_{\beta,v}(x,z)=\beta xz^2+x^2z+\beta z^4+v\beta x^2z^2+vx^3z $ is a polynomial 
        such that its every monomial has algebraic degree at least $ 3 $. 
        This gives $ (0 : 1 : 0) $ is a root of multiplicity $ 2 $. 
        
        For the point $ (1 : 0 : 0) $, \emph{i.e.}, $ X = 1 $, we have
        \begin{equation}\label{eq:points_100}
            \left( \frac{y}{z} \right)^2+\frac{y}{z}=\frac{z}{1+\beta z}+z+\frac{v}{z}.
        \end{equation}
        And multiplying \eqref{eq:points_100} by $ z^2(1+\beta z) $ gives 
        \[vz+v\beta z^2+y^2+zy+\beta y^2z+\beta yz^2+\beta z^4=0.\]
        Note that when $ v=0 $, we have 
        \[y^2+zy+\beta y^2z+\beta yz^2+\beta z^4=0,\]
        which implies that $ (1:0:0) $ is a root of multiplicity $ 2 $. 

        Therefore, equation \eqref{eq:trace_curve} has at most $ 4 $ points at infinity, \emph{i.e.} 
        $ \mathcal{M}_{\beta,v}\le4 $. 
        So combining \eqref{eq:tracesum_S},\eqref{eq:N_genus_inequality},\eqref{eq:N_S_M_equality} and the fact that 
        $ S(\alpha,\beta,v) $ is even, we can get our assertion
        \[\left\lvert S(\alpha,\beta,v)\right\rvert \le 2\left\lfloor 2^{\frac{k}{2}+1}\right\rfloor+4.\]
    \end{proof}

    \begin{lemma}\label{lemma:N_ij_trace}
        Assume  $ k\ge 3 $ be a positive integer, let 
        \[ N_{i,j} =\#\left\{x\in\F_{2^k}\middle|\TRACE\left(\theta_1x+\gamma_1\right)=i,\TRACE\left(\theta_2x+\gamma_2\right)=j\right\}, \]
        where  $ \gamma_1,\gamma_2\in\F_{2^k} $ and $ \theta_1,\theta_2\in\F_{2^k}^* $ are distinct. Then $ N_{0,0} =2^{k-2} $.
% 	 N_{0,1}=N_{1,0}=N_{1,1}= 
    \end{lemma}   
   
   \begin{proof}
    % $ N_{0,0}+N_{0,1}+N_{1,0}+N_{1,1}=2^k $. 
       We have 
       \begin{empheq}[left=\empheqbiglbrace]{align*}
            N_{0,0}+N_{0,1}&=\#\left\{x\in\F_{2^k}\middle|\TRACE\left(\theta_1x+\gamma_1\right)=0\right\}=2^{k-1}\\
            N_{1,1}+N_{0,1}&=\#\left\{x\in\F_{2^k}\middle|\TRACE\left(\theta_2x+\gamma_2\right)=1\right\}=2^{k-1}, 
       \end{empheq}
       then we get $ N_{0,0} = N_{1,1} $. 
       Besides, $ N_{0,0}+N_{1,1} = \#\left\{x\in\F_{2^k}\middle|\TRACE\left((\theta_1+\theta_2)x+(\gamma_1+\gamma_2)\right)=0\right\}=2^{k-1} $ 
       since the trace function is balanced if $ \theta_1\ne\theta_2 $. 
        Therefore $ N_{0,0}=2^{k-2} $. This completes the proof.
%    	 and $ N_{0,1}=N_{1,0}=2^{k-2} $.
   \end{proof}
   
  \begin{lemma}\label{lemma:N_ijk_trace}
     Assume  $ k\ge 3 $ be a positive integer, 
   % Assume $ V_i=\left\{x\in\F_{2^k}\middle| \TRACE\left(\theta_ix+c_i\right)=0 \right\} $ for $ i=1,2,3 $. 
   % It's obvious that $ \operatorname{dim}_{\F_2}(V_i)=k-1 $ for $ i=1,2,3 $ 
   % and $ \operatorname{dim}_{\F_2}(V_i\cap V_j)=k-2 $ for $ i\ne j $ 
   % and $ \operatorname{dim}_{\F_2}(V_1+V_2+V_3)=k $, then we have 
   % \begin{align*}
   %     \operatorname{dim}_{\F_2}(V_1\cap V_2\cap V_3)&=\operatorname{dim}_{\F_2}(V_1+V_2+V_3)-\sum_{i=1,2,3}\operatorname{dim}_{\F_2}(V_i)+\sum_{1\le i<j\le 3}\operatorname{dim}_{\F_2}(V_i\cap V_j)+\\
   %     &=k-3*(k-1)+3*(k-2)\\
   %     &=k-3.
   % \end{align*}
   let 
   \[ N_{i_1,i_2,i_3}=\#\left\{x\in\F_{2^k}\middle| \TRACE\left(\theta_1x+\gamma_1\right)=i_1,\TRACE\left(\theta_2x+\gamma_2\right)=i_2,\TRACE\left(\theta_3x+\gamma_3\right)=i_3 \right\},\] 
   where  $ \gamma_1,\gamma_2,\gamma_3\in\F_{2^k} $ and $ \theta_1,\theta_2,\theta_3\in\F_{2^k}^* $ are distinct and satisfy 
   $ \theta_3\ne\theta_1+\theta_2 $. Then $ N_{0,0,0}= 2^{k-3} $.
    \end{lemma}

    \begin{proof}
        Using Lemma \ref{lemma:N_ij_trace} we have  
        \begin{equation}\label{eq:from_lemma_1}\left\{\begin{alignedat}{3}
        &N_{0,0,0}+N_{0,0,1}=\#\left\{x\in\F_{2^k}\middle|\TRACE\left(\theta_1x+\gamma_1\right)=0, \TRACE\left(\theta_2x+\gamma_2\right)=0\right\}=2^{k-2}\\
        &N_{0,0,0}+N_{0,1,0}=\#\left\{x\in\F_{2^k}\middle|\TRACE\left(\theta_1x+\gamma_1\right)=0, \TRACE\left(\theta_3x+\gamma_3\right)=0\right\}=2^{k-2}\\
        &N_{0,0,0}+N_{1,0,0}=\#\left\{x\in\F_{2^k}\middle|\TRACE\left(\theta_2x+\gamma_2\right)=0, \TRACE\left(\theta_3x+\gamma_3\right)=0\right\}=2^{k-2}.\\
        \end{alignedat}\right.\end{equation}
        Thus, $ N_{0,0,1}=N_{0,1,0}=N_{1,0,0} $. With the same reason we can also obtain  $ N_{0,1,1}=N_{1,0,1}=N_{1,1,0} $. 

        As a result of $ \theta_1+\theta_2+\theta_3\ne 0 $, we arrive at  
        \begin{equation}\label{eq:sum_three_trace} \left\{\begin{alignedat}{2}
            &N_{0,0,1}+N_{0,1,0}+N_{1,0,0}+N_{1,1,1}=\#\left\{x\in\F_{2^k}\middle|\TRACE\left(\left(\theta_1+\theta_2+\theta_3\right)x+\left(\gamma_1+\gamma_2+\gamma_3\right)\right)=1\right\}=2^{k-1}\\
            &N_{0,1,1}+N_{1,0,1}+N_{1,1,0}+N_{0,0,0}=\#\left\{x\in\F_{2^k}\middle|\TRACE\left(\left(\theta_1+\theta_2+\theta_3\right)x+\left(\gamma_1+\gamma_2+\gamma_3\right)\right)=0\right\}=2^{k-1}.
        \end{alignedat}\right.\end{equation}
        Combining equations \eqref{eq:sum_three_trace} with  equations  
        \begin{equation}\label{eq:sum_N_0jk}\left\{\begin{alignedat}{2}
            &N_{0,0,0}+N_{0,0,1}+N_{0,1,0}+N_{0,1,1}=\#\left\{x\in\F_{2^k}\middle|\TRACE\left(\theta_1x+\gamma_1\right)=0\right\}=2^{k-1}\\
            &N_{1,0,0}+N_{1,0,1}+N_{1,1,0}+N_{1,1,1}=\#\left\{x\in\F_{2^k}\middle|\TRACE\left(\theta_1x+\gamma_1\right)=1\right\}=2^{k-1},\\
        \end{alignedat}\right.\end{equation}
        and by the facts $ N_{0,0,1}=N_{0,1,0}=N_{1,0,0} $, $ N_{0,1,1}=N_{1,0,1}=N_{1,1,0} $, we obtain $ N_{0,0,1}=N_{0,1,1} $. 
        Consequently, equations \eqref{eq:from_lemma_1} and equations \eqref{eq:sum_N_0jk} become  
        \begin{equation*}\left\{\begin{alignedat}{2}
            &N_{0,0,0}+N_{0,0,1}=2^{k-2}\\
            &N_{0,0,0}+3N_{0,0,1}=2^{k-1}.
        \end{alignedat}\right.\end{equation*}
        Clearly, $ N_{0,0,0}=N_{0,0,1}=2^{k-3} $ is the solution. This completes the proof.
    \end{proof}

    With two lemmas proved above, we can give the nonlinearities of all the second-order derivatives of the simplest $ \mathcal{PS} $ bent function.
    \begin{theorem}\label{thm:nl_DaDbf}
        Let $ k\ge 3 $ be an integer and $ n=2k $. 
        For the nonlinearity of the second-order derivative of 
        the simplest $ \mathcal{PS} $ bent function $ f(x,y)=\TRACE(\frac{\lambda x}{y}) $, 
        we have three cases based on the value of the derivative $ \alpha=(\alpha_1,\alpha_2) $ as follow: 
        \begin{enumerate}[label=(\arabic{*})]
            \item For every $ \alpha=(\alpha_1,\alpha_2)\in\F_{2^k}\times\F_{2^k} $ with $ \alpha_2\ne 0 $, 
            when $ \beta $ ranges over $ \F_{2^n} $, we have 
            \begin{equation}\label{res:nontrivil_nl}
                nl(D_{\beta}D_{\alpha}f)=\begin{cases}
                    2^{2k-1}-2^{k+2},&2^kL\text{~times}\\
                    2^{2k-1}-2^{k+1},&2^k(2^k-2-L)\text{~times}\\
                    0,&1\text{~time},%\footnote[1]{This happends if $ \beta=0 $},
                \end{cases}
            \end{equation}
            with $ nl(D_{\beta}D_{\alpha}f)\ge 2^{2k-1}-2^k\left\lfloor 2^{\frac{k}{2}+1}\right\rfloor-2^{k+1} $ occuring $ 2^{k+1}-1 $ times.
            \item For every $ \alpha=(\alpha_1,0)\in\F_{2^k}^*\times\{0\} $, when $ \beta $ ranges over $ \F_{2^n} $, 
            we have $ nl(D_{\beta}D_{\alpha}f)=0 $ occuring $ 2^k $ times,  
            otherwise, $ nl(D_{\beta}D_{\alpha}f)\ge 2^{2k-1}-2^k\left\lfloor 2^{\frac{k}{2}+1}\right\rfloor-2^{k+1} $ occurs $ 2^{2k}-2^k $ times. 
            \item For $ \alpha=(0,0) $, we have $ nl(D_{\beta}D_{\alpha}f) = 0 $ occuring $ 2^n $ times.
        \end{enumerate} 
    \end{theorem}

    \begin{proof}        
    Let us consider the Walsh transform of the second-order derivative of 
    $ f(x,y)=\TRACE\left(\frac{\lambda x}{y}\right) $ at the points 
    $ \alpha=(\alpha_1,\alpha_2),\beta=(\beta_1,\beta_2)\in\F_{2^k}^2 $ with $ \lambda\in\F_{2^k}^* $.
    We have 
    \begin{align*}\label{eq:secondordersum}
        &W_{D_{\beta}D_{\alpha}f}(\mu,\nu)\nonumber\\
        =&\sum_{x\in\F_{2^k}}\sum_{y\in\F_{2^k}}(-1)^{\TRACE\left(\frac{\lambda x}{y}+\frac{\lambda (x+\alpha_1)}{y+\alpha_2}+\frac{\lambda (x+\beta_1)}{y+\beta_2}+\frac{\lambda (x+\alpha_1+\beta_1)}{y+\alpha_2+\beta_2}+\mu x+\nu y\right)}\nonumber\\
        =&\sum_{y\in\F_{2^k}}(-1)^{\TRACE\left(\frac{\lambda\alpha_1}{y+\alpha_2}+\frac{\lambda\beta_1}{y+\beta_2}+\frac{\lambda(\alpha_1+\beta_1)}{y+\alpha_2+\beta_2}+\nu y\right)}\nonumber\\
        &\times \sum_{x\in\F_{2^k}}(-1)^{\TRACE\left(\left(\frac{\lambda}{y}+\frac{\lambda}{y+\alpha_2}+\frac{\lambda}{y+\beta_2}+\frac{\lambda}{y+\alpha_2+\beta_2}+\mu\right)x\right)}\nonumber\\
        =&\begin{cases}
            2^k\sum_{y\in S}(-1)^{\TRACE\left(\frac{\lambda\alpha_1}{y+\alpha_2}+\frac{\lambda\beta_1}{y+\beta_2}+\frac{\lambda(\alpha_1+\beta_1)}{y+\alpha_2+\beta_2}+\nu y\right)},&~\text{if}~\frac{\lambda}{y}+\frac{\lambda}{y+\alpha_2}+\frac{\lambda}{y+\beta_2}+\frac{\lambda}{y+\alpha_2+\beta_2}=\mu~\text{has solutions}\\
            0, &~\text{otherwise}, 
        \end{cases}
    \end{align*}
    where $ S $ is the set of solutions of equation 
    \begin{equation}\label{eq:coefficient}
        \frac{\lambda}{y}+\frac{\lambda}{y+\alpha_2}+\frac{\lambda}{y+\beta_2}+\frac{\lambda}{y+\alpha_2+\beta_2}=\mu.
    \end{equation}
    % $ \frac{\lambda}{y}+\frac{\lambda}{y+\alpha_2}+\frac{\lambda}{y+\beta_2}+\frac{\lambda}{y+\alpha_2+\beta_2}=\mu $.

    Note that  $ nl(D_{\beta}D_{\alpha}f)=2^{2k-1}-\frac{1}{2}\max_{\mu,\nu}\left\lvert W_{D_{\beta}D_{\alpha}f}(\mu,\nu)\rvert\right. $, 
    we only need to consider $ \max_{\mu,\nu}|W_{D_{\beta}D_{\alpha}f}(\mu,\nu)| $ for every points $ \alpha,\beta $. 
    % According to the proof of Lemma \ref{Secondderivativesolution} in \cite{tang2022invfunc}, 
    % for $ \alpha_2,\beta_2\in\F_{2^k}^* $ such that $ \alpha_2\ne\beta_2 $, 
    % we can always find some $ \mu\in\F_{2^k} $ such that equation
    % has at least $ 4 $ solutions. 
    % Besides, when $ \alpha_2=\beta_2\in\F_{2^k}^* $ or $ \alpha_2=0 $ or $ \beta_2=0 $, 
    % equation \eqref{eq:coefficient} has $ 2^k $ solutions for $ \mu=0 $.
    So we just consider the cases such that equation \eqref{eq:coefficient} has solutions, since we have 
    %  $ \max_{\mu,\nu}\left\lvert W_{D_{\beta}D_{\alpha}f}(\mu,\nu)\rvert\right. $ must occur in the case 
    $ 2^k\left\lvert \sum_{y\in S}(-1)^{\TRACE\left(\frac{\lambda\alpha_1}{y+\alpha_2}+\frac{\lambda\beta_1}{y+\beta_2}+\frac{\lambda(\alpha_1+\beta_1)}{y+\alpha_2+\beta_2}+\nu y\right)}\right\rvert \ge 0 $.
    Thanks to Lemma \ref{lemma:num_sol_second_dev}, it is enough to divide points $ \alpha,\beta $ into three cases by the number of solutions 
    of equation \eqref{eq:coefficient}: 

    % two steps are needed for all points 
    % $ \alpha=(\alpha_1,\alpha_2),\beta=(\beta_1,\beta_2)\in\F_{2^k}^2 $ with $ \lambda\in\F_{2^k}^* $:  
    % \begin{enumerate}[label=\roman{*})]
    %     \item Find all $ (\mu,\nu)\in\F_{2^k}^2 $ such that equation \eqref{eq:coefficient}
    %     has solutions.
    %     \item Calculate the value $ \max_{\mu,\nu}|W_{D_{\beta}D_{\alpha}f}(\mu,\nu)| $ among those $ (\mu,\nu) $.
    % \end{enumerate} 
    
    % After finding all $ (\mu,\nu)\in\F_{2^k}^2 $ such that equation \eqref{eq:coefficient} has solutions for every points 
    % $ \alpha=(\alpha_1,\alpha_2),\beta=(\beta_1,\beta_2) $, we need to calculate maxmial value 
    % $ 2^k\left\lvert \sum_{y\in S}(-1)^{\TRACE\left(\frac{\lambda\alpha_1}{y+\alpha_2}+\frac{\lambda\beta_1}{y+\beta_2}+\frac{\lambda(\alpha_1+\beta_1)}{y+\alpha_2+\beta_2}+\nu y\right)} \right\rvert$ 
    % between those $ (\mu,\nu) $.    
    
    % Therefore, we divide points $ \alpha,\beta $ into two parts  
    % owing to the number of solutions of equation \eqref{eq:coefficient}. 
    
    \begin{enumerate}[label=\textbf{Case \arabic*},wide = 0pt]
        \item If $ \alpha_2=\beta_2\in\F_{2^k}^* $ or $ \alpha_2=0 $ or $ \beta_2=0 $ 
    % with $ \mu\ne 0 $, 
    % then equation \eqref{eq:coefficient} has $ 0 $ solution, then
    % \[W_{D_{\beta}D_{\alpha}f}(\mu,\nu)=0.\]
    and $ \mu=0 $, equation \eqref{eq:coefficient} has $ 2^k $ solutions, 
    which actually are all elements of $ \F_{2^k} $, 
    then we have 
    \begin{equation}\label{eq:case2ksolutions}
        W_{D_{\beta}D_{\alpha}f}(0,\nu)=2^k\sum_{y\in\F_{2^k}}(-1)^{\TRACE\left(\frac{\lambda\alpha_1}{y+\alpha_2}+\frac{\lambda\beta_1}{y+\beta_2}+\frac{\lambda(\alpha_1+\beta_1)}{y+\alpha_2+\beta_2}+\nu y\right)}.
    \end{equation}



    % \textbf{CASE.2} ?(nontrivial) If $ \alpha_2=\beta_2 $ or $ \alpha_2=0 $ or $ \beta_2=0 $ with $ \mu=0 $, 
    % then equation \eqref{eq:coefficient} has $ 2^k $ solutions, we confirm that 
    % \begin{equation}\label{eq:case2ksolutions}
    %     W_{D_{\beta}D_{\alpha}f}(\mu,\nu)=2^k\sum_{y\in\F_{2^k}}(-1)^{\TRACE\left(\frac{\lambda\alpha_1}{y+\alpha_2}+\frac{\lambda\beta_1}{y+\beta_2}+\frac{\lambda(\alpha_1+\beta_1)}{y+\alpha_2+\beta_2}+\nu y\right)}.
    % \end{equation}
    
    For the simple cases, if $ \alpha=(\alpha_1,0),\beta=(\beta_1,0)\in\F_{2^k}^*\times\{0\} $ 
    or $ \alpha=(0,0) $ or $ \beta=(0,0) $, equation \eqref{eq:case2ksolutions} can be transformed into a simple form:
    \[W_{D_{\beta}D_{\alpha}f}(0,\nu)=2^k\sum_{y\in\F_{2^k}}(-1)^{\TRACE\left(\nu y\right)}.\]
    And $ \max_{\nu}|W_{D_{\beta}D_{\alpha}f}(0,\nu)|=|W_{D_{\beta}D_{\alpha}f}(0,0)|=2^{2k} $.

    For other cases, we will give the upper bounds for $ \max_{v}|W_{D_{\beta}D_{\alpha}f}(0,v)| $: 
    w.l.o.g. assume $ \alpha_2=\beta_2\in\F_{2^k}^* $ and $ \alpha_1\ne\beta_1 $, then we have 
    \[\left\lvert W_{D_{\beta}D_{\alpha}f}(0,v)\right\rvert =2^k\left\lvert \sum_{y\in\F_{2^k}}(-1)^{\TRACE\left(\frac{\lambda(\alpha_1+\beta_1)}{y+\alpha_2}+\frac{\lambda(\alpha_1+\beta_1)}{y}+vy\right)}\right\rvert\le 2^{k+1}\left\lfloor 2^{\frac{k}{2}+1}\right\rfloor+2^{k+2}.\]

    % \begin{enumerate}[label=(\arabic{*})]
    %     \item   If $ \alpha_1=\beta_1 $, then 
    %     \begin{equation}
    %         W_{D_{\beta}D_{\alpha}f}(\mu,\nu)=2^k\sum_{y\in\F_{2^k}}(-1)^{\TRACE\left(\frac{\lambda\alpha_1}{y+\alpha_2}+\frac{\lambda\alpha_1}{y+\beta_2}+\nu y\right)}.
    %     \end{equation}
    %     \item   If $ \alpha_1=\beta_2=0 $ or $ \alpha_2=\beta_1=0 $, without loss of 
    %     generality, we assume $ \alpha_1=\beta_2=0 $, then 
    %     \begin{equation}
    %         W_{D_{\beta}D_{\alpha}f}(\mu,\nu)=2^k\sum_{y\in\F_{2^k}}(-1)^{\TRACE\left(\frac{\lambda\beta_1}{y+\beta_2}+\frac{\lambda\beta_1}{y+\alpha_2+\beta_2}+\nu y\right)}.
    %     \end{equation}
    %     \item   If $ \beta_1=0 $, then 
    %     \begin{equation}
    %         W_{D_{\beta}D_{\alpha}f}(\mu,\nu)=2^k\sum_{y\in\F_{2^k}}(-1)^{\TRACE\left(\frac{\lambda\alpha_1}{y+\alpha_2}+\frac{\lambda\alpha_1}{y+\alpha_2+\beta_2}+\nu y\right)}.
    %     \end{equation}
    % \end{enumerate}
    Therefore, in the cases of $ \alpha_2=\beta_2\in\F_{2^k}^* $ or $ \alpha_2=0 $ or $ \beta_2=0 $, 
    we have 
    \[\max_{\mu,\nu}|W_{D_{\beta}D_{\alpha}f}(\mu,\nu)|\le 2^{k+1}\left\lfloor 2^{\frac{k}{2}+1}\right\rfloor+2^{k+2}.\]

    
    % \textbf{CASE.2} If $ \alpha_2\ne\beta_2 $ and $ \alpha_2,\beta_2\in\F_{2^k}^* $, 
    % we can always  
    % we confirm \eqref{eq:coefficient} has $ 0 $ solution, then
    % \[W_{D_{\beta}D_{\alpha}f}(\mu,\nu)=0.\]






    % \textbf{CASE.2} (trivial) If $ \alpha_2\ne\beta_2 $ and $ \alpha_2,\beta_2\in\F_{2^k}^* $, when $ \mu\ne 0 $, 
    % $ \lambda(\alpha_2^2+\beta_2^2+\alpha_2\beta_2)+\mu(\alpha_2^2\beta_2+\alpha_2\beta_2^2)\ne 0 $ with $ \TRACE\left(\frac{\lambda\alpha_2}{\mu\beta_2(\alpha_2+\beta_2)}\right)=1 $ or $ \TRACE\left(\frac{\lambda\beta_2}{\mu\alpha_2(\alpha_2+\beta_2)}\right)=1 $, 
    % then \eqref{eq:coefficient} has $ 0 $ solution, we get 
    % \[ W_{D_{\beta}D_{\alpha}f}(\mu,\nu)=0. \]
        \item 
    % \subsubsection{The case where there are at least four solutions}
    If $ \alpha_2,\beta_2\in\F_{2^k}^* $ such that $ \alpha_2\ne\beta_2 $ and   
    $ \mu= \frac{\lambda(\alpha_2^2+\beta_2^2+\alpha_2\beta_2)}{\alpha_2^2\beta_2+\alpha_2\beta_2^2}$, 
    we are sure that $ \{0,\alpha_2,\beta_2,\alpha_2+\beta_2\} $ are solutions of equations \eqref{eq:coefficient}, 
    then we have two subcases based on the number of solutions is $ 8 $ or $ 4 $: 
    \begin{enumerate}[label=(\arabic{*})]
        \item If $ \alpha_2,\beta_2 $ and $ \mu $ satisfy the system 
        \begin{equation}\label{eq:last_four_solution_condition}\left\{
            \begin{alignedat}{3}
                &\mu\ne 0\\
                &\TRACE\left(\frac{\lambda\alpha_2}{\mu\beta_2(\alpha_2+\beta_2)}\right)=0\\
                &\TRACE\left(\frac{\lambda\beta_2}{\mu\alpha_2(\alpha_2+\beta_2)}\right)=0,\\
            \end{alignedat}\right.
        \end{equation}
        then $ \{y_0,y_0+\alpha_2,y_0+\beta_2,y_0+\alpha_2+\beta_2\} $ are also solutions of equation \eqref{eq:coefficient}, 
        where $ y_0\notin\{0,\alpha_2,\beta_2,\alpha_2+\beta_2\} $, 
        therefore the number of solutions is $ 8 $.
        \item Otherwise, $ \{0,\alpha_2,\beta_2,\alpha_2+\beta_2\} $ are the only $ 4 $ solutions. 
    \end{enumerate}


    So we calculate $ W_{D_{\beta}D_{\alpha}f}(\mu,\nu) $ for some $ (\mu,\nu) $ in two cases.
    \begin{enumerate}[label=\textbf{Subcase \Alph{*}},itemindent=*,wide=\parindent]
        \item 
        We first consider the case where equation \eqref{eq:coefficient} has $ 4 $ solutions 
        $ \{0,\alpha_2,\beta_2,\alpha_2+\beta_2\} $, then $ S=\{0,\alpha_2,\beta_2,\alpha_2+\beta_2\} $. 
        Assume $ y\in S $, we have 
    % we confirm that \eqref{eq:coefficient} has $ 4 $ solution, assume $ 4 $ solutions are 
    % $ S_4=\{y^{\prime},y^{\prime}+\alpha_2,y^{\prime}+\beta_2,y^{\prime}+\alpha_2+\beta_2\} $ 
    % where $ y^{\prime}=0 $ or  $ y^{\prime}=y_0 $, then we have 
    % \begin{align}\label{eq:foursolutionsum}
    %     W_{D_{\beta}D_{\alpha}f}(\mu,\nu)=2^k&\sum_{y\in S_4}(-1)^{\TRACE\left(\frac{\lambda\alpha_1}{y+\alpha_2}+\frac{\lambda\beta_1}{y+\beta_2}+\frac{\lambda(\alpha_1+\beta_1)}{y+\alpha_2+\beta_2}+\nu y\right)}\nonumber\\
    %     =2^k&\left((-1)^{\TRACE\left(\frac{\lambda\alpha_1}{y^{\prime}+\alpha_2}+\frac{\lambda\beta_1}{y^{\prime}+\beta_2}+\frac{\lambda(\alpha_1+\beta_1)}{y^{\prime}+\alpha_2+\beta_2}+\nu y^{\prime}\right)}+(-1)^{\TRACE\left(\frac{\lambda\alpha_1}{y^{\prime}}+\frac{\lambda\beta_1}{y^{\prime}+\alpha_2+\beta_2}+\frac{\lambda(\alpha_1+\beta_1)}{y^{\prime}+\beta_2}+\nu (y^{\prime}+\alpha_2)\right)}\right.\nonumber\\
    %     &\left.+(-1)^{\TRACE\left(\frac{\lambda\alpha_1}{y^{\prime}+\alpha_2+\beta_2}+\frac{\lambda\beta_1}{y^{\prime}}+\frac{\lambda(\alpha_1+\beta_1)}{y^{\prime}+\alpha_2}+\nu (y^{\prime}+\beta_2)\right)}
    %     +(-1)^{\TRACE\left(\frac{\lambda\alpha_1}{y^{\prime}+\beta_2}+\frac{\lambda\beta_1}{y^{\prime}+\alpha_2}+\frac{\lambda(\alpha_1+\beta_1)}{y^{\prime}}+\nu (y^{\prime}+\alpha_2+\beta_2)\right)}\right).
    % \end{align}
    % We can sum the first and last part of equation \eqref{eq:foursolutionsum} to get 
    % \begin{align*}
    %     &(-1)^{\TRACE\left(\frac{\lambda\alpha_1}{y^{\prime}+\alpha_2}+\frac{\lambda\beta_1}{y^{\prime}+\beta_2}+\frac{\lambda(\alpha_1+\beta_1)}{y^{\prime}+\alpha_2+\beta_2}+\nu y^{\prime}\right)}
    %     +(-1)^{\TRACE\left(\frac{\lambda\alpha_1}{y^{\prime}+\beta_2}+\frac{\lambda\beta_1}{y^{\prime}+\alpha_2}+\frac{\lambda(\alpha_1+\beta_1)}{y^{\prime}}+\nu (y^{\prime}+\alpha_2+\beta_2)\right)}\\
    %     =&(-1)^{\TRACE\left(\frac{\lambda\alpha_1}{y^{\prime}+\alpha_2}+\frac{\lambda\beta_1}{y^{\prime}+\beta_2}+\frac{\lambda(\alpha_1+\beta_1)}{y^{\prime}+\alpha_2+\beta_2}+\nu y^{\prime}\right)}\cdot
    %     \left[1+(-1)^{\TRACE\left(\frac{\lambda(\alpha_1+\beta_1)}{y^{\prime}} \frac{\lambda(\alpha_1+\beta_1)}{y^{\prime}+\alpha_2}+\frac{\lambda(\alpha_1+\beta_1)}{y^{\prime}+\beta_2}+\frac{\lambda(\alpha_1+\beta_1)}{y^{\prime}+\alpha_2+\beta_2}+\nu (\alpha_2+\beta_2)\right)}\right]\\
    %     =&(-1)^{\TRACE\left(\frac{\lambda\alpha_1}{y^{\prime}+\alpha_2}+\frac{\lambda\beta_1}{y^{\prime}+\beta_2}+\frac{\lambda(\alpha_1+\beta_1)}{y^{\prime}+\alpha_2+\beta_2}+\nu y^{\prime}\right)}\cdot
    %     \left[1+(-1)^{\TRACE\left(\mu(\alpha_1+\beta_1)+\nu (\alpha_2+\beta_2)\right)}\right].
    % \end{align*}
    % We can also sum the second and third parts of equation \eqref{eq:foursolutionsum} to get
    % \begin{align*}
    %     &(-1)^{\TRACE\left(\frac{\lambda\alpha_1}{y^{\prime}}+\frac{\lambda\beta_1}{y^{\prime}+\alpha_2+\beta_2}+\frac{\lambda(\alpha_1+\beta_1)}{y^{\prime}+\beta_2}+\nu (y^{\prime}+\alpha_2)\right)}
    %     +(-1)^{\TRACE\left(\frac{\lambda\alpha_1}{y^{\prime}+\alpha_2+\beta_2}+\frac{\lambda\beta_1}{y^{\prime}}+\frac{\lambda(\alpha_1+\beta_1)}{y^{\prime}+\alpha_2}+\nu (y^{\prime}+\beta_2)\right)}\\
    %     =&(-1)^{\TRACE\left(\frac{\lambda\alpha_1}{y^{\prime}}+\frac{\lambda\beta_1}{y^{\prime}+\alpha_2+\beta_2}+\frac{\lambda(\alpha_1+\beta_1)}{y^{\prime}+\beta_2}+\nu (y^{\prime}+\alpha_2)\right)}\cdot
    %     \left[1+(-1)^{\TRACE\left(\mu(\alpha_1+\beta_1)+\nu (\alpha_2+\beta_2)\right)}\right].
    % \end{align*}
    % Then we have 
    \begin{align}\label{eq:simpleforms_4}
        &W_{D_{\beta}D_{\alpha}f}(\mu,\nu)\nonumber\\
        =&2^k\left[1+(-1)^{\TRACE\left((\alpha_1+\beta_1)\mu+ (\alpha_2+\beta_2)\nu\right)}\right]\nonumber\\
        &\cdot
        \left[(-1)^{\TRACE\left(\frac{\lambda\alpha_1}{y+\alpha_2}+\frac{\lambda\beta_1}{y+\beta_2}+\frac{\lambda(\alpha_1+\beta_1)}{y+\alpha_2+\beta_2}+ y\nu\right)}+
        (-1)^{\TRACE\left(\frac{\lambda\alpha_1}{y}+\frac{\lambda\beta_1}{y+\alpha_2+\beta_2}+\frac{\lambda(\alpha_1+\beta_1)}{y+\beta_2}+ (y+\alpha_2)\nu\right)}\right]\nonumber\\
        =&2^k\left[1+(-1)^{\TRACE\left((\alpha_1+\beta_1)\mu+ (\alpha_2+\beta_2)\nu\right)}\right]\nonumber\\
        &\cdot
        (-1)^{\TRACE\left(\frac{\lambda\alpha_1}{y+\alpha_2}+\frac{\lambda\beta_1}{y+\beta_2}+\frac{\lambda(\alpha_1+\beta_1)}{y+\alpha_2+\beta_2}+ y\nu\right)}\cdot
        \left[1+(-1)^{\TRACE\left(\frac{\lambda\alpha_1}{y}+\frac{\lambda\alpha_1}{y+\alpha_2}+\frac{\lambda\alpha_1}{y+\beta_2}+\frac{\lambda\alpha_1}{y+\alpha_2+\beta_2}+\nu\alpha_2\right)}\right]\nonumber\\
        =&2^k\left[1+(-1)^{\TRACE\left((\alpha_1+\beta_1)\mu+ (\alpha_2+\beta_2)\nu\right)}\right]\cdot
        \left[1+(-1)^{\TRACE\left(\alpha_1\mu+\alpha_2\nu\right)}\right]\cdot
        (-1)^{\TRACE\left(\frac{\lambda\alpha_1}{y+\alpha_2}+\frac{\lambda\beta_1}{y+\beta_2}+\frac{\lambda(\alpha_1+\beta_1)}{y+\alpha_2+\beta_2}+ y\nu\right)}\nonumber\\
        =&\begin{cases}
            2^{k+2}\cdot(-1)^{\TRACE\left(\frac{\lambda\alpha_1}{y+\alpha_2}+\frac{\lambda\beta_1}{y+\beta_2}+\frac{\lambda(\alpha_1+\beta_1)}{y+\alpha_2+\beta_2}+ y\nu\right)},&\text{if}~\TRACE\left(\alpha_2\nu+\alpha_1\mu\right)=0 ~
            \text{and}~\TRACE\left(\beta_2\nu+\beta_1 \mu\right)=0 \\
            0,~&\text{otherwise}.
        \end{cases}
    \end{align}
    Observing \eqref{eq:simpleforms_4} we can find $ |W_{D_{\beta}D_{\alpha}f}(\mu,\nu)| $ only 
    has values $ \{0,2^{k+2}\} $. 
    Furthermore, according to Lemma \ref{lemma:N_ij_trace}, 
    for all $ \alpha=(\alpha_1,\alpha_2),\beta=(\beta_1,\beta_2)\in\F_{2^k}\times\F_{2^k}^* $ such that 
    $ \alpha_2\ne\beta_2 $ and $ \mu=\frac{\lambda(\alpha_2^2+\beta_2^2+\alpha_2\beta_2)}{\alpha_2^2\beta_2+\alpha_2\beta_2^2} $, 
    the number of $ \nu\in\F_{2^k} $ satisfying the system
    \begin{equation}\label{eq:max_foursolution_condition}\left\{
        \begin{alignedat}{2}
            \TRACE\left(\alpha_2\nu+\alpha_1\mu\right)&=0\\
            \TRACE\left(\beta_2\nu +\beta_1 \mu\right)&=0
        \end{alignedat}\right.
    \end{equation}
    is $ 2^{k-2} $ greater than $ 0 $.  
    Thus, for all points $ \alpha,\beta\in\F_{2^k}^2 $ with $ \alpha_2,\beta_2\in\F_{2^k}^* $, $ \alpha_2\ne\beta_2 $ 
    and $ \mu=\frac{\lambda(\alpha_2^2+\beta_2^2+\alpha_2\beta_2)}{\alpha_2^2\beta_2+\alpha_2\beta_2^2} $ 
    such that don't satisfy equations \eqref{eq:last_four_solution_condition}, we have 
    \[\max_{\mu,\nu}|W_{D_{\beta}D_{\alpha}f}(\mu,\nu)|=2^{k+2}.\]
    
    % if $ \{0,\alpha_2,\beta_2,\alpha_2+\beta_2\} $ are only solutions of equation \eqref{eq:coefficient}, 
    % we can always 
    % the values $ \pm 2^{k+2} $, we conclude that $ \TRACE\left(\mu\alpha_1+\nu\alpha_2\right)=0 $ 
    % and $ \TRACE\left(\mu\beta_1+\nu\beta_2\right)=0 $. 

    \item
    The other is that if equation \eqref{eq:coefficient} has $ 8 $ solutions, 
    that is, $ \alpha_2,\beta_2 $ and $ \mu $ satisfy system \eqref{eq:last_four_solution_condition}. 
    Then we have 
    \begin{align*}
    &W_{D_{\beta}D_{\alpha}f}(\mu,\nu)\nonumber\\
        =&2^k\left[1+(-1)^{\TRACE\left((\alpha_1+\beta_1)\mu+ (\alpha_2+\beta_2)\nu\right)}\right]\cdot
        \left[1+(-1)^{\TRACE\left(\alpha_1\mu+\alpha_2\nu\right)}\right]\nonumber\\
        &\cdot
        \left[(-1)^{\TRACE\left(\frac{\lambda\alpha_1}{\alpha_2}+\frac{\lambda\beta_1}{\beta_2}+\frac{\lambda(\alpha_1+\beta_1)}{\alpha_2+\beta_2}\right)}+(-1)^{\TRACE\left(\frac{\lambda\alpha_1}{y_0+\alpha_2}+\frac{\lambda\beta_1}{y_0+\beta_2}+\frac{\lambda(\alpha_1+\beta_1)}{y_0+\alpha_2+\beta_2}+ y_0\nu\right)}\right]\nonumber\\
        =&(-1)^{c_0}2^k\cdot\left[1+(-1)^{\TRACE\left((\alpha_1+\beta_1)\mu+ (\alpha_2+\beta_2)\nu\right)}\right]\cdot
        \left[1+(-1)^{\TRACE\left(\alpha_1\mu+\alpha_2\nu\right)}\right]\cdot\left[1+(-1)^{c_0+c_1}\right]\nonumber\\
        =&\begin{cases}
            2^{k+3}\cdot(-1)^{c_0},~&\text{if}~ \TRACE\left(\alpha_1\mu+\alpha_2\nu\right)=0 , \TRACE\left(\beta_1\mu+\beta_2\nu\right)=0 ~\text{and}~ c_0+c_1=0\\
            0,~&\text{otherwise},
        \end{cases} 
    \end{align*}
    where $ y_0\notin\{0, \alpha_2, \beta_2, \alpha_2+\beta_2\} $ and 
    \begin{empheq}[left=\empheqlbrace]{align*}
        c_0&=\TRACE\left(\frac{\lambda\alpha_1}{\alpha_2}+\frac{\lambda\beta_1}{\beta_2}+\frac{\lambda(\alpha_1+\beta_1)}{\alpha_2+\beta_2}\right)\\
        c_1&= \TRACE\left(\frac{\lambda\alpha_1}{y_0+\alpha_2}+\frac{\lambda\beta_1}{y_0+\beta_2}+\frac{\lambda(\alpha_1+\beta_1)}{y_0+\alpha_2+\beta_2}+\nu y_0\right).
    \end{empheq}

    % then equation \eqref{eq:coefficient} has $ 8 $ distinct solutions 
    % $ \{0,\alpha_2,\beta_2,\alpha_2+\beta_2,y_0,y_0+\alpha_2,y_0+\beta_2,y_0+\alpha_2+\beta_2\} $.
    By Lemma \ref{lemma:N_ijk_trace}, 
    for all $ \alpha=(\alpha_1,\alpha_2),\beta=(\beta_1,\beta_2)\in\F_{2^k}\times\F_{2^k}^* $ such that 
    $ \alpha_2\ne\beta_2 $
    and $ y_0\notin\{0,\alpha_2,\beta_2,\alpha_2+\beta_2\} $, 
    there always exists $ \nu\in\F_{2^k} $ satisfying below equations,
    \begin{empheq}[left=\empheqbiglbrace]{align*}
        &\TRACE\left(\alpha_2\nu + \alpha_1\mu\right)=0\\
        &\TRACE\left(\beta_2 \nu + \beta_1\mu \right)=0\\
        &\TRACE\left(y_0\nu +\frac{\lambda\alpha_1}{\alpha_2}+\frac{\lambda\beta_1}{\beta_2}+\frac{\lambda(\alpha_1+\beta_1)}{\alpha_2+\beta_2}+\frac{\lambda\alpha_1}{y_0+\alpha_2}+\frac{\lambda\beta_1}{y_0+\beta_2}+\frac{\lambda(\alpha_1+\beta_1)}{y_0+\alpha_2+\beta_2} \right)=0
    \end{empheq}
    and the number of those $ \nu $ is $ 2^{k-3} $.
    So we conclude that for all points $ \alpha,\beta $ with $ \alpha_2,\beta_2\in\F_{2^k}^* $ 
    such that $ \alpha_2\ne\beta_2 $ 
    and $ \mu=\frac{\lambda(\alpha_2^2+\beta_2^2+\alpha_2\beta_2)}{\alpha_2^2\beta_2+\alpha_2\beta_2^2} $ 
    satisfying equations \eqref{eq:last_four_solution_condition}, we have 
    \[\max_{\mu,\nu}|W_{D_{\beta}D_{\alpha}f}(\mu,\nu)|=2^{k+3}.\]
\end{enumerate}
\item     
    For every $ \alpha_2,\beta_2\in\F_{2^k}^* $ such that $ \alpha_2\ne\beta_2 $, 
    there exist some $ \mu $ such that $ S=\{y_0,y_0+\alpha_2,y_0+\beta_2,y_0+\alpha_2+\beta_2\} $ 
    are the only $ 4 $ solutions of 
    equation \eqref{eq:coefficient}, where $ y_0\notin\{0, \alpha_2, \beta_2, \alpha_2+\beta_2\} $. Fortunately, 
    we don't need to treat with those $ \mu $.
    Indeed, in that case,  
    the maximal possible value is not greater than the result $ 2^{k+2} $ of Case 2 where equation \eqref{eq:coefficient} has $ 4 $ solutions 
    $ \{0,\alpha_2,\beta_2,\alpha_2+\beta_2\} $, that is, 
    \[ |W_{D_{\beta}D_{\alpha}f}(\mu,\nu)|=2^k\left\lvert\sum_{y\in S}(-1)^{\TRACE\left(\frac{\lambda\alpha_1}{y+\alpha_2}+\frac{\lambda\beta_1}{y+\beta_2}+\frac{\lambda(\alpha_1+\beta_1)}{y+\alpha_2+\beta_2}+ y\nu\right)}\right\rvert\le 2^{k+2}=|W_{D_{\beta}D_{\alpha}f}(\mu_0,\nu_0)|, \]   
    where $ \mu_0=\frac{\lambda(\alpha_2^2+\beta_2^2+\alpha_2\beta_2)}{\alpha_2^2\beta_2+\alpha_2\beta_2^2} $ 
    and $ \nu_0 $ satisfy the system \eqref{eq:max_foursolution_condition}. 
    % are the only four solutions of equation \eqref{eq:coefficient} for some $ \alpha_2,\beta_2 $ and $ \mu $, 
    % because in the case that $ \{0,\alpha_2,\beta_2, \alpha_2+\beta_2\} $ are only $ 4 $ solutions 
    % of equation \eqref{eq:coefficient},  
    % % when $ \{0,\alpha_2,\beta_2,\alpha_2+\beta_2\} $,  
    % That is, 
    % Note that in the case that $ S=\{0,\alpha_2,\beta_2,\alpha_2+\beta_2\} $, 
    % $ \max_{\mu,\nu}|W_{D_{\beta}D_{\alpha}f}(\mu,\nu)|=2^{k+2} $.

    % So if equation \eqref{eq:coefficient} has only $ 4 $ solutions, 
    % $ \max_{\mu,\nu}|W_{D_{\beta}D_{\alpha}f}(\mu,\nu)|=2^{k+2} $.
    \end{enumerate}
    \end{proof}
        
    

    % After taking the $ 8 $ solutions into equation \eqref{eq:secondordersum}, we get the summation
    % \begin{align}\label{eq:case2kplus3}
    %     &W_{D_{\beta}D_{\alpha}f}(\mu,\nu)\nonumber\\
    %     =&2^k\left[1+(-1)^{\TRACE\left(\mu(\alpha_1+\beta_1)+\nu (\alpha_2+\beta_2)\right)}\right]\cdot
    %     \left[1+(-1)^{\TRACE\left(\mu\alpha_1+\nu\alpha_2\right)}\right]\nonumber\\
    %     &\cdot
    %     \left[(-1)^{\TRACE\left(\frac{\lambda\alpha_1}{\alpha_2}+\frac{\lambda\beta_1}{\beta_2}+\frac{\lambda(\alpha_1+\beta_1)}{\alpha_2+\beta_2}\right)}+(-1)^{\TRACE\left(\frac{\lambda\alpha_1}{y_0+\alpha_2}+\frac{\lambda\beta_1}{y_0+\beta_2}+\frac{\lambda(\alpha_1+\beta_1)}{y_0+\alpha_2+\beta_2}+\nu y_0\right)}\right]\nonumber\\
    %     =&(-1)^{c_0}2^k\cdot\left[1+(-1)^{\TRACE\left(\mu(\alpha_1+\beta_1)+\nu (\alpha_2+\beta_2)\right)}\right]\cdot
    %     \left[1+(-1)^{\TRACE\left(\mu\alpha_1+\nu\alpha_2\right)}\right]\cdot\left[1+(-1)^{c_0+c_1}\right]\nonumber\\
    %     =&\begin{cases}
    %         \pm 2^{k+3},~\text{if}~ \TRACE\left(\mu\alpha_1+\nu\alpha_2\right)=0 , \TRACE\left(\mu\beta_1+\nu\beta_2\right)=0 ~\text{and}~ c_0+c_1=0;\\
    %         0,~\text{otherwise}.
    %     \end{cases}
    % \end{align}
    % where $ c_0=\TRACE\left(\frac{\lambda\alpha_1}{\alpha_2}+\frac{\lambda\beta_1}{\beta_2}+\frac{\lambda(\alpha_1+\beta_1)}{\alpha_2+\beta_2}\right) $ and 
    % $ c_1= \TRACE\left(\frac{\lambda\alpha_1}{y_0+\alpha_2}+\frac{\lambda\beta_1}{y_0+\beta_2}+\frac{\lambda(\alpha_1+\beta_1)}{y_0+\alpha_2+\beta_2}+\nu y_0\right)$.


    % If we fix the ratio of $ \beta_2 $ and $ \alpha_2 $: set $ \gamma=\frac{\beta_2}{\alpha_2}\in\F_{2^k}\setminus\F_{4} $ which holds $ \TRACE\left(\frac{\gamma^2}{\gamma^2+\gamma+1}\right)=\TRACE\left(\frac{\gamma^2}{\gamma^2+\gamma+1}\right)=0 $, then
    % we find $ \forall \alpha_1\in\F_{2^k}^* $, 
    % there always exists $ \beta_1\in\F_{2^k}^* $ s.t. $ W_{D_{\beta}D_{\alpha}f}(\mu,\nu)=\pm 2^{k+3} $, 
    % besides, we find that only $ \alpha_2,\beta_2,\alpha_1,\beta_1 $ can influence positive or negetive, and for every points 
    % $ \alpha,\beta $, there are $ 2^{k-3} $ $ \nu $'s leading to $ \pm 2^{k+3} $.

    % Note that $ c_0+c_1=\TRACE\left(\frac{\lambda\alpha_1}{\alpha_2}+\frac{\lambda\beta_1}{\beta_2}+\frac{\lambda(\alpha_1+\beta_1)}{\alpha_2+\beta_2}+\frac{\lambda\alpha_1}{y_0+\alpha_2}+\frac{\lambda\beta_1}{y_0+\beta_2}+\frac{\lambda(\alpha_1+\beta_1)}{y_0+\alpha_2+\beta_2}+\nu y_0\right) $.

    % Therefore we need to determine for every points $ \alpha=(\alpha_1,\alpha_2) $ and $ \beta=(\beta_1,\beta_2) $ 
    % with $ \gamma = \frac{\beta_2}{\alpha_2}\in\F_{2^k}\setminus\F_{2^2} $ satisfying 
    % $ \TRACE\left(\frac{1}{\gamma^2+\gamma+1}\right)=\TRACE\left(\frac{\gamma^2}{\gamma^2+\gamma+1}\right)=0 $ 
    % and $ \mu=\frac{\lambda(\alpha_2^2+\beta_2^2+\alpha_2\beta_2)}{\alpha_2^2\beta_2+\alpha_2\beta_2^2} $,  
    % whether or not there always exists $ \nu\in\F_{2^k} $ s.t. 
    % \begin{equation}\label{eq:3trace0}\left\{\begin{alignedat}{3}
    %     &\TRACE\left(\mu\alpha_1+\nu\alpha_2\right)=0\\ 
    %     &\TRACE\left(\mu\beta_1+\nu\beta_2\right)=0\\
    %     &\TRACE\left(\frac{\lambda\alpha_1}{\alpha_2}+\frac{\lambda\beta_1}{\beta_2}+\frac{\lambda(\alpha_1+\beta_1)}{\alpha_2+\beta_2}+\frac{\lambda\alpha_1}{y_0+\alpha_2}+\frac{\lambda\beta_1}{y_0+\beta_2}+\frac{\lambda(\alpha_1+\beta_1)}{y_0+\alpha_2+\beta_2}+\nu y_0\right)=0.
    % \end{alignedat}\right.\end{equation}

    % For every points $ \alpha=(\alpha_1,\alpha_2) $ and $ \beta=(\beta_1,\beta_2) $, equation 
    % \eqref{eq:3trace0} in fact are three trace functions about $ \nu $ 
    % with coefficients $ y_0,\alpha_2,\beta_2\in\F_{2^k}^* $ are distinct 
    % and satifying $ y_0+\alpha_2+\beta_2\ne 0 $. 
    % since $ \mu $ are fixed once $ \alpha_2,\beta_2 $ are fixed, and 
    % $ y_0 $ is also fixed since it's one of eight solutions of equation \eqref{eq:coefficient} and equation \eqref{eq:coefficient} is 
    % determined by $ \lambda,\alpha_2,\beta_2 $ and $ \mu $. 
    % Thus, using Lemma \ref{lemma:N_ijk_trace}, 
    % we confirm that equations \eqref{eq:3trace0} have $ 2^{k-3} $ solutions $ \nu\in\F_{2^k} $
    % for every points $ \alpha=(\alpha_1,\alpha_2)\in\F_{2^k}\times\F_{2^k}^* $ and $ \beta=(\beta_1,\beta_2)\in\F_{2^k}\times\F_{2^k}^* $ 
    % with $ \gamma=\frac{\beta_2}{\alpha_2}\in\F_{2^k}\setminus\F_{2^2} $ satisfying 
    % $ \TRACE\left(\frac{1}{\gamma^2+\gamma+1}\right)=\TRACE\left(\frac{\gamma^2}{\gamma^2+\gamma+1}\right)=0 $ 
    % and $ \mu=\frac{\lambda(\alpha_2^2+\beta_2^2+\alpha_2\beta_2)}{\alpha_2^2\beta_2+\alpha_2\beta_2^2} $.
    
    % So equation \eqref{eq:case2kplus3} will always have points $ (\mu,\nu) $ leading to values $ \pm 2^{k+3} $ for every points 
    % $ \alpha=(\alpha_1,\alpha_2)\in\F_{2^k}\times\F_{2^k}^* $ and $ \beta=(\beta_1,\beta_2)\in\F_{2^k}\times\F_{2^k}^* $ 
    % with $ \gamma=\frac{\beta_2}{\alpha_2}\in\F_{2^k}\setminus\F_{2^2} $ satisfying 
    % $ \TRACE\left(\frac{1}{\gamma^2+\gamma+1}\right)=\TRACE\left(\frac{\gamma^2}{\gamma^2+\gamma+1}\right)=0 $. 

    \subsection{A lower bound on the third-order nonlinearity of the simplest $ \mathcal{PS} $ bent function}
    Applying two times Lemma \ref{thm:High_order_nl_bound1}, that is, taking  
    \[nl_{r-1}(D_af) \ge 2^{n-1}-\frac{1}{2}\sqrt{2^{2n}-2\sum_{b\in\F_2^n}nl_{r-2}D_b(D_af)},\]
    into the summation of right-hand side, 
    \[nl_r(f) \ge 2^{n-1}-\frac{1}{2}\sqrt{2^{2n}-2\sum_{a\in\F_2^n}nl_{r-1}(D_af)},\]
    we obtain the relation between the third-order nonlinearity of $ f $ and the nonlinearities of the second-order derivatives of $ f $:
    \begin{equation}\label{eq:nl3_nlDaDbf}
        nl_3(f)\ge 2^{n-1}-\frac{1}{2}\sqrt{\sum_{\alpha\in\F_{2^n}}\sqrt{2^{2n}-2\sum_{\beta\in\F_{2^n}} nl(D_{\beta}D_{\alpha}f)}}. 
    \end{equation}
    % Thus we should calculate $ \nl(D_{\beta}D_{\alpha}f) $ for every the points $ \alpha,\beta\in\F_{2^n} $.
    Therefore, we can give a lower bound on the third-order nonlinearity of the simplest $ \mathcal{PS} $ bent function, which can be 
    directly deduced from inequality \eqref{eq:nl3_nlDaDbf} with results of Lemma \ref{thm:nl_DaDbf}:   
    \begin{theorem}\label{th:our_lower_bound}
        Let $ k\ge 3 $ be an integer and $ n=2k $. For the third-order nonlinearity of 
        the simplest $ \mathcal{PS} $ bent function 
        $ f(x,y)=\TRACE(\frac{\lambda x}{y}) $ with $ x,y\in\F_{2^k} $ and $ \lambda\in\F_{2^k}^* $, we have:
        \[nl_3(f)\ge 2^{n-1}-\frac{1}{2}\sqrt{A}\approx 2^{n-1}-2^{\frac{7n}{8}-\frac{1}{2}},\]
        % \approx 2^{n-1}-2^{\frac{7n}{8}+\frac{1}{2}}.
        where
        \begin{align*}
            A=2^n+&(2^{\frac{n}{2}}-1)\sqrt{(2^{\frac{3n}{2}+1}-2^{n+1})\left\lfloor 2^{\frac{n}{4}+1}\right\rfloor+5\cdot 2^{\frac{3n}{2}}-2^{n+2}}\\
            +&(2^n-2^{\frac{n}{2}})\sqrt{2^{\frac{3n}{2}+2}+2^n-2^{\frac{n}{2}+2}+(2^{n+2}-2^{\frac{n}{2}+1})\left\lfloor 2^{\frac{n}{4}+1}\right\rfloor+2^{n+2}L},
        \end{align*}
        and $ L $ is defined in Lemma \ref{L:SumInv00}. 
        % The low bound of the third-order nonlinearity of the simplest $ \mathcal{PS} $ 
        % bent function $ f(x,y)=\TRACE(\frac{\lambda x}{y}) $ 
    \end{theorem}
    \begin{proof}
        We have 
        \begin{align*}
            nl_3(f)&\ge 2^{n-1}-\frac{1}{2}\sqrt{\sum_{\alpha\in\F_{2^n}}\sqrt{2^{2n}-2\sum_{\beta\in\F_{2^n}} nl(D_{\beta}D_{\alpha}f)}}\\
            &=2^{n-1}-\frac{1}{2}\sqrt{\left( \sum_{\alpha=(0,0)}+\sum_{\alpha=(\alpha_1,0)\in\F_{2^k}^*\times\{0\}}+\sum_{\substack{\alpha=(\alpha_1,\alpha_2)\in\F_{2^k}^2\\\alpha_2\ne 0}} \right)\sqrt{2^{2n}-2\sum_{\beta\in\F_{2^n}} nl(D_{\beta}D_{\alpha}f)}}\\
            &\ge 2^{n-1}-\frac{1}{2}\left[2^n+(2^{\frac{n}{2}}-1)\sqrt{2^{2n}-2(2^n-2^{\frac{n}{2}})(2^{n-1}-2^{\frac{n}{2}}\left\lfloor 2^{\frac{n}{4}+1}\right\rfloor-2^{\frac{n}{2}+1})}\right.\\
            &\qquad\qquad\qquad\left.+(2^n-2^{\frac{n}{2}})\sqrt{2^{2n}-2\left( (2^{n-1}-2^{\frac{n}{2}+1})(2^n-1)-(2^{n+1}-2^{\frac{n}{2}})\left\lfloor 2^{\frac{n}{4}+1}\right\rfloor-2^{n+1}L \right)}\right]^{\frac{1}{2}},
        \end{align*}
        where the second sign of inequality comes from Lemma \ref{thm:nl_DaDbf}. 
        Then, we have 
        \begin{align*}
            A=2^n+&(2^{\frac{n}{2}}-1)\sqrt{(2^{\frac{3n}{2}+1}-2^{n+1})\left\lfloor 2^{\frac{n}{4}+1}\right\rfloor+5\cdot 2^{\frac{3n}{2}}-2^{n+2}}\\
            +&(2^n-2^{\frac{n}{2}})\sqrt{2^{\frac{3n}{2}+2}+2^n-2^{\frac{n}{2}+2}+(2^{n+2}-2^{\frac{n}{2}+1})\left\lfloor 2^{\frac{n}{4}+1}\right\rfloor+2^{n+2}L}.
        \end{align*}
        This completes the proof. 
    \end{proof}
    \begin{corollary}
        The third-order nonlinearity of the simplest $ \mathcal{PS} $ bent functions is lower bounded by approximately $ 2^{n-1}-2^{\frac{7n}{8}-\frac{1}{2}} $.
    \end{corollary}
    % \begin{remark}
    %     As for the second bound derived from inequality \eqref{eq:nl3_nlDaDbf_ng}, it is obvious that 
    %     \[2^{n-1}-\frac{1}{2}\sqrt{A}>\frac{1}{4}\left( 2^{n-1}-2^{\frac{n}{2}+1} \right),\] 
    %     implying that the second bound is not tigher than the bound in Theorem \ref{th:our_lower_bound}. 
    % \end{remark}
    \begin{remark}
        \newcommand{\rb}[1]{\raisebox{1.5ex}[0pt]{#1}}
        We only compare in Table \ref{table:MyTableLabel} the lower bounds on our third-order nonlinearity of the simplest $ \mathcal{PS} $ bent function with the previously known bounds for some small concrete values. 
        It can be seen that our lower bound on the third-order nonlinearity is always tighter than the bounds in \cite{TangCT2013NL_2bent} and \cite{Carlet2011NL_Profile_Dillon}. And when $ n $ is not too large, our lower bound is much more efficient than others. 
        \begin{table}                                           
            \centering                                              
            \caption{Comparison of the lower bounds on the third-order nonlinearity of $ f $}                       
            \begin{threeparttable}
                \begin{tabular}{|c|c|c|c|c|}                                      
                    \hline   
                            & Tang-Carlet-Tang bound      & Carlet bound                            & Our bound& \\  
                    \rb{$n$}& in \cite{TangCT2013NL_2bent}& in \cite{Carlet2011NL_Profile_Dillon} & in Theorem \ref{th:our_lower_bound}     &\rb{Difference\tnote{1}}   \\
                    \hline         
                    $8  $ &  $ -10         $       & $ -7       $     & $ 22       $     & $  29      $ \\  \hline     
                    $10 $ &  $ 55          $       & $ 63       $     & $ 178      $     & $  115     $ \\  \hline     
                    $12 $ &  $ 533         $       & $ 552      $     & $ 919      $     & $  367     $ \\  \hline     
                    $14 $ &  $ 3156        $       & $ 3204     $     & $ 4352     $     & $  1148    $ \\  \hline     
                    $16 $ &  $ 15984       $       & $ 16103    $     & $ 19952    $     & $  3849    $ \\  \hline     
                    $18 $ &  $ 75003       $       & $ 75291    $     & $ 88967    $     & $  13676   $ \\  \hline     
                    $20 $ &  $ 336633      $       & $ 337330   $     & $ 384819   $     & $  47489   $ \\  \hline     
                    $22 $ &  $ 1468218     $       & $ 1469893  $     & $ 1628591  $     & $  158698  $ \\  \hline     
                    $24 $ &  $ 6278535     $       & $ 6282550  $     & $ 6807016  $     & $  524466  $ \\  \hline     
                    $26 $ &  $ 26469867    $       & $ 26479472 $     & $ 28237579 $     & $  1758107 $ \\  \hline     
                \end{tabular}       
                \begin{tablenotes}
                    \footnotesize
                    \item[1] The values in the table are the difference of our bound's values with  Carlet bound's values.
                \end{tablenotes}                                     
            \end{threeparttable}
            \label{table:MyTableLabel}                              
        \end{table}    
     \end{remark}
    
    
    
   




    % \textbf{Nothing}:

    % We find in that case, $ c_1 $ don't change the value when $ y_0 $ is one of the four solutions such that
    % \begin{align*}
    %     c_1=&\TRACE\left(\frac{\lambda\alpha_1}{y_0+\alpha_2}+\frac{\lambda\beta_1}{y_0+\beta_2}+\frac{\lambda(\alpha_1+\beta_1)}{y_0+\alpha_2+\beta_2}+\nu y_0\right)\\
    %     =&\TRACE\left(\frac{\lambda(\alpha_1+\beta_1)}{y_0+\alpha_2}+\frac{\lambda(\alpha_1+\beta_1)}{y_0+\beta_2}+\frac{\lambda(\alpha_1+\beta_1)}{y_0+\alpha_2+\beta_2}+\frac{\lambda(\alpha_1+\beta_1)}{y_0}+\frac{\lambda(\alpha_1+\beta_1)}{y_0}+\frac{\lambda\alpha_1}{y_0+\beta_2}+\frac{\lambda\beta_1}{y_0+\alpha_2}+\nu y_0\right)\\
    %     =&\TRACE\left(\mu(\alpha_1+\beta_1)+\frac{\lambda(\alpha_1+\beta_1)}{y_0}+\frac{\lambda\alpha_1}{y_0+\beta_2}+\frac{\lambda\beta_1}{y_0+\alpha_2}+\nu y_0\right)\\
    %     =&\TRACE\left(\frac{\lambda(\alpha_1+\beta_1)}{y_0}+\frac{\lambda\alpha_1}{y_0+\beta_2}+\frac{\lambda\beta_1}{y_0+\alpha_2}+\nu (y_0+\alpha_2+\beta_2)\right).
    % \end{align*}
    % The last equation holds iff both two equations $ \TRACE\left(\mu\alpha_1+\nu\alpha_2\right)=0 $ 
    % and $ \TRACE\left(\mu\beta_1+\nu\beta_2\right)=0 $ holds.



    %         % \begin{empheq}[left=\empheqlbrace]{align}
    %         %     &F_1(\vb*{x})+F_1(\vb*{x}+\vb*{a})=\vb*{b}\label{eq:3-0-1}\\
    %         %     &f_1(\vb*{x})+f_1(\vb*{x}+\vb*{a})=b_n\label{eq:3-0-2}.
    %         % \end{empheq}
    % From the first condition we have $ \mu=\frac{\lambda(\alpha_2^2+\beta_2^2+\alpha_2\beta_2)}{\alpha_2^2\beta_2+\alpha_2\beta_2^2} $
    % and we can subsititute $ \mu $ into the second condition then we have 
    % \[\TRACE\left(\frac{\lambda\alpha_2}{\frac{\lambda(\alpha_2^2+\beta_2^2+\alpha_2\beta_2)}{\alpha_2^2\beta_2+\alpha_2\beta_2^2}\cdot\beta_2(\alpha_2+\beta_2)}\right)=\TRACE\left(\frac{\alpha_2^2}{\alpha_2^2+\beta_2^2+\alpha_2\beta_2}\right)=\TRACE\left(\frac{1}{\gamma^2+\gamma+1}\right)=0.\]
    % and 
    % \[\TRACE\left(\frac{\lambda\beta_2}{\frac{\lambda(\alpha_2^2+\beta_2^2+\alpha_2\beta_2)}{\alpha_2^2\beta_2+\alpha_2\beta_2^2}\cdot\alpha_2(\alpha_2+\beta_2)}\right)=\TRACE\left(\frac{\beta_2^2}{\alpha_2^2+\beta_2^2+\alpha_2\beta_2}\right)=\TRACE\left(\frac{\gamma^2}{\gamma^2+\gamma+1}\right)=0.\]
    % where $ \gamma=\frac{\beta_2}{\alpha_2}\in\F_{2^k}\setminus\F_{2^2} $ since $ \mu\ne 0 $ and $ \alpha_2\ne\beta_2 $ in condition 1 
    % means $ \alpha_2^2+\beta_2^2+\alpha_2\beta_2\ne 0 $.

    \bibliographystyle{plain} 
    \bibliography{mybib}

\end{document}



% for i in [1..2^5] do
%     inputvector:=Intseq(i-1,2,n);
%     eltvector:=&+[inputvector[j]*v^(j-1):j in [1..n]];
%     Append(~input,(eltvector));
% end for;

% for i in [1..2^n] do
%     inputvector:=Intseq(sbox[i],2,n);
%     eltvector:=&+[inputvector[j]*v^(j-1):j in [1..n]];
%     Append(~output,(eltvector));
% end for;

% function newsbox(x)
%     if x eq 0 then 
%         return 0; 
%     end if;
%     for i in [1..1024] do
%         if x eq input[i] then
%             return output[i];
%         end if;
%     end for;
% end function;



% G:=OrthoTest(newsbox,n);
% ddtG:=DDTexe(G,n);



%   \begin{tikzpicture}[
%     node distance = 5ex,
%     scale = 3,
%     thick,
%     > = latex,
%     % change the 
%     z = {(0.35, -0.4)},
%     edge/.style = {draw, thick, -, black},
%     sinal/.style = {inner sep = 1pt, thin, opacity = 0.4,
%       fill = blue, circle, text opacity = 1},
%     mtx/.style = {
%   %     matrix of math nodes,
%       matrix of nodes,
%       every node/.style = {
%         anchor = base,
%         text width = 2em,
%         text height = 1em,
%         align = center,
%       }
%     },
%     ]

%     \def\dist{0.1}
%     \def\cube{
%         % Vertices. (A,B,C), A x轴  B z轴  C y轴
        
        
%         % \node[left] (v0) at (0,0,0) {$ A $};
%         % note that command above can construct nodes and label them at the same time, 
%         % but sometimes you don't need the text, 
%         % so I just construct the coordinates and then label coordinates 
%         \coordinate (v0) at (0, 0, 0)  ;
%         \coordinate (v1) at (0, 1, 0)  ;
%         \coordinate (v2) at (1, 0, 0)  ;
%         \coordinate (v3) at (1, 1, 0)  ;
%         \coordinate (v4) at (0, 0, 1)  ;
%         \coordinate (v5) at (0, 1, 1)  ;
%         \coordinate (v6) at (1, 0, 1)  ;
%         \coordinate (v7) at (1, 1, 1)  ;
%         \coordinate (v8) at (0, 2, 0)  ;
%         \coordinate (v9) at (1, 2, 0)  ;
%         \coordinate (v10) at (0, 2, 1) ;
%         \coordinate (v11) at (1, 2, 1) ;
%     }
%     \begin{scope}[opacity=1] % opacity is the transparent 
%         \cube{};
%         % labeling verticals with text A B C at left\right\below\above\below left\below right\above left\above right
%         \node[left] at (v0) {$ A $};
%         \node[left] at (v1) {$ B $};
%         \node[right] at (v2) {$ C $};
%         \node[above right] at (v3) {$ D $};
%         \node[left] at (v4) {$ E $};
%         \node[below left] at (v5) {$ F $};
%         \node[right] at (v6) {$ G $};
%         \node[right] at (v7) {$ H $};
%         \node[left] at (v8) {$ I $};
%         \node[above right] at (v9) {$ J $};
%         \node[below left] at (v10) {$ K $};
%         \node[right] at (v11) {$ L $}; 
%         % Edges with some differential: alpha gamma beta theta
%         % arrow with direction from v1 to v0

%         \draw[->] (v2) -- (v3);
%         \draw[->] (v3) -- (v9);
%         \draw[->] (v6) -- (v7);
%         \draw[->] (v7) -- (v11);
%         % dotted line from v1 to v2 and the middle of line labeled is gamma
%         \draw[dashed] (v0) -- node[fill = white] {$ \gamma $} (v2) ;
%         \draw[dashed] (v4) -- node[fill = white] {$ \gamma $} (v6);
%         \draw[dashed] (v1) -- node[fill = white] {$ \beta $} (v5) -- node[fill = white] {$ \theta $} (v7) -- node[fill = white] {$ \beta $}(v3) -- node[fill = white] {$ \theta $} (v1);
%         \draw[dashed] (v8) -- node[fill = white] {$ \alpha $}(v10);
%         \draw[dashed] (v11) -- node[fill = white] {$ \alpha $}(v9);
%     \end{scope}
        
%     \begin{scope}[opacity=0.2]
%         % the pics in this part are transparent 0.2, 
%         % if not want this condition, delete the commands.
%         \draw[<-] (v0) -- (v1);
%         \draw[<-] (v1) -- (v8);
%         \draw[<-] (v4) -- (v5);
%         \draw[<-] (v5) -- (v10);
%     \end{scope}
        
%         % \foreach \i in {0, 1, ..., 11}{ \draw[fill = black] (v\i) circle (0.1pt); }
%         % } % boomerang attack model
        
%         \begin{scope}[]
%             % 
%             \coordinate (E0)  at (2, 0-0.4, 0);
%             \coordinate (E0L) at (2-0.3, 0-0.4, 0);
%             \coordinate (E0R) at (2+0.15, 0-0.4, 0);
%             \coordinate (E1)  at (2, 1-0.4, 0);
%             \coordinate (E1L) at (2-0.3, 1-0.4, 0);
%             \coordinate (E1R) at (2+0.15, 1-0.4, 0);
%             \coordinate (E2)  at (2, 2-0.4, 0);
%             \coordinate (E2L) at (2-0.3, 2-0.4, 0);
%             \coordinate (E2R) at (2+0.15, 2-0.4, 0);
%             \draw[->] (E0) -- node[right] {$ E_1^{-1} $} (E1);
%             \draw[->] (E2) -- node[right] {$ E_0 $} (E1);
%             % dotted line with transparent 
%             \draw[dashed,opacity=.5] (E0R) -- (E0) -- (E0L);
%             \draw[dashed,opacity=.5] (E1R) -- (E1) -- (E1L);
%             \draw[dashed,opacity=.5] (E2R) -- (E2) -- (E2L);
%         % \foreach \i in {0, 1, ..., 11}{
%         %   \node at (v\i) {\i}; 
%         % }
%       \end{scope}
    
%     \end{tikzpicture}%

F<v>:=GF(2,8);
G:=[x:x in F|x^8 + x^4 + x^3 + x^2 + 1 eq 0];
for g in G do
    g:=v;
    // sbox=[0, 152, 136, 56, 65, 217, 149, 189, 142, 95, 147, 214, 10, 103, 171, 213, 185, 4, 3, 20, 93, 54, 199, 202, 198, 116, 227, 33, 124, 181, 56, 196, 106, 5, 53, 40, 169, 194, 117, 56, 58, 122, 82, 103, 60, 9, 178, 182, 111, 194, 132, 67, 83, 248, 191, 27, 228, 122, 186, 157, 232, 61, 96, 24, 26, 6, 145, 122, 45, 76, 121, 25, 204, 64, 58, 52, 142, 177, 11, 37, 254, 244, 188, 143, 2, 206, 39, 59, 105, 140, 48, 101, 28, 161, 241, 228, 182, 193, 124, 211, 65, 13, 0, 116, 55, 254, 4, 9, 218, 12, 100, 24, 233, 116, 140, 54, 224, 53, 158, 51, 153, 73, 222, 88, 22, 68, 45, 182, 165, 33, 134, 12, 150, 118, 162, 210, 49, 123, 233, 9, 154, 56, 207, 176, 203, 54, 164, 210, 218, 159, 66, 222, 89, 139, 220, 22, 143, 187, 37, 83, 142, 112, 152, 202, 179, 251, 33, 15, 114, 126, 132, 163, 158, 222, 216, 110, 245, 153, 76, 173, 197, 226, 232, 254, 45, 109, 110, 55, 247, 95, 56, 107, 7, 102, 152, 14, 29, 231, 100, 169, 255, 145, 63, 22, 236, 121, 200, 154, 38, 237, 17, 76, 223, 46, 7, 92, 60, 6, 110, 38, 222, 149, 45, 69, 32, 242, 189, 200, 151, 193, 38, 176, 195, 88, 100, 204, 28, 232, 181, 193, 66, 146, 67, 137, 8, 86, 47, 216, 111, 218, 174, 6, 172, 113, 104, 103 ];

    nonlinearity v:=v^24   
    f:=func<x|x^3 + g^60*x^5 + g^191*x^6 + g^198*x^9 + g^232*x^10 + g^120*x^12+ g^54*x^17 + g^64*x^18 + g^159*x^20 + g^144*x^24 + g^248*x^33+ g^203*x^34 + g^32*x^36 + g^18*x^40 + g^216*x^48 + g^78*x^65+g^46*x^66 + g^91*x^68 + g^27*x^72 + g^70*x^80 + g^52*x^96+ g^224*x^129 + g^18*x^130 + g^197*x^136 + g^253*x^144 + x^160>;
    for b in F do 
        if b eq 0 then continue; end if;
        Walsh_spectra:=[];
        for a in F do 
            Append(~Walsh_spectra,Abs(&+[(-1)^(Integers()!Trace(b*f(x)+a*x)):x in F]));
        end for;
        Set(Walsh_spectra);
        //if Max(Walsh_spectra) eq 128 then 
        //    print "b=",b;
        //end if;
    end for;
end for;

F<v>:=GF(2,8);
g:=v;
for i in [10..255] do 
    i:=24;
    f:=func<x|Trace(v^i*(x^3 + g^60*x^5 + g^191*x^6 + g^198*x^9 + g^232*x^10 + g^120*x^12+ g^54*x^17 + g^64*x^18 + g^159*x^20 + g^144*x^24 + g^248*x^33+ g^203*x^34 + g^32*x^36 + g^18*x^40 + g^216*x^48 + g^78*x^65+g^46*x^66 + g^91*x^68 + g^27*x^72 + g^70*x^80 + g^52*x^96+ g^224*x^129 + g^18*x^130 + g^197*x^136 + g^253*x^144 + x^160))>;
    if Max([&+[(-1)^(Integers()!(f(x)+Trace(a*x))):x in F]:a in F]) eq 128 then
    wf:=[&+[(-1)^(Integers()!(f(x)+Trace(a*x))):x in F]:a in F];
    i;
    end if;
end for;

subspace_set:={};
for a,b in GF(2,4) do 
    if a eq b then continue; end if;
    flag:=0;
    for x in F do 
        if f(x)+f(x+a)+f(x+b)+f(x+a+b) ne 0 then 
            flag:=1;
            break;
        end if;
    end for;
    if flag eq 1 then 
        print "not a subspace";
        break;
    end if;
end for;




% for g=v, we have b=v^24 s.t. bf is an affine function. 
======================================================================================

diff_spec:=[
    [ 0, 2, 6, 595386, 416361, 35805 ],
    [ 0, 2, 4, 6, 8, 12, 713031, 211761, 92070, 15345, 5115, 10230 ],
    [ 0, 2, 4, 6, 8, 629331, 330336, 72540, 13020, 2325 ],
    [ 0, 2, 4, 6, 8, 628401, 329871, 75330, 12555, 1395 ],
    [ 0, 2, 4, 6, 8, 10, 633636, 322701, 75045, 13980, 1905, 285 ],
    [ 0, 2, 4, 6, 8, 10, 12, 630216, 327081, 76215, 12150, 1665, 195, 30 ],
    [ 0, 2, 4, 6, 8, 10, 635314, 317626, 80290, 11780, 2480, 62 ],
    [ 0, 2, 4, 6, 8, 10, 631811, 322617, 80197, 11098, 1674, 155 ],
    [ 0, 2, 4, 6, 8, 10, 633733, 320695, 78399, 12803, 1736, 186 ],
    [ 0, 2, 4, 6, 8, 10, 641514, 307706, 81375, 14880, 1705, 372 ],
    [ 0, 2, 4, 6, 8, 10, 634260, 321036, 76353, 13857, 1767, 279 ],
    [ 0, 2, 4, 6, 8, 10, 12, 14, 630664, 324942, 78647, 11842, 1364, 31, 31, 31 ],
    [ 0, 2, 4, 6, 8, 22, 636306, 315018, 82335, 11715, 2145, 33 ],
    [ 0, 2, 4, 6, 8, 637701, 313131, 80910, 14415, 1395 ],
    [ 0, 2, 4, 6, 8, 626541, 330336, 79515, 10230, 930 ],
    [ 0, 2, 4, 6, 8, 10, 634291, 318401, 81995, 10385, 2170, 310 ],
    [ 0, 2, 4, 6, 8, 640491, 304296, 89280, 13020, 465 ],
    [ 0, 2, 4, 6, 8, 10, 12, 632431, 322958, 77562, 13020, 1333, 186, 62 ],
    [ 0, 2, 4, 6, 8, 624216, 334986, 76725, 11160, 465 ],
    [ 0, 2, 4, 6, 8, 10, 639045, 311861, 80410, 13860, 2365, 11 ]
];

=====================================================================================
// we want to find 10 bit quadratic APN, and test whether they are ccz-eq to some APN classes or instances
// f(x)=x^d+x^i+beta*x^j is a trinomial, it ccz eq to x^d +alpha*x^i+beta*x^j, so we confirm that no new APN instance for trinomial. 
// d = 5,6,9,10,12,17,33,34,40,48,65,66,68,72,80,96,129,130,132,136,144,160,192,257,258,260,264, 272, 288, 320, 384, 513, 514, 516, 520, 528, 544, 576, 640, 768 no APN instance 

// test x^d + x^i + beta x^j, exclude the case of EA s.t. L2 . F . L1 
n:=10;
Z:=Integers();
F<v>:=GF(2,n);
F5<w>:=GF(2,n div 2);
Fstar:=[v^i:i in [0..2^n-2]];
P<x>:=PolynomialRing(F);
index_wt_2:={2^i+2^j:i,j in [0..n-1]|i ne j};

load "DiffSpecOdExe.m";
for d in D do 
    for i,j in index_wt_2 do
        if i eq d or j eq d or i eq j then continue; end if;
        //exclude the case of L2=a x^2i + b x^2j + x and F = x^3
        if i in {2^k*3:k in [0..n-1]} and j in {2^k*3:k in [0..n-1]} then continue; end if;
        if i in {2^k*3^2:k in [0..n-1]} and j in {2^k*3^2:k in [0..n-1]} then continue; end if;
        ccz_coef_set:={};
        i,j;
        for beta in Fstar do
            if #{x^(2^i):x in ccz_coef_set,i in [1..n-1]} eq 2^n-1 then break; end if;
            // exclude the case of L2=2^{n-i},L1=2^i 
            if beta in {x^(2^i):x in ccz_coef_set,i in [1..n-1]} then continue; end if;
            f:=x^d+x^i+beta*x^j;
            if IsAPN(f) eq true then 
                print "f is APN and parameters [i,j,beta] are:";
                i,j,beta;
                ddtS:=DiffSpecOd(func<x|x^d+x^i+beta*x^j>,n);
                ddtS;
                if ddtS notin diff_spec then
                    print "new diff_spec_od,i=",i;
                    Write("APN_10bit_new_instance.txt",[i,j]); 
                    Write("APN_10bit_new_instance.txt",beta);
                else
                    print "existing instance!";
                    // exclude the case of L2=a^d x,L1=x/a
                    Include(~ccz_coef_set,beta);
                end if;
            else 
                //print "not APN";
                Include(~ccz_coef_set,beta);
            end if;
        end for;
    end for;
end for;


//another f(x) is a quadrinomial, f = x^9 + a*Tr_2^n(b*x^i)

n:=10;
d:=9;
Z:=Integers();
F<v>:=GF(2,n);
Fstar:=[v^i:i in [0..2^n-2]];
P<x>:=PolynomialRing(F);
index_wt_2:={2^i+2^j:i,j in [0..n-1]|i ne j};

for i in index_wt_2 do
    if i in [2^j*d:j in [0..n-1]] then continue; end if;
    i;
    ccz_coef_set:={};
    for beta in Fstar do
        if #{x^(2^i):x in ccz_coef_set,i in [1..n-1]} eq 2^n-1 then break; end if;
        // exclude the case of L2=2^{n-i},L1=2^i 
        if beta in {x^(2^i):x in ccz_coef_set,i in [1..n-1]} then continue; end if;
        f:=func<x|x^d+ v^2*(&+[(beta*x^i)^((2^2)^(j-1)):j in [1..n div 2]])>;
        if IsAPN(f,n) eq true then 
            print "f is APN and parameters [i,j,beta] are:";
            i,beta;
            ddtS:=DiffSpecOd(f,n);
            ddtS;
            if ddtS notin diff_spec then
                print "new diff_spec_od,i=",i,"beta=",beta;
                Write("APN_10bit_new_instance.txt",i); 
                Write("APN_10bit_new_instance.txt",beta);
            else
                print "existing instance!";
                // exclude the case of L2=a^d x,L1=x/a
                Include(~ccz_coef_set,beta);
            end if;
        else 
            //print "not APN";
            Include(~ccz_coef_set,beta);
        end if;
    end for;
end for;



=====================================================================================



function Sbox2ff(inputsbox)
    Z:=Integers();
    n:=Ilog2(#inputsbox);
    F<v>:=GF(2,n);
    list:=[];
    for i in [0..2^n-2] do
        Append(~list,&+[Intseq(inputsbox[&+[Z!Eltseq(v^i)[j]*2^(j-1):j in [1..n]]+1],2,n)[k]*v^(k-1):k in [1..n]]);
    end for;
    Append(~list,F!0);
    function ff_fromsbox(x)
        return (x eq 0) select F!0 else list[Log(x)+1];
    end function;
    return ff_fromsbox;
end function;


======================================================================================

function IsAPN(f,variables)
    n:=variables;
    F<al>:=GF(2,n);
    for i in [0..2^n-2] do
        a:=al^i;
        set_b:={}; set_a:={};
        for y in F do 
            if not y in set_a then
                b:=f(y+a)-f(y);
                if b notin set_b then 
                    Include(~set_b,b); 
                else 
                    return false; 
                end if;
                Include(~set_a,y+a); 
            end if;
        end for; 
    end for;
    return true; 
end function;


function IsAPN(f)
    P:=Parent(f);
    F<al>:=BaseRing(P);
    n:=Degree(F);
    for i in [0..2^n-2] do
    a:=al^i;
    set_b:={}; set_a:={};
    for y in F do if not y in set_a then
    b:=Evaluate(f,y+a)-Evaluate(f,y);
    if b notin set_b then Include(~set_b,b); 
    else 
    return false; end if;
    Include(~set_a,y+a); end if;
    end for; end for;
    return true; 
end function;


Z:=Integers();
F<v>:=GF(2,8);
[&+[Z!Eltseq(x)[i]*2^(i-1):i in [1..8]]:x in F];