\documentclass{article}
\usepackage{fullpage,enumitem,amsthm,amsmath,amssymb,graphicx,empheq,bm}
\usepackage{physics}
\usepackage{tcolorbox}
\usepackage[citecolor=blue]{hyperref}
\usepackage{tikz}
\usepackage[noadjust]{cite}
\usepackage{multirow}
\usepackage{threeparttable}
\usepackage{authblk}

\newcommand{\Z}{\mathbf{Z}}
\newcommand{\F}{\mathbb{F}}
\newcommand{\Com}{\mathbf{C}}
\newcommand{\ord}{\operatorname{ord}}
\newcommand{\Q}{\mathbf{Q}}
\newcommand{\R}{\mathbf{R}}
\newcommand{\E}{\mathbb{E}}
\newcommand{\0}{\textbf{0}}
\newcommand{\1}{\textbf{1}}
\newcommand{\wt}{\operatorname{wt}}
\newcommand{\B}{\mathcal{B}}
\newcommand{\nl}{\mathrm{nl}}
\newcommand{\TRACE}{\operatorname{Tr}_1^k}
\newcommand{\TrN}{\operatorname{Tr}_1^n}
\theoremstyle{plain}

\newtheorem{lemma}{Lemma}
\newtheorem{theorem}{Theorem}
\newtheorem{remark}{Remark}
\newtheorem{corollary}{Corollary}
\newtheorem{definition}{Definition}
% \theoremstyle{nonumberplain}



\newtheorem{construction}{Construction}
% \newcommand{\Tr}{\mathrm{Tr}_1^n}
% \newcommand{\tr}{\mathrm{Tr}_1^k}

\title{A Lower Bound on the Third-order Nonlinearity of the Simplest $\mathcal{PS}_{ap}$ Bent Functions}
\author[a]{Zhaole Li}
\author[b]{Bing Shen}
\author[a,b]{Deng Tang}



\affil[a]{School of Electronic Information and Electrical Engineering, Shanghai Jiao Tong University, Shanghai 200240, China}
\affil[b]{Science and Technology on Communication Security Laboratory, Chengdu 610041, Sichuan, China}
% Zhaole Li, Bing Shen, and Deng Tang

% z.l.li@foxmail.com (Z. Li)

% dtang@foxmail.com (D. Tang)

\begin{document}
 \maketitle
  \noindent
  \rule{\linewidth}{0.4pt}

\begin{abstract}
Boolean functions used in symmetric-key encryption should have high higher-order nonlinearity to resist several known cryptographic attacks,
such as algebraic attacks and low-degree approximation attacks.
The higher-order nonlinearity also play an important role in coding theory and theoretical computer science, since they relate to the covering radius of Reed-Muller codes and the Gowers norm, respectively.
It is well-known that bent functions have the highest nonlinearity in even number of variables and thus they possess
the best ability to withstand the best affine approximation attack and fast correlation attack.
However, there is currently limited knowledge regarding the higher-order nonlinearity of bent functions
because computing the higher-order nonlinearity, or even providing tight lower bounds, is an extremely hard task.
In 1974, Dillon proposed two well-known classes of bent functions based on partial spread (in brief, $\mathcal{PS}$), called
$\mathcal{PS}^-$ and $\mathcal{PS}^+$, respectively. He also exhibited a subclass of bent functions in $\mathcal{PS}^-$, known as
partial spread affine plane ($\mathcal{PS}_{ap}$ for short).
In this paper, we provide a lower bound on the third-order nonlinearity of the $n$-variable, where $n\ge 6$ is even, simplest $\mathcal{PS}_{ap}$ bent functions by calculating the nonlinearity of all second-order derivatives of this kind of bent functions.
Compared to the two known lower bounds on the third-order nonlinearity given by Carlet and Tang et al. respectively,
our lower bound is much better than these two ones.
\end{abstract}

\section{Introduction}
    Boolean functions play an important role in the design of symmetric-key encryption, coding theory and theoretical computer science, see \cite{Carlet2020book,BhattacharyyaKSSZ2010gowers,CohenHLL1997RMcodecover} for instances.
    Let $f$ be an $n$-variable Boolean function. The $r$-th order nonlinearity of $f$, denoted by $nl_r(f)$,
    is defined as the minimum Hamming distance from $f$ to all the $n$-variable Boolean functions of degree at most $r$.
    The first-order nonlinearity of $f$ is simply called the nonlinearity of $f$ and
    the $r$-th order nonlinearity with $r\ge 2$ is called the higher-order nonlinearity of $f$ if $r$ is omitted.
    Indeed, the higher-order nonlinearity is an important parameter for symmetric-key encryption since
    it contributes to the knowledge on the Boolean functions  to resist  several known cryptographic attacks, for examples,
    the (fast) algebraic attacks \cite{CM03,CFAA03,FAA06,WT10} and low-degree approximation attacks \cite{Golic1996lower_order_approximation,IwataK1999highorderbentfunction,KnudsenR1996nonlinear_approximation,Courtois2002XL_algorithm_and_NL_r}.
    In coding theory, the value of $nl_r(f)$ equals the minimum Hamming distance from $f$ to the Reed-Muller code $\mathcal{RM}(r,n)$ of length $2^n$ and of order $r$.
    Thus, the maximum $r$-th order nonlinearity of all Boolean functions in $n$ variables equals the covering radius of $\mathcal{RM}(r,n)$ \cite{CohenHLL1997RMcodecover}.
    The higher-order nonlinearity of Boolean functions is also related to the Gowers norm in theoretical computer science, since the correlation between a
    Boolean function $f$ to its closest polynomial of degree $d$ is at most the $(d+1)$-th Gowers norm of $f$ \cite{BhattacharyyaKSSZ2010gowers}.

    % Computing algorithmically the $r$-th order nonlinearity of an $n$-variable Boolean function is hard for $r>1$.
    As mentioned before, Boolean functions used in symmetric-key encryption should have high higher-order nonlinearity to resist several known attacks.
    Till date, however, only a few lower bounds on the higher-order nonlinearity of several classes of Boolean functions have been obtained.
    Indeed, it is a challenging task to compute the $r$-th order nonlinearity of a general Boolean function of algebraic degree strictly greater than $r$ for $r>1$,
    both theoretically and algorithmically.
    Even for $r=2$,
    % determining the second-order nonlinearity of Boolean functions is only possible for those in special forms or with a small number of variables.
    the second-order nonlinearity can only be determined for Boolean functions that are in special forms or have a small number of variables.
    % only functions, which are in special forms or in a small number of variables, can be  to determine their second-order nonlinearity.
    % determining the second-order nonlinearity is possible only for a few functions in special forms and functions in small number of variables.
    With regards to algorithms, the nonlinearity is related with the Walsh spectrum, which can be efficiently computed by a divide-and-conquer butterfly algorithm.
    As for the second-order nonlinearity of Boolean functions, Kabatiansky and Tavernier \cite{KabatianskyT2005listdecoding_RM_2_n} proposed a nice algorithm by using list decoding of second-order Reed-Muller codes.
    Soon, this algorithm was improved and implemented to quadratic Boolean functions up to $n=11$ in general and up to $n=13$ in some special cases \cite{FourquetT2008improved_listdecoding_RM_2_n}.
    This algorithm can also be applied to the higher-order nonlinearity but only for small number of variables.
    There is currently a lack of knowledge regarding algorithms that are related to the $r$-th order nonlinearity for $r\ge 3$.
    It is, therefore, important to provide lower bounds on the $r$-th order nonlinearity of Boolean functions.
    However, it is still a difficult task, even for $r=2$.
    Fortunately, Carlet \cite{Carlet2008lowbound_NL_profile} gave two lemmas about lower bounds on the $r$-th order nonlinearity of a given Boolean function.
    Based on those two lemmas, he derived some lower bounds on the $r$-th order nonlinearity of Maiorana-McFarland, Welch, Kasami and inverse functions.
    Owing to Carlet's lemmas for efficiently lower bounding the $r$-th order nonlinearity, numerous authors have obtained lower bounds on
    the $r$-th order nonlinearity of different functions, mostly for $r=2,3$, see for \cite{Carlet2011NL_Profile_Dillon,YanT2020NL_2,Liu2023NL_2,TangYZZ2020NL_2bent,SihemKJ2020NL_2cubic,SunW2009NL_2,SarkarG2009NL_2MM,GangopadhyayST2010NL_2,SunW2011NL_2,GaoT2017NL_2_MM,TangCT2013NL_2bent,GodeG2010NL_3Kasami,Singh2014NL_3_biquadratic}.

    Bent functions, which achieve maximum possible nonlinearity in even number of variables,
    have attracted a lot of researches due to their best resistance against fast correlation attack \cite{MS1988fast_correlation_attack} and best affine approximation attack \cite{DingXS1991book} in cryptography and relations to several mathematical objects, such as difference sets, strongly regular graphs and complementary sequences \cite{Carlet2020book}.
    However, there is limited knowledge about the higher-order nonlinearity of bent functions due to the challenge of providing lower bounds on the higher-order nonlinearity.
    Additionally, all functions possess explicit mathematical expressions in the $\mathcal{PS}_{ap}$ bent functions (where ``$ap$ '' means ``affine plane''), which is a subclass of the $\mathcal{PS}^-$ class, one of the most well-known bent function classes based on a kind of partial spread proposed by Dillon \cite{Dillon1974PSbent}.
    The explicit and simplest expressions of this kind of bent functions will ensure that the proof of our lower bound remains low complexity and high feasibility.
    % This is another reason why we concentrate on them.
    Due to the two aforementioned reasons,
    %  our attention is directed towards the third-order nonlinearity of bent functions in this class.
    % More specifically,
    in this paper, we derive a lower bound on the third-order nonlinearity of the simplest $\mathcal{PS}_{ap}$ bent functions.


    In this paper, we aim to derive a lower bound on the third-order nonlinearity of the simplest $\mathcal{PS}_{ap}$ bent functions.
    To this end, we only need to determine the nonlinearity of the second-order derivatives of the $n$-variable Boolean function $f(x,y)$ for all possible pairs $(\alpha,\beta)\in\F_{2^k}\times\F_{2^k}$, where $n=2k$ and $x,y\in\F_{2^k}$, according to Carlet's lemma.
    To determine the nonlinearity of $D_{\beta}D_{\alpha}f(x,y)$ with the pair $(\alpha,\beta)\in\F_{2^k}^2$, it is equivalent to find the maximal absolute values of the corresponding character sums with $(\mu,\nu)$ ranging over $\F_{2^k}^2$.
    By simple algebraic geometry lemmas and some calculations, a lower bound on the nonlinearity of $D_{\beta}D_{\alpha}f(x,y)$ for all trivial pairs $(\alpha,\beta)$ is derived.
    While for the general pairs $(\alpha,\beta)$, we obtain the corresponding nonlinearity by determining the number of solutions for systems of trace functions.
    We derive then a lower bound on the third-order nonlinearity of the simplest $\mathcal{PS}_{ap}$ bent functions.
    As a result,
    %In the present paper,
    % we derive a lower bound on the third-order nonlinearity of the simplest $\mathcal{PS}_{ap}$ bent function.
    we improve the previous two lower bounds in \cite{TangCT2013NL_2bent,Carlet2011NL_Profile_Dillon}, and the comparison of three lower bounds can be found in Table \ref{table:MyTableLabel} for small concrete values.
    Specifically, our lower bound is asymptotically equivalent to $2^{n-1}-2^{\frac{7n}{8}-\frac{1}{2}}$, whose
    % comparing with the best known lower bound, Carlet bound in \cite{Carlet2011NL_Profile_Dillon},
    improvement is approximately equal to $(\sqrt{2}-1)2^{\frac{7n}{8}-\frac{1}{2}}$, compared with Carlet's lower bound.
    Additionally, our lower bound is much more better than the previous two lower bounds when $n\le 18$ as we have showed.

    The remainder of this paper is organized as follows:
    Section 2 introduces some basic notions which will be used in the sequel.
    In Section 3, we  obtain a lower bound on the third-order nonlinearity of the simplest $\mathcal{PS}_{ap}$ bent functions.
    Finally, Section 4 concludes this paper.
    % Throughout this work, for the integer $n$, denote $|n|$ the absolute value of $n$.

\section{Preliminaries}
    Let $\F_2$ be the field with two elements $\{0,1\}$ and $\F_2^n$ be the vector space of $n$-tuples over $\F_2$.
    Let $\F_{2^n}$ be the finite field of order $2^n$.
    The Hamming weight of an arbitrary vector $a=(a_1,a_2,\dots,a_n)\in\F_2^n$ is defined as $\wt(a)=\#\left\{ 1\le i\le n:a_i\ne 0 \right\}$, where $\#S$ denotes the cardinality of a given set $S$.
    For any integer $0\le i \le 2^n-1$, we denote by $\overline{i}=(i_1,i_2,\dots,i_n)\in\F_2^n$ the binary expansion of $i$, \emph{i.e.}, $i=\sum_{j=0}^{n-1}i_{j+1}2^j$.
    In the rest of this paper, we use $+$ (resp. $\Sigma$) to denote the addition (resp. multiple sums) in the set of the integers $\Z$ or the finite field $\F_{2^n}$, and use $\oplus$ (resp. $\bigoplus$) to denote the addition (resp. multiple sums) in $\F_2$ or $\F_2^n$.
    If the context is clear, we shall use $+$ (resp. $\Sigma$) to replace $\oplus$ (resp. $\bigoplus$).
    The Hamming distance between any two vectors $a=(a_1,a_2,\dots,a_n),b=(b_1,b_2,\dots,b_n)\in\F_2^n$ is defined as $d_H(a,b)=\#\left\{ 1\le i\le n : a_i\ne b_i \right\}$, and it can be easily checked that $d_H(a,b)=\wt(a\oplus b)$.

    A Boolean function of $n$ variables is a mapping from $\F_2^n$ to $\F_2$.
    The set of all $n$-variable Boolean functions from $\F_2^n$ to $\F_2$ is denoted by $\mathcal{B}_n$.
    The most well-known representation of an $n$-variable Boolean function $f$ is by its truth table (in brief, TT), \emph{i.e.},
    \[f=\left[ f(0,0,\dots,0),f(1,0,\dots,0),\dots,f(0,1,\dots,1),f(1,1,\dots,1) \right].\]
    Clearly, this representation is unique.
    The support of $f$, denoted by $\operatorname{supp}(f)$, is defined as the set $\left\{ x\in\F_2^n : f(x)\ne 0 \right\}$.
    The Hamming weight of $f$, denoted by $\wt(f)$, is defined as the cardinality of the support of $f$.
    It is well-known that any $f\in\mathcal{B}_n$ in variables $x_1,x_2,\dots,x_n$ can also be represented by its algebraic normal form (in brief, ANF)
    \[f(x_1,x_2,\dots,x_n) = \bigoplus_{u\in\F_2^n}a_ux^u,\]
    where $x=(x_1,x_2,\dots,x_n)\in\F_2^n$, $u=(u_1,u_2,\dots,u_n)\in\F_2^n$, $a_u\in\F_2$ and the term $x^{u}=\prod_{i=1}^nx_i^{u_i}$ is called a monomial.
    The algebraic degree of $f$ is then denoted by $\deg(f)=\max\left\{ \wt(u) :  u\in\F_2^n, a_u\ne 0 \right\}$.
    If a Boolean function has algebraic degree at most $1$, we say that it is an affine function.
    The set of all affine functions in $n$ variables is denoted by $A_n$.
    And we shall call quadratic functions the Boolean functions of algebraic degree at most $2$.
    It is known that the vector space $\F_2^n$ is isomorphic to the finite field $\F_{2^n}$ through a choice of a basis of $\F_{2^n}$ over $\F_2$.
    Indeed, if $(\lambda_1,\lambda_2,\dots,\lambda_n)$ is a basis of $\F_{2^n}$ over $\F_2$, then every vector $x=(x_1,x_2,\dots,x_n)\in\F_2^n$ can be
    identified with the element $x_1\lambda_1+x_2\lambda_2+\cdots+x_n\lambda_n\in\F_{2^n}$ and thus $\F_{2^n}$ can be viewed as an $n$-dimensional vector space over $\F_2$.
    Therefore, any $n$-variable Boolean function can also be defined over $\F_{2^n}$ and uniquely represented by a univariate polynomial in variable $x\in\F_{2^n}$ over $\F_{2^n}$
    \[f(x) = \sum_{i=0}^{2^n-1}\sigma_ix^i,\]
    where $\sigma_0,\sigma_{2^n-1}\in\F_2$ and, for every $i\in\{1,2,\dots,2^n-2\}$, $\sigma_{2i\pmod{2^n-1}}=\sigma_i^2$.
    In this case, the algebraic degree of $f$ can be computed by $\max\left\{ \wt(\overline{i}) : \sigma_i\ne 0, 0\le i\le 2^n-1 \right\}$ and the set $A_n$ is constituted by the functions $\Tr_1^n(\alpha x)\oplus c$, where $\alpha\in\F_{2^n}, c\in\F_2$ and $ \Tr_m^n(x)=x+x^{2^m}+x^{2^{2m}}+\cdots+x^{2^{n-m}}$ is the (absolute) trace function from $\F_{2^n}$ to $\F_{2^m}$ with $m\mid n$.
    Particularly, when $n$ is even, any $n$-variable Boolean function $f$ can be uniquely expressed by a bivariate polynomial in variables $x,y\in\F_{2^{n/2}}$ over $\F_{2^{n/2}}$
    \[f(x,y)=\sum_{0\le i,j\le 2^{n/2}-1}f_{i,j}x^iy^j,\]
    where $f_{i,j}$'s are elements in the finite field $\F_{2^{n/2}}$.
    In this case, the algebraic degree of $f$ can be computed by $\max\left\{ \wt(\overline{i})+\wt(\overline{j}) : f_{i,j}\ne 0, 0\le i,j\le 2^{n/2}-1 \right\}$ and the set $A_n$ is constituted by the function $\Tr_1^{n/2}(\alpha x+\beta y)\oplus c$, where $\alpha,\beta\in\F_{2^{n/2}}$ and $c\in\F_2$.

    In coding theory, a binary linear code $\mathcal{C}$ with length $N$ is a subspace of $\F_2^N$ and $\F_2$-dimension of this subspace is called the dimension of $\mathcal{C}$.
    The minimum distance of $\mathcal{C}$ is defined as the minimum Hamming weight of all nonzero vectors (called codewords in coding theory).
    A binary linear code has length $N$, dimension $k$ and minimum distance $d$ is called an $\left[ N,k,d \right]$-code.
    The Reed-Muller code of order $r$, denoted by $\mathcal{RM}(r,n)$, where $1\le r\le n$, is formed by the truth tables of all $n$-variable Boolean functions of algebraic degree at most $r$. The covering radius of a binary $\left[ N,k,d \right]$-code is defined as the smallest integer $\rho$ such that all
    vectors in $\F_2^N$ are within Hamming distance $\rho$ of some codewords (see more details in \cite{MS1977}).
    Thus, the maximum $r$-th order nonlinearity of all Boolean functions in $n$ variables equals the covering radius of $\mathcal{RM}(r,n)$, \emph{i.e.},
    \[\max_{f\in \mathcal{B}_n}nl_r(f)=\max_{f\in\mathcal{B}_n}d_H(f,\mathcal{RM}(r,n)).\]
    % For more details about the covering radius, see \cite{CohenHLL1997RMcodecover}.
    \begin{definition}
        The $r$-th order nonlinearity of a Boolean function $f\in\mathcal{B}_n$ is defined as its minimum Hamming distance from $f$ to all $n$-variable Boolean functions of degree at most $r$
        \[nl_r(f)=\min_{g\in\mathcal{B}_n,\deg(g)\le r} d_H(f,g),\]
        where $d_H(f,g)$ denotes the Hamming distance between $f$ and $g$, i.e., $d_H(f,g)=\#\left\{ x\in\F_2^n : f(x)\ne g(x) \right\}$.
    \end{definition}
    The first-order nonlinearity of $f$ is simply called the nonlinearity of $f$ and is denoted by $nl(f)$.
    The nonlinearity $nl(f)$ is the minimum Hamming distance from $f$ to all affine functions.
    In order to be resistant against the best affine approximation attack \cite{DingXS1991book} and the fast correlation attack \cite{MS1988fast_correlation_attack}, Boolean functions used in symmetric-key encryption must have high nonlinearity.
    % It should be noted that the maximal $r$-th order nonlinearity of all Boolean functions in $n$ variables is equal to the covering radius of $\mathcal{RM}(r, n)$ (see more details in \cite{CohenHLL1997RMcodecover}).
    The nonlinearity of any Boolean function can also be expressed by means of its Walsh transform.
    Let $x=(x_1,x_2,\dots,x_n)$ and $a=(a_1,a_2,\dots,a_n)$ be two vectors in $\F_2^n$ and $a\cdot x=a_1x_1\oplus a_2x_2\oplus\cdots\oplus a_nx_n$ be the usual inner product in $\F_2^n$.
    The Walsh transform of any $n$-variable Boolean function $f$ at a point $a\in\F_2^n$ is defined as
    \[W_f(a)=\sum_{x\in\F_2^n}(-1)^{f(x)+a\cdot x}.\]
    The multiset formed by the values of the Walsh transform is called the Walsh spectrum of $f$.
    If $f$ is defined over $\F_{2^n}$, the Walsh transform of $f$ at a point $\alpha\in\F_{2^n}$ is defined as
    \[W_f(\alpha)=\sum_{x\in\F_{2^n}}(-1)^{f(x)+\TrN(\alpha x)}.\]
    In addition, if $n=2k$ is even and $f(x,y)\in\mathcal{B}_n$ is the bivariate polynomial form of $f$, the Walsh transform of $f$ at any pair $(\alpha,\beta)\in\F_{2^k}^2$ is defined as
    \[W_f(\alpha,\beta)=\sum_{(x,y)\in\F_{2^k}^2}(-1)^{f(x,y)+\Tr_1^k(\alpha x+\beta y)}.\]
    Observing the Walsh transform, we can easily get that the nonlinearity of any Boolean function $f\in\mathcal{B}_n$ can be computed as
    \begin{align*}
        nl(f) &= 2^{n-1} - \frac{1}{2}\max_{a\in\F_2^n}|W_f(a)|\\
              &= 2^{n-1} - \frac{1}{2}\max_{\alpha\in\F_{2^n}}|W_f(\alpha)|\\
              &= 2^{n-1} - \frac{1}{2}\max_{(\alpha,\beta)\in\F_{2^k}^2}|W_f(\alpha,\beta)|~(\text{if }n=2k\text{ is even}).
    \end{align*}
    As mentioned before, Boolean functions used in symmetric-key encryption should have high nonlinearity to resist the best affine approximation attack and the fast correlation attack.
   Let $f$ be any $n$-variables Boolean function. It follows from the well-known Parseval relation that $\sum_{a\in\F_2^n}W_f^2(a)=2^{2n}$.
   This implies that $\max_{a\in\F_2^n}\left\lvert W_f(a)\right\rvert \geq  2^{n/2}$ and thus we can see that
   the nonlinearity of any $n$-variable Boolean function is upper-bounded by $2^{n-1}-2^{\frac{n}{2}-1}$.
    This bound is tight for even $n$ but is not tight for odd $n\geq 3$.
    If a Boolean function $f$ has nonlinearity achieving the upper bound with the equality,
    we say that $f$ is a bent function and clearly bent functions only exist in even number of variables.

    There exists a simple divide-and-conquer butterfly algorithm with complexity $O(n2^n)$ to compute the Walsh spectrum of any $n$-variable Boolean function from its truth table.
    Hence, it is easy to compute the nonlinearity of a Boolean function from its truth table in a large number of variables (around $30$ variables in a personal computer).
    However, as mentioned in Introduction, there is a limited amount of results on the higher-order nonlinearity of Boolean functions
    because computing the higher-order nonlinearity, or even providing a tight lower bound, of a general Boolean function with high algebraic degree is an extremely hard task.
    In 2008, Carlet showed in \cite{Carlet2008lowbound_NL_profile} that a lower bound on the $r$-th order nonlinearity of a Boolean function can be deduced from its $(r-1)$th-order nonlinearity of the first-order derivatives of this Boolean function.
    Thus, to obtain a lower bound on the third-order nonlinearity of a Boolean function, we need to introduce the first-order and second-order derivatives of Boolean functions.
    \begin{definition}
        The first-order derivative of a Boolean function $f$ over $\F_{2^n}$ at a point $\alpha\in\F_{2^n}$ is defined as $D_{\alpha}f(x)=f(x)+f(x+\alpha)$.
        And the second-order derivative of a Boolean function $f$ at a pair $(\alpha,\beta)\in\F_{2^n}^2$ is defined as $D_{\beta}D_{\alpha}f(x)=f(x)+f(x+\alpha)+f(x+\beta)+f(x+\alpha+\beta)$.
    \end{definition}
    With the notions above, Carlet proposed the following lemma which can efficiently provide a lower bound on the $r$-th order nonlinearity of a given Boolean function,
    provided that the $(r-1)$-th order nonlinearity of all first-order derivatives of $f$ are known.
    \begin{lemma}[\cite{Carlet2008lowbound_NL_profile}]\label{thm:High_order_nl_bound1}
        Let $f$ be an $n$-variable Boolean function, and let $r<n$ be a positive integer.
        We have
        \[nl_r(f)\ge 2^{n-1}-\frac{1}{2}\sqrt{2^{2n}-2\sum_{a\in\F_2^n}nl_{r-1}(D_af)}.\]
    \end{lemma}

    There is a class of character sums with polynomial arguments needed to be treated for proving the main result of this paper.
    To this end, we need the following two lemmas about the existence of solutions of the quadratic polynomial over the finite field $\F_{2^n}$ and algebraic geometry (we refer the reader to the original book~\cite{Stichtenoth2008book_algebraicfunctionfieldsandcodes} for more details).
    \begin{lemma}[\cite{Lidl1997FiniteFieldBook}]\label{lemma:hilbert90}
        Let $\F_{2^n}$ be the finite field of order $2^n$.
        Then $x^2+x=\alpha$ with $\alpha\in\F_{2^n}$ has a solution (actually two solutions) in $\F_{2^n}$, if and only if $\Tr_1^n(\alpha)=0$.
    \end{lemma}
    % the genus of the algebraic function field.
    \begin{lemma}[\cite{Stichtenoth2008book_algebraicfunctionfieldsandcodes}]\label{L:genus_K_F}
        Let $K=F(X,Y)$, where $X,Y$ are transcendentals over $F$.
        Then the genus $g$ of the function field $K/F$ satisfies
        \[g\le ([K : F(X)] - 1)([K : F(Y)] - 1).\]
    \end{lemma}

\section{A lower bound on the third-order nonlinearity of the simplest $\mathcal{PS}_{ap}$ bent functions}

In this section, we focus on providing a lower bound on the third-order nonlinearity of the simplest
partial spread affine plane ($\mathcal{PS}_{ap}$ for short) bent functions.
We first introduce the definition of the $\mathcal{PS}_{ap}$ bent functions.
Then we determine the nonlinearities of all second-order derivatives of the simplest $\mathcal{PS}_{ap}$ bent functions
except for the trivial case, and a lower bound on the nonlinearities of the second-order derivatives is given in trivial case.
So we can obtain a lower bound on the third-order nonlinearity of the simplest $\mathcal{PS}_{ap}$
bent functions.

  \subsection{The simplest $\mathcal{PS}_{ap}$ bent functions}

In this subsection, we present the basic definition of the $\mathcal{PS}_{ap}$ bent functions.
Let $\left\{ E_1,E_2,\dots,E_s \right\}$ be a set of pairwise disjoint $k$-dimensional subspaces of $\F_2^n$, {\em i.e.,}
$E_i\cap E_j = \{\bm{0}_n\}$ for $1\le i<j\le s$, where $\bm{0}_n$ is the all-zero vector in $\F_2^n$.
Any collection of $\left\{ E_1,E_2,\dots,E_s \right\}$ is called a partial $k$-spread of $\F_2^n$.
Further, a partial spread is called a spread, if it covers $\F_2^n$, \emph{i.e.}, the union of $E_i$ is equal to $\F_2^n$, where $1\le i\le s$.
Particularly, when $n=2k$, a partial $k$-spread of $\F_2^n$ is a set of pairwise supplementary of $k$-dimensional subspaces of $\F_2^n$,
\emph{i.e.},  the sum of any two of them is direct and equals $\F_2^n$.
Assume $n=2k$ and we then have $\F_{2^n}$ being isomorphic to the affine plane $\F_{2^k}\times\F_{2^k}$,
in which  an affine plane is a vector space of dimension two in which one has ``forgotten'' where the origin is.
Considering $\F_{2^k}\times\F_{2^k}$, we then have $2^k+1$ pairwise disjoint $n/2$-dimensional $\F_2$-subspaces of $\F_2^n$, which are
$E_a=\left\{(x,ax) : x\in\F_{2^k}\right\}$ for $a\in\F_{2^k}$ and $E_{\emptyset}=\left\{(0,x) : x\in\F_{2^k}\right\}$.
It can be seen that all elements $(x,y)$ of $E_a$ constitute a line $y=ax$ over $\F_{2^k}$ for every $a\in\F_{2^k}$ and
$E_{\emptyset}$ can be viewed as a line $y\in\F_{2^k}$ and all $2^k+1$ lines intersect only at the origin.
This is the reason why these subspaces are called ``affine plane''.
%That is why we use the term ``affine plane''.
These $2^k+1$ $n/2$-dimensional $\F_2$-subspaces of $\F_2^n$ constitute a spread, which is the well-known Desarguesian spread.

In \cite{Dillon1974PSbent}, Dillon proposed constructions of bent functions in the $\mathcal{PS}^-$ (resp. $\mathcal{PS}^+$) class in $n$ variables, whose supports are the unions of $2^{k-1}$ (resp. $2^{k-1}+1$) different elements of a partial $k$-spread, where $n=2k$.
Especially, Dillon exhibited the so-called $\mathcal{PS}_{ap}$ bent functions over finite fields, which is a subclass of the $\mathcal{PS}^-$ class.
%By taking any $2^{k-1}$ elements of the Desarguesian spread except $E_0$ and $E_{\emptyset}$ as the supports of a Boolean function will lead to a bent function in the $\mathcal{PS}_{ap}$ class, in the form
Any $\mathcal{PS}_{ap}$ bent function can be written as
\begin{equation*}\label{Eqn_PS_bent}
    f_0(x,y)=g\left(xy^{2^k-2}\right)=g\left(\frac{x}{y}\right),
\end{equation*}
where $g$ is a balanced Boolean function on $\F_{2^{k}}$ with $g(0)=0$, and $\frac{x}{y}$ is defined to be $0$ if $y=0$ (we shall always assume this kind of convention in the sequel).
We can see that the support of $f_0$ is constituted by $2^{k-1}$ elements of the Desarguesian spread except $E_0$ and $E_{\emptyset}$.
In this paper, let $g$ be the simplest balanced Boolean function on $\F_{2^k}$, \emph{i.e.}, the trace function $\TRACE(x)$, then
$f_0$ can be written as
\begin{equation*}\label{sub-bent}
    f(x,y)=\TRACE\left(\frac{\lambda x}{y}\right),
\end{equation*}
where $(x,y)\in\F_{2^k}^2$, $\lambda\in\F_{2^k}^{*}$ and $\TRACE(x)=\sum\limits_{i=0}^{k-1}x^{2^i}$ is the trace function from $\F_{2^k}$ to $\F_2$.
In this paper, $f$ is called a simplest $\mathcal{PS}_{ap}$ bent function.

    % Note that all $f_{\lambda}$, $\lambda\ne 0$, are affinely equivalent to each others and then they have the same $r$-th order nonlinearity.
    % Thus, we can use $f$ instead of $f_{\lambda}$ for simplicity.
    % In this paper, our aim is to obtain a lower bound on the third-order nonlinearity of the simplest $\mathcal{PS}_{ap}$ bent functions.
    \subsection{The nonlinearity of the second-order derivatives of the simplest $\mathcal{PS}_{ap}$ bent functions}
    As mentioned in Introduction,
    we need to determine the nonlinearities of all second-order derivatives of the simplest $\mathcal{PS}_{ap}$ bent functions to derive a lower bound on the third-order nonlinearity.
    To this end, we begin by introducing two lemmas involved to the multiplicative inverse functions,
    as well as providing a lower bound on a class of character sums by algebraic geometry.
    Additionally, we present two lemmas about determining the number of solutions for systems of trace functions.
    These results will be important in the sequel for proving our main result.

    For the finite field $\F_{2^k}$, the multiplicative inverse function over $\F_{2^k}$ is defined as $I(x)=x^{2^k-2}$ (we shall use $x^{-1}$ to denote $x^{2^k-2}$ with the convention that $x^{-1}=\frac{1}{x}=0$ if $x=0$).
    Besides, for any $v\in\F_{2^k}^*$ we also define $I_v(x)=\Tr_1^k(vx^{-1})$.
    Regarding the multiplicative inverse function, the following two lemmas will be used in the sequel.

    \begin{lemma}[\cite{TangMM2022inversefunction}]\label{L:SumInv00}
        Let $k\geq 3$ be an arbitrary integer.
        We define
        $$L=\#\left\{c\in\F_{2^k}  :  \mathrm{Tr}_1^k\left(\frac{1}{c^2+c+1}\right)=\mathrm{Tr}_1^k\left(\frac{c^2}{c^2+c+1}\right)=0\right\}.$$
        Then we have $L=2^{k-2}+\frac{3}{4}(-1)^kW_{I_1}(1)+\frac{1}{2}\left(1-(-1)^k\right)$, where $W_{I_1}(1)=1-\sum_{t=0}^{\lfloor k/2\rfloor}(-1)^{k-t}\frac{k}{k-t}{{k-t}\choose {t}}2^t$.
    \end{lemma}

    % Let $f$ be a function from $\F_{2^n}$ to $\F_{2^m}$.
    % For any $\gamma,\eta\in\F_{2^n}$ and $\omega\in\F_{2^m}$, let us define
    % \[\mathcal{S}_f(\gamma,\eta,\omega),\]
    % associated with $\mathcal{N}_f(\gamma,\eta,\omega)=\#\mathcal{S}_f(\gamma,\eta,\omega)$.

    \begin{lemma}\label{lemma:num_sol_second_dev}
        Let $k\ge 3$ be an integer.
        For $a,b,\mu\in\F_{2^k}$ with $\lambda\in\F_{2^k}^*$, define $F_{\lambda}(x)=\lambda x^{-1}$ and
        \begin{align*}
            U_{F_{\lambda}}(a,b,\mu)=&\left\{x\in\F_{2^k} : F_{\lambda}(x)+F_{\lambda}(x+a)+F_{\lambda}(x+b)+F_{\lambda}(x+a+b)=\mu\right\}\\
            =&\left\{ x\in\F_{2^k} : \frac{\lambda}{x}+\frac{\lambda}{x+a}+\frac{\lambda}{x+b}+\frac{\lambda}{x+a+b}=\mu \right\},
        \end{align*}
        then $\# U_{F_{\lambda}}(a,b,\mu)$ can be determined as follows:
        \[\# U_{F_{\lambda}}(a,b,\mu) = \left\{
            \begin{alignedat}{2}
                &2^k,&&\text{ if }(a,b)\in M\text{ and }\mu = 0,\\
                &8,&&\text{ if both two conditions }\ref{item_a}\text{ and }\ref{item_b}\text{ hold,}\\
                &4,&&\text{ if either condition }\ref{item_a}\text{ or }\ref{item_b}\text{ holds,}\\
                &0,&&\text{ otherwise},
            \end{alignedat}\right.\]
        where $M=\left\{ (x,x):x\in\F_{2^k}^* \right\}\cup\left\{ (x,0):x\in\F_{2^k} \right\}\cup\left\{ (0,y):y\in\F_{2^k}^* \right\}$, and two conditions are listed as follows:
        \begin{enumerate}[label=(\arabic{*})]
            \item $a\ne b\in\F_{2^k}^*$ and $\lambda(a^2+b^2+ab)+\mu(a^2b+ab^2)=0$.\label{item_a}
            \item $a\ne b\in\F_{2^k}^*$, $\mu\ne 0$, $\TRACE\left(\frac{\lambda a}{\mu b(a+b)}\right)=0$ and
            $\TRACE\left(\frac{\lambda b}{\mu a(a+b)}\right)=0$.\label{item_b}
        \end{enumerate}
    \end{lemma}
    \begin{proof}
        The proof is analogue to the proof of Lemma 13 in \cite{TangMM2022inversefunction} and we omit it.
    \end{proof}
    \begin{remark}
        Note that if Condition \ref{item_a} hold, we have $\{0,a,b,a+b\}\subseteq U_{F_{\lambda} }(a,b,\mu)$. 
        If Condition \ref{item_b} hold, we have $\{y_0,y_0+a,y_0+b,y_0+a+b\}\subseteq U_{F_{\lambda}}(a,b,\mu)$ where $y_0\notin\{0,a,b,a+b\}$.
    \end{remark}
    \begin{remark}
        For any $a\in\F_{2^k}^*$, it follows from  Condition \ref{item_a} and Condition \ref{item_b} that there exist $L-4$ distinct values of $b$
        such that $\#U_{F_{\lambda}}(a,b,\mu)=8$ for some fixed $\lambda\in\F_{2^k}^*$ and $\mu\in\F_{2^k}$,
        where $L$ is given by Lemma~\ref{L:SumInv00}.
        %Let us see more details for Condition \ref{item_a} and Condition \ref{item_b}.
       Indeed, note that $a,b$ be two distinct elements of $\F_{2^k}^*$ and $\mu\ne 0$,
        we have $\mu(a^2b+ab^2)\ne 0$.
        In Condition \ref{item_a}, this indicates $\lambda(a^2+b^2+ab)\ne 0$,
        which implies $\frac{b}{a}\notin\F_4$.
        So assume $\mu=\frac{\lambda(a^2+b^2+ab)}{a^2b+ab^2}$,
        then by  $\TRACE\left(\frac{\lambda a}{\mu b(a+b)}\right)=0$
        and $\TRACE\left(\frac{\lambda b}{\mu a(a+b)}\right)=0$
        we obtain $\TRACE\left(\frac{1}{\gamma^2+\gamma+1}\right)=0$ and $\TRACE\left(\frac{\gamma^2}{\gamma^2+\gamma+1}\right)=0$ respectively,
        where $\gamma=\frac{b}{a}\in\F_{2^k}\setminus\F_{4}$.
        Therefore, according to Lemma \ref{L:SumInv00},
        the number of $\gamma=\frac{b}{a}\in\F_{2^k}\setminus\F_{4}$ satisfying
        $\TRACE\left(\frac{1}{\gamma^2+\gamma+1}\right)=0$ and $\TRACE\left(\frac{\gamma^2}{\gamma^2+\gamma+1}\right)=0$
        is $L-4$ since we need to exclude the cases $\gamma\in\F_4$.
    \end{remark}


The following lemma provides a lower bound on a class of character sums by some basic knowledge about algebraic geometry (see more details in \cite{Stichtenoth2008book_algebraicfunctionfieldsandcodes}).
    \begin{lemma}\label{lemma:charactersums}
        Let $k\ge 3$ be a positive integer and assume
        \[S(\alpha,\beta,v)=\sum_{x\in\F_{2^k}}(-1)^{\TRACE\left( \frac{\alpha}{x+\beta}+\frac{\alpha}{x}+vx \right)},\]
        where $\alpha,\beta,v\in\F_{2^k}^*$.
        We have
        \[\left\lvert S(\alpha,\beta,v)\right\rvert\le 2\left\lfloor 2^{\frac{k}{2}+1}\right\rfloor+4 .\]
    \end{lemma}
    \begin{proof}
        Note that $S(1,\beta/\alpha,v\alpha)=S(\alpha,\beta,v)$.
        So we only need to determine the values of $S(1,\beta,v)$.
        In other words, we need to obtain the number of $x\in\F_{2^k}$ such that $\TRACE\left( \frac{1}{x+\beta}+\frac{1}{x}+vx \right)=0$.
        By Lemma $\ref{lemma:hilbert90}$, it is equivalent to calculate the number of solutions $(x,y)\in\F_{2^k}^2$ of $y^2+y=\frac{1}{x+\beta}+\frac{1}{x}+vx$.

        Let us define
        \[\mathcal{S}_{\beta,v}=\left\{(x,y)\in\F_{2^k}\times\F_{2^k} : y^2+y=\frac{1}{x+\beta}+\frac{1}{x}+vx\right\}.\]
        Note that $y\mapsto y^2+y$ is $2$-to-$1$ mapping, then we have
        \begin{equation}\label{eq:tracesum_S}
            S(\alpha,\beta,v)=\frac{\#\mathcal{S}_{\beta,v}}{2}-\left(2^k-\frac{\#\mathcal{S}_{\beta,v}}{2}\right)=\#\mathcal{S}_{\beta,v}-2^k.
        \end{equation}
        Note that $\#\mathcal{S}_{\beta,v}$ is even and then $S(\alpha,\beta,v)$ must be even as well.
        Consider the function field $K=\F_{2^k}(x,y)$ with defining equation
        \begin{equation}\label{eq:trace_curve}
            y^2+y=\frac{1}{x+\beta}+\frac{1}{x}+vx.
        \end{equation}
        By Lemma \ref{L:genus_K_F}, we can easily obtain that the genus of $K$ is at most $2-\delta_v$,
        where $\delta_v=1$ if $v=0$ and $\delta_v=0$ otherwise.
        Denote by $\mathcal{N}$ the number of the places with degree one of $K/\F_{2^k}$.
        Then by Serre bound \cite{Serre1982serrebound}, we have
        \begin{equation}\label{eq:N_genus_inequality}
            \left\lvert \mathcal{N}-(2^k+1)\right\rvert\le g\left\lfloor 2^{\frac{k}{2}+1}\right\rfloor,
        \end{equation}
        where $g$ is the genus of the function field $K/\F_{2^k}$.
        To estimate the value of $|S(\alpha,\beta,v)|$,
        we also need the well-known equation
        \begin{equation}\label{eq:N_S_M_equality}
            \mathcal{N}=\#\mathcal{S}_{\beta,v}-\mathcal{M}_{\beta,v},
        \end{equation}
        where $\mathcal{M}_{\beta,v}$ is the number of the points at infinity of Equation \eqref{eq:trace_curve}.
        So we homogenize Equation \eqref{eq:trace_curve} to
        \begin{equation}\label{eq:homogenize}
            \left( \frac{Y}{Z} \right)^2+\frac{Y}{Z}=\frac{Z}{X+\beta Z}+\frac{Z}{X}+\frac{vX}{Z}.
        \end{equation}
        Multiplying both sides of Equation \eqref{eq:homogenize} by $Z^2X\left( X+\beta Z \right)$ and then let $Z=0$,
        we have
        \[X^2Y^2=0,\]
        hence the points at infinite are $(1:0:0)$ and $(0:1:0)$.

        We now consider the multiplicity of roots of $(0 : 1 : 0)$ and $(1 : 0 : 0)$, respectively.
        For the point $(0 : 1 : 0 )$, \emph{i.e.}, $Y = 1$, we have
        \begin{equation}\label{eq:points_010}
            \left( \frac{1}{z} \right)^2+\frac{1}{z}=\frac{z}{x+\beta z}+\frac{z}{x}+\frac{vx}{z}.
        \end{equation}
        And multiplying Equation \eqref{eq:points_010} by $z^2x(x+\beta z)$ gives
        \[x^2+\beta xz+R_{\beta,v}(x,z)=0,\]
        where $R_{\beta,v}(x,z)=\beta xz^2+x^2z+\beta z^4+v\beta x^2z^2+vx^3z$ is a polynomial
        such that its every monomial has algebraic degree at least $3$.
        This gives $(0 : 1 : 0)$ is a root of multiplicity $2$.

        For the point $(1 : 0 : 0)$, \emph{i.e.}, $X = 1$, we have
        \begin{equation}\label{eq:points_100}
            \left( \frac{y}{z} \right)^2+\frac{y}{z}=\frac{z}{1+\beta z}+z+\frac{v}{z}.
        \end{equation}
        And multiplying Equation \eqref{eq:points_100} by $z^2(1+\beta z)$ gives
        \[vz+v\beta z^2+y^2+zy+\beta y^2z+\beta yz^2+\beta z^4=0.\]
        Note that when $v=0$, we have
        \[y^2+zy+\beta y^2z+\beta yz^2+\beta z^4=0,\]
        which implies that $(1:0:0)$ is a root of multiplicity $2$.

        Therefore, Equation \eqref{eq:trace_curve} has at most $4$ points at infinity, \emph{i.e.},
        $\mathcal{M}_{\beta,v}\le 4$.
        Thus, combining \eqref{eq:tracesum_S},\eqref{eq:N_genus_inequality},\eqref{eq:N_S_M_equality} and the fact that
        $S(\alpha,\beta,v)$ is even we can get our assertion
        \[\left\lvert S(\alpha,\beta,v)\right\rvert \le 2\left\lfloor 2^{\frac{k}{2}+1}\right\rfloor+4.\]
        This finishes the proof.
    \end{proof}

    \begin{lemma}\label{lemma:N_ij_trace}
        Assume  $k\ge 3$ be a positive integer, let
        \[ N_{i,j} =\#\left\{x\in\F_{2^k} : \TRACE\left(\theta_1x+\gamma_1\right)=i,\TRACE\left(\theta_2x+\gamma_2\right)=j\right\}, \]
        where  $\gamma_1,\gamma_2\in\F_{2^k}$ and $\theta_1\ne\theta_2\in\F_{2^k}^*$. Then $N_{0,0} =2^{k-2}$.
    \end{lemma}

   \begin{proof}
        Clearly, we can see that
        \[N_{0,0}+N_{0,1}=\#\left\{x\in\F_{2^k} : \TRACE\left(\theta_1x+\gamma_1\right)=0\right\}=2^{k-1}\]
        and
        \[N_{1,1}+N_{0,1}=\#\left\{x\in\F_{2^k} : \TRACE\left(\theta_2x+\gamma_2\right)=1\right\}=2^{k-1}.\]
        Then we have $N_{0,0} = N_{1,1}$.
        Besides, since $\theta_1\ne\theta_2$, the trace function $\TRACE\left((\theta_1+\theta_2)x+(\gamma_1+\gamma_2)\right)$ is balanced.
        So we have $N_{0,0}+N_{1,1} = \#\left\{x\in\F_{2^k} : \TRACE\left((\theta_1+\theta_2)x+(\gamma_1+\gamma_2)\right)=0\right\}=2^{k-1}$.
        Therefore $N_{0,0}=2^{k-2}$. This completes the proof.
   \end{proof}

    \begin{lemma}\label{lemma:N_ijk_trace}
        Assume $k\ge 3$ be a positive integer, let
        \[ N_{i_1,i_2,i_3}=\#\left\{x\in\F_{2^k} :  \TRACE\left(\theta_1x+\gamma_1\right)=i_1,\TRACE\left(\theta_2x+\gamma_2\right)=i_2,\TRACE\left(\theta_3x+\gamma_3\right)=i_3 \right\},\]
        where  $\gamma_1,\gamma_2,\gamma_3\in\F_{2^k}$, and $\theta_1,\theta_2,\theta_3$ are three pairwise distinct elements of $\F_{2^k}^*$ such that $\theta_3\ne\theta_1+\theta_2$. Then $N_{0,0,0}= 2^{k-3}$.
    \end{lemma}

    \begin{proof}
        By Lemma \ref{lemma:N_ij_trace} we have
        \begin{equation}\label{eq:from_lemma_1}\left\{\begin{alignedat}{3}
        &N_{0,0,0}+N_{0,0,1}=\#\left\{x\in\F_{2^k} : \TRACE\left(\theta_1x+\gamma_1\right)=0, \TRACE\left(\theta_2x+\gamma_2\right)=0\right\}=2^{k-2}\\
        &N_{0,0,0}+N_{0,1,0}=\#\left\{x\in\F_{2^k} : \TRACE\left(\theta_1x+\gamma_1\right)=0, \TRACE\left(\theta_3x+\gamma_3\right)=0\right\}=2^{k-2}\\
        &N_{0,0,0}+N_{1,0,0}=\#\left\{x\in\F_{2^k} : \TRACE\left(\theta_2x+\gamma_2\right)=0, \TRACE\left(\theta_3x+\gamma_3\right)=0\right\}=2^{k-2}.\\
        \end{alignedat}\right.\end{equation}
        Thus, $N_{0,0,1}=N_{0,1,0}=N_{1,0,0}$. With the same reason we can also obtain  $N_{0,1,1}=N_{1,0,1}=N_{1,1,0}$.
        Recall that $\theta_1+\theta_2+\theta_3\ne 0$, we obtain equations
        \begin{equation}\label{eq:sum_three_trace_1}
            N_{0,0,1}+N_{0,1,0}+N_{1,0,0}+N_{1,1,1}=\#\left\{x\in\F_{2^k} : \TRACE\left(\left(\theta_1+\theta_2+\theta_3\right)x+\left(\gamma_1+\gamma_2+\gamma_3\right)\right)=1\right\}=2^{k-1}
        \end{equation}
        and
        \begin{equation}\label{eq:sum_three_trace_2}
            N_{0,1,1}+N_{1,0,1}+N_{1,1,0}+N_{0,0,0}=\#\left\{x\in\F_{2^k} : \TRACE\left(\left(\theta_1+\theta_2+\theta_3\right)x+\left(\gamma_1+\gamma_2+\gamma_3\right)\right)=0\right\}=2^{k-1}.
        \end{equation}
        Combining Equation \eqref{eq:sum_three_trace_1}, Equation \eqref{eq:sum_three_trace_2} and the following equations
        \begin{equation}\label{eq:sum_N_0jk}\left\{\begin{alignedat}{2}
            &N_{0,0,0}+N_{0,0,1}+N_{0,1,0}+N_{0,1,1}=\#\left\{x\in\F_{2^k} : \TRACE\left(\theta_1x+\gamma_1\right)=0\right\}=2^{k-1}\\
            &N_{1,0,0}+N_{1,0,1}+N_{1,1,0}+N_{1,1,1}=\#\left\{x\in\F_{2^k} : \TRACE\left(\theta_1x+\gamma_1\right)=1\right\}=2^{k-1}\\
            &N_{0,0,1}=N_{0,1,0}=N_{1,0,0}\\
            &N_{0,1,1}=N_{1,0,1}=N_{1,1,0},
        \end{alignedat}\right.\end{equation}
        we obtain $N_{0,0,1}=N_{0,1,1}$.
        Consequently, Equations \eqref{eq:from_lemma_1} and Equations \eqref{eq:sum_N_0jk} can be transformed into
        $N_{0,0,0}+N_{0,0,1}=2^{k-2}$ and $N_{0,0,0}+3N_{0,0,1}=2^{k-1}$, respectively.
        So we have $N_{0,0,0}=N_{0,0,1}=2^{k-3}$. This completes the proof.
    \end{proof}

    With the five lemmas above, we can determine or provide a lower bound on the nonlinearities of all second-order derivatives of the simplest $\mathcal{PS}_{ap}$ bent functions.
    \begin{theorem}\label{thm:nl_DaDbf}
        Let $k\ge 3$ be an integer and $n=2k$ and $f(x,y)=\TRACE\left(\frac{\lambda x}{y}\right)$ be a simplest $\mathcal{PS}_{ap}$ bent function, $\lambda\in\F_{2^k}^*$.
        Then for $\alpha=(\alpha_1,\alpha_2),\beta=(\beta_1,\beta_2)\in\F_{2^k}\times\F_{2^k}$, we have 
        \begin{equation}\label{res:nontrivil_nl}
            nl(D_{\beta}D_{\alpha}f)=\begin{cases}
                2^{2k-1}-2^{k+2},&\text{if both Condition (1) and Condition (2) hold},\\
                2^{2k-1}-2^{k+1},&\text{if either Condition (1) or Condition (2) holds},\\
                0,&\text{if }(\alpha,\beta)\in \mathcal{C},%\footnote[1]{This happends if $ \beta=0 $},
            \end{cases}
        \end{equation}
        and $nl(D_{\beta}D_{\alpha}f)\ge 2^{2k-1}-2^k\left\lfloor 2^{\frac{k}{2}+1}\right\rfloor-2^{k+1}$ otherwise,
        where $\mathcal{C}=\{(x_1,x_2,x_1,x_2):x_1\in\F_{2^k},x_2\in\F_{2^k}^*\}\cup\{(x_1,0,x_2,0):x_1,x_2\in\F_{2^k}\}$.
    \end{theorem}

    \begin{proof}
        Let us consider the Walsh transform of the second-order derivative of $f(x,y)=\TRACE\left(\frac{\lambda x}{y}\right)$ at  
        points $\alpha=(\alpha_1,\alpha_2),\beta=(\beta_1,\beta_2)\in\F_{2^k}\times\F_{2^k}$ with $\lambda\in\F_{2^k}^*$.
        We have
        \begin{align*}\label{eq:secondordersum}
            &W_{D_{\beta}D_{\alpha}f}(\mu,\nu)\nonumber\\
            =&\sum_{x\in\F_{2^k}}\sum_{y\in\F_{2^k}}(-1)^{\TRACE\left(\frac{\lambda x}{y}+\frac{\lambda (x+\alpha_1)}{y+\alpha_2}+\frac{\lambda (x+\beta_1)}{y+\beta_2}+\frac{\lambda (x+\alpha_1+\beta_1)}{y+\alpha_2+\beta_2}+\mu x+\nu y\right)}\nonumber\\
            =&\sum_{y\in\F_{2^k}}(-1)^{\TRACE\left(\frac{\lambda\alpha_1}{y+\alpha_2}+\frac{\lambda\beta_1}{y+\beta_2}+\frac{\lambda(\alpha_1+\beta_1)}{y+\alpha_2+\beta_2}+\nu y\right)}\nonumber\\
            &\times \sum_{x\in\F_{2^k}}(-1)^{\TRACE\left(\left(\frac{\lambda}{y}+\frac{\lambda}{y+\alpha_2}+\frac{\lambda}{y+\beta_2}+\frac{\lambda}{y+\alpha_2+\beta_2}+\mu\right)x\right)}\nonumber\\
            =&\begin{cases}
                2^k\sum_{y\in U}(-1)^{\TRACE\left(\frac{\lambda\alpha_1}{y+\alpha_2}+\frac{\lambda\beta_1}{y+\beta_2}+\frac{\lambda(\alpha_1+\beta_1)}{y+\alpha_2+\beta_2}+\nu y\right)},&~\text{if}~\frac{\lambda}{y}+\frac{\lambda}{y+\alpha_2}+\frac{\lambda}{y+\beta_2}+\frac{\lambda}{y+\alpha_2+\beta_2}=\mu~\text{has solutions},\\
                0, &~\text{otherwise},
            \end{cases}
        \end{align*}
        where $U$ is the set of solutions of equation
        \begin{equation}\label{eq:coefficient}
            \frac{\lambda}{y}+\frac{\lambda}{y+\alpha_2}+\frac{\lambda}{y+\beta_2}+\frac{\lambda}{y+\alpha_2+\beta_2}=\mu.
        \end{equation}
        Note that the number of solutions of Equation \eqref{eq:coefficient} is fully determined in Lemma \ref{lemma:num_sol_second_dev}. 

        For the nonlinearity of $D_{\beta}D_{\alpha}f$, we only need to consider $\max_{\mu,\nu}|W_{D_{\beta}D_{\alpha}f}(\mu,\nu)|$ for every pair $(\alpha,\beta)$.
        Note that if Equation \eqref{eq:coefficient} has no solutions we have $\left\lvert W_{D_{\beta}D_{\alpha}f}(\mu,\nu)\right\rvert=0$.
        So we only need to consider the case that Equation \eqref{eq:coefficient} has solutions.
        According to values of $\alpha_2$ and $\beta_2$ in Lemma \ref{lemma:num_sol_second_dev}, the values of $W_{D_{\beta}D_{\alpha}f}(\mu,\nu)$
        can be divided into the following three cases.

        \begin{enumerate}[label=\textbf{Case \arabic*},wide = 0pt]
            \item The trivial case $\mu=0$, $(\alpha_1,\beta_1)\in\F_{2^k}\times\F_{2^k}$ and $(\alpha_2,\beta_2)\in M$ (where $M$ is defined in Lemma \ref{lemma:num_sol_second_dev}).
            It can be easily verified that any $y\in \F_{2^k}$ is a solution of Equation \eqref{eq:coefficient}.
            Thus, Equation \eqref{eq:coefficient} has $2^k$ solutions in this case.
            So we have
            \begin{equation}\label{eq:case2ksolutions}
                W_{D_{\beta}D_{\alpha}f}(0,\nu)=2^k\sum_{y\in\F_{2^k}}(-1)^{\TRACE\left(\frac{\lambda\alpha_1}{y+\alpha_2}+\frac{\lambda\beta_1}{y+\beta_2}+\frac{\lambda(\alpha_1+\beta_1)}{y+\alpha_2+\beta_2}+\nu y\right)}.
            \end{equation}
            
            
            For simple case, that is 
            $(\alpha_1,\alpha_2,\beta_1,\beta_2)\in\{(x_1,x_2,x_1,x_2):x_1\in\F_{2^k},x_2\in\F_{2^k}^*\}\cup\{(x_1,0,x_2,0):x_1,x_2\in\F_{2^k}\}$,
            % $(\alpha_1,\beta_1)\in\{(x,x):x\in\F_{2^k}\}$ with $(\alpha_2,\beta_2)\in\left\{ (x,x):x\in\F_{2^k}^* \right\}$, 
            Equation \eqref{eq:case2ksolutions} will be in a simple form:
            \[W_{D_{\beta}D_{\alpha}f}(0,\nu)=2^k\sum_{y\in\F_{2^k}}(-1)^{\TRACE\left(\nu y\right)}.\]
            Clearly $\max_{\nu}|W_{D_{\beta}D_{\alpha}f}(0,\nu)|=W_{D_{\beta}D_{\alpha}f}(0,0)=2^{2k}$.
            % Note that another case $(\alpha_1,\beta_1)\in\F_{2^k}\times\F_{2^k}$ with $(\alpha_2,\beta_2)=(0,0)$ will 
            % $\cup\left\{ (x,y):x\in\F_{2^k}\times\{0\},y\in\F_{2^k}\times\{0\} \right\}$, 
            For other cases, we will give an upper bound on $\max_{v}|W_{D_{\beta}D_{\alpha}f}(0,v)|$.
            Consider the case $(\alpha_1,\alpha_2,\beta_1,\beta_2)\in\{(x_1,x_2,x_3,x_2):x_1\ne x_3\in\F_{2^k},x_2\in\F_{2^k}^*\}$, 
            % $\alpha_2=\beta_2\in\F_{2^k}^*$ and $\alpha_1\ne\beta_1$, 
            by Lemma \ref{lemma:charactersums} we then have
            \[\left\lvert W_{D_{\beta}D_{\alpha}f}(0,v)\right\rvert =2^k\left\lvert \sum_{y\in\F_{2^k}}(-1)^{\TRACE\left(\frac{\lambda(\alpha_1+\beta_1)}{y+\alpha_2}+\frac{\lambda(\alpha_1+\beta_1)}{y}+vy\right)}\right\rvert\le 2^{k+1}\left\lfloor 2^{\frac{k}{2}+1}\right\rfloor+2^{k+2}.\]
            
            The case $(\alpha_1,\alpha_2,\beta_1,\beta_2)\in\F_{2^k}^*\times\{0\}\times\F_{2^k}\times\F_{2^k}^*$ and the case $(\alpha_1,\alpha_2,\beta_1,\beta_2)\in\F_{2^k}\times\F_{2^k}^*\times\F_{2^k}^*\times\{0\}$ will lead to analogue proofs with the same upper bounds and we omit them.
            % Therefore, in the cases of $\alpha_2=\beta_2\in\F_{2^k}^*$ or $\alpha_2=0$ or $\beta_2=0$, we have
            % \[\max_{\mu,\nu}|W_{D_{\beta}D_{\alpha}f}(\mu,\nu)|\le 2^{k+1}\left\lfloor 2^{\frac{k}{2}+1}\right\rfloor+2^{k+2}.\]

            \item If $\alpha_2,\beta_2$ are two distinct elements of $\F_{2^k}^*$ and both Condition \ref{item_a} and Condition \ref{item_b} hold,
            it can be guaranteed that Equation~\eqref{eq:coefficient} has $8$ solutions $\{0,\alpha_2,\beta_2,\alpha_2+\beta_2,y_0,y_0+\alpha_2,y_0+\beta_2,y_0+\alpha_2+\beta_2\}$, where $y_0\notin\{0,\alpha_2,\beta_2,\alpha_2+\beta_2\}$, according to Lemma \ref{lemma:num_sol_second_dev}.
            % Then we have two subcases based on the number of solutions of Equation \eqref{eq:coefficient} is $8$ or $4$:
            % \begin{enumerate}[label=(\arabic{*})]
            %     \item When $\mu\ne 0$, if two different elements $\alpha_2,\beta_2\in\F_{2^k}^*$ satisfy
            %     \begin{equation}\label{eq:last_four_solution_condition}
            %         \TRACE\left(\frac{\lambda\alpha_2}{\mu\beta_2(\alpha_2+\beta_2)}\right)=\TRACE\left(\frac{\lambda\beta_2}{\mu\alpha_2(\alpha_2+\beta_2)}\right)=0,
            %     \end{equation}
            %     then $\{y_0,y_0+\alpha_2,y_0+\beta_2,y_0+\alpha_2+\beta_2\}$ are another $4$ solutions of Equation \eqref{eq:coefficient}, where $y_0\notin\{0,\alpha_2,\beta_2,\alpha_2+\beta_2\}$, that is, the number of solutions of Equation \eqref{eq:coefficient} is $8$.
            %     \item Otherwise, $\{0,\alpha_2,\beta_2,\alpha_2+\beta_2\}$ are the only $4$ solutions of Equation \eqref{eq:coefficient}.
            % \end{enumerate}
            % So we can calculate $W_{D_{\beta}D_{\alpha}f}(\mu,\nu)$ for two subcases based on whether or not Condition \ref{item_b} holds.
            % \begin{enumerate}[label=\textbf{Subcase \Alph{*}},itemindent=*,wide=\parindent]
            %     \item Consider the case where Equation \eqref{eq:coefficient} has $8$ solutions,
                % that is, both Condition \ref{item_a} and Condition \ref{item_b} holds. 
            Then we have
            \begin{align*}
            &W_{D_{\beta}D_{\alpha}f}(\mu,\nu)\nonumber\\
                =&2^k\left[1+(-1)^{\TRACE\left((\alpha_1+\beta_1)\mu+ (\alpha_2+\beta_2)\nu\right)}\right]\cdot
                \left[1+(-1)^{\TRACE\left(\alpha_1\mu+\alpha_2\nu\right)}\right]\nonumber\\
                &\cdot
                \left[(-1)^{\TRACE\left(\frac{\lambda\alpha_1}{\alpha_2}+\frac{\lambda\beta_1}{\beta_2}+\frac{\lambda(\alpha_1+\beta_1)}{\alpha_2+\beta_2}\right)}+(-1)^{\TRACE\left(\frac{\lambda\alpha_1}{y_0+\alpha_2}+\frac{\lambda\beta_1}{y_0+\beta_2}+\frac{\lambda(\alpha_1+\beta_1)}{y_0+\alpha_2+\beta_2}+ y_0\nu\right)}\right]\nonumber\\
                =&(-1)^{c_0}2^k\cdot\left[1+(-1)^{\TRACE\left((\alpha_1+\beta_1)\mu+ (\alpha_2+\beta_2)\nu\right)}\right]\cdot
                \left[1+(-1)^{\TRACE\left(\alpha_1\mu+\alpha_2\nu\right)}\right]\cdot\left[1+(-1)^{c_0+c_1}\right]\nonumber\\
                =&\begin{cases}
                    2^{k+3}\cdot(-1)^{c_0},&\text{if }\TRACE\left(\alpha_1\mu+\alpha_2\nu\right)=\TRACE\left(\beta_1\mu+\beta_2\nu\right)=c_0+c_1=0,\\
                    0,&\text{otherwise},
                \end{cases}
            \end{align*}
            where 
            $c_0=\TRACE\left(\frac{\lambda\alpha_1}{\alpha_2}+\frac{\lambda\beta_1}{\beta_2}+\frac{\lambda(\alpha_1+\beta_1)}{\alpha_2+\beta_2}\right)$
            and
            $c_1= \TRACE\left(\frac{\lambda\alpha_1}{y_0+\alpha_2}+\frac{\lambda\beta_1}{y_0+\beta_2}+\frac{\lambda(\alpha_1+\beta_1)}{y_0+\alpha_2+\beta_2}+\nu y_0\right)$.
            By Lemma \ref{lemma:N_ijk_trace},
            for any points $\alpha=(\alpha_1,\alpha_2),\beta=(\beta_1,\beta_2)\in\F_{2^k}\times\F_{2^k}^*$ such that
            $\alpha_2\ne\beta_2$ and satisfying both Condition \ref{item_a} and Condition \ref{item_b}, 
            the number of solutions $\nu\in\F_{2^k}$ for the system
            \begin{empheq}[left=\empheqbiglbrace]{align*}
                &\TRACE\left(\alpha_2\nu + \alpha_1\mu\right)=0\\
                &\TRACE\left(\beta_2 \nu + \beta_1\mu \right)=0\\
                &\TRACE\left(y_0\nu +\frac{\lambda\alpha_1}{\alpha_2}+\frac{\lambda\beta_1}{\beta_2}+\frac{\lambda(\alpha_1+\beta_1)}{\alpha_2+\beta_2}+\frac{\lambda\alpha_1}{y_0+\alpha_2}+\frac{\lambda\beta_1}{y_0+\beta_2}+\frac{\lambda(\alpha_1+\beta_1)}{y_0+\alpha_2+\beta_2} \right)=0,
            \end{empheq}
            is $2^{k-3}\ge 1$.
            This implies there exists $(\mu,\nu)\in\F_{2^k}\times\F_{2^k}$ such that $W_{D_{\beta}D_{\alpha}f}(\mu,\nu)=2^{k+3}\cdot(-1)^{c_0}$.
            So in this case, we have
            \[\max_{\mu,\nu}|W_{D_{\beta}D_{\alpha}f}(\mu,\nu)|=2^{k+3}.\]
            \item If $\alpha_2,\beta_2$ are two distinct elements of $\F_{2^k}^*$ and only one Condition \ref{item_a} or Condition \ref{item_b} holds,
            it can be seen that Equation \eqref{eq:coefficient} has only $4$ solutions.
            Assume Condition \ref{item_a} is satisfied, then $\{0,\alpha_2,\beta_2,\alpha_2+\beta_2\}$ are the only $4$ solutions of Equation \eqref{eq:coefficient}.
            So we have
            \begin{align}\label{eq:simpleforms_4}
                &W_{D_{\beta}D_{\alpha}f}(\mu,\nu)\nonumber\\
                =&2^k\left[1+(-1)^{\TRACE\left((\alpha_1+\beta_1)\mu+ (\alpha_2+\beta_2)\nu\right)}\right]\nonumber\\
                &\cdot
                \left[(-1)^{\TRACE\left(\frac{\lambda\alpha_1}{y+\alpha_2}+\frac{\lambda\beta_1}{y+\beta_2}+\frac{\lambda(\alpha_1+\beta_1)}{y+\alpha_2+\beta_2}+ y\nu\right)}+
                (-1)^{\TRACE\left(\frac{\lambda\alpha_1}{y}+\frac{\lambda\beta_1}{y+\alpha_2+\beta_2}+\frac{\lambda(\alpha_1+\beta_1)}{y+\beta_2}+ (y+\alpha_2)\nu\right)}\right]\nonumber\\
                =&2^k\left[1+(-1)^{\TRACE\left((\alpha_1+\beta_1)\mu+ (\alpha_2+\beta_2)\nu\right)}\right]\nonumber\\
                &\cdot
                (-1)^{\TRACE\left(\frac{\lambda\alpha_1}{y+\alpha_2}+\frac{\lambda\beta_1}{y+\beta_2}+\frac{\lambda(\alpha_1+\beta_1)}{y+\alpha_2+\beta_2}+ y\nu\right)}\cdot
                \left[1+(-1)^{\TRACE\left(\frac{\lambda\alpha_1}{y}+\frac{\lambda\alpha_1}{y+\alpha_2}+\frac{\lambda\alpha_1}{y+\beta_2}+\frac{\lambda\alpha_1}{y+\alpha_2+\beta_2}+\nu\alpha_2\right)}\right]\nonumber\\
                =&2^k\left[1+(-1)^{\TRACE\left((\alpha_1+\beta_1)\mu+ (\alpha_2+\beta_2)\nu\right)}\right]\cdot
                \left[1+(-1)^{\TRACE\left(\alpha_1\mu+\alpha_2\nu\right)}\right]\cdot
                (-1)^{\TRACE\left(\frac{\lambda\alpha_1}{y+\alpha_2}+\frac{\lambda\beta_1}{y+\beta_2}+\frac{\lambda(\alpha_1+\beta_1)}{y+\alpha_2+\beta_2}+ y\nu\right)}\nonumber\\
                =&\begin{cases}
                    2^{k+2}\cdot(-1)^{\TRACE\left(\frac{\lambda\alpha_1}{y+\alpha_2}+\frac{\lambda\beta_1}{y+\beta_2}+\frac{\lambda(\alpha_1+\beta_1)}{y+\alpha_2+\beta_2}+ y\nu\right)},&\text{if}~\TRACE\left(\alpha_2\nu+\alpha_1\mu\right)=0 ~
                    \text{and}~\TRACE\left(\beta_2\nu+\beta_1 \mu\right)=0 \\
                    0,~&\text{otherwise}.
                \end{cases}
            \end{align}
            % Observing \eqref{eq:simpleforms_4} we can find that $|W_{D_{\beta}D_{\alpha}f}(\mu,\nu)|$ only has two possible values $\{0,2^{k+2}\}$.
            According to Lemma \ref{lemma:N_ij_trace}, for any $\alpha=(\alpha_1,\alpha_2),\beta=(\beta_1,\beta_2)\in\F_{2^k}\times\F_{2^k}^*$ such that $\alpha_2\ne\beta_2$, the number of solutions $\nu\in\F_{2^k}$ for the system
            \begin{equation}\label{eq:max_foursolution_condition}
                \left\{
                \begin{alignedat}{2}
                    \TRACE\left(\alpha_2\nu+\alpha_1\mu\right)&=0\\
                    \TRACE\left(\beta_2\nu +\beta_1 \mu\right)&=0
                \end{alignedat}
                \right.
            \end{equation}
            is $2^{k-2}$.
            Thus, in the case where Condition \ref{item_a} holds, we have
            \[\max_{\mu,\nu}|W_{D_{\beta}D_{\alpha}f}(\mu,\nu)|=2^{k+2}.\]
            Assume Condition \ref{item_b} holds, 
            it can be seen that $\{y_0,y_0+\alpha_2,y_0+\beta_2,y_0+\alpha_2+\beta_2\}$
            are the only $4$ solutions of
            Equation \eqref{eq:coefficient}, where $y_0\notin\{0, \alpha_2, \beta_2, \alpha_2+\beta_2\}$. 
            The rest of proof  is similar to the case where only Condition \ref{item_a} holds, so we omit it and 
            straightforwardly present the result $\max_{\mu,\nu}|W_{D_{\beta}D_{\alpha}f}(\mu,\nu)|=2^{k+2}$. 
            Therefore, in the case where Equation \eqref{eq:coefficient} has only $4$ solutions, we have 
            \[\max_{\mu,\nu}|W_{D_{\beta}D_{\alpha}f}(\mu,\nu)|=2^{k+2}.\] 
        \end{enumerate}
        This completes the proof.
    \end{proof}

\subsection{A lower bound on the third-order nonlinearity of the simplest $\mathcal{PS}_{ap}$ bent functions}
    After obtaining the nonlinearities of all second-order derivatives of the simplest $\mathcal{PS}_{ap}$ bent functions, we can derive a lower bound on the third-order nonlinearity by Lemma \ref{thm:High_order_nl_bound1}.
    Applying twice Lemma \ref{thm:High_order_nl_bound1}, that is, taking
    \[nl_{r-1}(D_{\alpha}f) \ge 2^{n-1}-\frac{1}{2}\sqrt{2^{2n}-2\sum_{\beta\in\F_2^n}nl_{r-2}(D_{\beta}D_{\alpha}f)},\]
    into the right-hand side of the following inequality
    \[nl_r(f) \ge 2^{n-1}-\frac{1}{2}\sqrt{2^{2n}-2\sum_{\alpha\in\F_2^n}nl_{r-1}(D_{\alpha}f)},\]
    we obtain the relation between the $r$-th order nonlinearity of $f$ and the $(r-2)$-th order nonlinearities of all second-order derivatives of $f$:
    \begin{equation}\label{eq:nl3_nlDaDbf}
        nl_r(f)\ge 2^{n-1}-\frac{1}{2}\sqrt{\sum_{\alpha\in\F_{2^n}}\sqrt{2^{2n}-2\sum_{\beta\in\F_{2^n}} nl_{r-2}(D_{\beta}D_{\alpha}f)}}.
    \end{equation}
    Therefore, consider $r=3$ in Inequality \eqref{eq:nl3_nlDaDbf} with Theorem \ref{thm:nl_DaDbf}, we can provide a lower bound on the third-order nonlinearity of the simplest $\mathcal{PS}_{ap}$ bent functions.
    \begin{theorem}\label{th:our_lower_bound}
        Let $k\ge 3$ be an integer and $n=2k$. For the third-order nonlinearity of the simplest $\mathcal{PS}_{ap}$ bent function $f(x,y)=\TRACE(\frac{\lambda x}{y})$ with $x,y\in\F_{2^k}$ and $\lambda\in\F_{2^k}^*$, we have:
        \[nl_3(f)\ge 2^{n-1}-\frac{1}{2}\sqrt{A},\]
        % \approx 2^{n-1}-2^{\frac{7n}{8}+\frac{1}{2}}.
        where
        \begin{align*}
            A=2^n+&(2^{\frac{n}{2}}-1)\sqrt{(2^{\frac{3n}{2}+1}-2^{n+1})\left\lfloor 2^{\frac{n}{4}+1}\right\rfloor+5\cdot 2^{\frac{3n}{2}}-2^{n+2}}\\
            +&(2^n-2^{\frac{n}{2}})\sqrt{2^{\frac{3n}{2}+2}-15\cdot 2^n-2^{\frac{n}{2}+2}+(2^{n+2}-2^{\frac{n}{2}+1})\left\lfloor 2^{\frac{n}{4}+1}\right\rfloor+2^{n+2}L},
        \end{align*}
        and $L$ is defined in Lemma \ref{L:SumInv00}.
    \end{theorem}
    \begin{proof}
        We have
        \begin{align*}
            nl_3(f)&\ge 2^{n-1}-\frac{1}{2}\sqrt{\sum_{\alpha\in\F_{2^n}}\sqrt{2^{2n}-2\sum_{\beta\in\F_{2^n}} nl(D_{\beta}D_{\alpha}f)}}\\
            &=2^{n-1}-\frac{1}{2}\sqrt{\left( \sum_{\alpha=(0,0)}+\sum_{\alpha=(\alpha_1,0)\in\F_{2^k}^*\times\{0\}}+\sum_{\substack{\alpha=(\alpha_1,\alpha_2)\in\F_{2^k}\times\F_{2^k}^*}} \right)\sqrt{2^{2n}-2\sum_{\beta\in\F_{2^n}} nl(D_{\beta}D_{\alpha}f)}}\\
            &\ge 2^{n-1}-\frac{1}{2}\left[2^n+(2^{\frac{n}{2}}-1)\sqrt{2^{2n}-2(2^n-2^{\frac{n}{2}})(2^{n-1}-2^{\frac{n}{2}}\left\lfloor 2^{\frac{n}{4}+1}\right\rfloor-2^{\frac{n}{2}+1})}\right.\\
            &\qquad\qquad\qquad\left.+(2^n-2^{\frac{n}{2}})\sqrt{2^{2n}-2\left( (2^{n-1}-2^{\frac{n}{2}+1})(2^n-1)-(2^{n+1}-2^{\frac{n}{2}})\left\lfloor 2^{\frac{n}{4}+1}\right\rfloor+2^{n+3}-2^{n+1}L \right)}\right]^{\frac{1}{2}},
        \end{align*}
        where the second sign of inequality comes from Lemma \ref{thm:nl_DaDbf}.
        Then, we have
        \begin{align*}
            A=2^n+&(2^{\frac{n}{2}}-1)\sqrt{(2^{\frac{3n}{2}+1}-2^{n+1})\left\lfloor 2^{\frac{n}{4}+1}\right\rfloor+5\cdot 2^{\frac{3n}{2}}-2^{n+2}}\\
            +&(2^n-2^{\frac{n}{2}})\sqrt{2^{\frac{3n}{2}+2}-15\cdot 2^n-2^{\frac{n}{2}+2}+(2^{n+2}-2^{\frac{n}{2}+1})\left\lfloor 2^{\frac{n}{4}+1}\right\rfloor+2^{n+2}L}.
        \end{align*}
        This completes the proof.
    \end{proof}
    \begin{corollary}
        Let $n\ge 6$ be an arbitrary even integer. The third-order nonlinearity of the simplest $\mathcal{PS}_{ap}$ bent functions in $n$ variables is lower bounded by approximately $2^{n-1}-2^{\frac{7n}{8}-\frac{1}{2}}$.
    \end{corollary}
    \begin{corollary}
        Let $n=2k\ge 6$ be an arbitrary even integer and $4\le r\le n-2$ be an arbitrary integer. Then for the $r$-th order nonlinearity of any simplest $\mathcal{PS}_{ap}$ bent function $f$ in $n$ variables, we have 
        \[nl_r(f)\ge 2^{n-1}-\frac{1}{2}\sqrt{2^{2n+1}-2^{\frac{n}{2}+2}(2^{\frac{n}{2}}-1)nl_{r-1}(f)+2^{n+1}l_{r-1}},\]
        where $l_r$ is defined by $l_1=2^{\frac{n}{4}}$ and $l_r=\sqrt{(2^{\frac{n}{2}}-1)(l_{r-1}-1)+2^{\frac{n}{2}-2 }}$. 
    \end{corollary}
    \begin{proof}
        In \cite{Carlet2008lowbound_NL_profile}, Carlet proved that 
        \[\forall\alpha=(\alpha_1,\alpha_2)\in\F_{2^k}\times\F_{2^k}^*, nl_r(D_{\alpha}f)\ge 2nl_r(f)-2^{n-1}-2^{\frac{n}{2}},\]
        and 
        \[\forall\alpha=(\alpha_1,0)\in\F_{2^k}\times\{0\}, nl_r(D_{\alpha}f)\ge 2^{n-1}-2^{\frac{n}{2}}(l_r+1),\]
        where $l_r$ is defined by $l_1=2^{\frac{n}{4}}$ and $l_r=\sqrt{(2^{\frac{n}{2}}-1)(l_{r-1}-1)+2^{\frac{n}{2}-2 }}$. 
        Therefore, by Lemma \ref{thm:High_order_nl_bound1}, we have 
        \begin{align*}
            &nl_r(f)\\
            \ge& 2^{n-1}-\frac{1}{2}\sqrt{2^{2n}-2\sum_{\alpha\in\F_{2^n}} nl_{r-1}(D_{\alpha}f)}\\
            =& 2^{n-1}-\frac{1}{2}\sqrt{2^{2n}-2\sum_{\alpha=(\alpha_1,\alpha_2)\in\F_{2^k}\times\F_{2^k}}nl_{r-1}(D_{\alpha}f)}\\
            =& 2^{n-1}-\frac{1}{2}\sqrt{2^{2n}-2\sum_{\alpha=(\alpha_1,\alpha_2)\in\F_{2^k}\times\F_{2^k}^*}nl_{r-1}(D_{\alpha}f)-2\sum_{\alpha=(\alpha_1,0)\in\F_{2^k}\times\{0\}}nl_{r-1}(D_{\alpha}f)} \\
            \ge& 2^{n-1}-\frac{1}{2}\sqrt{2^{2n}-2\sum_{\alpha=(\alpha_1,\alpha_2)\in\F_{2^k}\times\F_{2^k}^*}\left(2nl_{r-1}(f)-2^{n-1}-2^{\frac{n}{2}}\right)-2\sum_{\alpha=(\alpha_1,0)\in\F_{2^k}\times\{0\}}\left(2^{n-1}-2^{\frac{n}{2}}(l_{r-1}+1)\right)}\\
            =& 2^{n-1}-\frac{1}{2}\sqrt{2^{2n}-2^{k+1}(2^k-1)\left(2nl_{r-1}(f)-2^{n-1}-2^{\frac{n}{2}}\right)-2^{k+1}\left(2^{n-1}-2^{\frac{n}{2}}(l_{r-1}+1)\right)} \\
            =& 2^{n-1}-\frac{1}{2}\sqrt{2^{2n+1}-2^{\frac{n}{2}+2}(2^{\frac{n}{2}}-1)nl_{r-1}(f)+2^{n+1}l_{r-1}}. 
        \end{align*}
        This completes the proof.
    \end{proof}
    \begin{remark}
        Note that lower bounds \cite{TangCT2013NL_2bent,Carlet2011NL_Profile_Dillon} rely on the imprecise nonlinearities of all second-order derivatives of the simplest $\mathcal{PS}_{ap}$ bent functions deduced from some inequalities.
        In contrast, we are able to determine the nonlinearity of all second-order derivatives of the simplest $\mathcal{PS}_{ap}$ bent functions, except the trivial cases for which only a lower bound on the nonlinearities of the second-order derivatives can be derived.
        Fortunately, compared with the number of nontrivial cases, the number of the trivial cases is negligible.
        This means our lower bound is almost determined by those nontrivial cases.
        That means our lower bound almost relies on the precise nonlinearities of the second-order derivatives.
        Therefore, our lower bound is tighter than others.
    \end{remark}
    \begin{remark}
        \newcommand{\rb}[1]{\raisebox{1.5ex}[0pt]{#1}}
        We only compare in Table \ref{table:MyTableLabel} our lower bound on the third-order nonlinearity of the simplest $\mathcal{PS}_{ap}$ bent functions with two previously known bounds  in \cite{TangCT2013NL_2bent,Carlet2011NL_Profile_Dillon} for some small concrete values.
        It can be seen that our lower bound on the third-order nonlinearity is always tighter than other two lower bounds.
        And when $n$ is not too large, our lower bound is much more better than others.
        \begin{table}
            \centering
            \caption{Comparison of lower bounds on the third-order nonlinearity of the simplest $\mathcal{PS}_{ap}$ bent functions}
            \begin{threeparttable}
                \begin{tabular}{|c|c|c|c|c|}
                    \hline
                            & Tang-Carlet-Tang bound      & Carlet bound                            & Our bound& \\
                    \rb{$n$}& in \cite{TangCT2013NL_2bent}& in \cite{Carlet2011NL_Profile_Dillon} & in Theorem \ref{th:our_lower_bound}     &\rb{Difference\tnote{1}}   \\
                    \hline
                    $6  $ &  $ -           $       & $ -        $     & $ 1        $     & $  -       $ \\  \hline
                    $8  $ &  $ -           $       & $ -        $     & $ 26       $     & $  -       $ \\  \hline
                    $10 $ &  $ 55          $       & $ 63       $     & $ 186      $     & $  123     $ \\  \hline
                    $12 $ &  $ 533         $       & $ 553      $     & $ 932      $     & $  379     $ \\  \hline
                    $14 $ &  $ 3156        $       & $ 3205     $     & $ 4372     $     & $  1167    $ \\  \hline
                    $16 $ &  $ 15985       $       & $ 16103    $     & $ 19986    $     & $  3883    $ \\  \hline
                    $18 $ &  $ 75003       $       & $ 75292    $     & $ 89028    $     & $  13736   $ \\  \hline
                    $20 $ &  $ 336634      $       & $ 337330   $     & $ 384926   $     & $  47596   $ \\  \hline
                    $22 $ &  $ 1468219     $       & $ 1469893  $     & $ 1628771  $     & $  158878  $ \\  \hline
                    $24 $ &  $ 6278535     $       & $ 6282551  $     & $ 6807315  $     & $  524764  $ \\  \hline
                    $26 $ &  $ 26469867    $       & $ 26479472 $     & $ 28238083 $     & $  1758611 $ \\  \hline
                \end{tabular}
                \begin{tablenotes}
                    \footnotesize
                    \item[$\star$] All concrete values of lower bounds in this table are approximated to nearest  integers greater than or equal to values.
                    \item[1] The values in the table are the difference of our bound's values with  Carlet bound's values.
                \end{tablenotes}
            \end{threeparttable}
            \label{table:MyTableLabel}
        \end{table}
    \end{remark}


\section{Conclusion}
    The ability of symmetric-key encryption to withstand some known attacks, such as algebraic attacks and low-degree approximation attacks, can be evaluated by the higher-order nonlinearity of Boolean functions used in symmetric-key encryption.
    % The higher-order nonlinearity of Boolean functions measure the ability of cryptosystems to resist some attacks like algebraic attacks and low-degree approximation.
    Consequently, it is crucial that Boolean functions employed in the design of symmetric-key encryption meet this cryptographic criterion.
    In this paper, we derived a lower bound on the third-order nonlinearity of the simplest $\mathcal{PS}_{ap}$ bent functions.
    Our lower bound improved the previous two lower bounds in earlier works \cite{TangCT2013NL_2bent,Carlet2011NL_Profile_Dillon}.
    We hope that the result obtained in this paper will play an important role with respect to countermeasures against these known attacks.

\bibliographystyle{plain}
\bibliography{mybib}

\end{document}



% for i in [1..2^5] do
%     inputvector:=Intseq(i-1,2,n);
%     eltvector:=&+[inputvector[j]*v^(j-1):j in [1..n]];
%     Append(~input,(eltvector));
% end for;

% for i in [1..2^n] do
%     inputvector:=Intseq(sbox[i],2,n);
%     eltvector:=&+[inputvector[j]*v^(j-1):j in [1..n]];
%     Append(~output,(eltvector));
% end for;

% function newsbox(x)
%     if x eq 0 then
%         return 0;
%     end if;
%     for i in [1..1024] do
%         if x eq input[i] then
%             return output[i];
%         end if;
%     end for;
% end function;



% G:=OrthoTest(newsbox,n);
% ddtG:=DDTexe(G,n);



%   \begin{tikzpicture}[
%     node distance = 5ex,
%     scale = 3,
%     thick,
%     > = latex,
%     % change the
%     z = {(0.35, -0.4)},
%     edge/.style = {draw, thick, -, black},
%     sinal/.style = {inner sep = 1pt, thin, opacity = 0.4,
%       fill = blue, circle, text opacity = 1},
%     mtx/.style = {
%   %     matrix of math nodes,
%       matrix of nodes,
%       every node/.style = {
%         anchor = base,
%         text width = 2em,
%         text height = 1em,
%         align = center,
%       }
%     },
%     ]

%     \def\dist{0.1}
%     \def\cube{
%         % Vertices. (A,B,C), A x轴  B z轴  C y轴


%         % \node[left] (v0) at (0,0,0) {$ A $};
%         % note that command above can construct nodes and label them at the same time,
%         % but sometimes you don't need the text,
%         % so I just construct the coordinates and then label coordinates
%         \coordinate (v0) at (0, 0, 0)  ;
%         \coordinate (v1) at (0, 1, 0)  ;
%         \coordinate (v2) at (1, 0, 0)  ;
%         \coordinate (v3) at (1, 1, 0)  ;
%         \coordinate (v4) at (0, 0, 1)  ;
%         \coordinate (v5) at (0, 1, 1)  ;
%         \coordinate (v6) at (1, 0, 1)  ;
%         \coordinate (v7) at (1, 1, 1)  ;
%         \coordinate (v8) at (0, 2, 0)  ;
%         \coordinate (v9) at (1, 2, 0)  ;
%         \coordinate (v10) at (0, 2, 1) ;
%         \coordinate (v11) at (1, 2, 1) ;
%     }
%     \begin{scope}[opacity=1] % opacity is the transparent
%         \cube{};
%         % labeling verticals with text A B C at left\right\below\above\below left\below right\above left\above right
%         \node[left] at (v0) {$ A $};
%         \node[left] at (v1) {$ B $};
%         \node[right] at (v2) {$ C $};
%         \node[above right] at (v3) {$ D $};
%         \node[left] at (v4) {$ E $};
%         \node[below left] at (v5) {$ F $};
%         \node[right] at (v6) {$ G $};
%         \node[right] at (v7) {$ H $};
%         \node[left] at (v8) {$ I $};
%         \node[above right] at (v9) {$ J $};
%         \node[below left] at (v10) {$ K $};
%         \node[right] at (v11) {$ L $};
%         % Edges with some differential: alpha gamma beta theta
%         % arrow with direction from v1 to v0

%         \draw[->] (v2) -- (v3);
%         \draw[->] (v3) -- (v9);
%         \draw[->] (v6) -- (v7);
%         \draw[->] (v7) -- (v11);
%         % dotted line from v1 to v2 and the middle of line labeled is gamma
%         \draw[dashed] (v0) -- node[fill = white] {$ \gamma $} (v2) ;
%         \draw[dashed] (v4) -- node[fill = white] {$ \gamma $} (v6);
%         \draw[dashed] (v1) -- node[fill = white] {$ \beta $} (v5) -- node[fill = white] {$ \theta $} (v7) -- node[fill = white] {$ \beta $}(v3) -- node[fill = white] {$ \theta $} (v1);
%         \draw[dashed] (v8) -- node[fill = white] {$ \alpha $}(v10);
%         \draw[dashed] (v11) -- node[fill = white] {$ \alpha $}(v9);
%     \end{scope}

%     \begin{scope}[opacity=0.2]
%         % the pics in this part are transparent 0.2,
%         % if not want this condition, delete the commands.
%         \draw[<-] (v0) -- (v1);
%         \draw[<-] (v1) -- (v8);
%         \draw[<-] (v4) -- (v5);
%         \draw[<-] (v5) -- (v10);
%     \end{scope}

%         % \foreach \i in {0, 1, ..., 11}{ \draw[fill = black] (v\i) circle (0.1pt); }
%         % } % boomerang attack model

%         \begin{scope}[]
%             %
%             \coordinate (E0)  at (2, 0-0.4, 0);
%             \coordinate (E0L) at (2-0.3, 0-0.4, 0);
%             \coordinate (E0R) at (2+0.15, 0-0.4, 0);
%             \coordinate (E1)  at (2, 1-0.4, 0);
%             \coordinate (E1L) at (2-0.3, 1-0.4, 0);
%             \coordinate (E1R) at (2+0.15, 1-0.4, 0);
%             \coordinate (E2)  at (2, 2-0.4, 0);
%             \coordinate (E2L) at (2-0.3, 2-0.4, 0);
%             \coordinate (E2R) at (2+0.15, 2-0.4, 0);
%             \draw[->] (E0) -- node[right] {$ E_1^{-1} $} (E1);
%             \draw[->] (E2) -- node[right] {$ E_0 $} (E1);
%             % dotted line with transparent
%             \draw[dashed,opacity=.5] (E0R) -- (E0) -- (E0L);
%             \draw[dashed,opacity=.5] (E1R) -- (E1) -- (E1L);
%             \draw[dashed,opacity=.5] (E2R) -- (E2) -- (E2L);
%         % \foreach \i in {0, 1, ..., 11}{
%         %   \node at (v\i) {\i};
%         % }
%       \end{scope}

%     \end{tikzpicture}%

F<v>:=GF(2,8);
G:=[x:x in F|x^8 + x^4 + x^3 + x^2 + 1 eq 0];
for g in G do
    g:=v;
    // sbox=[0, 152, 136, 56, 65, 217, 149, 189, 142, 95, 147, 214, 10, 103, 171, 213, 185, 4, 3, 20, 93, 54, 199, 202, 198, 116, 227, 33, 124, 181, 56, 196, 106, 5, 53, 40, 169, 194, 117, 56, 58, 122, 82, 103, 60, 9, 178, 182, 111, 194, 132, 67, 83, 248, 191, 27, 228, 122, 186, 157, 232, 61, 96, 24, 26, 6, 145, 122, 45, 76, 121, 25, 204, 64, 58, 52, 142, 177, 11, 37, 254, 244, 188, 143, 2, 206, 39, 59, 105, 140, 48, 101, 28, 161, 241, 228, 182, 193, 124, 211, 65, 13, 0, 116, 55, 254, 4, 9, 218, 12, 100, 24, 233, 116, 140, 54, 224, 53, 158, 51, 153, 73, 222, 88, 22, 68, 45, 182, 165, 33, 134, 12, 150, 118, 162, 210, 49, 123, 233, 9, 154, 56, 207, 176, 203, 54, 164, 210, 218, 159, 66, 222, 89, 139, 220, 22, 143, 187, 37, 83, 142, 112, 152, 202, 179, 251, 33, 15, 114, 126, 132, 163, 158, 222, 216, 110, 245, 153, 76, 173, 197, 226, 232, 254, 45, 109, 110, 55, 247, 95, 56, 107, 7, 102, 152, 14, 29, 231, 100, 169, 255, 145, 63, 22, 236, 121, 200, 154, 38, 237, 17, 76, 223, 46, 7, 92, 60, 6, 110, 38, 222, 149, 45, 69, 32, 242, 189, 200, 151, 193, 38, 176, 195, 88, 100, 204, 28, 232, 181, 193, 66, 146, 67, 137, 8, 86, 47, 216, 111, 218, 174, 6, 172, 113, 104, 103 ];

    nonlinearity v:=v^24
    f:=func<x|x^3 + g^60*x^5 + g^191*x^6 + g^198*x^9 + g^232*x^10 + g^120*x^12+ g^54*x^17 + g^64*x^18 + g^159*x^20 + g^144*x^24 + g^248*x^33+ g^203*x^34 + g^32*x^36 + g^18*x^40 + g^216*x^48 + g^78*x^65+g^46*x^66 + g^91*x^68 + g^27*x^72 + g^70*x^80 + g^52*x^96+ g^224*x^129 + g^18*x^130 + g^197*x^136 + g^253*x^144 + x^160>;
    for b in F do
        if b eq 0 then continue; end if;
        Walsh_spectra:=[];
        for a in F do
            Append(~Walsh_spectra,Abs(&+[(-1)^(Integers()!Trace(b*f(x)+a*x)):x in F]));
        end for;
        Set(Walsh_spectra);
        //if Max(Walsh_spectra) eq 128 then
        //    print "b=",b;
        //end if;
    end for;
end for;

F<v>:=GF(2,8);
g:=v;
for i in [10..255] do
    i:=24;
    f:=func<x|Trace(v^i*(x^3 + g^60*x^5 + g^191*x^6 + g^198*x^9 + g^232*x^10 + g^120*x^12+ g^54*x^17 + g^64*x^18 + g^159*x^20 + g^144*x^24 + g^248*x^33+ g^203*x^34 + g^32*x^36 + g^18*x^40 + g^216*x^48 + g^78*x^65+g^46*x^66 + g^91*x^68 + g^27*x^72 + g^70*x^80 + g^52*x^96+ g^224*x^129 + g^18*x^130 + g^197*x^136 + g^253*x^144 + x^160))>;
    if Max([&+[(-1)^(Integers()!(f(x)+Trace(a*x))):x in F]:a in F]) eq 128 then
    wf:=[&+[(-1)^(Integers()!(f(x)+Trace(a*x))):x in F]:a in F];
    i;
    end if;
end for;

subspace_set:={};
for a,b in GF(2,4) do
    if a eq b then continue; end if;
    flag:=0;
    for x in F do
        if f(x)+f(x+a)+f(x+b)+f(x+a+b) ne 0 then
            flag:=1;
            break;
        end if;
    end for;
    if flag eq 1 then
        print "not a subspace";
        break;
    end if;
end for;




% for g=v, we have b=v^24 s.t. bf is an affine function.
======================================================================================

diff_spec:=[
    [ 0, 2, 6, 595386, 416361, 35805 ],
    [ 0, 2, 4, 6, 8, 12, 713031, 211761, 92070, 15345, 5115, 10230 ],
    [ 0, 2, 4, 6, 8, 629331, 330336, 72540, 13020, 2325 ],
    [ 0, 2, 4, 6, 8, 628401, 329871, 75330, 12555, 1395 ],
    [ 0, 2, 4, 6, 8, 10, 633636, 322701, 75045, 13980, 1905, 285 ],
    [ 0, 2, 4, 6, 8, 10, 12, 630216, 327081, 76215, 12150, 1665, 195, 30 ],
    [ 0, 2, 4, 6, 8, 10, 635314, 317626, 80290, 11780, 2480, 62 ],
    [ 0, 2, 4, 6, 8, 10, 631811, 322617, 80197, 11098, 1674, 155 ],
    [ 0, 2, 4, 6, 8, 10, 633733, 320695, 78399, 12803, 1736, 186 ],
    [ 0, 2, 4, 6, 8, 10, 641514, 307706, 81375, 14880, 1705, 372 ],
    [ 0, 2, 4, 6, 8, 10, 634260, 321036, 76353, 13857, 1767, 279 ],
    [ 0, 2, 4, 6, 8, 10, 12, 14, 630664, 324942, 78647, 11842, 1364, 31, 31, 31 ],
    [ 0, 2, 4, 6, 8, 22, 636306, 315018, 82335, 11715, 2145, 33 ],
    [ 0, 2, 4, 6, 8, 637701, 313131, 80910, 14415, 1395 ],
    [ 0, 2, 4, 6, 8, 626541, 330336, 79515, 10230, 930 ],
    [ 0, 2, 4, 6, 8, 10, 634291, 318401, 81995, 10385, 2170, 310 ],
    [ 0, 2, 4, 6, 8, 640491, 304296, 89280, 13020, 465 ],
    [ 0, 2, 4, 6, 8, 10, 12, 632431, 322958, 77562, 13020, 1333, 186, 62 ],
    [ 0, 2, 4, 6, 8, 624216, 334986, 76725, 11160, 465 ],
    [ 0, 2, 4, 6, 8, 10, 639045, 311861, 80410, 13860, 2365, 11 ]
];

=====================================================================================
// we want to find 10 bit quadratic APN, and test whether they are ccz-eq to some APN classes or instances
// f(x)=x^d+x^i+beta*x^j is a trinomial, it ccz eq to x^d +alpha*x^i+beta*x^j, so we confirm that no new APN instance for trinomial.
// d = 5,6,9,10,12,17,33,34,40,48,65,66,68,72,80,96,129,130,132,136,144,160,192,257,258,260,264, 272, 288, 320, 384, 513, 514, 516, 520, 528, 544, 576, 640, 768 no APN instance

// test x^d + x^i + beta x^j, exclude the case of EA s.t. L2 . F . L1
n:=10;
Z:=Integers();
F<v>:=GF(2,n);
F5<w>:=GF(2,n div 2);
Fstar:=[v^i:i in [0..2^n-2]];
P<x>:=PolynomialRing(F);
index_wt_2:={2^i+2^j:i,j in [0..n-1]|i ne j};

load "DiffSpecOdExe.m";
for d in D do
    for i,j in index_wt_2 do
        if i eq d or j eq d or i eq j then continue; end if;
        //exclude the case of L2=a x^2i + b x^2j + x and F = x^3
        if i in {2^k*3:k in [0..n-1]} and j in {2^k*3:k in [0..n-1]} then continue; end if;
        if i in {2^k*3^2:k in [0..n-1]} and j in {2^k*3^2:k in [0..n-1]} then continue; end if;
        ccz_coef_set:={};
        i,j;
        for beta in Fstar do
            if #{x^(2^i):x in ccz_coef_set,i in [1..n-1]} eq 2^n-1 then break; end if;
            // exclude the case of L2=2^{n-i},L1=2^i
            if beta in {x^(2^i):x in ccz_coef_set,i in [1..n-1]} then continue; end if;
            f:=x^d+x^i+beta*x^j;
            if IsAPN(f) eq true then
                print "f is APN and parameters [i,j,beta] are:";
                i,j,beta;
                ddtS:=DiffSpecOd(func<x|x^d+x^i+beta*x^j>,n);
                ddtS;
                if ddtS notin diff_spec then
                    print "new diff_spec_od,i=",i;
                    Write("APN_10bit_new_instance.txt",[i,j]);
                    Write("APN_10bit_new_instance.txt",beta);
                else
                    print "existing instance!";
                    // exclude the case of L2=a^d x,L1=x/a
                    Include(~ccz_coef_set,beta);
                end if;
            else
                //print "not APN";
                Include(~ccz_coef_set,beta);
            end if;
        end for;
    end for;
end for;


//another f(x) is a quadrinomial, f = x^9 + a*Tr_2^n(b*x^i)

n:=10;
d:=9;
Z:=Integers();
F<v>:=GF(2,n);
Fstar:=[v^i:i in [0..2^n-2]];
P<x>:=PolynomialRing(F);
index_wt_2:={2^i+2^j:i,j in [0..n-1]|i ne j};

for i in index_wt_2 do
    if i in [2^j*d:j in [0..n-1]] then continue; end if;
    i;
    ccz_coef_set:={};
    for beta in Fstar do
        if #{x^(2^i):x in ccz_coef_set,i in [1..n-1]} eq 2^n-1 then break; end if;
        // exclude the case of L2=2^{n-i},L1=2^i
        if beta in {x^(2^i):x in ccz_coef_set,i in [1..n-1]} then continue; end if;
        f:=func<x|x^d+ v^2*(&+[(beta*x^i)^((2^2)^(j-1)):j in [1..n div 2]])>;
        if IsAPN(f,n) eq true then
            print "f is APN and parameters [i,j,beta] are:";
            i,beta;
            ddtS:=DiffSpecOd(f,n);
            ddtS;
            if ddtS notin diff_spec then
                print "new diff_spec_od,i=",i,"beta=",beta;
                Write("APN_10bit_new_instance.txt",i);
                Write("APN_10bit_new_instance.txt",beta);
            else
                print "existing instance!";
                // exclude the case of L2=a^d x,L1=x/a
                Include(~ccz_coef_set,beta);
            end if;
        else
            //print "not APN";
            Include(~ccz_coef_set,beta);
        end if;
    end for;
end for;



=====================================================================================



function Sbox2ff(inputsbox)
    Z:=Integers();
    n:=Ilog2(#inputsbox);
    F<v>:=GF(2,n);
    list:=[];
    for i in [0..2^n-2] do
        Append(~list,&+[Intseq(inputsbox[&+[Z!Eltseq(v^i)[j]*2^(j-1):j in [1..n]]+1],2,n)[k]*v^(k-1):k in [1..n]]);
    end for;
    Append(~list,F!0);
    function ff_fromsbox(x)
        return (x eq 0) select F!0 else list[Log(x)+1];
    end function;
    return ff_fromsbox;
end function;


======================================================================================

function IsAPN(f,variables)
    n:=variables;
    F<al>:=GF(2,n);
    for i in [0..2^n-2] do
        a:=al^i;
        set_b:={}; set_a:={};
        for y in F do
            if not y in set_a then
                b:=f(y+a)-f(y);
                if b notin set_b then
                    Include(~set_b,b);
                else
                    return false;
                end if;
                Include(~set_a,y+a);
            end if;
        end for;
    end for;
    return true;
end function;


function IsAPN(f)
    P:=Parent(f);
    F<al>:=BaseRing(P);
    n:=Degree(F);
    for i in [0..2^n-2] do
    a:=al^i;
    set_b:={}; set_a:={};
    for y in F do if not y in set_a then
    b:=Evaluate(f,y+a)-Evaluate(f,y);
    if b notin set_b then Include(~set_b,b);
    else
    return false; end if;
    Include(~set_a,y+a); end if;
    end for; end for;
    return true;
end function;


Z:=Integers();
F<v>:=GF(2,8);
[&+[Z!Eltseq(x)[i]*2^(i-1):i in [1..8]]:x in F]; 