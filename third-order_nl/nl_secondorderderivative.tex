\documentclass{article}
\usepackage{fullpage,enumitem,amsthm,amsmath,amssymb,graphicx,empheq}
\usepackage{physics}
\usepackage{tcolorbox}
\usepackage[citecolor=blue]{hyperref}
\usepackage{tikz}
\usepackage{cite}

\newcommand{\Z}{\mathbf{Z}}
\newcommand{\F}{\mathbb{F}}
\newcommand{\Com}{\mathbf{C}}
\newcommand{\ord}{\operatorname{ord}}
\newcommand{\Q}{\mathbf{Q}}
\newcommand{\R}{\mathbf{R}}
\newcommand{\E}{\mathbb{E}}
\newcommand{\0}{\textbf{0}}
\newcommand{\1}{\textbf{1}}
\newcommand{\B}{\mathcal{B}}
\newcommand{\nl}{\mathrm{nl}}
\newcommand{\TRACE}{\operatorname{Tr}_1^k}
\newcommand{\TrN}{\operatorname{Tr}_1^n}
\theoremstyle{plain}

\newtheorem{lemma}{Lemma}
\newtheorem{theorem}{Theorem}
\newtheorem{remark}{Remark}
\theoremstyle{nonumberplain}



\newtheorem{construction}{Construction}
% \newcommand{\Tr}{\mathrm{Tr}_1^n}
% \newcommand{\tr}{\mathrm{Tr}_1^k}

\title{ 2022 second-order of simplest \mathcal{PS} bent function}




\begin{document}
%   \maketitle
  \noindent
  \rule{\linewidth}{0.4pt}


\section{Introduction}
    The nonlinearities of Boolean functions are of great interest, for being the most important cryptographic 
    criteria for the symmetric cryptography, which is 
    
    The Boolean functions used in stream ciphers and block ciphers must have high nonlinearities to provide confusion and avoid the fast correlation attack. 
    
    Several papers have shown the role of higher-order nonlinearity of Boolean functions against some 
    cryptanalyses 

    The set $ RM(r,n) $ is another well-known  
    
    However, the profile of nonlinearity of Boolean functions 

    For more details, we refer to .
    Fisrt we recall the notions of the well-known result of 


\section{Preliminaries}

    Throughout this work, for $ n\in\Z $, let $ \F_{2^n} $ denote the finite field with $ 2^n $ elements.
    Every $ n $-dimensional $ \F_2 $-vector space can be equiped with a multiplication and be viewed as 
    $ \F_{2^n} $. 

    The algebraic degree of the Boolean function $ f $   is defined as 
    \[\max_{(u_1,...,u_j)|c_{u_1,\dots,u_j}\ne 0}\left( \operatorname{wt}(j_1)+\cdots+\operatorname{wt}(j_t) \right),\]
    where $ \operatorname{wt}(k) $ denotes the Hamming weight of the binary string of $ k\in\Z $. 
    Functions with algebraic degree equals to $ 2 $ are always called quadratic functions. 

    Computing the $ r $-th order nonlinearity of a given Boolean function with algebraic
    degree strictly larger than $ r $ is difficult for $ r > 1 $. 
    Even the second-order nonlinearity is known only for a few special functions or functions in small 
    numbers of variables. 


    The Walsh transform of $ f $ at point $\alpha \in \F_{2^n}$ is defined as
    \begin{equation*}
        \widehat{f}(\alpha)=\sum_{x \in \F_{2^n}}(-1)^{f(x)+\Tr(\alpha x)}.
    \end{equation*}


% \begin{lemma}\label{L:Parsevalrelation}
% For any $n$-variable Boolean function $f$, we have  $$\sum_{a\in\F_{2^n}}\widehat{f}\;^2(a)=2^{2n}.$$
% \end{lemma}

% \begin{lemma}\label{L:autoshift}
% Let $f$ be an arbitrary $n$-variable Boolean function. For any $\alpha,\beta\in\F_{2^n}*$, we have
% $$\sum_{x\in\F_{2^n}}(-1)^{f(\alpha x)+f(\beta x)}=2^{-n}\sum_{u\in\F_{2^n}}\widehat{f}(\alpha^{-1}u)\widehat{f}(\beta^{-1}u).$$
% \end{lemma}

\section{The Walsh spectra of the derivatives of the inverse function}

For any integer $n>0$, let us define $I_\nu(x)=\TrN(\nu x^{-1})$ over $\B_n$.
The Kloosterman sums over $\F_{2^n}$ are defined as
$\mathcal{K}(a)=\widehat{I_1}(\alpha)=\sum_{x\in\F_{2^n}}(-1)^{\TrN(x^{-1}+\alpha x)}$, where $\alpha\in\F_{2^n}$.
In fact, the Kloosterman sums are generally defined on the multiplicative
group $\F_{2^n}^*$. We extend them to $0$ by assuming $(-1)^0=1$.
%\begin{lemma}[\cite{CarlitzKloo1969}]\label{L:Kloostermansumsone}
%For any integer $m>0$, $\mathcal{K}(1)=1-\sum_{t=0}^{\lfloor m/2\rfloor}(-1)^{m-t}\frac{m}{m-t}{{m-t}\choose {t}}2^t$.
%\end{lemma}
% \begin{lemma}[\cite{LW90}] \label{inverse-nl}
% For any positive integer $n$ and arbitrary $a\in\F_{2^n}^*$,  the
% Walsh spectrum of $I_1(x)$ defined on $\F_{2^n}$ can take
% any value divisible by $4$ in the range
% $[-2^{{n/2}+1}+1,2^{{n/2}+1}+1]$.
% \end{lemma}








% \begin{lemma}\label{inverse-df}
%      Let $n\ge 3$ be an integer. Define
%      $C_{{\mu,\nu}}(\tau)=\sum_{x\in\F_{2^n}}(-1)^{\Tr(\mu x^{-1}+\nu(x+\tau)^{-1})}$,
%      where $\mu,\nu,\tau\in\F_{2^n}^*$. Then the value of
%      $C_{{\mu,\nu}}(\tau)$ belongs to $[-2^{{n/2}+1}-3, 2^{{n/2}+1}+1]$ and is divisible by $4$.
%       More preciously,
% $$C_{{\mu,\nu}}(\tau)=(-1)^{\Tr(\frac{\mu}{\tau}+\frac{\nu}{\tau})}W_{I_1}\left(\frac{\mu \nu}{\tau^2}\right)-2\Big((-1)^{\Tr(\frac{\mu}{\tau})\Tr(\frac{\nu}{\tau})}-1\Big).$$
% \end{lemma}
    \begin{proof} 
    For any $\mu,\nu,\tau\in\F_{2^n}^*$,
    we have (still using the convention $\frac 10=0$)
    \begin{eqnarray*}
    &&C_{{\mu,\nu}}(\tau)\\
    &=&\sum_{x\in\F_{2^n}}(-1)^{\TrN(\frac{\mu}{x}+\frac{\nu}{x+\tau})}\\
    &=&\sum_{x\in\F_{2^n}\setminus\{0,\tau\}}(-1)^{\TrN(\frac{\mu}{x}+\frac{\nu}{x+\tau})}+(-1)^{\TrN(\frac{\mu}{\tau})}+(-1)^{\TrN(\frac{\nu}{\tau})}\\
    &=&\sum_{x\in\F_{2^n}\setminus\{0,\tau^{-1}\}}(-1)^{\TrN(\mu x+\frac{\nu x}{1+\tau x})}+(-1)^{\TrN(\frac{\mu}{\tau})}+(-1)^{\TrN(\frac{\nu}{\tau})}\\
    &=&\sum_{x\in\F_{2^n}\setminus\{0,\tau^{-1}\}}(-1)^{\TrN(\mu x+\frac{1}{1+\tau x}\cdot\frac{\nu}{\tau}+\frac{\nu}{\tau})}+(-1)^{\TrN(\frac{\mu}{\tau})}+(-1)^{\TrN(\frac{\nu}{\tau})}\\
    &=&\sum_{x\in\F_{2^n}\setminus\{0,1\}}(-1)^{\TrN(\frac{\mu x}{\tau}+\frac{\nu}{\tau x}+\frac{\mu}{\tau}+\frac{\nu}{\tau})}+(-1)^{\TrN(\frac{\mu}{\tau})}+(-1)^{\TrN(\frac{\nu}{\tau})}\\
    &=&\sum_{x\in\F_{2^n}\setminus\{0,\frac{\tau}{\nu}\}}(-1)^{\TrN(\frac{1}{x}+\frac{\mu \nu}{\tau^2}x)+\TrN(\frac{\mu}{\tau}+\frac{\nu}{\tau})}+(-1)^{\TrN(\frac{\mu}{\tau})}+(-1)^{\TrN(\frac{\nu}{\tau})}\\
    &=&\sum_{x\in\F_{2^n}}(-1)^{\TrN(\frac{1}{x}+\frac{\mu \nu}{\tau^2}x)+\TrN(\frac{\mu}{\tau}+\frac{\nu}{\tau})}-
    (-1)^{\TrN(\frac{\mu}{\tau}+\frac{\nu}{\tau})}-(-1)^{\TrN(0)}+(-1)^{\TrN(\frac{\mu}{\tau})}+(-1)^{\TrN(\frac{\nu}{\tau})}
   \end{eqnarray*}
   where the third, fifth, and sixth identities hold by changing $x$ to ${1\over x}$, ${x+1\over \tau}$, and ${\nu x\over \tau}$ respectively.
   Note that $-(-1)^{\TrN(\frac{\mu}{\tau}+\frac{\nu}{\tau})}-(-1)^{\TrN(0)}+(-1)^{\TrN(\frac{\mu}{\tau})}+(-1)^{\TrN(\frac{\nu}{\tau})}$
   equals $0$ or $-4$. According to Lemma~\ref{inverse-nl}, we can see that $C_{{\mu,\nu}}(\tau)$ belongs to $[-2^{{n/2}+1}-3, 2^{{n/2}+1}+1]$ and is divisible by $4$.
   This finishes the proof.
   \end{proof}



%\begin{lemma}\label{L:solution}
%(Nyberg-Inverse)
%the solutions of $$\frac1x+\frac{1}{x+\alpha}=\beta$$
%\end{lemma}



\subsection{The multiplicative inverse function}
For any finite field $\F_{2^n}$, the multiplicative inverse function of $\F_{2^n}$, denoted by $I$, is defined as $I(x)=x^{2^n-2}$. In the sequel, we will use $x^{-1}$ or $\frac{1}{x}$ to
denote $x^{2^n-2}$ with the convention that $x^{-1}=\frac{1}{x}=0$ when $x=0$. We recall that, for any $v \neq 0$, $I_v(x) = \mathrm{Tr}_1^n(vx^{-1})$ is a component function of $I$.
The Walsh--Hadamard transform of $I_1$ at any point $\alpha$ is commonly known as Kloosterman sum over $\F_{2^n}$ at $\alpha$, which is usually denoted by $\mathcal{K}(\alpha)$,
i.e., $\mathcal{K}(\alpha)=\widehat{I_1}(\alpha)=\sum_{x\in\F_{2^n}}(-1)^{\mathrm{Tr}_1^n(x^{-1}+\alpha x)}$.
The original Kloosterman sums are generally defined on the multiplicative group $\F_{2^n}^*$. We extend them to $0$ by assuming $(-1)^0=1$. Regarding the Kloosterman sums,
the following results are well known and we will use them in the sequel.
% \begin{lemma}\cite{CarlitzKloo1969}
% \label{L:Kloostermansumsone}
% For any integer $n>0$, $\widehat{I_1}(1)=1-\sum_{t=0}^{\lfloor n/2\rfloor}(-1)^{n-t}\frac{n}{n-t}{{n-t}\choose {t}}2^t$.
% \end{lemma}
% \begin{lemma}\cite{LW90}
% \label{inverse-nl}
% For any positive integer $n$ and arbitrary $a\in\F_{2^n}^*$, the Walsh--Hadamard spectrum of $I_1(x)$ defined on $\F_{2^n}$ can take any value divisible by $4$ in the range
% $[-2^{{n/2}+1}+1,2^{{n/2}+1}+1]$.
% \end{lemma}
% Let $n=2t+1$ be an odd integer and $P$ be the largest positive integer such that $P \equiv 0 \pmod 4$ and $P\leq 2^{t+1}\sqrt{2}+1$.
% \begin{remark}
% \label{rem-max-min}
% The possible maximum absolute value of Walsh--Hadamard spectrum of $I_1$ over $\mathbb F_{2^n}$ is
% \begin{eqnarray*}
% \max_{\alpha\in\mathbb F_{2^n}^*}|\widehat{I_1}(\alpha)|=\left\{
% \begin{array}{llll}
% 2^{\frac{n}{2}+1}, &\mbox{ if } n \mbox{ is even}\\
% P, &\mbox{ if } n \mbox{ is odd}
% \end{array}
% \right.,
% \end{eqnarray*}
% where $P$ is as defined above.
% \end{remark}
% \subsection{Basic Lemmas}
% \begin{lemma}[\cite{MS1977}]\label{L:solutiondegree2}
% For any $(\alpha,\beta)\in\F_{2^n}^*\times\F_{2^n}$, we define a polynomial
% $\mu(x)=\alpha x^2+\beta x+\gamma\in\F_{2^n}[x]$.
% Then the equation $\mu(x)=0$ has 2 solutions if and only if $\mathrm{Tr}_1^n(\beta^{-2}\alpha\gamma)=0$.
% \end{lemma}
% \begin{lemma}\label{L:tr0sum}
% Let $n$ be a positive integer and $T_0=\{\upsilon^2+\upsilon : \upsilon\in\F_{2^n}\}$.
% Then we have
% \begin{eqnarray*}
% \sum_{x\in T_0}(-1)^{\Tr\left(\frac{1}{x+1}\right)}=\frac12(-1)^{n}\widehat{I_1}(1).
% \end{eqnarray*}
% \end{lemma}


% \begin{lemma}\label{L:rootssum}
% Let $n$ be a positive integer. We have
% \begin{eqnarray*}
% \sum_{\upsilon\in\F_{2^n}\setminus\F_2}(-1)^{\mathrm{Tr}_1^n
% \left(\frac{{\upsilon}^2}{\upsilon^2+\upsilon+1}\right)}=\sum_{\upsilon\in\F_{2^n}\setminus\F_2}(-1)^{\mathrm{Tr}_1^n\left(\frac{{\upsilon}^2+1}{\upsilon^2+\upsilon+1}\right)}=(-1)^n\left(\widehat{I}_1(1)-2\right).
% \end{eqnarray*}
% \end{lemma}

\begin{lemma}\cite{tang2022invfunc}\label{L:SumInv00}
Let $n\geq 3$ be an arbitrary integer. We define
$$L=\#\left\{c\in\F_{2^n} : \mathrm{Tr}_1^n\left(\frac{1}{c^2+c+1}\right)=\mathrm{Tr}_1^n\left(\frac{c^2}{c^2+c+1}\right)=0\right\}.$$
Then we have $L=2^{n-2}+\frac{3}{4}(-1)^n\widehat{I_1}(1)+\frac{1}{2}\left(1-(-1)^n\right)$, where $ \widehat{I_1}(1)=1-\sum_{t=0}^{\lfloor n/2\rfloor}(-1)^{n-t}\frac{n}{n-t}{{n-t}\choose {t}}2^t $.
% where $\widehat{I_1}(1)$ can be computed using Lemma~\ref{L:Kloostermansumsone}.
\end{lemma}

Let $F$ be an $(n,m)$-function. For any $\gamma,\eta\in\F_{2^n}$ and
$\omega\in\F_{2^m}$, let us define
\begin{equation}
\label{2nddrivative}
\mathcal{N}_F(\gamma,\eta,\omega)=\#\left\{x\in\F_{2^n} : F(x)+F(x+\gamma)+F(x+\eta)+F(x+\eta+\gamma)=\omega\right\}.
\end{equation}
It is clear that for $\gamma=0$ or $\eta=0$ or $\gamma=\eta$, we have $\mathcal{N}_F(\gamma,\eta,0)=2^n$, and when $\omega\neq 0$, $\mathcal{N}_F(\gamma,\eta,\omega)=0$. If $F$ is the multiplicative inverse function over $\mathbb F_{2^n}$, we denote $\mathcal{N}_I(\gamma,\eta,\omega)$ by $\mathcal{N}(\gamma,\eta,\omega)$.

\begin{lemma}\cite{tang2022invfunc}\label{Secondderivativesolution}
Let $n\geq 3$ be a positive integer and $\mathcal{N}(\gamma,\eta,\omega)$ be defined as in \eqref{2nddrivative}.
Let $\gamma,\eta$ be two elements of $\F_{2^n}^*$ such that $\gamma\neq \eta$. Then for any $\omega\in\F_{2^n}$,  we have $\mathcal{N}(\gamma,\eta,\omega)\in \{0,4,8\}$.
Moreover, the number of $(\gamma,\eta,\omega)\in\F_{2^n}^3$ such that $\mathcal{N}(\gamma,\eta,\omega)=8$ is
$$\left(2^{n-2}+\frac{3}{4}(-1)^{n}\widehat{I_1}(1)-\frac{5}{2}(-1)^{n}-\frac{3}{2}\right)\left(2^n-1\right).$$
\end{lemma}

%   \begin{tikzpicture}[
%     node distance = 5ex,
%     scale = 3,
%     thick,
%     > = latex,
%     % change the 
%     z = {(0.35, -0.4)},
%     edge/.style = {draw, thick, -, black},
%     sinal/.style = {inner sep = 1pt, thin, opacity = 0.4,
%       fill = blue, circle, text opacity = 1},
%     mtx/.style = {
%   %     matrix of math nodes,
%       matrix of nodes,
%       every node/.style = {
%         anchor = base,
%         text width = 2em,
%         text height = 1em,
%         align = center,
%       }
%     },
%     ]

%     \def\dist{0.1}
%     \def\cube{
%         % Vertices. (A,B,C), A x轴  B z轴  C y轴
        
        
%         % \node[left] (v0) at (0,0,0) {$ A $};
%         % note that command above can construct nodes and label them at the same time, 
%         % but sometimes you don't need the text, 
%         % so I just construct the coordinates and then label coordinates 
%         \coordinate (v0) at (0, 0, 0)  ;
%         \coordinate (v1) at (0, 1, 0)  ;
%         \coordinate (v2) at (1, 0, 0)  ;
%         \coordinate (v3) at (1, 1, 0)  ;
%         \coordinate (v4) at (0, 0, 1)  ;
%         \coordinate (v5) at (0, 1, 1)  ;
%         \coordinate (v6) at (1, 0, 1)  ;
%         \coordinate (v7) at (1, 1, 1)  ;
%         \coordinate (v8) at (0, 2, 0)  ;
%         \coordinate (v9) at (1, 2, 0)  ;
%         \coordinate (v10) at (0, 2, 1) ;
%         \coordinate (v11) at (1, 2, 1) ;
%     }
%     \begin{scope}[opacity=1] % opacity is the transparent 
%         \cube{};
%         % labeling verticals with text A B C at left\right\below\above\below left\below right\above left\above right
%         \node[left] at (v0) {$ A $};
%         \node[left] at (v1) {$ B $};
%         \node[right] at (v2) {$ C $};
%         \node[above right] at (v3) {$ D $};
%         \node[left] at (v4) {$ E $};
%         \node[below left] at (v5) {$ F $};
%         \node[right] at (v6) {$ G $};
%         \node[right] at (v7) {$ H $};
%         \node[left] at (v8) {$ I $};
%         \node[above right] at (v9) {$ J $};
%         \node[below left] at (v10) {$ K $};
%         \node[right] at (v11) {$ L $}; 
%         % Edges with some differential: alpha gamma beta theta
%         % arrow with direction from v1 to v0

%         \draw[->] (v2) -- (v3);
%         \draw[->] (v3) -- (v9);
%         \draw[->] (v6) -- (v7);
%         \draw[->] (v7) -- (v11);
%         % dotted line from v1 to v2 and the middle of line labeled is gamma
%         \draw[dashed] (v0) -- node[fill = white] {$ \gamma $} (v2) ;
%         \draw[dashed] (v4) -- node[fill = white] {$ \gamma $} (v6);
%         \draw[dashed] (v1) -- node[fill = white] {$ \beta $} (v5) -- node[fill = white] {$ \theta $} (v7) -- node[fill = white] {$ \beta $}(v3) -- node[fill = white] {$ \theta $} (v1);
%         \draw[dashed] (v8) -- node[fill = white] {$ \alpha $}(v10);
%         \draw[dashed] (v11) -- node[fill = white] {$ \alpha $}(v9);
%     \end{scope}
        
%     \begin{scope}[opacity=0.2]
%         % the pics in this part are transparent 0.2, 
%         % if not want this condition, delete the commands.
%         \draw[<-] (v0) -- (v1);
%         \draw[<-] (v1) -- (v8);
%         \draw[<-] (v4) -- (v5);
%         \draw[<-] (v5) -- (v10);
%     \end{scope}
        
%         % \foreach \i in {0, 1, ..., 11}{ \draw[fill = black] (v\i) circle (0.1pt); }
%         % } % boomerang attack model
        
%         \begin{scope}[]
%             % 
%             \coordinate (E0)  at (2, 0-0.4, 0);
%             \coordinate (E0L) at (2-0.3, 0-0.4, 0);
%             \coordinate (E0R) at (2+0.15, 0-0.4, 0);
%             \coordinate (E1)  at (2, 1-0.4, 0);
%             \coordinate (E1L) at (2-0.3, 1-0.4, 0);
%             \coordinate (E1R) at (2+0.15, 1-0.4, 0);
%             \coordinate (E2)  at (2, 2-0.4, 0);
%             \coordinate (E2L) at (2-0.3, 2-0.4, 0);
%             \coordinate (E2R) at (2+0.15, 2-0.4, 0);
%             \draw[->] (E0) -- node[right] {$ E_1^{-1} $} (E1);
%             \draw[->] (E2) -- node[right] {$ E_0 $} (E1);
%             % dotted line with transparent 
%             \draw[dashed,opacity=.5] (E0R) -- (E0) -- (E0L);
%             \draw[dashed,opacity=.5] (E1R) -- (E1) -- (E1L);
%             \draw[dashed,opacity=.5] (E2R) -- (E2) -- (E2L);
%         % \foreach \i in {0, 1, ..., 11}{
%         %   \node at (v\i) {\i}; 
%         % }
%       \end{scope}
    
%     \end{tikzpicture}%

    A theorem is introduced for efficiently bounding from below 
    the nonlinearity profile of a given function when lower bounds exist
    for the $ (r-1) $-th order nonlinearities of the derivatives of $ f $:
   \begin{theorem}\cite{C2007book}\label{thm:High_order_nl_bound}
        Let $ f $ be a $ n $-variable Boolean function, and let $ 0<r<n $ be an integer. We have 
        \[nl_r(f)\ge 2^{n-1}-\frac{1}{2}\sqrt{2^{2n}-2\sum_{a\in\F_2^n}nl_{r-1}(D_af)}.\] 
   \end{theorem}
  
\section{The third-order nonlinearity of the simplest $ \mathcal{PS} $ bent function}

    Dillon presented a $ \mathcal{PS} $ bent function class $ f(x,y) $ from $ \F_{2^n}=\F_{2^k}^2 $ 
    to $ \F_2 $ as 
    \begin{eqnarray*}\label{Eqn_PS_bent}
        \mathcal{D}(x,y)=g\left({x\over y}\right)
    \end{eqnarray*}
    where $ g $ is a balanced Boolean function on $ \F_{2^{k}} $ with $ g(0)=0 $, 
    and ${ x\over y }$ is defined to be $ 0 $ if $ y=0 $ 
    (we shall always assume this kind of convention in the sequel).

    In this paper, our goal is to give a lower bound on the third-order nonlinearity of the simplest 
    $ \mathcal{PS} $ bent function, \emph{i.e.}
    \begin{eqnarray}\label{sub-bent}
        f(x,y)=\TRACE\left(\frac{\lambda x}{y}\right)
    \end{eqnarray}
    where $ (x,y)\in\F_{2^k}^2 $, $ \lambda\in\F_{2^k}^{*} $ 
    and $ \TRACE(x)=\sum\limits_{i=0}^{n-1}x^{2^i} $ is the trace function from
    $ \F_{2^k} $ to $ \F_2 $.

    \subsection{A lower bound on the third-order nonlinearity of the simplest $ \mathcal{PS} $ bent function}

    Before giving the lower bound of third-order nonlinearity of the simplest $ \mathcal{PS} $ bent function, 
    We first introduce two useful lemmas that are needed in the sequel.

    % We explictly give the number of solutions of systems of two or three trace functions since  
    % trace functions are balanced and $ \F_2 $-linear in finite field $ \F_{2^n} $.
    \begin{lemma}\label{lemma:N_ij_trace}
        Assume  $ k\ge 3 $, let 
        \[ N_{i,j} =\left\lvert\left\{x\in\F_{2^k}\middle|\TRACE\left(\theta_1x+\gamma_1\right)=i,\TRACE\left(\theta_2x+\gamma_2\right)=j\right\}\right\rvert, \]
        where  $ \gamma_1,\gamma_2\in\F_{2^k} $ and $ \theta_1,\theta_2\in\F_{2^k}^* $ are distinct. Then $ N_{0,0} =2^{k-2} $.
% 	 N_{0,1}=N_{1,0}=N_{1,1}= 
    \end{lemma}   
   
   \begin{proof}
    % $ N_{0,0}+N_{0,1}+N_{1,0}+N_{1,1}=2^k $. 
       We have 
       \begin{empheq}[left=\empheqbiglbrace]{align*}
            N_{0,0}+N_{0,1}&=\left\lvert\left\{x\in\F_{2^k}\middle|\TRACE\left(\theta_1x+\gamma_1\right)=0\right\}\right\rvert=2^{k-1}\\
            N_{1,1}+N_{0,1}&=\left\lvert\left\{x\in\F_{2^k}\middle|\TRACE\left(\theta_2x+\gamma_2\right)=1\right\}\right\rvert=2^{k-1}, 
       \end{empheq}
       then we get $ N_{0,0} = N_{1,1} $. 
       Besides, $ N_{0,0}+N_{1,1} = \left\lvert\left\{x\in\F_{2^k}\middle|\TRACE\left((\theta_1+\theta_2)x+(\gamma_1+\gamma_2)\right)=0\right\}\right\rvert=2^{k-1} $ 
       since the trace function is balanced if $ \theta_1\ne \theta_2 $ . 
        Therefore $ N_{0,0}=2^{k-2} $. This completes the proof.
%    	 and $ N_{0,1}=N_{1,0}=2^{k-2} $.
   \end{proof}
   
  \begin{lemma}\label{lemma:N_ijk_trace}
     Assume  $ k\ge 3 $, 
   % Assume $ V_i=\left\{x\in\F_{2^k}\middle| \TRACE\left(\theta_ix+c_i\right)=0 \right\} $ for $ i=1,2,3 $. 
   % It's obvious that $ \operatorname{dim}_{\F_2}(V_i)=k-1 $ for $ i=1,2,3 $ 
   % and $ \operatorname{dim}_{\F_2}(V_i\cap V_j)=k-2 $ for $ i\ne j $ 
   % and $ \operatorname{dim}_{\F_2}(V_1+V_2+V_3)=k $, then we have 
   % \begin{align*}
   %     \operatorname{dim}_{\F_2}(V_1\cap V_2\cap V_3)&=\operatorname{dim}_{\F_2}(V_1+V_2+V_3)-\sum_{i=1,2,3}\operatorname{dim}_{\F_2}(V_i)+\sum_{1\le i<j\le 3}\operatorname{dim}_{\F_2}(V_i\cap V_j)+\\
   %     &=k-3*(k-1)+3*(k-2)\\
   %     &=k-3.
   % \end{align*}
   let 
   \[ N_{i_1,i_2,i_3}=\left\lvert\left\{x\in\F_{2^k}\middle| \TRACE\left(\theta_1x+\gamma_1\right)=i_1,\TRACE\left(\theta_2x+\gamma_2\right)=i_2,\TRACE\left(\theta_3x+\gamma_3\right)=i_3 \right\} \right\rvert,\] 
   where  $ \gamma_1,\gamma_2,\gamma_3\in\F_{2^k} $ and $ \theta_1,\theta_2,\theta_3\in\F_{2^k}^* $ are distinct and satisfy 
   $ \theta_3\ne\theta_1+\theta_2 $. Then $ N_{0,0,0}= 2^{k-3} $.
    \end{lemma}

    \begin{proof}
        Using Lemma \ref{lemma:N_ij_trace} we have  
        \begin{equation}\label{eq:from_lemma_1}\left\{\begin{alignedat}{3}
        &N_{0,0,0}+N_{0,0,1}=\left\lvert\left\{x\in\F_{2^k}\middle|\TRACE\left(\theta_1x+\gamma_1\right)=0, \TRACE\left(\theta_2x+\gamma_2\right)=0\right\}\right\rvert=2^{k-2}\\
        &N_{0,0,0}+N_{0,1,0}=\left\lvert\left\{x\in\F_{2^k}\middle|\TRACE\left(\theta_1x+\gamma_1\right)=0, \TRACE\left(\theta_3x+\gamma_3\right)=0\right\}\right\rvert=2^{k-2}\\
        &N_{0,0,0}+N_{1,0,0}=\left\lvert\left\{x\in\F_{2^k}\middle|\TRACE\left(\theta_2x+\gamma_2\right)=0, \TRACE\left(\theta_3x+\gamma_3\right)=0\right\}\right\rvert=2^{k-2}.\\
        \end{alignedat}\right.\end{equation}
        Thus, $ N_{0,0,1}=N_{0,1,0}=N_{1,0,0} $. With the same reason we can also obtain  $ N_{0,1,1}=N_{1,0,1}=N_{1,1,0} $. 

        Because of $ \theta_1+\theta_2+\theta_3\ne 0 $, we have 
        \begin{equation}\label{eq:sum_three_trace} \left\{\begin{alignedat}{2}
            &N_{0,0,1}+N_{0,1,0}+N_{1,0,0}+N_{1,1,1}=\left\lvert\left\{x\in\F_{2^k}\middle|\TRACE\left(\left(\theta_1+\theta_2+\theta_3\right)x+\left(\gamma_1+\gamma_2+\gamma_3\right)\right)=1\right\}\right\rvert=2^{k-1}\\
            &N_{0,1,1}+N_{1,0,1}+N_{1,1,0}+N_{0,0,0}=\left\lvert\left\{x\in\F_{2^k}\middle|\TRACE\left(\left(\theta_1+\theta_2+\theta_3\right)x+\left(\gamma_1+\gamma_2+\gamma_3\right)\right)=0\right\}\right\rvert=2^{k-1}.
        \end{alignedat}\right.\end{equation}
        Combine equations \eqref{eq:sum_three_trace} with $ N_{0,0,1}=N_{0,1,0}=N_{1,0,0} $, $ N_{0,1,1}=N_{1,0,1}=N_{1,1,0} $ and equations  
        \begin{equation}\label{eq:sum_N_0jk}\left\{\begin{alignedat}{2}
            &N_{0,0,0}+N_{0,0,1}+N_{0,1,0}+N_{0,1,1}=\left\lvert\left\{x\in\F_{2^k}\middle|\TRACE\left(\theta_1x+\gamma_1\right)=0\right\}\right\rvert=2^{k-1}\\
            &N_{1,0,0}+N_{1,0,1}+N_{1,1,0}+N_{1,1,1}=\left\lvert\left\{x\in\F_{2^k}\middle|\TRACE\left(\theta_1x+\gamma_1\right)=1\right\}\right\rvert=2^{k-1},\\
        \end{alignedat}\right.\end{equation}
        we obtain $ N_{0,0,1}=N_{0,1,1} $. 
        Therefore from equations \eqref{eq:from_lemma_1} and equations \eqref{eq:sum_N_0jk}, the system 
        \begin{equation}\left\{\begin{alignedat}{2}
            &N_{0,0,0}+N_{0,0,1}=2^{k-2}\\
            &N_{0,0,0}+3N_{0,0,1}=2^{k-1} 
        \end{alignedat}\right.\end{equation}
        has the solution $ N_{0,0,0}=N_{0,0,1}=2^{k-3} $. This completes the proof.
    \end{proof}

    \begin{theorem}\label{thm:nl_DaDbf}
        Let $ k\ge 3 $ be an integer and $ n=2k $. 
        For the nonlinearity of the second-order derivative of 
        the simplest $ \mathcal{PS} $ bent function $ f(x,y)=\TRACE(\frac{\lambda x}{y}) $, 
        we have three cases based on the value of $ \alpha $:
        \begin{enumerate}[label=(\arabic{*})]
            \item For every $ \alpha=(\alpha_1,\alpha_2)\in\F_{2^k}\times\F_{2^k} $ with $ \alpha_2\ne 0 $, 
            when $ \beta $ ranges over $ \F_{2^n} $, we have 
            \begin{equation}\label{res:nontrivil_nl}
                nl(D_{\beta}D_{\alpha}f)=\begin{cases}
                    2^{2k-1}-2^{k+2},&2^kL\text{~times}\\
                    2^{2k-1}-2^{k+1},&2^k(2^k-2-L)\text{~times}\\
                    0,&1\text{~time},%\footnote[1]{This happends if $ \beta=0 $},
                \end{cases}
            \end{equation}
            with $ nl(D_{\beta}D_{\alpha}f)\ge 2^{2k-1}-** $ occuring $ 2^{k+1}-1 $ times.
            \item For every $ \alpha=(\alpha_1,0)\in\F_{2^k}^*\times\{0\} $, when $ \beta $ ranges over $ \F_{2^n} $, 
            we have $ nl(D_{\beta}D_{\alpha}f)=0 $ for $ \beta=(\beta_1,0)\in\F_{2^k}\times\{0\} $, 
            otherwise, $ nl(D_{\beta}D_{\alpha}f)\ge *** $. 
            \item For $ \alpha=(0,0) $, we have $ nl(D_{\beta}D_{\alpha}f) = 0 $ for all $ \beta\in\F_{2^n} $.
        \end{enumerate} 
    \end{theorem}

    \begin{proof}        
    Let us consider the Walsh transform of the second-order derivative of 
    $ f(x,y)=\TRACE\left(\frac{\lambda x}{y}\right) $ at the points 
    $ \alpha=(\alpha_1,\alpha_2),\beta=(\beta_1,\beta_2)\in\F_{2^k}^2 $ with $ \lambda\in\F_{2^k}^* $.
    We have 
    \begin{align}\label{eq:secondordersum}
        &W_{D_{\beta}D_{\alpha}f}(\mu,\nu)\nonumber\\
        =&\sum_{x\in\F_{2^k}}\sum_{y\in\F_{2^k}}(-1)^{\TRACE\left(\frac{\lambda x}{y}+\frac{\lambda (x+\alpha_1)}{y+\alpha_2}+\frac{\lambda (x+\beta_1)}{y+\beta_2}+\frac{\lambda (x+\alpha_1+\beta_1)}{y+\alpha_2+\beta_2}+\mu x+\nu y\right)}\nonumber\\
        =&\sum_{y\in\F_{2^k}}(-1)^{\TRACE\left(\frac{\lambda\alpha_1}{y+\alpha_2}+\frac{\lambda\beta_1}{y+\beta_2}+\frac{\lambda(\alpha_1+\beta_1)}{y+\alpha_2+\beta_2}+\nu y\right)}\nonumber\\
        &\times \sum_{x\in\F_{2^k}}(-1)^{\TRACE\left(\left(\frac{\lambda}{y}+\frac{\lambda}{y+\alpha_2}+\frac{\lambda}{y+\beta_2}+\frac{\lambda}{y+\alpha_2+\beta_2}+\mu\right)x\right)}\nonumber\\
        =&\begin{cases}
            2^k\sum_{y\in S}(-1)^{\TRACE\left(\frac{\lambda\alpha_1}{y+\alpha_2}+\frac{\lambda\beta_1}{y+\beta_2}+\frac{\lambda(\alpha_1+\beta_1)}{y+\alpha_2+\beta_2}+\nu y\right)},&~\text{if}~\frac{\lambda}{y}+\frac{\lambda}{y+\alpha_2}+\frac{\lambda}{y+\beta_2}+\frac{\lambda}{y+\alpha_2+\beta_2}=\mu~\text{has solutions}\\
            0, &~\text{otherwise}, 
        \end{cases}
    \end{align}
    where $ S $ is the set of solutions of equation 
    \begin{equation}\label{eq:coefficient}
        \frac{\lambda}{y}+\frac{\lambda}{y+\alpha_2}+\frac{\lambda}{y+\beta_2}+\frac{\lambda}{y+\alpha_2+\beta_2}=\mu.
    \end{equation}
    % $ \frac{\lambda}{y}+\frac{\lambda}{y+\alpha_2}+\frac{\lambda}{y+\beta_2}+\frac{\lambda}{y+\alpha_2+\beta_2}=\mu $.

    Note that  $ nl(D_{\beta}D_{\alpha}f)=2^{2k-1}-\frac{1}{2}\max_{\mu,\nu}\left\lvert W_{D_{\beta}D_{\alpha}f}(\mu,\nu)\rvert\right. $, 
    we only need to consider $ \max_{\mu,\nu}|W_{D_{\beta}D_{\alpha}f}(\mu,\nu)| $ for every points $ \alpha,\beta $. 
    % According to the proof of Lemma \ref{Secondderivativesolution} in \cite{tang2022invfunc}, 
    % for $ \alpha_2,\beta_2\in\F_{2^k}^* $ such that $ \alpha_2\ne\beta_2 $, 
    % we can always find some $ \mu\in\F_{2^k} $ such that equation
    % has at least $ 4 $ solutions. 
    % Besides, when $ \alpha_2=\beta_2\in\F_{2^k}^* $ or $ \alpha_2=0 $ or $ \beta_2=0 $, 
    % equation \eqref{eq:coefficient} has $ 2^k $ solutions for $ \mu=0 $.
    So we only consider $ \left\lvert W_{D_{\beta}D_{\alpha}f}(\mu,\nu)\rvert\right. $ 
    for some $ \mu $ such that equation \eqref{eq:coefficient} has solutions, since we have 
    %  $ \max_{\mu,\nu}\left\lvert W_{D_{\beta}D_{\alpha}f}(\mu,\nu)\rvert\right. $ must occur in the case 
    $ 2^k\left\lvert \sum_{y\in S}(-1)^{\TRACE\left(\frac{\lambda\alpha_1}{y+\alpha_2}+\frac{\lambda\beta_1}{y+\beta_2}+\frac{\lambda(\alpha_1+\beta_1)}{y+\alpha_2+\beta_2}+\nu y\right)}\right\rvert \ge 0 $.
    Therefore, two steps are needed for all points 
    $ \alpha=(\alpha_1,\alpha_2),\beta=(\beta_1,\beta_2)\in\F_{2^k}^2 $ with $ \lambda\in\F_{2^k}^* $:  
    \begin{enumerate}[label=\roman{*})]
        \item Find all $ (\mu,\nu)\in\F_{2^k}^2 $ such that equation \eqref{eq:coefficient}
        has solutions.
        \item Calculate the value $ \max_{\mu,\nu}|W_{D_{\beta}D_{\alpha}f}(\mu,\nu)| $ among those $ (\mu,\nu) $.
    \end{enumerate} 
    
    So we first give the conditions such that equation \eqref{eq:coefficient} has solutions, 
    whose proof is analogue to the proof of Lemma 13 in \cite{tang2022invfunc} and we omit it: 
    \begin{enumerate}[label=\arabic{*})]
        \item If $ \alpha_2=\beta_2\in\F_{2^k}^* $ or $ \alpha_2=0 $ or $ \beta_2=0 $, 
        then \eqref{eq:coefficient} has $ 2^k $ solution when $ \mu = 0 $.
        \item If $ \alpha_2,\beta_2\in\F_{2^k}^* $ such that $ \alpha_2\ne\beta_2 $, then we have: 
        \begin{enumerate}
            \item If $ \lambda(\alpha_2^2+\beta_2^2+\alpha_2\beta_2)+\mu(\alpha_2^2\beta_2+\alpha_2\beta_2^2)=0 $, 
            $ \{0,\alpha_2,\beta_2,\alpha_2+\beta_2\} $ are four solutions of \eqref{eq:coefficient}.
            % \item when $ \mu=0 $, we have \eqref{eq:coefficient} in the form 
            % \[\frac{\lambda\alpha_2}{y(y+\alpha_2)}+\frac{\lambda\alpha_2}{y(y+\alpha_2)+\alpha_2\beta_2+\beta_2^2}=0.\]
            % It has solutions only when $ \alpha_2=0 $ or $ \beta_2=0 $ or $ \alpha_2=\beta_2 $, contradiction, so it has 
            % $ 0 $ solution;
            \item If $ \mu\ne 0 $, $ \TRACE\left(\frac{\lambda\alpha_2}{\mu\beta_2(\alpha_2+\beta_2)}\right)=0 $ and 
            % $ \TRACE\left(u\left(\left(\frac{\beta_2}{\alpha_2}\right)^2+\frac{\beta_2}{\alpha_2}\right)\right)=0\Leftrightarrow
            $ \TRACE\left(\frac{\lambda\beta_2}{\mu\alpha_2(\alpha_2+\beta_2)}\right)=0 $, 
            $ \{y_0,y_0+\alpha_2,y_0+\beta_2,y_0+\alpha_2+\beta_2\} $ are four solutions of \eqref{eq:coefficient}, where $ y_0 $ is a solution of \eqref{eq:coefficient} 
            and $ y_0\notin\{0,\alpha_2,\beta_2,\alpha_2+\beta_2\} $. 
            % $ u $ is the solution of $ t^2+t+\frac{\lambda\alpha_2}{\mu\beta_2(\alpha_2+\beta_2)}=0 $ 
            % with $ t=\frac{y(y+\alpha_2)}{\alpha_2\beta_2+\beta_2^2} $.
        \end{enumerate}
    \end{enumerate} 
    After finding all $ (\mu,\nu)\in\F_{2^k}^2 $ such that equation \eqref{eq:coefficient} has solutions for every points 
    $ \alpha=(\alpha_1,\alpha_2),\beta=(\beta_1,\beta_2) $, we need to calculate maxmial value 
    $ 2^k\left\lvert \sum_{y\in S}(-1)^{\TRACE\left(\frac{\lambda\alpha_1}{y+\alpha_2}+\frac{\lambda\beta_1}{y+\beta_2}+\frac{\lambda(\alpha_1+\beta_1)}{y+\alpha_2+\beta_2}+\nu y\right)} \right\rvert$ 
    between those $ (\mu,\nu) $.    
    
    % Therefore, we divide points $ \alpha,\beta $ into two parts  
    % owing to the number of solutions of equation \eqref{eq:coefficient}. 
    
    \begin{enumerate}[label=\textbf{Case \arabic*}]
        \item If $ \alpha_2=\beta_2\in\F_{2^k}^* $ or $ \alpha_2=0 $ or $ \beta_2=0 $ 
    % with $ \mu\ne 0 $, 
    % then equation \eqref{eq:coefficient} has $ 0 $ solution, then
    % \[W_{D_{\beta}D_{\alpha}f}(\mu,\nu)=0.\]
    and $ \mu=0 $, equation \eqref{eq:coefficient} has $ 2^k $ solutions, 
    which are all elements of $ \F_{2^k} $, 
    then we have 
    \begin{equation}\label{eq:case2ksolutions}
        W_{D_{\beta}D_{\alpha}f}(0,\nu)=2^k\sum_{y\in\F_{2^k}}(-1)^{\TRACE\left(\frac{\lambda\alpha_1}{y+\alpha_2}+\frac{\lambda\beta_1}{y+\beta_2}+\frac{\lambda(\alpha_1+\beta_1)}{y+\alpha_2+\beta_2}+\nu y\right)}.
    \end{equation}

    % \textbf{CASE.2} ?(nontrivial) If $ \alpha_2=\beta_2 $ or $ \alpha_2=0 $ or $ \beta_2=0 $ with $ \mu=0 $, 
    % then equation \eqref{eq:coefficient} has $ 2^k $ solutions, we confirm that 
    % \begin{equation}\label{eq:case2ksolutions}
    %     W_{D_{\beta}D_{\alpha}f}(\mu,\nu)=2^k\sum_{y\in\F_{2^k}}(-1)^{\TRACE\left(\frac{\lambda\alpha_1}{y+\alpha_2}+\frac{\lambda\beta_1}{y+\beta_2}+\frac{\lambda(\alpha_1+\beta_1)}{y+\alpha_2+\beta_2}+\nu y\right)}.
    % \end{equation}
    
    For the simple cases, if $ \alpha=(\alpha_1,0),\beta=(\beta_1,0)\in\F_{2^k}^*\times\{0\} $ 
    or $ \alpha=(0,0) $ or $ \beta=(0,0) $, equation \eqref{eq:case2ksolutions} can be transformed into a simple form:
    \[W_{D_{\beta}D_{\alpha}f}(0,\nu)=2^k\sum_{y\in\F_{2^k}}(-1)^{\TRACE\left(\nu y\right)}.\]
    And $ \max_{\nu}|W_{D_{\beta}D_{\alpha}f}(0,\nu)|=|W_{D_{\beta}D_{\alpha}f}(0,0)|=2^{2k} $.

    For other cases we will give the upper bounds of 
    $ \max_{\nu}|W_{D_{\beta}D_{\alpha}f}(0,\nu)| $: assume 
    $ \alpha_2=\beta_2\in\F_{2^k}^* $ and $ \alpha_1\ne\beta_1 $, then we have 
    \[ W_{D_{\beta}D_{\alpha}f}(0,\nu)=2^k\sum_{y\in\F_{2^k}}(-1)^{\TRACE\left(\frac{\lambda(\alpha_1+\beta_1)}{y+\alpha_2}+\frac{\lambda(\alpha_1+\beta_1)}{y}+\nu y\right)}.\]

    % \begin{enumerate}[label=(\arabic{*})]
    %     \item   If $ \alpha_1=\beta_1 $, then 
    %     \begin{equation}
    %         W_{D_{\beta}D_{\alpha}f}(\mu,\nu)=2^k\sum_{y\in\F_{2^k}}(-1)^{\TRACE\left(\frac{\lambda\alpha_1}{y+\alpha_2}+\frac{\lambda\alpha_1}{y+\beta_2}+\nu y\right)}.
    %     \end{equation}
    %     \item   If $ \alpha_1=\beta_2=0 $ or $ \alpha_2=\beta_1=0 $, without loss of 
    %     generality, we assume $ \alpha_1=\beta_2=0 $, then 
    %     \begin{equation}
    %         W_{D_{\beta}D_{\alpha}f}(\mu,\nu)=2^k\sum_{y\in\F_{2^k}}(-1)^{\TRACE\left(\frac{\lambda\beta_1}{y+\beta_2}+\frac{\lambda\beta_1}{y+\alpha_2+\beta_2}+\nu y\right)}.
    %     \end{equation}
    %     \item   If $ \beta_1=0 $, then 
    %     \begin{equation}
    %         W_{D_{\beta}D_{\alpha}f}(\mu,\nu)=2^k\sum_{y\in\F_{2^k}}(-1)^{\TRACE\left(\frac{\lambda\alpha_1}{y+\alpha_2}+\frac{\lambda\alpha_1}{y+\alpha_2+\beta_2}+\nu y\right)}.
    %     \end{equation}
    % \end{enumerate}
    Therefore, in the cases of $ \alpha_2=\beta_2\in\F_{2^k}^* $ or $ \alpha_2=0 $ or $ \beta_2=0 $, 
    we have the upper bound of the maximial absolute values 
    \[\max_{\mu,\nu}|W_{D_{\beta}D_{\alpha}f}(\mu,\nu)|\le ***.\]

    
    % \textbf{CASE.2} If $ \alpha_2\ne\beta_2 $ and $ \alpha_2,\beta_2\in\F_{2^k}^* $, 
    % we can always  
    % we confirm \eqref{eq:coefficient} has $ 0 $ solution, then
    % \[W_{D_{\beta}D_{\alpha}f}(\mu,\nu)=0.\]






    % \textbf{CASE.2} (trivial) If $ \alpha_2\ne\beta_2 $ and $ \alpha_2,\beta_2\in\F_{2^k}^* $, when $ \mu\ne 0 $, 
    % $ \lambda(\alpha_2^2+\beta_2^2+\alpha_2\beta_2)+\mu(\alpha_2^2\beta_2+\alpha_2\beta_2^2)\ne 0 $ with $ \TRACE\left(\frac{\lambda\alpha_2}{\mu\beta_2(\alpha_2+\beta_2)}\right)=1 $ or $ \TRACE\left(\frac{\lambda\beta_2}{\mu\alpha_2(\alpha_2+\beta_2)}\right)=1 $, 
    % then \eqref{eq:coefficient} has $ 0 $ solution, we get 
    % \[ W_{D_{\beta}D_{\alpha}f}(\mu,\nu)=0. \]
        \item 
    % \subsubsection{The case where there are at least four solutions}
    If $ \alpha_2,\beta_2\in\F_{2^k}^* $ such that $ \alpha_2\ne\beta_2 $ and   
    $ \mu= \frac{\lambda(\alpha_2^2+\beta_2^2+\alpha_2\beta_2)}{\alpha_2^2\beta_2+\alpha_2\beta_2^2}$, 
    we are sure that $ \{0,\alpha_2,\beta_2,\alpha_2+\beta_2\} $ are solutions of equations \eqref{eq:coefficient}, 
    then we have two subcases in the following, that is: 
    \begin{enumerate}[label=\arabic{*})]
        \item If $ \alpha_2,\beta_2 $ and $ \mu $ satisfy the system 
        \begin{equation}\label{eq:last_four_solution_condition}\left\{
            \begin{alignedat}{3}
                &\mu\ne 0\\
                &\TRACE\left(\frac{\lambda\alpha_2}{\mu\beta_2(\alpha_2+\beta_2)}\right)=0\\
                &\TRACE\left(\frac{\lambda\beta_2}{\mu\alpha_2(\alpha_2+\beta_2)}\right)=0,\\
            \end{alignedat}\right.
        \end{equation}
        then $ \{y_0,y_0+\alpha_2,y_0+\beta_2,y_0+\alpha_2+\beta_2\} $ are also solutions of equation \eqref{eq:coefficient}, 
        where $ y_0\notin\{0,\alpha_2,\beta_2,\alpha_2+\beta_2\} $, 
        therefore the number of solutions is $ 8 $.
        \item Otherwise, $ \{0,\alpha_2,\beta_2,\alpha_2+\beta_2\} $ are the only $ 4 $ solutions. 
    \end{enumerate}


    So we calculate $ W_{D_{\beta}D_{\alpha}f}(\mu,\nu) $ for some $ (\mu,\nu) $ in two cases.
    \begin{enumerate}[label=\textbf{Case \Alph{*}},itemindent=*,wide=\parindent]
        \item 
        We first consider the case equation \eqref{eq:coefficient} has $ 4 $ solutions 
        $ \{0,\alpha_2,\beta_2,\alpha_2+\beta_2\} $. 
        Then $ S=\{0,\alpha_2,\beta_2,\alpha_2+\beta_2\} $ and $ y\in S $, we have 
    % we confirm that \eqref{eq:coefficient} has $ 4 $ solution, assume $ 4 $ solutions are 
    % $ S_4=\{y^{\prime},y^{\prime}+\alpha_2,y^{\prime}+\beta_2,y^{\prime}+\alpha_2+\beta_2\} $ 
    % where $ y^{\prime}=0 $ or  $ y^{\prime}=y_0 $, then we have 
    % \begin{align}\label{eq:foursolutionsum}
    %     W_{D_{\beta}D_{\alpha}f}(\mu,\nu)=2^k&\sum_{y\in S_4}(-1)^{\TRACE\left(\frac{\lambda\alpha_1}{y+\alpha_2}+\frac{\lambda\beta_1}{y+\beta_2}+\frac{\lambda(\alpha_1+\beta_1)}{y+\alpha_2+\beta_2}+\nu y\right)}\nonumber\\
    %     =2^k&\left((-1)^{\TRACE\left(\frac{\lambda\alpha_1}{y^{\prime}+\alpha_2}+\frac{\lambda\beta_1}{y^{\prime}+\beta_2}+\frac{\lambda(\alpha_1+\beta_1)}{y^{\prime}+\alpha_2+\beta_2}+\nu y^{\prime}\right)}+(-1)^{\TRACE\left(\frac{\lambda\alpha_1}{y^{\prime}}+\frac{\lambda\beta_1}{y^{\prime}+\alpha_2+\beta_2}+\frac{\lambda(\alpha_1+\beta_1)}{y^{\prime}+\beta_2}+\nu (y^{\prime}+\alpha_2)\right)}\right.\nonumber\\
    %     &\left.+(-1)^{\TRACE\left(\frac{\lambda\alpha_1}{y^{\prime}+\alpha_2+\beta_2}+\frac{\lambda\beta_1}{y^{\prime}}+\frac{\lambda(\alpha_1+\beta_1)}{y^{\prime}+\alpha_2}+\nu (y^{\prime}+\beta_2)\right)}
    %     +(-1)^{\TRACE\left(\frac{\lambda\alpha_1}{y^{\prime}+\beta_2}+\frac{\lambda\beta_1}{y^{\prime}+\alpha_2}+\frac{\lambda(\alpha_1+\beta_1)}{y^{\prime}}+\nu (y^{\prime}+\alpha_2+\beta_2)\right)}\right).
    % \end{align}
    % We can sum the first and last part of equation \eqref{eq:foursolutionsum} to get 
    % \begin{align*}
    %     &(-1)^{\TRACE\left(\frac{\lambda\alpha_1}{y^{\prime}+\alpha_2}+\frac{\lambda\beta_1}{y^{\prime}+\beta_2}+\frac{\lambda(\alpha_1+\beta_1)}{y^{\prime}+\alpha_2+\beta_2}+\nu y^{\prime}\right)}
    %     +(-1)^{\TRACE\left(\frac{\lambda\alpha_1}{y^{\prime}+\beta_2}+\frac{\lambda\beta_1}{y^{\prime}+\alpha_2}+\frac{\lambda(\alpha_1+\beta_1)}{y^{\prime}}+\nu (y^{\prime}+\alpha_2+\beta_2)\right)}\\
    %     =&(-1)^{\TRACE\left(\frac{\lambda\alpha_1}{y^{\prime}+\alpha_2}+\frac{\lambda\beta_1}{y^{\prime}+\beta_2}+\frac{\lambda(\alpha_1+\beta_1)}{y^{\prime}+\alpha_2+\beta_2}+\nu y^{\prime}\right)}\cdot
    %     \left[1+(-1)^{\TRACE\left(\frac{\lambda(\alpha_1+\beta_1)}{y^{\prime}} \frac{\lambda(\alpha_1+\beta_1)}{y^{\prime}+\alpha_2}+\frac{\lambda(\alpha_1+\beta_1)}{y^{\prime}+\beta_2}+\frac{\lambda(\alpha_1+\beta_1)}{y^{\prime}+\alpha_2+\beta_2}+\nu (\alpha_2+\beta_2)\right)}\right]\\
    %     =&(-1)^{\TRACE\left(\frac{\lambda\alpha_1}{y^{\prime}+\alpha_2}+\frac{\lambda\beta_1}{y^{\prime}+\beta_2}+\frac{\lambda(\alpha_1+\beta_1)}{y^{\prime}+\alpha_2+\beta_2}+\nu y^{\prime}\right)}\cdot
    %     \left[1+(-1)^{\TRACE\left(\mu(\alpha_1+\beta_1)+\nu (\alpha_2+\beta_2)\right)}\right].
    % \end{align*}
    % We can also sum the second and third parts of equation \eqref{eq:foursolutionsum} to get
    % \begin{align*}
    %     &(-1)^{\TRACE\left(\frac{\lambda\alpha_1}{y^{\prime}}+\frac{\lambda\beta_1}{y^{\prime}+\alpha_2+\beta_2}+\frac{\lambda(\alpha_1+\beta_1)}{y^{\prime}+\beta_2}+\nu (y^{\prime}+\alpha_2)\right)}
    %     +(-1)^{\TRACE\left(\frac{\lambda\alpha_1}{y^{\prime}+\alpha_2+\beta_2}+\frac{\lambda\beta_1}{y^{\prime}}+\frac{\lambda(\alpha_1+\beta_1)}{y^{\prime}+\alpha_2}+\nu (y^{\prime}+\beta_2)\right)}\\
    %     =&(-1)^{\TRACE\left(\frac{\lambda\alpha_1}{y^{\prime}}+\frac{\lambda\beta_1}{y^{\prime}+\alpha_2+\beta_2}+\frac{\lambda(\alpha_1+\beta_1)}{y^{\prime}+\beta_2}+\nu (y^{\prime}+\alpha_2)\right)}\cdot
    %     \left[1+(-1)^{\TRACE\left(\mu(\alpha_1+\beta_1)+\nu (\alpha_2+\beta_2)\right)}\right].
    % \end{align*}
    % Then we have 
    \begin{align}\label{eq:simpleforms_4}
        &W_{D_{\beta}D_{\alpha}f}(\mu,\nu)\nonumber\\
        =&2^k\left[1+(-1)^{\TRACE\left((\alpha_1+\beta_1)\mu+ (\alpha_2+\beta_2)\nu\right)}\right]\nonumber\\
        &\cdot
        \left[(-1)^{\TRACE\left(\frac{\lambda\alpha_1}{y+\alpha_2}+\frac{\lambda\beta_1}{y+\beta_2}+\frac{\lambda(\alpha_1+\beta_1)}{y+\alpha_2+\beta_2}+ y\nu\right)}+
        (-1)^{\TRACE\left(\frac{\lambda\alpha_1}{y}+\frac{\lambda\beta_1}{y+\alpha_2+\beta_2}+\frac{\lambda(\alpha_1+\beta_1)}{y+\beta_2}+ (y+\alpha_2)\nu\right)}\right]\nonumber\\
        =&2^k\left[1+(-1)^{\TRACE\left((\alpha_1+\beta_1)\mu+ (\alpha_2+\beta_2)\nu\right)}\right]\nonumber\\
        &\cdot
        (-1)^{\TRACE\left(\frac{\lambda\alpha_1}{y+\alpha_2}+\frac{\lambda\beta_1}{y+\beta_2}+\frac{\lambda(\alpha_1+\beta_1)}{y+\alpha_2+\beta_2}+ y\nu\right)}\cdot
        \left[1+(-1)^{\TRACE\left(\frac{\lambda\alpha_1}{y}+\frac{\lambda\alpha_1}{y+\alpha_2}+\frac{\lambda\alpha_1}{y+\beta_2}+\frac{\lambda\alpha_1}{y+\alpha_2+\beta_2}+\nu\alpha_2\right)}\right]\nonumber\\
        =&2^k\left[1+(-1)^{\TRACE\left((\alpha_1+\beta_1)\mu+ (\alpha_2+\beta_2)\nu\right)}\right]\cdot
        \left[1+(-1)^{\TRACE\left(\alpha_1\mu+\alpha_2\nu\right)}\right]\cdot
        (-1)^{\TRACE\left(\frac{\lambda\alpha_1}{y+\alpha_2}+\frac{\lambda\beta_1}{y+\beta_2}+\frac{\lambda(\alpha_1+\beta_1)}{y+\alpha_2+\beta_2}+ y\nu\right)}\nonumber\\
        =&\begin{cases}
            2^{k+2}\cdot(-1)^{\TRACE\left(\frac{\lambda\alpha_1}{y+\alpha_2}+\frac{\lambda\beta_1}{y+\beta_2}+\frac{\lambda(\alpha_1+\beta_1)}{y+\alpha_2+\beta_2}+ y\nu\right)},&\text{if}~\TRACE\left(\alpha_2\nu+\alpha_1\mu\right)=0 ~
            \text{and}~\TRACE\left(\beta_2\nu+\beta_1 \mu\right)=0 \\
            0,~&\text{otherwise}.
        \end{cases}
    \end{align}
    Observing \eqref{eq:simpleforms_4} we can find $ |W_{D_{\beta}D_{\alpha}f}(\mu,\nu)| $ only 
    has values $ \{0,2^{k+2}\} $. 
    Furthermore, by Lemma \ref{lemma:N_ij_trace}, 
    for all $ \alpha=(\alpha_1,\alpha_2),\beta=(\beta_1,\beta_2)\in\F_{2^k}\times\F_{2^k}^* $ such that 
    $ \alpha_2\ne\beta_2 $ and $ \mu=\frac{\lambda(\alpha_2^2+\beta_2^2+\alpha_2\beta_2)}{\alpha_2^2\beta_2+\alpha_2\beta_2^2} $, 
    there always exists $ 2^{k-2} $  $ \nu\in\F_{2^k} $ satisfying the system
    
        \begin{equation}\label{eq:max_foursolution_condition}\left\{
        \begin{alignedat}{2}
            \TRACE\left(\alpha_2\nu+\alpha_1\mu\right)&=0\\
            \TRACE\left(\beta_2\nu +\beta_1 \mu\right)&=0.
        \end{alignedat}\right.
\end{equation}
    Thus, for all points $ \alpha,\beta\in\F_{2^k}^2 $ with $ \alpha_2,\beta_2\in\F_{2^k}^* $, $ \alpha_2\ne\beta_2 $ 
    and $ \mu=\frac{\lambda(\alpha_2^2+\beta_2^2+\alpha_2\beta_2)}{\alpha_2^2\beta_2+\alpha_2\beta_2^2} $ 
    such that don't satisfy equations \eqref{eq:last_four_solution_condition}, we have 
    \[\max_{\mu,\nu}|W_{D_{\beta}D_{\alpha}f}(\mu,\nu)|=2^{k+2}.\]
    
    % if $ \{0,\alpha_2,\beta_2,\alpha_2+\beta_2\} $ are only solutions of equation \eqref{eq:coefficient}, 
    % we can always 
    % the values $ \pm 2^{k+2} $, we conclude that $ \TRACE\left(\mu\alpha_1+\nu\alpha_2\right)=0 $ 
    % and $ \TRACE\left(\mu\beta_1+\nu\beta_2\right)=0 $. 

    \item
    Next case is that if equation \eqref{eq:coefficient} has $ 8 $ solutions, 
    that is, $ \alpha_2,\beta_2 $ and $ \mu $ satisfy system \eqref{eq:last_four_solution_condition}. 

    Then we have 
    \begin{align}
    &W_{D_{\beta}D_{\alpha}f}(\mu,\nu)\nonumber\\
        =&2^k\left[1+(-1)^{\TRACE\left((\alpha_1+\beta_1)\mu+ (\alpha_2+\beta_2)\nu\right)}\right]\cdot
        \left[1+(-1)^{\TRACE\left(\alpha_1\mu+\alpha_2\nu\right)}\right]\nonumber\\
        &\cdot
        \left[(-1)^{\TRACE\left(\frac{\lambda\alpha_1}{\alpha_2}+\frac{\lambda\beta_1}{\beta_2}+\frac{\lambda(\alpha_1+\beta_1)}{\alpha_2+\beta_2}\right)}+(-1)^{\TRACE\left(\frac{\lambda\alpha_1}{y_0+\alpha_2}+\frac{\lambda\beta_1}{y_0+\beta_2}+\frac{\lambda(\alpha_1+\beta_1)}{y_0+\alpha_2+\beta_2}+ y_0\nu\right)}\right]\nonumber\\
        =&(-1)^{c_0}2^k\cdot\left[1+(-1)^{\TRACE\left((\alpha_1+\beta_1)\mu+ (\alpha_2+\beta_2)\nu\right)}\right]\cdot
        \left[1+(-1)^{\TRACE\left(\alpha_1\mu+\alpha_2\nu\right)}\right]\cdot\left[1+(-1)^{c_0+c_1}\right]\nonumber\\
        =&\begin{cases}
            2^{k+3}\cdot(-1)^{c_0},~&\text{if}~ \TRACE\left(\alpha_1\mu+\alpha_2\nu\right)=0 , \TRACE\left(\beta_1\mu+\beta_2\nu\right)=0 ~\text{and}~ c_0+c_1=0\\
            0,~&\text{otherwise},
        \end{cases} 
    \end{align}
    where $ y_0\notin\{0, \alpha_2, \beta_2, \alpha_2+\beta_2\} $ and 
    \begin{empheq}[left=\empheqlbrace]{align*}
        c_0&=\TRACE\left(\frac{\lambda\alpha_1}{\alpha_2}+\frac{\lambda\beta_1}{\beta_2}+\frac{\lambda(\alpha_1+\beta_1)}{\alpha_2+\beta_2}\right)\\
        c_1&= \TRACE\left(\frac{\lambda\alpha_1}{y_0+\alpha_2}+\frac{\lambda\beta_1}{y_0+\beta_2}+\frac{\lambda(\alpha_1+\beta_1)}{y_0+\alpha_2+\beta_2}+\nu y_0\right).
    \end{empheq}

    % then equation \eqref{eq:coefficient} has $ 8 $ distinct solutions 
    % $ \{0,\alpha_2,\beta_2,\alpha_2+\beta_2,y_0,y_0+\alpha_2,y_0+\beta_2,y_0+\alpha_2+\beta_2\} $.
    By Lemma \ref{lemma:N_ijk_trace}, 
    for all $ \alpha=(\alpha_1,\alpha_2),\beta=(\beta_1,\beta_2)\in\F_{2^k}\times\F_{2^k}^* $ such that 
    $ \alpha_2\ne\beta_2 $
    and $ y_0\notin\{0,\alpha_2,\beta_2,\alpha_2+\beta_2\} $, 
    there always exists $ 2^{k-3} $ $ \nu\in\F_{2^k} $ satisfying below equations,
    \begin{empheq}[left=\empheqbiglbrace]{align*}
        &\TRACE\left(\alpha_2\nu + \alpha_1\mu\right)=0\\
        &\TRACE\left(\beta_2 \nu + \beta_1\mu \right)=0\\
        &\TRACE\left(y_0\nu +\frac{\lambda\alpha_1}{\alpha_2}+\frac{\lambda\beta_1}{\beta_2}+\frac{\lambda(\alpha_1+\beta_1)}{\alpha_2+\beta_2}+\frac{\lambda\alpha_1}{y_0+\alpha_2}+\frac{\lambda\beta_1}{y_0+\beta_2}+\frac{\lambda(\alpha_1+\beta_1)}{y_0+\alpha_2+\beta_2} \right)=0.
    \end{empheq}
    So we conclude that for all points $ \alpha,\beta $ with $ \alpha_2,\beta_2\in\F_{2^k}^* $ 
    such that $ \alpha_2\ne\beta_2 $ 
    and $ \mu=\frac{\lambda(\alpha_2^2+\beta_2^2+\alpha_2\beta_2)}{\alpha_2^2\beta_2+\alpha_2\beta_2^2} $ 
    satisfying equations \eqref{eq:last_four_solution_condition}, we have 
    \[\max_{\mu,\nu}|W_{D_{\beta}D_{\alpha}f}(\mu,\nu)|=2^{k+3}.\]

    

    \begin{remark}
        There must exist some points $ \alpha,\beta $ 
        such that $ \max_{\mu,\nu}|W_{D_{\beta}D_{\alpha}f}(\mu,\nu)|=2^{k+3} $.
        Indeed, the conditions $ \alpha_2,\beta_2\in\F_{2^k}^* $, $ \alpha_2\ne\beta_2 $ and $ \mu\ne 0 $ 
        can tell us $ \mu(\alpha_2^2\beta_2+\alpha_2\beta_2^2)\ne 0 $, 
        resulting in $ \lambda(\alpha_2^2+\beta_2^2+\alpha_2\beta_2)\ne 0 $, 
        which implies $ \frac{\beta_2}{\alpha_2}\notin\F_4 $.
        So take $ \mu=\lambda(\alpha_2^2+\beta_2^2+\alpha_2\beta_2)/(\alpha_2^2\beta_2+\alpha_2\beta_2^2) $ 
        into $ \TRACE\left(\frac{\lambda\alpha_2}{\mu\beta_2(\alpha_2+\beta_2)}\right)=0 $ 
        and $ \TRACE\left(\frac{\lambda\beta_2}{\mu\alpha_2(\alpha_2+\beta_2)}\right)=0 $ respectively, 
        we can transform two equations into $ \TRACE\left(\frac{1}{\gamma^2+\gamma+1}\right)=0 $ 
        and $ \TRACE\left(\frac{\gamma^2}{\gamma^2+\gamma+1}\right)=0 $, 
        where $ \gamma=\frac{\beta_2}{\alpha_2}\in\F_{2^k}\setminus\F_{4} $. 
        Furthermore, according to Lemma \ref{L:SumInv00}, 
        the number of $ \gamma=\frac{\beta_2}{\alpha_2}\in\F_{2^k}\setminus\F_{4} $ satisfying 
        $ \TRACE\left(\frac{1}{\gamma^2+\gamma+1}\right)=0 $ and $ \TRACE\left(\frac{\gamma^2}{\gamma^2+\gamma+1}\right)=0 $ 
        is $ L $, which means that for points $ \alpha=(\alpha_1,\alpha_2)\in\F_{2^k}^2 $ with $ \alpha_2\ne 0 $, there exist 
        $ L $ $ \beta_2 $ such that $ \max_{\mu,\nu}|W_{D_{\beta}D_{\alpha}f}(\mu,\nu)|=2^{k+3} $. 
    \end{remark}
\end{enumerate}
\item     
    For every $ \alpha_2,\beta_2\in\F_{2^k}^* $ such that $ \alpha_2\ne\beta_2 $, 
    there exist some $ \mu $ satifying that $ S=\{y_0,y_0+\alpha_2,y_0+\beta_2,y_0+\alpha_2+\beta_2\} $ 
    are the only $ 4 $ solutions of 
    equation \eqref{eq:coefficient}, where $ y_0\notin\{0, \alpha_2, \beta_2, \alpha_2+\beta_2\} $. Fortunately, 
    we don't need to treat with those $ \mu $ since in that case, 
    the maximal possible value is not greater than the result of Case 1 where equation \eqref{eq:coefficient} has $ 4 $ solutions 
    $ \{0,\alpha_2,\beta_2,\alpha_2+\beta_2\} $, that is, 
    \[ |W_{D_{\beta}D_{\alpha}f}(\mu,\nu)|=2^k\left\lvert\sum_{y\in S}(-1)^{\TRACE\left(\frac{\lambda\alpha_1}{y+\alpha_2}+\frac{\lambda\beta_1}{y+\beta_2}+\frac{\lambda(\alpha_1+\beta_1)}{y+\alpha_2+\beta_2}+ y\nu\right)}\right\rvert\le 2^{k+2}=|W_{D_{\beta}D_{\alpha}f}(\mu_0,\nu_0)|, \]   
    where $ \mu_0=\frac{\lambda(\alpha_2^2+\beta_2^2+\alpha_2\beta_2)}{\alpha_2^2\beta_2+\alpha_2\beta_2^2} $ 
    and $ \nu_0 $ satisfy the system \eqref{eq:max_foursolution_condition}. 
    % are the only four solutions of equation \eqref{eq:coefficient} for some $ \alpha_2,\beta_2 $ and $ \mu $, 
    % because in the case that $ \{0,\alpha_2,\beta_2, \alpha_2+\beta_2\} $ are only $ 4 $ solutions 
    % of equation \eqref{eq:coefficient},  
    % % when $ \{0,\alpha_2,\beta_2,\alpha_2+\beta_2\} $,  
    % That is, 
    % Note that in the case that $ S=\{0,\alpha_2,\beta_2,\alpha_2+\beta_2\} $, 
    % $ \max_{\mu,\nu}|W_{D_{\beta}D_{\alpha}f}(\mu,\nu)|=2^{k+2} $.

    % So if equation \eqref{eq:coefficient} has only $ 4 $ solutions, 
    % $ \max_{\mu,\nu}|W_{D_{\beta}D_{\alpha}f}(\mu,\nu)|=2^{k+2} $.
    \end{enumerate}
    \end{proof}
        
    

    % After taking the $ 8 $ solutions into equation \eqref{eq:secondordersum}, we get the summation
    % \begin{align}\label{eq:case2kplus3}
    %     &W_{D_{\beta}D_{\alpha}f}(\mu,\nu)\nonumber\\
    %     =&2^k\left[1+(-1)^{\TRACE\left(\mu(\alpha_1+\beta_1)+\nu (\alpha_2+\beta_2)\right)}\right]\cdot
    %     \left[1+(-1)^{\TRACE\left(\mu\alpha_1+\nu\alpha_2\right)}\right]\nonumber\\
    %     &\cdot
    %     \left[(-1)^{\TRACE\left(\frac{\lambda\alpha_1}{\alpha_2}+\frac{\lambda\beta_1}{\beta_2}+\frac{\lambda(\alpha_1+\beta_1)}{\alpha_2+\beta_2}\right)}+(-1)^{\TRACE\left(\frac{\lambda\alpha_1}{y_0+\alpha_2}+\frac{\lambda\beta_1}{y_0+\beta_2}+\frac{\lambda(\alpha_1+\beta_1)}{y_0+\alpha_2+\beta_2}+\nu y_0\right)}\right]\nonumber\\
    %     =&(-1)^{c_0}2^k\cdot\left[1+(-1)^{\TRACE\left(\mu(\alpha_1+\beta_1)+\nu (\alpha_2+\beta_2)\right)}\right]\cdot
    %     \left[1+(-1)^{\TRACE\left(\mu\alpha_1+\nu\alpha_2\right)}\right]\cdot\left[1+(-1)^{c_0+c_1}\right]\nonumber\\
    %     =&\begin{cases}
    %         \pm 2^{k+3},~\text{if}~ \TRACE\left(\mu\alpha_1+\nu\alpha_2\right)=0 , \TRACE\left(\mu\beta_1+\nu\beta_2\right)=0 ~\text{and}~ c_0+c_1=0;\\
    %         0,~\text{otherwise}.
    %     \end{cases}
    % \end{align}
    % where $ c_0=\TRACE\left(\frac{\lambda\alpha_1}{\alpha_2}+\frac{\lambda\beta_1}{\beta_2}+\frac{\lambda(\alpha_1+\beta_1)}{\alpha_2+\beta_2}\right) $ and 
    % $ c_1= \TRACE\left(\frac{\lambda\alpha_1}{y_0+\alpha_2}+\frac{\lambda\beta_1}{y_0+\beta_2}+\frac{\lambda(\alpha_1+\beta_1)}{y_0+\alpha_2+\beta_2}+\nu y_0\right)$.


    % If we fix the ratio of $ \beta_2 $ and $ \alpha_2 $: set $ \gamma=\frac{\beta_2}{\alpha_2}\in\F_{2^k}\setminus\F_{4} $ which holds $ \TRACE\left(\frac{\gamma^2}{\gamma^2+\gamma+1}\right)=\TRACE\left(\frac{\gamma^2}{\gamma^2+\gamma+1}\right)=0 $, then
    % we find $ \forall \alpha_1\in\F_{2^k}^* $, 
    % there always exists $ \beta_1\in\F_{2^k}^* $ s.t. $ W_{D_{\beta}D_{\alpha}f}(\mu,\nu)=\pm 2^{k+3} $, 
    % besides, we find that only $ \alpha_2,\beta_2,\alpha_1,\beta_1 $ can influence positive or negetive, and for every points 
    % $ \alpha,\beta $, there are $ 2^{k-3} $ $ \nu $'s leading to $ \pm 2^{k+3} $.

    % Note that $ c_0+c_1=\TRACE\left(\frac{\lambda\alpha_1}{\alpha_2}+\frac{\lambda\beta_1}{\beta_2}+\frac{\lambda(\alpha_1+\beta_1)}{\alpha_2+\beta_2}+\frac{\lambda\alpha_1}{y_0+\alpha_2}+\frac{\lambda\beta_1}{y_0+\beta_2}+\frac{\lambda(\alpha_1+\beta_1)}{y_0+\alpha_2+\beta_2}+\nu y_0\right) $.

    % Therefore we need to determine for every points $ \alpha=(\alpha_1,\alpha_2) $ and $ \beta=(\beta_1,\beta_2) $ 
    % with $ \gamma = \frac{\beta_2}{\alpha_2}\in\F_{2^k}\setminus\F_{2^2} $ satisfying 
    % $ \TRACE\left(\frac{1}{\gamma^2+\gamma+1}\right)=\TRACE\left(\frac{\gamma^2}{\gamma^2+\gamma+1}\right)=0 $ 
    % and $ \mu=\frac{\lambda(\alpha_2^2+\beta_2^2+\alpha_2\beta_2)}{\alpha_2^2\beta_2+\alpha_2\beta_2^2} $,  
    % whether or not there always exists $ \nu\in\F_{2^k} $ s.t. 
    % \begin{equation}\label{eq:3trace0}\left\{\begin{alignedat}{3}
    %     &\TRACE\left(\mu\alpha_1+\nu\alpha_2\right)=0\\ 
    %     &\TRACE\left(\mu\beta_1+\nu\beta_2\right)=0\\
    %     &\TRACE\left(\frac{\lambda\alpha_1}{\alpha_2}+\frac{\lambda\beta_1}{\beta_2}+\frac{\lambda(\alpha_1+\beta_1)}{\alpha_2+\beta_2}+\frac{\lambda\alpha_1}{y_0+\alpha_2}+\frac{\lambda\beta_1}{y_0+\beta_2}+\frac{\lambda(\alpha_1+\beta_1)}{y_0+\alpha_2+\beta_2}+\nu y_0\right)=0.
    % \end{alignedat}\right.\end{equation}

    % For every points $ \alpha=(\alpha_1,\alpha_2) $ and $ \beta=(\beta_1,\beta_2) $, equation 
    % \eqref{eq:3trace0} in fact are three trace functions about $ \nu $ 
    % with coefficients $ y_0,\alpha_2,\beta_2\in\F_{2^k}^* $ are distinct 
    % and satifying $ y_0+\alpha_2+\beta_2\ne 0 $. 
    % since $ \mu $ are fixed once $ \alpha_2,\beta_2 $ are fixed, and 
    % $ y_0 $ is also fixed since it's one of eight solutions of equation \eqref{eq:coefficient} and equation \eqref{eq:coefficient} is 
    % determined by $ \lambda,\alpha_2,\beta_2 $ and $ \mu $. 
    % Thus, using Lemma \ref{lemma:N_ijk_trace}, 
    % we confirm that equations \eqref{eq:3trace0} have $ 2^{k-3} $ solutions $ \nu\in\F_{2^k} $
    % for every points $ \alpha=(\alpha_1,\alpha_2)\in\F_{2^k}\times\F_{2^k}^* $ and $ \beta=(\beta_1,\beta_2)\in\F_{2^k}\times\F_{2^k}^* $ 
    % with $ \gamma=\frac{\beta_2}{\alpha_2}\in\F_{2^k}\setminus\F_{2^2} $ satisfying 
    % $ \TRACE\left(\frac{1}{\gamma^2+\gamma+1}\right)=\TRACE\left(\frac{\gamma^2}{\gamma^2+\gamma+1}\right)=0 $ 
    % and $ \mu=\frac{\lambda(\alpha_2^2+\beta_2^2+\alpha_2\beta_2)}{\alpha_2^2\beta_2+\alpha_2\beta_2^2} $.
    
    % So equation \eqref{eq:case2kplus3} will always have points $ (\mu,\nu) $ leading to values $ \pm 2^{k+3} $ for every points 
    % $ \alpha=(\alpha_1,\alpha_2)\in\F_{2^k}\times\F_{2^k}^* $ and $ \beta=(\beta_1,\beta_2)\in\F_{2^k}\times\F_{2^k}^* $ 
    % with $ \gamma=\frac{\beta_2}{\alpha_2}\in\F_{2^k}\setminus\F_{2^2} $ satisfying 
    % $ \TRACE\left(\frac{1}{\gamma^2+\gamma+1}\right)=\TRACE\left(\frac{\gamma^2}{\gamma^2+\gamma+1}\right)=0 $. 

    
    Applying two times Theorem \ref{thm:High_order_nl_bound}, we obtain the relation between the third-order nonlinearity of $ f $ and 
    the nonlinearity of the second-order derivative of $ f $:
    \begin{equation}\label{eq:nl3_nlDaDbf}
        nl_3(f)\ge 2^{n-1}-\frac{1}{2}\sqrt{\sum_{\alpha\in\F_{2^n}}\sqrt{2^{2n}-2\sum_{\beta\in\F_{2^n}} nl(D_{\beta}D_{\alpha}f)}}.
    \end{equation} 
    % Thus we should calculate $ \nl(D_{\beta}D_{\alpha}f) $ for every the points $ \alpha,\beta\in\F_{2^n} $.
    Therefore, we can give the lower bound of third-order nonlinearity of the simplest $ \mathcal{PS} $ bent function:   
    \begin{theorem}
        Let $ k\ge 3 $ be an integer and $ n=2k $. For the third-order nonlinearity of 
        the simplest $ \mathcal{PS} $ bent function 
        $ f(x,y)=\TRACE(\frac{\lambda x}{y}) $ with $ x,y\in\F_{2^k} $ and $ \lambda\in\F_{2^k}^* $, we have:
        \[nl_3(f)\ge 2^{n-1}-\frac{1}{2}\sqrt{A}\]
        where
        \begin{align*}
            A=2^{2n}-.
        \end{align*}
        
        % The low bound of the third-order nonlinearity of the simplest $ \mathcal{PS} $ 
        % bent function $ f(x,y)=\TRACE(\frac{\lambda x}{y}) $ 
    \end{theorem}
    \begin{proof}
        We have 
        \begin{align*}
            nl_3(f)&\ge 2^{n-1}-\frac{1}{2}\sqrt{\sum_{\alpha\in\F_{2^n}}\sqrt{2^{2n}-2\sum_{\beta\in\F_{2^n}} nl(D_{\beta}D_{\alpha}f)}}\\
            &=2^{n-1}-\frac{1}{2}\sqrt{\sum_{\alpha=(\alpha_1,0)\in\F_{2^k}\times\{0\}}\sqrt{2^{2n}-2\sum_{\beta\in\F_{2^n}} nl(D_{\beta}D_{\alpha}f)}+\sum_{\substack{\alpha=(\alpha_1,\alpha_2)\in\F_{2^k}^2\\\alpha_2\ne 0}}\sqrt{2^{2n}-2\sum_{\beta\in\F_{2^n}} nl(D_{\beta}D_{\alpha}f)}}\\
            &\ge 
            % 2^{n-1}-\frac{1}{2}\sqrt{\sum_{\alpha_1\in\F_{2^k}}\sqrt{2^{2n}-2\sum_{\beta\in\F_{2^n}} \nl(D_{\beta}D_{(\alpha_1,0)}f)}+\sum_{\alpha=(\alpha_1,\alpha_2)\in\F_{2^k}\times\F_{2^k}^*}\sqrt{2^{2n}-2\sum_{\beta=(\beta_1,\beta_2)\in\F_{2^k}^2} \nl(D_{\beta}D_{\alpha}f)}}
        \end{align*}
        where the second sign of inequality comes from Theorem \ref{thm:nl_DaDbf}.
    \end{proof}
 
    \subsection{Comparison with the known results}
        Carlet has deduced that the $ r $th-order nonlinearity of an $ (n,n) $ Dillon function is bounded from 
        below by.... 
        Therefore, the lower bound on
        % We give a lower bound 
    
    
    
   




    % \textbf{Nothing}:

    % We find in that case, $ c_1 $ don't change the value when $ y_0 $ is one of the four solutions such that
    % \begin{align*}
    %     c_1=&\TRACE\left(\frac{\lambda\alpha_1}{y_0+\alpha_2}+\frac{\lambda\beta_1}{y_0+\beta_2}+\frac{\lambda(\alpha_1+\beta_1)}{y_0+\alpha_2+\beta_2}+\nu y_0\right)\\
    %     =&\TRACE\left(\frac{\lambda(\alpha_1+\beta_1)}{y_0+\alpha_2}+\frac{\lambda(\alpha_1+\beta_1)}{y_0+\beta_2}+\frac{\lambda(\alpha_1+\beta_1)}{y_0+\alpha_2+\beta_2}+\frac{\lambda(\alpha_1+\beta_1)}{y_0}+\frac{\lambda(\alpha_1+\beta_1)}{y_0}+\frac{\lambda\alpha_1}{y_0+\beta_2}+\frac{\lambda\beta_1}{y_0+\alpha_2}+\nu y_0\right)\\
    %     =&\TRACE\left(\mu(\alpha_1+\beta_1)+\frac{\lambda(\alpha_1+\beta_1)}{y_0}+\frac{\lambda\alpha_1}{y_0+\beta_2}+\frac{\lambda\beta_1}{y_0+\alpha_2}+\nu y_0\right)\\
    %     =&\TRACE\left(\frac{\lambda(\alpha_1+\beta_1)}{y_0}+\frac{\lambda\alpha_1}{y_0+\beta_2}+\frac{\lambda\beta_1}{y_0+\alpha_2}+\nu (y_0+\alpha_2+\beta_2)\right).
    % \end{align*}
    % The last equation holds iff both two equations $ \TRACE\left(\mu\alpha_1+\nu\alpha_2\right)=0 $ 
    % and $ \TRACE\left(\mu\beta_1+\nu\beta_2\right)=0 $ holds.



    %         % \begin{empheq}[left=\empheqlbrace]{align}
    %         %     &F_1(\vb*{x})+F_1(\vb*{x}+\vb*{a})=\vb*{b}\label{eq:3-0-1}\\
    %         %     &f_1(\vb*{x})+f_1(\vb*{x}+\vb*{a})=b_n\label{eq:3-0-2}.
    %         % \end{empheq}
    % From the first condition we have $ \mu=\frac{\lambda(\alpha_2^2+\beta_2^2+\alpha_2\beta_2)}{\alpha_2^2\beta_2+\alpha_2\beta_2^2} $
    % and we can subsititute $ \mu $ into the second condition then we have 
    % \[\TRACE\left(\frac{\lambda\alpha_2}{\frac{\lambda(\alpha_2^2+\beta_2^2+\alpha_2\beta_2)}{\alpha_2^2\beta_2+\alpha_2\beta_2^2}\cdot\beta_2(\alpha_2+\beta_2)}\right)=\TRACE\left(\frac{\alpha_2^2}{\alpha_2^2+\beta_2^2+\alpha_2\beta_2}\right)=\TRACE\left(\frac{1}{\gamma^2+\gamma+1}\right)=0.\]
    % and 
    % \[\TRACE\left(\frac{\lambda\beta_2}{\frac{\lambda(\alpha_2^2+\beta_2^2+\alpha_2\beta_2)}{\alpha_2^2\beta_2+\alpha_2\beta_2^2}\cdot\alpha_2(\alpha_2+\beta_2)}\right)=\TRACE\left(\frac{\beta_2^2}{\alpha_2^2+\beta_2^2+\alpha_2\beta_2}\right)=\TRACE\left(\frac{\gamma^2}{\gamma^2+\gamma+1}\right)=0.\]
    % where $ \gamma=\frac{\beta_2}{\alpha_2}\in\F_{2^k}\setminus\F_{2^2} $ since $ \mu\ne 0 $ and $ \alpha_2\ne\beta_2 $ in condition 1 
    % means $ \alpha_2^2+\beta_2^2+\alpha_2\beta_2\ne 0 $.

    \bibliographystyle{plain} 
    \bibliography{mybib}

\end{document}



% for i in [1..2^5] do
%     inputvector:=Intseq(i-1,2,n);
%     eltvector:=&+[inputvector[j]*v^(j-1):j in [1..n]];
%     Append(~input,(eltvector));
% end for;

% for i in [1..2^n] do
%     inputvector:=Intseq(sbox[i],2,n);
%     eltvector:=&+[inputvector[j]*v^(j-1):j in [1..n]];
%     Append(~output,(eltvector));
% end for;

% function newsbox(x)
%     if x eq 0 then 
%         return 0; 
%     end if;
%     for i in [1..1024] do
%         if x eq input[i] then
%             return output[i];
%         end if;
%     end for;
% end function;



% G:=OrthoTest(newsbox,n);
% ddtG:=DDTexe(G,n);
