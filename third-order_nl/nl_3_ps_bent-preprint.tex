\documentclass[preprint,10pt]{elsarticle}
\linespread{1}
\usepackage{fullpage,enumitem,amsthm,amsmath,amssymb,graphicx,empheq,bm}
\usepackage{physics}
\usepackage{tcolorbox}
\usepackage[citecolor=blue]{hyperref}
% \usepackage[noadjust]{cite}
\usepackage{multirow}
\usepackage{threeparttable}
% \usepackage{authblk}
\usepackage{float}
\newcommand{\Z}{\mathbf{Z}}
\newcommand{\F}{\mathbb{F}}
\newcommand{\Com}{\mathbf{C}}
\newcommand{\ord}{\operatorname{ord}}
\newcommand{\Q}{\mathbf{Q}}
\newcommand{\R}{\mathbf{R}}
\newcommand{\E}{\mathbb{E}}
\newcommand{\0}{\textbf{0}}
\newcommand{\1}{\textbf{1}}
\newcommand{\wt}{\operatorname{wt}}
\newcommand{\B}{\mathcal{B}}
\newcommand{\nl}{\mathrm{nl}}
\newcommand{\TRACE}{\operatorname{Tr}_1^k}
\newcommand{\TrN}{\operatorname{Tr}_1^n}
\theoremstyle{plain}

\newtheorem{lemma}{Lemma}
\newtheorem{theorem}{Theorem}
\newtheorem{remark}{Remark}
\newtheorem{corollary}{Corollary}
\newtheorem{definition}{Definition}
\newtheorem{construction}{Construction}

% \newcommand{\Tr}{\mathrm{Tr}_1^n}
% \newcommand{\tr}{\mathrm{Tr}_1^k}
\begin{document}

\begin{frontmatter}
\title{A Lower Bound on the Third-order Nonlinearity of the Simplest $\mathcal{PS}_{ap}$ Bent Functions}
\author[1]{Zhaole Li}
\ead{lizhaole@sjtu.edu.cn}
\affiliation[1]{organization={School of Electronic Information and Electrical Engineering},%Department and Organization
            addressline={Shanghai Jiao Tong University}, 
            city={Shanghai},
            postcode={200240}, 
            % state={Shanghai},
            country={China}}

\author[2]{Bing Shen}
\ead{shenbing1115@qq.com}
\affiliation[2]{organization={Science and Technology on Communication Security Laboratory},%Department and Organization
            addressline={Institute of Southwest Communication Research}, 
            city={Chengdu},
            postcode={610041}, 
            state={Sichuan},
            country={China}}

\author[1,2]{Deng Tang\corref{cor1}}
\ead{dtang@foxmail.com}
\cortext[cor1]{Corresponding author}
% \affiliation[1,2]{organization={Shanghai Jiao Tong University},%Department and Organization
%             addressline={School of Electronic Information and Electrical Engineering}, 
%             city={Shanghai},
%             postcode={200240}, 
%             state={Shanghai},
%             country={China}}

\begin{abstract}
Boolean functions used in symmetric-key encryption should have high higher-order nonlinearity to resist several known cryptographic attacks,
such as algebraic attacks and low-degree approximation attacks.
The higher-order nonlinearity also plays an important role in coding theory and theoretical computer science, since it relates to the covering radius of Reed-Muller codes and the Gowers norm, respectively.
It is well-known that bent functions have the highest nonlinearity in an even number of variables and thus they possess
the best ability to withstand fast correlation attacks and best affine approximation attacks.
However, there is currently limited knowledge regarding the higher-order nonlinearity of bent functions
because computing the higher-order nonlinearity, or even providing tight lower bounds, is an extremely hard task.
In 1974, Dillon proposed two well-known classes of bent functions based on partial spread (in brief, $\mathcal{PS}$), called
$\mathcal{PS}^-$ and $\mathcal{PS}^+$, respectively. He also exhibited a subclass of bent functions in $\mathcal{PS}^-$, known as
partial spread affine plane ($\mathcal{PS}_{ap}$ for short).
In this paper, we provide a lower bound on the third-order nonlinearity of the simplest $\mathcal{PS}_{ap}$ bent functions in $n$ variables, where $n\ge 6$ is even, by calculating the nonlinearities of all second-order derivatives of this kind of bent functions.
Compared to the two known lower bounds on the third-order nonlinearity given by Carlet and Tang et al. respectively,
our lower bound is much better than these two ones.
\end{abstract}
\begin{keyword}
    Boolean functions \sep $\mathcal{PS}_{ap}$ bent functions \sep Third-order nonlinearity \sep Derivatives
\end{keyword}            
\end{frontmatter}
% 06E30  \sep 94A60 \sep 94C10 \sep 14G50
% \affil[a]{School of Electronic Information and Electrical Engineering, Shanghai Jiao Tong University, Shanghai 200240, China}
% \affil[b]{Science and Technology on Communication Security Laboratory, Chengdu 610041, Sichuan, China}
% Zhaole Li, Bing Shen, and Deng Tang
\section{Introduction}
    Boolean functions are of great interest in the design of symmetric-key encryption, coding theory and theoretical computer science,
    % The importance of Boolean functions can be seen in their applications to the design of symmetric-key encryption, coding theory, and theoretical computer science, 
    see \cite{Carlet2020book,CohenHLL1997RMcodecover,BhattacharyyaKSSZ2010gowers} for instances.
    Let $f$ be an $n$-variable Boolean function. The $r$-th order nonlinearity of $f$, denoted by $nl_r(f)$,
    is defined as the minimum Hamming distance from $f$ to all $n$-variable Boolean functions of degree at most $r$.
    The first-order nonlinearity of $f$ is simply called the nonlinearity of $f$ and
    the $r$-th order nonlinearity with $r\ge 2$ is called the higher-order nonlinearity of $f$ if $r$ is omitted.
    Indeed, the higher-order nonlinearity plays a crucial role in symmetric-key encryption as it enhances the understanding of the resistance of Boolean functions against several known cryptographic attacks, for examples,
    % is an important parameter for symmetric-key encryption since
    % it contributes to the knowledge on the Boolean functions to resist several known cryptographic attacks, for examples,
    (fast) algebraic attacks \cite{FAA06,CFAA03,CM03,WT10} and low-degree approximation attacks \cite{Courtois2002XL_algorithm_and_NL_r,Golic1996lower_order_approximation,IwataK1999highorderbentfunction,KnudsenR1996nonlinear_approximation}.
    In coding theory, the value of $nl_r(f)$ equals the minimum Hamming distance from $f$ to the Reed-Muller code $\mathcal{RM}(r,n)$ of length $2^n$ and of order $r$.
    Thus, the maximum $r$-th order nonlinearity of all Boolean functions in $n$ variables equals the covering radius of $\mathcal{RM}(r,n)$ \cite{CohenHLL1997RMcodecover}.
    The higher-order nonlinearity of Boolean functions is also related to the Gowers norm in theoretical computer science, since the correlation between a
    Boolean function $f$ and its closest polynomial of degree $d$ is at most the $(d+1)$-th Gowers norm of $f$ \cite{BhattacharyyaKSSZ2010gowers}.

    As mentioned in Abstract, 
    in order to withstand several known attacks, Boolean functions employed in symmetric-key encryption should have high higher-order nonlinearity.
    % Boolean functions used in symmetric-key encryption should have high higher-order nonlinearity to resist several known attacks.
    Till date, however, 
    % only a few lower bounds on the higher-order nonlinearity of several classes of Boolean functions have been obtained.
    there have been limited results in obtaining the lower bounds on the higher-order nonlinearity for certain classes of Boolean functions. 
    Indeed, it is a challenging task to compute the $r$-th order nonlinearity of a general Boolean function of algebraic degree strictly greater than $r$ for $r>1$, both theoretically and algorithmically.
    In theoretical calculations, 
    the second-order nonlinearity can only be determined for Boolean functions that are in special forms or have a small number of variables.
    Determining the $r$-th order nonlinearity of a general Boolean function becomes even more challenging when $r\ge 3$.
    % The $r$-th order nonlinearity of a general Boolean function is even more challenging to determine for $r\ge 3$. 
    With regards to algorithms, the nonlinearity is related with the Walsh spectrum, which can be efficiently computed by a divide-and-conquer butterfly algorithm.
    % For the second-order nonlinearity of Boolean functions, 
    For $r=2$, 
    Kabatiansky and Tavernier \cite{KabatianskyT2005listdecoding_RM_2_n} proposed a list decoding algorithm for the second-order Reed-Muller codes.
    Soon, this algorithm was improved and implemented to quadratic Boolean functions up to $n=11$ in general and up to $n=13$ in some special cases \cite{FourquetT2008improved_listdecoding_RM_2_n}.
    This algorithm can also be applied to the higher-order nonlinearity but only for small number of variables.
    There is currently a lack of knowledge regarding algorithms that are related to the $r$-th order nonlinearity, where $r$ is greater than or equal to $3$.
    %  for $r\ge 3$.
    It is, therefore, important to provide lower bounds on the $r$-th order nonlinearity of Boolean functions.
    However, it is still a difficult task, even for $r=2$.
    Fortunately, Carlet \cite{Carlet2008lowbound_NL_profile} gave two lemmas about lower bounds on the $r$-th order nonlinearity of a given Boolean function.
    Based on those two lemmas, he derived lower bounds on the $r$-th order nonlinearity of Maiorana-McFarland, Welch, Kasami and inverse functions.
    Owing to Carlet's lemmas for efficiently lower bounding the $r$-th order nonlinearity, many authors have obtained lower bounds on
    the $r$-th order nonlinearity of different functions, mostly for $r=2,3$, see for \cite{Carlet2011NL_Profile_Dillon,GangopadhyayST2010NL_2,GaoT2017NL_2_MM,GodeG2010NL_3Kasami,Liu2023NL_2,SihemKJ2020NL_2cubic,Singh2014NL_3_biquadratic,SunW2009NL_2,SunW2011NL_2,TangCT2013NL_2bent,TangYZZ2020NL_2bent,YanT2020NL_2}.

    Bent functions, which achieve maximum possible nonlinearity in even number of variables,
    have attracted a lot of researches due to their best resistance against fast correlation attacks \cite{MS1988fast_correlation_attack} and best affine approximation attacks \cite{DingXS1991book} in cryptography and relations to several mathematical objects, such as difference sets, strongly regular graphs and complementary sequences \cite{Carlet2020book}.
    However, there is limited knowledge about the higher-order nonlinearity of bent functions due to the challenge of providing lower bounds on the higher-order nonlinearity.
    Additionally, all functions possess explicit mathematical expressions in the $\mathcal{PS}_{ap}$ bent functions (where ``$ap$ '' means ``affine plane''), which is a subclass of the $\mathcal{PS}^-$ class, one of the most well-known bent function classes proposed by Dillon \cite{Dillon1974PSbent}.
    The explicit and simplest expressions of this kind of bent functions will ensure that the proof of our lower bound remains low complexity and high feasibility.
    Due to the two aforementioned reasons,
    % in this paper, we derive a lower bound on the third-order nonlinearity of the simplest $\mathcal{PS}_{ap}$ bent functions.
    the focus of this paper is to obtain a lower bound on the third-order nonlinearity of the simplest $\mathcal{PS}_{ap}$ bent functions.

    In this paper, we aim to derive a lower bound on the third-order nonlinearity of the simplest $\mathcal{PS}_{ap}$ bent functions.
    % To this end, 
    In order to achieve this objective, we only need to determine the nonlinearity of the second-order derivatives of the $n$-variable Boolean function $f(x,y)$ for all possible pairs $(\alpha,\beta)\in\F_{2^k}\times\F_{2^k}$, where $n=2k$ and $x,y\in\F_{2^k}$, according to Carlet's lemma.
    
    To determine the nonlinearity of $D_{\beta}D_{\alpha}f(x,y)$ with the pair $(\alpha,\beta)\in\F_{2^k}\times\F_{2^k}$, it is equivalent to find the maximal absolute values of the corresponding character sums with $(\mu,\nu)$ ranging over $\F_{2^k}\times\F_{2^k}$.
    By Serre bound and some knowledge about algebraic curves, we can give an upper bound on the absolute value of a class of character sums, therefore a lower bound on the nonlinearity of $D_{\beta}D_{\alpha}f(x,y)$ for all trivial pairs $(\alpha,\beta)$ is  derived.
    While for the general pairs $(\alpha,\beta)$, we obtain the corresponding nonlinearity by determining the number of solutions for systems of trace functions.
    We derive then a lower bound on the third-order nonlinearity of the simplest $\mathcal{PS}_{ap}$ bent functions.
    As a result,
    we improve the previous two lower bounds in \cite{Carlet2011NL_Profile_Dillon,TangCT2013NL_2bent}, and the comparison of three lower bounds can be found in Table \ref{table:MyTableLabel} for small concrete values.
    Specifically, our lower bound is asymptotically equivalent to $2^{n-1}-2^{\frac{7n}{8}-\frac{1}{2}}$.

    The remainder of this paper is organized as follows:
    Section 2 introduces some basic notions which will be used in the sequel.
    In Section 3, we  obtain a lower bound on the third-order nonlinearity of the simplest $\mathcal{PS}_{ap}$ bent functions followed by the conclusion in Section 4.
    % Finally, Section 4 concludes this paper.

\section{Preliminaries}
    Let $\F_2$ be the field with two elements $\{0,1\}$ and $\F_2^n$ be the vector space of $n$-tuples over $\F_2$.
    Let $\F_{2^n}$ be the finite field of order $2^n$.
    The Hamming weight of an arbitrary vector $a=(a_1,a_2,\dots,a_n)\in\F_2^n$ is defined as $\wt(a)=\#\left\{ 1\le i\le n:a_i\ne 0 \right\}$, where $\#S$ denotes the cardinality of a given set $S$.
    For any integer $0\le i \le 2^n-1$, we denote by $\overline{i}=(i_1,i_2,\dots,i_n)\in\F_2^n$ the binary expansion of $i$, \emph{i.e.}, $i=\sum_{j=0}^{n-1}i_{j+1}2^j$.
    In the rest of this paper, we use $+$ (resp. $\Sigma$) to denote the addition (resp. multiple sums) in the set of the integers $\Z$ or the finite field $\F_{2^n}$, and use $\oplus$ (resp. $\bigoplus$) to denote the addition (resp. multiple sums) in $\F_2$ or $\F_2^n$.
    If the context is clear, we shall use $+$ (resp. $\Sigma$) to replace $\oplus$ (resp. $\bigoplus$).
    The Hamming distance between any two vectors $a=(a_1,a_2,\dots,a_n),b=(b_1,b_2,\dots,b_n)\in\F_2^n$ is defined as $d_H(a,b)=\#\left\{ 1\le i\le n : a_i\ne b_i \right\}$, and it can be easily checked that $d_H(a,b)=\wt(a\oplus b)$.

    A Boolean function in $n$ variables is a mapping from $\F_2^n$ to $\F_2$.
    The set of all $n$-variable Boolean functions is denoted by $\mathcal{B}_n$.
    The truth table (TT) is the most well-known way to represent Boolean functions, that is, 
    for an $n$-variable Boolean function $f$, we have 
    % The most well-known representation of an $n$-variable Boolean function $f$ is by its truth table (in brief, TT), \emph{i.e.},
    \[f=\left[ f(0,0,\dots,0),f(1,0,\dots,0),\dots,f(0,1,\dots,1),f(1,1,\dots,1) \right].\]
    It is clear that this representation is unique.
    The support of $f$, denoted by $\operatorname{supp}(f)$, is defined as the set $\left\{ x\in\F_2^n : f(x)\ne 0 \right\}$.
    The Hamming weight of $f$, denoted by $\wt(f)$, is defined as the cardinality of the support of $f$.
    It is well-known that any $f\in\mathcal{B}_n$ in variables $x_1,x_2,\dots,x_n$ can also be expressed through its algebraic normal form (in brief, ANF)
    \[f(x_1,x_2,\dots,x_n) = \bigoplus_{u\in\F_2^n}a_ux^u,\]
    where $x=(x_1,x_2,\dots,x_n)\in\F_2^n$, $u=(u_1,u_2,\dots,u_n)\in\F_2^n$, $a_u\in\F_2$ and the term $x^{u}=\prod_{i=1}^nx_i^{u_i}$ is called a monomial.
    The algebraic degree of $f$ is then denoted by $\deg(f)=\max\left\{ \wt(u) :  u\in\F_2^n, a_u\ne 0 \right\}$.
    If a Boolean function has algebraic degree at most $1$, we say that it is an affine function.
    The set of all affine functions in $n$ variables is denoted by $A_n$.
    And quadratic (Boolean) functions are those Boolean functions of algebraic degree at most $2$.
    % And we shall call quadratic functions the Boolean functions of algebraic degree at most $2$.
    % It is known that the vector space $\F_2^n$ is isomorphic to the finite field $\F_{2^n}$ through a choice of a basis of $\F_{2^n}$ over $\F_2$.
    An important result of finite field tells us that by choosing a basis of $\F_{2^n}$ over $\F_2$, the vector space $\F_2^n$ and the finite field $\F_{2^n}$ are isomorphic.
    Indeed, assuming that $(\alpha_1,\alpha_2,\dots,\alpha_n)$ is a basis of $\F_{2^n}$ over $\F_2$, then every vector $x=(x_1,x_2,\dots,x_n)\in\F_2^n$ corresponds to the element $x_1\alpha_1+x_2\alpha_2+\cdots+x_n\alpha_n\in\F_{2^n}$. 
    Therefore, $\F_{2^n}$ can be regarded as an $n$-dimensional vector space over $\F_2$, enabling the definition of $n$-variable Boolean functions over $\F_{2^n}$.
    As a result, any Boolean function in $n$ variables can be uniquely represented by 
    % As a result, any $n$-variable Boolean function can also be defined over $\F_{2^n}$ and uniquely represented by 
    a univariate polynomial in variable $x\in\F_{2^n}$ over $\F_{2^n}$
    \[f(x) = \sum_{i=0}^{2^n-1}\sigma_ix^i,\]
    where $\sigma_0,\sigma_{2^n-1}\in\F_2$ and, for every $i\in\{1,2,\dots,2^n-2\}$, $\sigma_{2i\pmod{2^n-1}}=\sigma_i^2$.
    In this case, the algebraic degree of $f$ can be computed by $\max\left\{ \wt(\overline{i}) : \sigma_i\ne 0, 0\le i\le 2^n-1 \right\}$ and the set $A_n$ is constituted by the functions $\Tr_1^n(\alpha x)\oplus c$, where $\alpha\in\F_{2^n}, c\in\F_2$ and $\Tr_m^n(x)=x+x^{2^m}+x^{2^{2m}}+\cdots+x^{2^{n-m}}$ is the (absolute) trace function from $\F_{2^n}$ to $\F_{2^m}$ with $m\mid n$.
    Particularly, for even $n$, any $n$-variable Boolean function $f$ can be uniquely expressed by a bivariate polynomial in variables $x,y\in\F_{2^{n/2}}$ over $\F_{2^{n/2}}$
    \[f(x,y)=\sum_{0\le i,j\le 2^{n/2}-1}f_{i,j}x^iy^j,\]
    where $f_{i,j}$'s are elements in the finite field $\F_{2^{n/2}}$.
    In this case, the algebraic degree of $f$ can be computed by $\max\left\{ \wt(\overline{i})+\wt(\overline{j}) : f_{i,j}\ne 0, 0\le i,j\le 2^{n/2}-1 \right\}$ and the set $A_n$ is constituted by the function $\Tr_1^{n/2}(\alpha x+\beta y)\oplus c$, where $\alpha,\beta\in\F_{2^{n/2}}$ and $c\in\F_2$.

    In coding theory, a binary linear code $\mathcal{C}$ with length $N$ is a subspace of $\F_2^N$ and $\F_2$-dimension of this subspace is called the dimension of $\mathcal{C}$.
    The minimum distance of $\mathcal{C}$ is defined as the minimum Hamming weight of all nonzero vectors (called codewords in coding theory).
    A binary linear code has length $N$, dimension $k$ and minimum distance $d$ is called an $\left[ N,k,d \right]$-code.
    The Reed-Muller code of order $r$, denoted by $\mathcal{RM}(r,n)$, where $1\le r\le n$, is formed by the truth tables of all $n$-variable Boolean functions of algebraic degree at most $r$. The covering radius of a binary $\left[ N,k,d \right]$-code is defined as the smallest integer $\rho$ such that all
    vectors in $\F_2^N$ are within Hamming distance $\rho$ of some codewords (see more details in \cite{MS1977}).
    Thus, the maximum $r$-th order nonlinearity of all Boolean functions in $n$ variables equals the covering radius of $\mathcal{RM}(r,n)$, \emph{i.e.},
    \[\max_{f\in \mathcal{B}_n}nl_r(f)=\max_{f\in\mathcal{B}_n}d_H(f,\mathcal{RM}(r,n)).\]
    \begin{definition}[\cite{Carlet2020book}]
        The $r$-th order nonlinearity of a Boolean function $f\in\mathcal{B}_n$ is defined as its minimum Hamming distance from $f$ to all $n$-variable Boolean functions of degree at most $r$
        \[nl_r(f)=\min_{g\in\mathcal{B}_n,\deg(g)\le r} d_H(f,g),\]
        where $d_H(f,g)$ denotes the Hamming distance between $f$ and $g$, i.e., $d_H(f,g)=\#\left\{ x\in\F_2^n : f(x)\ne g(x) \right\}$.
    \end{definition}
    The nonlinearity of $f$, denoted as $nl(f)$, refers to the first-order nonlinearity of $f$. 
    It is determined as the minimum Hamming distance from $f$ to all affine functions.
    % The first-order nonlinearity of $f$ is simply called the nonlinearity of $f$ and is denoted by $nl(f)$.
    % The nonlinearity $nl(f)$ is the minimum Hamming distance from $f$ to all affine functions.
    % Boolean functions employed in symmetric-key encryption must have high nonlinearity.
    For symmetric-key encryption, the higher the nonlinearity of the employed Boolean functions, the better resistance against fast correlation attacks \cite{MS1988fast_correlation_attack} and best affine approximation attacks \cite{DingXS1991book}.
    By Walsh transform, we can determine the nonlinearity of a given Boolean function. 
    % The nonlinearity of any Boolean function can also be expressed by means of its Walsh transform.
    % Let $x=(x_1,x_2,\dots,x_n)$ and $a=(a_1,a_2,\dots,a_n)$ be two vectors in $\F_2^n$ and $a\cdot x=a_1x_1\oplus a_2x_2\oplus\cdots\oplus a_nx_n$ be the usual inner product in $\F_2^n$.
    Actually, the Walsh transform of any $n$-variable Boolean function $f$ over $\F_2^n$ at a point $a\in\F_2^n$ is defined as
    \[W_f(a)=\sum_{x\in\F_2^n}(-1)^{f(x)+a\cdot x},\]
    where $a\cdot x$ is the usual inner product of two vectors $x=(x_1,x_2,\dots,x_n)$ and $a=(a_1,a_2,\dots,a_n)$ in $\F_2^n$, \emph{i.e.}, $a\cdot x=a_1x_1\oplus a_2x_2\oplus\cdots\oplus a_nx_n$.
    The multiset consisting of values of the Walsh transform is called the Walsh spectrum of $f$.
    % If $f$ is defined over $\F_{2^n}$,
    Meanwhile, the Walsh transform of $f$ over $\F_{2^n}$ at a point $\alpha\in\F_{2^n}$ is defined as
    \[W_f(\alpha)=\sum_{x\in\F_{2^n}}(-1)^{f(x)+\TrN(\alpha x)}.\]
    Additionally, when $n=2k$ is even, if $f(x,y)\in\mathcal{B}_n$ is the bivariate polynomial form of $f$, the Walsh transform of $f$ at any pair $(\alpha,\beta)\in\F_{2^k}^2$ is then defined as
    \[W_f(\alpha,\beta)=\sum_{(x,y)\in\F_{2^k}^2}(-1)^{f(x,y)+\Tr_1^k(\alpha x+\beta y)}.\]
    % Observing the Walsh transform, we can easily get that the nonlinearity of any Boolean function $f\in\mathcal{B}_n$ can be computed as
    Due to the Walsh transform, it becomes clear that for the nonlinearity of any Boolean function $f\in \mathcal{B}_n$, we have 
    \begin{align*}
        nl(f) &= 2^{n-1} - \frac{1}{2}\max_{a\in\F_2^n}|W_f(a)|\\
              &= 2^{n-1} - \frac{1}{2}\max_{\alpha\in\F_{2^n}}|W_f(\alpha)|\\
              &= 2^{n-1} - \frac{1}{2}\max_{(\alpha,\beta)\in\F_{2^k}^2}|W_f(\alpha,\beta)|~(\text{if }n=2k\text{ is even}).
    \end{align*}

    As mentioned before, Boolean functions utilized in symmetric-key encryption should have nonlinearity as high as possible to resist fast correlation attacks and best affine approximation attacks.
    Let $f$ be any $n$-variable Boolean function. It follows from the well-known Parseval relation that $\sum_{a\in\F_2^n}W_f^2(a)=2^{2n}$.
    This implies that $\max_{a\in\F_2^n}\left\lvert W_f(a)\right\rvert \geq  2^{n/2}$ and thus we can see that
    the nonlinearity of any $n$-variable Boolean function is limited to an upper bound of $2^{n-1}-2^{\frac{n}{2}-1}$.
    This bound is tight for even $n$ but is not tight for odd $n\geq 3$.
    If a Boolean function $f$ has nonlinearity achieving the upper bound with the equality,
    we say that $f$ is a bent function and clearly bent functions in $n$ variables exist only for even $n$.

    Given the truth table of an $n$-variable Boolean function, there exists an efficient divide-and-conquer algorithm for computing the Walsh spectrum with complexity $O(n2^n)$. 
    % There exists a simple divide-and-conquer butterfly algorithm with complexity $O(n2^n)$ to compute the Walsh spectrum of any $n$-variable Boolean function from its truth table.
    Hence, it is easy to compute the nonlinearity of a Boolean function from its truth table in a large number of variables (around $30$ variables in a personal computer).
    However, as mentioned in Introduction, there is a limited amount of results on the higher-order nonlinearity of Boolean functions. 
    This is due to the difficulty of computing the higher-order nonlinearity, or even providing a tight lower bound, of a general Boolean function with a high algebraic degree.
    In 2008, Carlet showed in \cite{Carlet2008lowbound_NL_profile} that a lower bound on the $r$-th order nonlinearity of a Boolean function can be deduced from its all $(r-1)$th-order nonlinearities of the first-order derivatives of this Boolean function.
    Thus, to obtain a lower bound on the third-order nonlinearity of a Boolean function, we need to introduce the first-order and second-order derivatives of Boolean functions.
    \begin{definition}
        The first-order derivative of a Boolean function $f$ over $\F_{2^n}$ at a point $\alpha\in\F_{2^n}$ is defined as $D_{\alpha}f(x)=f(x)+f(x+\alpha)$.
        And the second-order derivative of a Boolean function $f$ at a pair $(\alpha,\beta)\in\F_{2^n}\times\F_{2^n}$ is defined as $D_{\beta}D_{\alpha}f(x)=f(x)+f(x+\alpha)+f(x+\beta)+f(x+\alpha+\beta)$.
    \end{definition}
    With the notions above, Carlet proposed the following lemma which can efficiently provide a lower bound on the $r$-th order nonlinearity of a given Boolean function,
    provided that the $(r-1)$-th order nonlinearities of all first-order derivatives of $f$ are known.
    \begin{lemma}[Proposition 3 of \cite{Carlet2008lowbound_NL_profile}]\label{thm:High_order_nl_bound1}
        Let $f$ be any Boolean function in $n$ variables, and let $0<r<n$ be an integer.
        We have
        \[nl_r(f)\ge 2^{n-1}-\frac{1}{2}\sqrt{2^{2n}-2\sum_{a\in\F_2^n}nl_{r-1}(D_af)}.\]
    \end{lemma}

    There is a class of character sums with polynomial arguments needed to be treated for proving the main result of this paper.
    To this end, we need the following three lemmas about the existence of solutions of the quadratic polynomial over the finite field $\F_{2^n}$, the number of the genus of a function field and the Serre bound (we refer the reader to the original book~\cite{Stichtenoth2008book_algebraicfunctionfieldsandcodes} for more details).
    \begin{lemma}[\cite{Lidl1997FiniteFieldBook}]\label{lemma:hilbert90}
        Let $\F_{2^n}$ be the finite field of order $2^n$.
        Then $x^2+x=\alpha$ with $\alpha\in\F_{2^n}$ has a solution (actually two solutions) in $\F_{2^n}$, if and only if $\Tr_1^n(\alpha)=0$.
    \end{lemma}

    Before introducing the other two lemmas about the genus of a function field and the Serre bound, we need the following basic knowledge refer to algebraic geometry. 

    % If $\mathcal{C}$ is such a curve, then 
    % we write $g(\mathcal{C})$  for the genus of $\mathcal{C}$. 
    \begin{definition}[\cite{Stichtenoth2008book_algebraicfunctionfieldsandcodes}]
        An algebraic function field $K/F$ of one variables over a field $F$ is an extension field $K\supseteq F$ such that $K$ is a finite algebraic extension of $F(x)$ for some elements $x\in K$ which is transcendental over $F$.
    \end{definition}
    For brevity we shall simply refer to $K/F$ as a function field.
    A place $P$ of the function field $K/F$ is the maximal ideal of some valuation ring $\mathcal{O}$ of $K/F$ and $\deg(P):=\left[K:F\right]$ is called the degree of $P$. 
    We write $g(K)$ for the genus of $K/F$ and $N(K)$ for the number of places of degree one of $K/F$.
    % It will be convenient to replace the language of algebraic curves over finite fields by the equivalent language of global function fields, \emph{i.e.}, of algebraic function fields with finite constant fields.
    % For an algebraic curve $\mathcal{C}$ over $\F_q$, the field $K$ of $\F_q$-rational functions on $\mathcal{C}$ is a function field with full constant field $\F_q$, that is, with $\F_q$ algebraically closed in $K$.
    % We use the notation $K/\F_q$ to emphasize the fact that $\F_q$ is the full constant field of $K$.
    With every function field $K/\F_q$ we can associate an algebraic curve $\mathcal{C}$ over $\F_q$. 
    When we speak of an algebraic curve $\mathcal{C}$ over the finite field $\F_q$, we always mean a smooth, projective and absolutely irreducible algebraic curve defined over $\F_q$. 
    A point of $\mathcal{C}$ is called $\F_q$-rational if it has has homogeneous coordinates which all belong to $\F_q$.
    Let $N(\mathcal{C})$ denote the number of $\F_q$-rational points of $\mathcal{C}$. 
    In the correspondence between the curve $\mathcal{C}$ and its function field $K/\F_q$, 
    the $\F_q$-rational points of $\mathcal{C}$ can be identified with the places of degree one of $K/\F_q$, so that $N(\mathcal{C}) = N(K)$ (more details see for instance \cite{Serre1982serrebound}). 
    
    % the closed points of $\mathcal{C}$ can be identified with the places of $K$ and 
    % Furthermore, we have $g(\mathcal{C})=g(K)$.  
    \begin{lemma}[Corollary 3.11.4 of \cite{Stichtenoth2008book_algebraicfunctionfieldsandcodes}]\label{L:genus_K_F}
        Let $K=F(X,Y)$, where $X,Y$ are transcendentals over $F$.
        Then the genus $g$ of the function field $K/F$ satisfies
        \[g\le ([K : F(X)] - 1)([K : F(Y)] - 1).\]
    \end{lemma}

    In the last of this section, we briefly present a well-known bound, the Serre bound, which is useful in deriving an upper bound on the absolute value of a class of character sums.
    It provides a bound for the number of places of degree one of a function field $K/F$ in terms of its genus and the ground field size $|F|$.
    \begin{lemma}[Theorem 5.3.1 of \cite{Stichtenoth2008book_algebraicfunctionfieldsandcodes}]\label{L:Serrebound}
        For a function field $K/\F_q$ of genus $g$, the number $N(K)$ of places of degree one is bounded by 
        \[\lvert N(K)-(q+1)\rvert\le g\lfloor 2q^{1/2}\rfloor,\]
        where $\lfloor x\rfloor$ denotes the greatest integer not exceeding the real number $x$.
    \end{lemma}


     

\section{A lower bound on the third-order nonlinearity of the simplest $\mathcal{PS}_{ap}$ bent functions}

In this section, we focus on providing a lower bound on the third-order nonlinearity of the simplest
partial spread affine plane ($\mathcal{PS}_{ap}$ for short) bent functions.
We first introduce the definition of the $\mathcal{PS}_{ap}$ bent functions.
Then we determine the nonlinearities of all second-order derivatives of the simplest $\mathcal{PS}_{ap}$ bent functions
except for the trivial case, and a lower bound on the nonlinearities of the second-order derivatives is given in trivial case.
So we can obtain a lower bound on the third-order nonlinearity of the simplest $\mathcal{PS}_{ap}$
bent functions.

  \subsection{The simplest $\mathcal{PS}_{ap}$ bent functions}

In this subsection, we present the basic definition of the $\mathcal{PS}_{ap}$ bent functions.
Let $\left\{ E_1,E_2,\dots,E_s \right\}$ be a set of pairwise disjoint $k$-dimensional subspaces of $\F_2^n$, {\em i.e.,}
$E_i\cap E_j = \{\bm{0}_n\}$ for $1\le i<j\le s$, where $\bm{0}_n$ is the all-zero vector in $\F_2^n$.
Any collection of $\left\{ E_1,E_2,\dots,E_s \right\}$ is called a partial $k$-spread of $\F_2^n$.
Further, a partial spread is called a spread, if it covers $\F_2^n$, \emph{i.e.}, the union of $E_i$ is equal to $\F_2^n$, where $1\le i\le s$.
Particularly, when $n=2k$, a partial $k$-spread of $\F_2^n$ is a set of pairwise supplementary of $k$-dimensional subspaces of $\F_2^n$,
\emph{i.e.},  the sum of any two of them is direct and equals $\F_2^n$.
Assume $n=2k$ and we then have $\F_{2^n}$ being isomorphic to the affine plane $\F_{2^k}\times\F_{2^k}$,
in which  an affine plane is a vector space of dimension two in which one has ``forgotten'' where the origin is.
Considering $\F_{2^k}\times\F_{2^k}$, we then have $2^k+1$ pairwise disjoint $n/2$-dimensional $\F_2$-subspaces of $\F_2^n$, which are
$E_a=\left\{(x,\alpha x) : x\in\F_{2^k}\right\}$ for $\alpha\in\F_{2^k}$ and $E_{\emptyset}=\left\{(0,x) : x\in\F_{2^k}\right\}$.
It can be seen that all elements $(x,y)$ of $E_{\alpha}$ constitute a line $y=\alpha x$ over $\F_{2^k}$ for every $\alpha\in\F_{2^k}$ and
$E_{\emptyset}$ can be viewed as a line $y\in\F_{2^k}$ and all $2^k+1$ lines intersect only at the origin.
This is the reason why these subspaces are called ``affine plane''.
%That is why we use the term ``affine plane''.
These $2^k+1$ pairwise disjoint $n/2$-dimensional $\F_2$-subspaces of $\F_2^n$ constitute a spread, which is the well-known Desarguesian spread.

In \cite{Dillon1974PSbent}, Dillon proposed constructions of bent functions in the $\mathcal{PS}^-$ (resp. $\mathcal{PS}^+$) class in $n$ variables, whose supports are the unions of $2^{k-1}$ (resp. $2^{k-1}+1$) different elements of a partial $k$-spread, where $n=2k$.
Especially, Dillon exhibited the so-called $\mathcal{PS}_{ap}$ bent functions over finite fields, which is a subclass of the $\mathcal{PS}^-$ class.
Any $\mathcal{PS}_{ap}$ bent function can be written as
\begin{equation*}\label{Eqn_PS_bent}
    f_0(x,y)=g\left(xy^{2^k-2}\right)=g\left(\frac{x}{y}\right),
\end{equation*}
where $g$ is a $k$-variable balanced Boolean function on $\F_{2^{k}}$ with the condition $g(0)=0$, and we assume the convention that $x/y$ equals $0$ if $y=0$ in the rest of this paper.
We can see that the support of $f_0$ is constituted by $2^{k-1}$ elements of the Desarguesian spread except $E_0$ and $E_{\emptyset}$.
In this paper, let $g$ be the simplest balanced Boolean function on $\F_{2^k}$, \emph{i.e.}, the trace function $\TRACE(x)$, then
$f_0$ can be written as
\begin{equation*}\label{sub-bent}
    f(x,y)=\TRACE\left(\frac{\lambda x}{y}\right),
\end{equation*}
where $(x,y)\in\F_{2^k}^2$, $\lambda\in\F_{2^k}^{*}$ and $\TRACE(x)=\sum\limits_{i=0}^{k-1}x^{2^i}$ is the trace function from $\F_{2^k}$ to $\F_2$.
In this paper, $f$ is called a simplest $\mathcal{PS}_{ap}$ bent function.

    \subsection{The nonlinearity of the second-order derivatives of the simplest $\mathcal{PS}_{ap}$ bent functions}
    As mentioned in Introduction,
    we need to determine the nonlinearities of all second-order derivatives of the simplest $\mathcal{PS}_{ap}$ bent functions to derive a lower bound on the third-order nonlinearity.
    To this end, we begin by introducing two lemmas involved to the multiplicative inverse functions,
    as well as providing a lower bound on a class of character sums by algebraic geometry.
    Additionally, we present two lemmas about determining the number of solutions for systems of trace functions.
    These results will be important in the sequel for proving our main result.

    For the finite field $\F_{2^k}$, the multiplicative inverse function over $\F_{2^k}$ is defined as $I(x)=x^{-1}$ (we shall use $x^{-1}$ to denote $x^{2^k-2}$ and clearly $x^{-1}=x^{2^k-2}=0$ if $x=0$).
    % with the convention that $x^{-1}=\frac{1}{x}=0$ if $x=0$).
    Besides, for any $v\in\F_{2^k}^*$ we also define $I_v(x)=\Tr_1^k(vx^{-1})$.
    Regarding the multiplicative inverse function, the following two lemmas will be used in the sequel.

    \begin{lemma}[Lemma 12 of \cite{TangMM2022inversefunction}]\label{L:SumInv00}
        Let $k\geq 3$ be an arbitrary integer.
        We define
        $$L=\#\left\{c\in\F_{2^k}  :  \mathrm{Tr}_1^k\left(\frac{1}{c^2+c+1}\right)=\mathrm{Tr}_1^k\left(\frac{c^2}{c^2+c+1}\right)=0\right\}.$$
        Then we have $L=2^{k-2}+\frac{3}{4}(-1)^kW_{I_1}(1)+\frac{1}{2}\left(1-(-1)^k\right)$, where $W_{I_1}(1)=1-\sum_{t=0}^{\lfloor k/2\rfloor}(-1)^{k-t}\frac{k}{k-t}{{k-t}\choose {t}}2^t$.
    \end{lemma}

    \begin{lemma}\label{lemma:num_sol_second_dev}
        Let $k\ge 3$ be an integer.
        For $a,b,\mu\in\F_{2^k}$ with $\lambda\in\F_{2^k}^*$, define $F_{\lambda}(x)=\lambda x^{-1}$ and
        \begin{align*}
            U_{F_{\lambda}}(a,b,\mu)=&\left\{x\in\F_{2^k} : F_{\lambda}(x)+F_{\lambda}(x+a)+F_{\lambda}(x+b)+F_{\lambda}(x+a+b)=\mu\right\}\\
            =&\left\{ x\in\F_{2^k} : \frac{\lambda}{x}+\frac{\lambda}{x+a}+\frac{\lambda}{x+b}+\frac{\lambda}{x+a+b}=\mu \right\},
        \end{align*}
        then we have
        \[\# U_{F_{\lambda}}(a,b,\mu) = \left\{
            \begin{alignedat}{2}
                &2^k,&&\text{ if }(a,b)\in M\text{ and }\mu = 0,\\
                &8,&&\text{ if both two Conditions~A}\text{ and~B}\text{ hold,}\\
                &4,&&\text{ if either Condition~A}\text{ or Condition B} \text{ holds,}\\
                &0,&&\text{ otherwise},
            \end{alignedat}\right.\]
        where $M=\left\{ (x,x):x\in\F_{2^k}^* \right\}\cup\left\{ (x,0):x\in\F_{2^k} \right\}\cup\left\{ (0,y):y\in\F_{2^k}^* \right\}$, and the two conditions are listed as follows:

        \noindent $-$  Condition A: $a\ne b\in\F_{2^k}^*$ and $\lambda(a^2+b^2+ab)+\mu(a^2b+ab^2)=0$.\label{item_a}

        \noindent $-$  Condition B: $a\ne b\in\F_{2^k}^*$, $\mu\ne 0$, $\TRACE\left(\frac{\lambda a}{\mu b(a+b)}\right)=0$ and $\TRACE\left(\frac{\lambda b}{\mu a(a+b)}\right)=0$.\label{item_b}

    \end{lemma}
    \begin{proof}
        For any fixed $\lambda\in\F_{2^k}^*$, we have $U_{F_{\lambda}}(a,b,\mu)=U_{F_1}(a,b,\mu/\lambda)=\left\{ x\in\F_{2^k} : \frac{1}{x}+\frac{1}{x+a}+\frac{1}{x+b}+\frac{1}{x+a+b}=\frac{\mu}{\lambda} \right\}$.
        And the rest of the proof is the same as the proof of Lemma 13 in \cite{TangMM2022inversefunction} if we treat $\mu/\lambda$ as an entry, so we omit it. 
    \end{proof}
    \begin{remark}
        Note that if \hyperref[item_a]{Condition A} holds, we have $\{0,a,b,a+b\}\subseteq U_{F_{\lambda} }(a,b,\mu)$.
        If \hyperref[item_b]{Condition B} holds, we have $\{y_0,y_0+a,y_0+b,y_0+a+b\}\subseteq U_{F_{\lambda}}(a,b,\mu)$ where $y_0\notin\{0,a,b,a+b\}$.
    \end{remark}
    \begin{remark}
        For any $a\in\F_{2^k}^*$, it follows from \hyperref[item_a]{Conditions A} and \hyperref[item_b]{B} that there exist $L-4$ distinct values of $b$
        such that $\#U_{F_{\lambda}}(a,b,\mu)=8$ for some fixed $\lambda\in\F_{2^k}^*$ and $\mu\in\F_{2^k}$,
        where $L$ is given by Lemma~\ref{L:SumInv00}.
       Indeed, note that $a,b$ be two distinct elements of $\F_{2^k}^*$ and $\mu\ne 0$,
        we have $\mu(a^2b+ab^2)\ne 0$.
        In \hyperref[item_a]{Condition A}, this indicates $\lambda(a^2+b^2+ab)\ne 0$,
        which implies $\frac{b}{a}\notin\F_4$.
        So assume $\mu=\frac{\lambda(a^2+b^2+ab)}{a^2b+ab^2}$,
        then by  $\TRACE\left(\frac{\lambda a}{\mu b(a+b)}\right)=0$
        and $\TRACE\left(\frac{\lambda b}{\mu a(a+b)}\right)=0$
        we obtain $\TRACE\left(\frac{1}{\gamma^2+\gamma+1}\right)=0$ and $\TRACE\left(\frac{\gamma^2}{\gamma^2+\gamma+1}\right)=0$ respectively,
        where $\gamma=\frac{b}{a}\in\F_{2^k}\setminus\F_{4}$.
        Therefore, according to Lemma \ref{L:SumInv00},
        the number of $\gamma=\frac{b}{a}\in\F_{2^k}\setminus\F_{4}$ satisfying
        $\TRACE\left(\frac{1}{\gamma^2+\gamma+1}\right)=0$ and $\TRACE\left(\frac{\gamma^2}{\gamma^2+\gamma+1}\right)=0$
        is $L-4$ since we need to exclude the cases $\gamma\in\F_4$.
    \end{remark}


The following lemma provides a lower bound on a class of character sums by some basic knowledge about algebraic geometry (see more details in \cite{Stichtenoth2008book_algebraicfunctionfieldsandcodes}).
    \begin{lemma}\label{lemma:charactersums}
        Let $k\ge 3$ be a positive integer and assume
        \[S(\alpha,\beta,v)=\sum_{x\in\F_{2^k}}(-1)^{\TRACE\left( \frac{\alpha}{x+\beta}+\frac{\alpha}{x}+vx \right)},\]
        where $\alpha,\beta,v\in\F_{2^k}^*$.
        We have
        \[\left\lvert S(\alpha,\beta,v)\right\rvert\le 2\left\lfloor 2^{\frac{k}{2}+1}\right\rfloor+4 .\]
    \end{lemma}
    \begin{proof}
        Note that $S(1,\beta/\alpha,v\alpha)=S(\alpha,\beta,v)$.
        So we only need to determine the values of $S(1,\beta,v)$.
        In other words, we need to obtain the number of solution $x\in\F_{2^k}$ of $\TRACE\left( \frac{1}{x+\beta}+\frac{1}{x}+vx \right)=0$.
        By Lemma $\ref{lemma:hilbert90}$, it is equivalent to calculate the number of solutions $(x,y)\in\F_{2^k}^2$ of $y^2+y=\frac{1}{x+\beta}+\frac{1}{x}+vx$.

        Let us define  
        \[\mathcal{S}_{\beta,v}=\left\{(x,y)\in\F_{2^k}\times\F_{2^k} : y^2+y=\frac{1}{x+\beta}+\frac{1}{x}+vx\right\}.\]
        Since $y\mapsto y^2+y$ is a $2$-to-$1$ mapping, we have
        \begin{equation}\label{eq:tracesum_S}
            S(\alpha,\beta,v)=\frac{\#\mathcal{S}_{\beta,v}}{2}-\left(2^k-\frac{\#\mathcal{S}_{\beta,v}}{2}\right)=\#\mathcal{S}_{\beta,v}-2^k.
        \end{equation}
        Note that $\#\mathcal{S}_{\beta,v}$ is even and then $S(\alpha,\beta,v)$ must be even as well.
        Consider  the function field $K=\F_{2^k}(x,y)$ with defining equation
        \begin{equation}\label{eq:trace_curve}
            y^2+y=\frac{1}{x+\beta}+\frac{1}{x}+vx.
        \end{equation}
        Using Lemma \ref{L:genus_K_F}, it can be deduced that the maximum possible genus of $K$ is $2-\delta_v$,
        where $\delta_v=1$ if $v=0$ and $\delta_v=0$ otherwise.
        % We denote by $\mathcal{N}$ the number of the places with degree one of $K/\F_{2^k}$.
        The number of the places of degree one of $K/\F_{2^k}$ is denoted by $N(K)$.
        Then we have
        \begin{equation}\label{eq:N_genus_inequality}
            \left\lvert N(K)-(2^k+1)\right\rvert\le g\left\lfloor 2^{\frac{k}{2}+1}\right\rfloor,
        \end{equation}
        where $g$ is the genus of the function field $K/\F_{2^k}$, according to Lemma \ref{L:Serrebound}.
        It is known that
         \begin{equation}\label{eq:N_S_M_equality}
            N(K)=\#\mathcal{S}_{\beta,v}-\mathcal{M}_{\beta,v},
        \end{equation}
        where $\mathcal{M}_{\beta,v}$ is the number of the points at infinity of Equation \eqref{eq:trace_curve}.
        Thus, to estimate the value of $|S(\alpha,\beta,v)|$, we need to compute the value of $\mathcal{M}_{\beta,v}$.
        We homogenize Equation \eqref{eq:trace_curve} to
        \begin{equation}\label{eq:homogenize}
            \left( \frac{Y}{Z} \right)^2+\frac{Y}{Z}=\frac{Z}{X+\beta Z}+\frac{Z}{X}+\frac{vX}{Z}.
        \end{equation}
        Multiplying both sides of Equation \eqref{eq:homogenize} by $Z^2X\left( X+\beta Z \right)$ and then let $Z=0$,
        we have
        \[X^2Y^2=0,\]
        hence the points at infinity are $(1:0:0)$ and $(0:1:0)$.

        Next, we will determine the multiplicity of roots of $(0 : 1 : 0)$ and $(1 : 0 : 0)$, respectively.
        For the point at infinity $(1 : 0 : 0)$, \emph{i.e.}, $X = 1$, we have
        \begin{equation}\label{eq:points_100}
            \left( \frac{y}{z} \right)^2+\frac{y}{z}=\frac{z}{1+\beta z}+z+\frac{v}{z}.
        \end{equation}
        And multiplying Equation \eqref{eq:points_100} by $z^2(1+\beta z)$ gives
        \[vz+v\beta z^2+y^2+zy+\beta y^2z+\beta yz^2+\beta z^4=0.\]
        Note that when $v=0$, we have
        \[y^2+zy+\beta y^2z+\beta yz^2+\beta z^4=0,\]
        which implies that the multiplicity of the root $(1:0:0)$ is $2$.
        For the point at infinity $(0 : 1 : 0 )$, \emph{i.e.}, $Y = 1$, we have
        \begin{equation}\label{eq:points_010}
            \left( \frac{1}{z} \right)^2+\frac{1}{z}=\frac{z}{x+\beta z}+\frac{z}{x}+\frac{vx}{z}.
        \end{equation}
        And multiplying Equation \eqref{eq:points_010} by $z^2x(x+\beta z)$ gives
        \[x^2+\beta xz+R_{\beta,v}(x,z)=0,\]
        where $R_{\beta,v}(x,z)=\beta xz^2+x^2z+\beta z^4+v\beta x^2z^2+vx^3z$ is a polynomial
        such that its every monomial has algebraic degree at least $3$.
        This gives $(0 : 1 : 0)$ is a root of multiplicity $2$.

        Therefore, there exists at most $4$ points at infinity for Equation \eqref{eq:trace_curve}, \emph{i.e.},
        $\mathcal{M}_{\beta,v}\le 4$.
        Thus, combining Equations \eqref{eq:tracesum_S},\eqref{eq:N_genus_inequality},\eqref{eq:N_S_M_equality} and the fact that
        $S(\alpha,\beta,v)$ is even we can get our assertion
        \[\left\lvert S(\alpha,\beta,v)\right\rvert \le 2\left\lfloor 2^{\frac{k}{2}+1}\right\rfloor+4.\]
        This ends the proof.
    \end{proof}

    To verify our upper bound on the values of $|S(\alpha,\beta,\mu)|$, we give the Table \ref{table:Lemma6table} about the exact values of $\max_{\alpha,\beta,\mu\in\F_{2^k}^*}|S(\alpha,\beta,\mu)|$ and the estimation provided by our upper bound on small number of variables.

    \begin{table}
        \centering
        \caption{The exact values of $\max_{\alpha,\beta,\mu\in\F_{2^k}^*}|S(\alpha,\beta,\mu)|$ for $6\le k\le 14$}
        \begin{threeparttable}
            \begin{tabular}{|c|c|c|c|c|c|c|c|c|c|}
                \hline
                $k$ &$6$ &  $ 7 $ & $ 8 $     & $ 9 $     & $  10 $ & $11$ &$12$ &$13$ &$14$\\  \hline 
                Exact value  &  $32$  & $40$ & $64$ &$88$  &$128^*$ &$176^*$ &$256^*$ & $360^*$ & $504^*$\\ \hline
                Our upper bound in Lemma \ref{lemma:charactersums}& $36$  & $48$ & $68$ & $94$ & $132$ & $184$& $260$ & $366$ & $516$\\ \hline
                Difference$^{\dagger}$ & $4$ & $8$& $4$& $6$ &$-$ &$-$ &$-$&$-$ &$-$ \\ \hline
            \end{tabular}
            \begin{tablenotes}
                \footnotesize
                % \item[$\star$] All concrete values of lower bounds in this table have been rounded up to the nearest integer.
                \item[$\dagger$]  The values in the table are the difference between the exact values of $\max_{\alpha,\beta,\mu\in\F_{2^k}^*}|S(\alpha,\beta,\mu)|$ and our upper bound.
                \item[*] The entries marked with an asterisk mean that we have not yet exhausted all possible values of $|S(\alpha,\beta,\mu)|$ due to the large computational space.
            \end{tablenotes}
        \end{threeparttable}
        \label{table:Lemma6table}
    \end{table}


    The next two lemmas are intended to determine the number of solutions for systems of trace functions by properties of the trace functions over finite fields.
    \begin{lemma}\label{lemma:N_ij_trace}
        Let $k\ge 3$ be a positive integer.  We define
        \[ N_{i,j} =\#\left\{x\in\F_{2^k} : \TRACE\left(\theta_1x+\gamma_1\right)=i,\TRACE\left(\theta_2x+\gamma_2\right)=j\right\}, \]
        where  $\gamma_1,\gamma_2\in\F_{2^k}$ and $\theta_1\ne\theta_2\in\F_{2^k}^*$. Then $N_{0,0} =2^{k-2}$.
    \end{lemma}

   \begin{proof}
        Clearly, we can see that
        \[N_{0,0}+N_{0,1}=\#\left\{x\in\F_{2^k} : \TRACE\left(\theta_1x+\gamma_1\right)=0\right\}=2^{k-1}\]
        and
        \[N_{1,1}+N_{0,1}=\#\left\{x\in\F_{2^k} : \TRACE\left(\theta_2x+\gamma_2\right)=1\right\}=2^{k-1}.\]
        Then we have $N_{0,0} = N_{1,1}$.
        Besides, since $\theta_1\ne\theta_2$, the trace function $\TRACE\left((\theta_1+\theta_2)x+(\gamma_1+\gamma_2)\right)$ is balanced.
        So we have $N_{0,0}+N_{1,1} = \#\left\{x\in\F_{2^k} : \TRACE\left((\theta_1+\theta_2)x+(\gamma_1+\gamma_2)\right)=0\right\}=2^{k-1}$.
        Therefore $N_{0,0}=2^{k-2}$. This completes the proof.
   \end{proof}

    \begin{lemma}\label{lemma:N_ijk_trace}
        Assume $k\ge 3$ be a positive integer, let
        \[ N_{i_1,i_2,i_3}=\#\left\{x\in\F_{2^k} :  \TRACE\left(\theta_1x+\gamma_1\right)=i_1,\TRACE\left(\theta_2x+\gamma_2\right)=i_2,\TRACE\left(\theta_3x+\gamma_3\right)=i_3 \right\},\]
        where  $\gamma_1,\gamma_2,\gamma_3\in\F_{2^k}$, and $\theta_1,\theta_2,\theta_3$ are three pairwise distinct elements of $\F_{2^k}^*$ such that $\theta_3\ne\theta_1+\theta_2$. Then $N_{0,0,0}= 2^{k-3}$.
    \end{lemma}

    \begin{proof}
        By Lemma \ref{lemma:N_ij_trace} we have
        \begin{equation}\label{eq:from_lemma_1}\left\{\begin{alignedat}{3}
        &N_{0,0,0}+N_{0,0,1}=\#\left\{x\in\F_{2^k} : \TRACE\left(\theta_1x+\gamma_1\right)=0, \TRACE\left(\theta_2x+\gamma_2\right)=0\right\}=2^{k-2}\\
        &N_{0,0,0}+N_{0,1,0}=\#\left\{x\in\F_{2^k} : \TRACE\left(\theta_1x+\gamma_1\right)=0, \TRACE\left(\theta_3x+\gamma_3\right)=0\right\}=2^{k-2}\\
        &N_{0,0,0}+N_{1,0,0}=\#\left\{x\in\F_{2^k} : \TRACE\left(\theta_2x+\gamma_2\right)=0, \TRACE\left(\theta_3x+\gamma_3\right)=0\right\}=2^{k-2}.\\
        \end{alignedat}\right.\end{equation}
        Thus, $N_{0,0,1}=N_{0,1,0}=N_{1,0,0}$. With the same reason we can also obtain  $N_{0,1,1}=N_{1,0,1}=N_{1,1,0}$.
        Recall that $\theta_1+\theta_2+\theta_3\ne 0$, we obtain equations
        \begin{equation}\label{eq:sum_three_trace_1}
            N_{0,0,1}+N_{0,1,0}+N_{1,0,0}+N_{1,1,1}=\#\left\{x\in\F_{2^k} : \TRACE\left(\left(\theta_1+\theta_2+\theta_3\right)x+\left(\gamma_1+\gamma_2+\gamma_3\right)\right)=1\right\}=2^{k-1}
        \end{equation}
        and
        \begin{equation}\label{eq:sum_three_trace_2}
            N_{0,1,1}+N_{1,0,1}+N_{1,1,0}+N_{0,0,0}=\#\left\{x\in\F_{2^k} : \TRACE\left(\left(\theta_1+\theta_2+\theta_3\right)x+\left(\gamma_1+\gamma_2+\gamma_3\right)\right)=0\right\}=2^{k-1}.
        \end{equation}
        Combining Equation \eqref{eq:sum_three_trace_1}, \eqref{eq:sum_three_trace_2} and the following equations
        \begin{equation}\label{eq:sum_N_0jk}\left\{\begin{alignedat}{2}
            &N_{0,0,0}+N_{0,0,1}+N_{0,1,0}+N_{0,1,1}=\#\left\{x\in\F_{2^k} : \TRACE\left(\theta_1x+\gamma_1\right)=0\right\}=2^{k-1}\\
            &N_{1,0,0}+N_{1,0,1}+N_{1,1,0}+N_{1,1,1}=\#\left\{x\in\F_{2^k} : \TRACE\left(\theta_1x+\gamma_1\right)=1\right\}=2^{k-1}\\
            &N_{0,0,1}=N_{0,1,0}=N_{1,0,0}\\
            &N_{0,1,1}=N_{1,0,1}=N_{1,1,0},
        \end{alignedat}\right.\end{equation}
        we obtain $N_{0,0,1}=N_{0,1,1}$.
        Consequently, Equations \eqref{eq:from_lemma_1} and \eqref{eq:sum_N_0jk} can be transformed into
        $N_{0,0,0}+N_{0,0,1}=2^{k-2}$ and $N_{0,0,0}+3N_{0,0,1}=2^{k-1}$, respectively.
        So we have $N_{0,0,0}=N_{0,0,1}=2^{k-3}$. This completes the proof.
    \end{proof}

    With the five lemmas above, we can determine or provide a lower bound on the nonlinearities of all second-order derivatives of the simplest $\mathcal{PS}_{ap}$ bent functions.
    \begin{theorem}\label{thm:nl_DaDbf}
        Let $k\ge 3$ be an integer and $n=2k$ and $f(x,y)=\TRACE\left(\frac{\lambda x}{y}\right)$ be a simplest $\mathcal{PS}_{ap}$ bent function, $\lambda\in\F_{2^k}^*$.
        Then for $\alpha=(\alpha_1,\alpha_2),\beta=(\beta_1,\beta_2)\in\F_{2^k}\times\F_{2^k}$, we have
        \begin{equation}\label{res:nontrivil_nl}
            nl(D_{\beta}D_{\alpha}f)=\begin{cases}
                2^{2k-1}-2^{k+2},&\text{if both }\hyperref[item_a]{{Conditions~A}}\text{ and }\hyperref[item_b]{{B}} \text{ hold},\\
                2^{2k-1}-2^{k+1},&\text{if either }\hyperref[item_a]{{Condition~A}}\text{ or }\hyperref[item_b]{Condition~B}\text{ holds},\\
                0,&\text{if }(\alpha,\beta)\in \mathcal{E},%\footnote[1]{This happends if $ \beta=0 $},
            \end{cases}
        \end{equation}
        and $nl(D_{\beta}D_{\alpha}f)\ge 2^{2k-1}-2^k\left\lfloor 2^{\frac{k}{2}+1}\right\rfloor-2^{k+1}$ otherwise,
        where $\mathcal{E}=\{(x_1,x_2,x_1,x_2):x_1\in\F_{2^k},x_2\in\F_{2^k}^*\}\cup\{(x_1,0,x_2,0):x_1,x_2\in\F_{2^k}\}$.
    \end{theorem}

    \begin{proof}
        Let us consider the Walsh transform of the second-order derivative of $f(x,y)=\TRACE\left(\frac{\lambda x}{y}\right)$ at
        pair $(\alpha, \beta)\in\F_{2^n}\times\F_{2^n}$, where $\lambda\in\F_{2^k}^*$.
        Assume that $\alpha=(\alpha_1,\alpha_2),\beta=(\beta_1,\beta_2)\in\F_{2^k}\times\F_{2^k}$, then
        we have
        \begin{align*}\label{eq:secondordersum}
            &W_{D_{\beta}D_{\alpha}f}(\mu,\nu)\nonumber\\
            =&\sum_{x\in\F_{2^k}}\sum_{y\in\F_{2^k}}(-1)^{\TRACE\left(\frac{\lambda x}{y}+\frac{\lambda (x+\alpha_1)}{y+\alpha_2}+\frac{\lambda (x+\beta_1)}{y+\beta_2}+\frac{\lambda (x+\alpha_1+\beta_1)}{y+\alpha_2+\beta_2}+\mu x+\nu y\right)}\nonumber\\
            =&\sum_{y\in\F_{2^k}}(-1)^{\TRACE\left(\frac{\lambda\alpha_1}{y+\alpha_2}+\frac{\lambda\beta_1}{y+\beta_2}+\frac{\lambda(\alpha_1+\beta_1)}{y+\alpha_2+\beta_2}+\nu y\right)}\nonumber\\
            &\times \sum_{x\in\F_{2^k}}(-1)^{\TRACE\left(\left(\frac{\lambda}{y}+\frac{\lambda}{y+\alpha_2}+\frac{\lambda}{y+\beta_2}+\frac{\lambda}{y+\alpha_2+\beta_2}+\mu\right)x\right)}\nonumber\\
            =&\begin{cases}
                2^k\sum_{y\in U}(-1)^{\TRACE\left(\frac{\lambda\alpha_1}{y+\alpha_2}+\frac{\lambda\beta_1}{y+\beta_2}+\frac{\lambda(\alpha_1+\beta_1)}{y+\alpha_2+\beta_2}+\nu y\right)},&~\text{if}~\frac{\lambda}{y}+\frac{\lambda}{y+\alpha_2}+\frac{\lambda}{y+\beta_2}+\frac{\lambda}{y+\alpha_2+\beta_2}=\mu~\text{has solutions},\\
                0, &~\text{otherwise},
            \end{cases}
        \end{align*}
        where $U$ is the set of solutions of equation
        \begin{equation}\label{eq:coefficient}
            \frac{\lambda}{y}+\frac{\lambda}{y+\alpha_2}+\frac{\lambda}{y+\beta_2}+\frac{\lambda}{y+\alpha_2+\beta_2}=\mu.
        \end{equation}
        Note that the number of solutions of Equation \eqref{eq:coefficient} is fully determined in Lemma \ref{lemma:num_sol_second_dev}.

        For the nonlinearity of $D_{\beta}D_{\alpha}f$, we only need to consider $\max_{\mu,\nu}|W_{D_{\beta}D_{\alpha}f}(\mu,\nu)|$ for every pair $(\alpha,\beta)$.
        Note that if Equation \eqref{eq:coefficient} has no solutions we have $\left\lvert W_{D_{\beta}D_{\alpha}f}(\mu,\nu)\right\rvert=0$.
        So we only need to consider the case that Equation \eqref{eq:coefficient} has solutions.
        According to values of $\alpha_2$ and $\beta_2$ in Lemma \ref{lemma:num_sol_second_dev}, the values of $W_{D_{\beta}D_{\alpha}f}(\mu,\nu)$
        can be divided into the following three cases.

        \begin{enumerate}[label=\textbf{Case \arabic*},wide = 0pt]
            \item The trivial case $\mu=0$, $(\alpha_1,\beta_1)\in\F_{2^k}\times\F_{2^k}$ and $(\alpha_2,\beta_2)\in M$, where $M$ is defined in Lemma \ref{lemma:num_sol_second_dev}.
            % It can be easily verified that any $y\in \F_{2^k}$ is a solution of Equation \eqref{eq:coefficient}.
            Thus, Equation \eqref{eq:coefficient} has $2^k$ solutions in this case, \emph{i.e.}, any $y\in\F_{2^k}$ is a solution of Equation \eqref{eq:coefficient}.
            So we have
            \begin{equation}\label{eq:case2ksolutions}
                W_{D_{\beta}D_{\alpha}f}(0,\nu)=2^k\sum_{y\in\F_{2^k}}(-1)^{\TRACE\left(\frac{\lambda\alpha_1}{y+\alpha_2}+\frac{\lambda\beta_1}{y+\beta_2}+\frac{\lambda(\alpha_1+\beta_1)}{y+\alpha_2+\beta_2}+\nu y\right)}.
            \end{equation}

            In the subcase of
            $(\alpha_1,\alpha_2,\beta_1,\beta_2)\in\{(x_1,x_2,x_1,x_2):x_1\in\F_{2^k},x_2\in\F_{2^k}^*\}\cup\{(x_1,0,x_2,0):x_1,x_2\in\F_{2^k}\}$,
            Equation \eqref{eq:case2ksolutions} will be in a simple form:
            \[W_{D_{\beta}D_{\alpha}f}(0,\nu)=2^k\sum_{y\in\F_{2^k}}(-1)^{\TRACE\left(\nu y\right)}.\]
            Clearly $\max_{\nu}|W_{D_{\beta}D_{\alpha}f}(0,\nu)|=W_{D_{\beta}D_{\alpha}f}(0,0)=2^{2k}$.
            For other subcases, we will give an upper bound on $\max_{v}|W_{D_{\beta}D_{\alpha}f}(0,v)|$.
            Consider the case $(\alpha_1,\alpha_2,\beta_1,\beta_2)\in\{(x_1,x_2,x_3,x_2):x_1\ne x_3\in\F_{2^k},x_2\in\F_{2^k}^*\}$,
            by Lemma \ref{lemma:charactersums} we then have
            \[\left\lvert W_{D_{\beta}D_{\alpha}f}(0,v)\right\rvert =2^k\left\lvert \sum_{y\in\F_{2^k}}(-1)^{\TRACE\left(\frac{\lambda(\alpha_1+\beta_1)}{y+\alpha_2}+\frac{\lambda(\alpha_1+\beta_1)}{y}+vy\right)}\right\rvert\le 2^{k+1}\left\lfloor 2^{\frac{k}{2}+1}\right\rfloor+2^{k+2}.\]

            The case $(\alpha_1,\alpha_2,\beta_1,\beta_2)\in\F_{2^k}^*\times\{0\}\times\F_{2^k}\times\F_{2^k}^*$ and the case $(\alpha_1,\alpha_2,\beta_1,\beta_2)\in\F_{2^k}\times\F_{2^k}^*\times\F_{2^k}^*\times\{0\}$ will lead to analogue proofs with the same upper bounds and we omit them.

            \item If both \hyperref[item_a]{Conditions~A} and \hyperref[item_b]{B} hold,\label{case_2}
            it can be guaranteed that Equation~\eqref{eq:coefficient} has $8$ solutions $\{0,\alpha_2,\beta_2,\alpha_2+\beta_2,y_0,y_0+\alpha_2,y_0+\beta_2,y_0+\alpha_2+\beta_2\}$, where $y_0\notin\{0,\alpha_2,\beta_2,\alpha_2+\beta_2\}$, according to Lemma \ref{lemma:num_sol_second_dev}.
            Then we have
            \begin{align*}
            &W_{D_{\beta}D_{\alpha}f}(\mu,\nu)\nonumber\\
                =&2^k\left[1+(-1)^{\TRACE\left((\alpha_1+\beta_1)\mu+ (\alpha_2+\beta_2)\nu\right)}\right]\cdot
                \left[1+(-1)^{\TRACE\left(\alpha_1\mu+\alpha_2\nu\right)}\right]\nonumber\\
                &\cdot
                \left[(-1)^{\TRACE\left(\frac{\lambda\alpha_1}{\alpha_2}+\frac{\lambda\beta_1}{\beta_2}+\frac{\lambda(\alpha_1+\beta_1)}{\alpha_2+\beta_2}\right)}+(-1)^{\TRACE\left(\frac{\lambda\alpha_1}{y_0+\alpha_2}+\frac{\lambda\beta_1}{y_0+\beta_2}+\frac{\lambda(\alpha_1+\beta_1)}{y_0+\alpha_2+\beta_2}+ y_0\nu\right)}\right]\nonumber\\
                =&(-1)^{c_0}2^k\cdot\left[1+(-1)^{\TRACE\left((\alpha_1+\beta_1)\mu+ (\alpha_2+\beta_2)\nu\right)}\right]\cdot
                \left[1+(-1)^{\TRACE\left(\alpha_1\mu+\alpha_2\nu\right)}\right]\cdot\left[1+(-1)^{c_0+c_1}\right]\nonumber\\
                =&\begin{cases}
                    2^{k+3}\cdot(-1)^{c_0},&\text{if }\TRACE\left(\alpha_1\mu+\alpha_2\nu\right)=\TRACE\left(\beta_1\mu+\beta_2\nu\right)=c_0+c_1=0,\\
                    0,&\text{otherwise},
                \end{cases}
            \end{align*}
            where
            $c_0=\TRACE\left(\frac{\lambda\alpha_1}{\alpha_2}+\frac{\lambda\beta_1}{\beta_2}+\frac{\lambda(\alpha_1+\beta_1)}{\alpha_2+\beta_2}\right)$
            and
            $c_1= \TRACE\left(\frac{\lambda\alpha_1}{y_0+\alpha_2}+\frac{\lambda\beta_1}{y_0+\beta_2}+\frac{\lambda(\alpha_1+\beta_1)}{y_0+\alpha_2+\beta_2}+\nu y_0\right)$.
            By Lemma \ref{lemma:N_ijk_trace},
            for any points $\alpha=(\alpha_1,\alpha_2),\beta=(\beta_1,\beta_2)\in\F_{2^k}\times\F_{2^k}^*$ such that
            $\alpha_2\ne\beta_2$ and satisfying both \hyperref[item_a]{{Conditions~A}} and \hyperref[item_b]{{B}},
            the number of solutions $\nu\in\F_{2^k}$ for the system
            \begin{empheq}[left=\empheqbiglbrace]{align*}
                &\TRACE\left(\alpha_2\nu + \alpha_1\mu\right)=0\\
                &\TRACE\left(\beta_2 \nu + \beta_1\mu \right)=0\\
                &\TRACE\left(y_0\nu +\frac{\lambda\alpha_1}{\alpha_2}+\frac{\lambda\beta_1}{\beta_2}+\frac{\lambda(\alpha_1+\beta_1)}{\alpha_2+\beta_2}+\frac{\lambda\alpha_1}{y_0+\alpha_2}+\frac{\lambda\beta_1}{y_0+\beta_2}+\frac{\lambda(\alpha_1+\beta_1)}{y_0+\alpha_2+\beta_2} \right)=0,
            \end{empheq}
            is $2^{k-3}\ge 1$.
            This implies there exists $(\mu,\nu)\in\F_{2^k}\times\F_{2^k}$ such that $W_{D_{\beta}D_{\alpha}f}(\mu,\nu)=2^{k+3}\cdot(-1)^{c_0}$.
            So in this case, we have
            \[\max_{\mu,\nu}|W_{D_{\beta}D_{\alpha}f}(\mu,\nu)|=2^{k+3}.\]
            \item If either \hyperref[item_a]{{Condition A}} or \hyperref[item_b]{{Condition B}} holds,\label{case_3}
            it can be seen that Equation \eqref{eq:coefficient} has only $4$ solutions.
            Assume \hyperref[item_a]{{Condition A}} is satisfied, then $\{0,\alpha_2,\beta_2,\alpha_2+\beta_2\}$ are the only $4$ solutions of Equation \eqref{eq:coefficient}.
            So we have
            \begin{align*}\label{eq:simpleforms_4}
                &W_{D_{\beta}D_{\alpha}f}(\mu,\nu)\nonumber\\
                =&2^k\left[1+(-1)^{\TRACE\left((\alpha_1+\beta_1)\mu+ (\alpha_2+\beta_2)\nu\right)}\right]\nonumber\\
                &\cdot
                \left[(-1)^{\TRACE\left(\frac{\lambda\alpha_1}{y+\alpha_2}+\frac{\lambda\beta_1}{y+\beta_2}+\frac{\lambda(\alpha_1+\beta_1)}{y+\alpha_2+\beta_2}+ y\nu\right)}+
                (-1)^{\TRACE\left(\frac{\lambda\alpha_1}{y}+\frac{\lambda\beta_1}{y+\alpha_2+\beta_2}+\frac{\lambda(\alpha_1+\beta_1)}{y+\beta_2}+ (y+\alpha_2)\nu\right)}\right]\nonumber\\
                =&2^k\left[1+(-1)^{\TRACE\left((\alpha_1+\beta_1)\mu+ (\alpha_2+\beta_2)\nu\right)}\right]\nonumber\\
                &\cdot
                (-1)^{\TRACE\left(\frac{\lambda\alpha_1}{y+\alpha_2}+\frac{\lambda\beta_1}{y+\beta_2}+\frac{\lambda(\alpha_1+\beta_1)}{y+\alpha_2+\beta_2}+ y\nu\right)}\cdot
                \left[1+(-1)^{\TRACE\left(\frac{\lambda\alpha_1}{y}+\frac{\lambda\alpha_1}{y+\alpha_2}+\frac{\lambda\alpha_1}{y+\beta_2}+\frac{\lambda\alpha_1}{y+\alpha_2+\beta_2}+\nu\alpha_2\right)}\right]\nonumber\\
                =&2^k\left[1+(-1)^{\TRACE\left((\alpha_1+\beta_1)\mu+ (\alpha_2+\beta_2)\nu\right)}\right]\cdot
                \left[1+(-1)^{\TRACE\left(\alpha_1\mu+\alpha_2\nu\right)}\right]\cdot
                (-1)^{\TRACE\left(\frac{\lambda\alpha_1}{y+\alpha_2}+\frac{\lambda\beta_1}{y+\beta_2}+\frac{\lambda(\alpha_1+\beta_1)}{y+\alpha_2+\beta_2}+ y\nu\right)}\nonumber\\
                =&\begin{cases}
                    2^{k+2}\cdot(-1)^{\TRACE\left(\frac{\lambda\alpha_1}{y+\alpha_2}+\frac{\lambda\beta_1}{y+\beta_2}+\frac{\lambda(\alpha_1+\beta_1)}{y+\alpha_2+\beta_2}+ y\nu\right)},&\text{if}~\TRACE\left(\alpha_2\nu+\alpha_1\mu\right)=0 ~
                    \text{and}~\TRACE\left(\beta_2\nu+\beta_1 \mu\right)=0, \\
                    0,~&\text{otherwise}.
                \end{cases}
            \end{align*}
            According to Lemma \ref{lemma:N_ij_trace}, for any $\alpha=(\alpha_1,\alpha_2),\beta=(\beta_1,\beta_2)\in\F_{2^k}\times\F_{2^k}^*$ such that $\alpha_2\ne\beta_2$, the number of solutions $\nu\in\F_{2^k}$ for the system
            \begin{equation*}\label{eq:max_foursolution_condition}
                \left\{
                \begin{alignedat}{2}
                    \TRACE\left(\alpha_2\nu+\alpha_1\mu\right)&=0\\
                    \TRACE\left(\beta_2\nu +\beta_1 \mu\right)&=0
                \end{alignedat}
                \right.
            \end{equation*}
            is $2^{k-2}\ge 1$.
            This implies there exists $(\mu,\nu)\in\F_{2^k}\times\F_{2^k}$ such that $|W_{D_{\beta}D_{\alpha}f}(\mu,\nu)|=2^{k+2}$.
            Thus, in the case where \hyperref[item_a]{{Condition A}} holds, we have
            \[\max_{\mu,\nu}|W_{D_{\beta}D_{\alpha}f}(\mu,\nu)|=2^{k+2}.\]
            Assume \hyperref[item_b]{{Condition B}} holds,
            it can be seen that $\{y_0,y_0+\alpha_2,y_0+\beta_2,y_0+\alpha_2+\beta_2\}$
            are the only $4$ solutions of
            Equation \eqref{eq:coefficient}, where $y_0\notin\{0, \alpha_2, \beta_2, \alpha_2+\beta_2\}$.
            The rest of proof  is similar to the case where only Condition \hyperref[item_a]{{Condition A}} holds, so we omit it and
            straightforwardly present the result $\max_{\mu,\nu}|W_{D_{\beta}D_{\alpha}f}(\mu,\nu)|=2^{k+2}$.
            Therefore, in the case where Equation \eqref{eq:coefficient} has only $4$ solutions, we have
            \[\max_{\mu,\nu}|W_{D_{\beta}D_{\alpha}f}(\mu,\nu)|=2^{k+2}.\]
        \end{enumerate}
        This completes the proof.
    \end{proof}

\subsection{A lower bound on the third-order nonlinearity of the simplest $\mathcal{PS}_{ap}$ bent functions}

In this subsection,  we can derive a lower bound on the third-order nonlinearity of the simplest $\mathcal{PS}_{ap}$ bent functions.
Applying twice Lemma \ref{thm:High_order_nl_bound1}, that is, taking
    \[nl_{r-1}(D_{\alpha}f) \ge 2^{n-1}-\frac{1}{2}\sqrt{2^{2n}-2\sum_{\beta\in\F_2^n}nl_{r-2}(D_{\beta}D_{\alpha}f)},\]
    into the right-hand side of the following equation
    \[nl_r(f) \ge 2^{n-1}-\frac{1}{2}\sqrt{2^{2n}-2\sum_{\alpha\in\F_2^n}nl_{r-1}(D_{\alpha}f)}.\]
    Then we can obtain a relation between the $r$-th order nonlinearity of $f$ and the $(r-2)$-th order nonlinearities of all second-order derivatives of $f$:
    \begin{equation}\label{eq:nl3_nlDaDbf}
        nl_r(f)\ge 2^{n-1}-\frac{1}{2}\sqrt{\sum_{\alpha\in\F_{2^n}}\sqrt{2^{2n}-2\sum_{\beta\in\F_{2^n}} nl_{r-2}(D_{\beta}D_{\alpha}f)}}.
    \end{equation}
    Therefore, combining Theorem \ref{thm:nl_DaDbf} and Equation \eqref{eq:nl3_nlDaDbf} we can obtain the following theorem.
    \begin{theorem}\label{th:our_lower_bound}
        Let $k\ge 3$ be an integer and $n=2k$. For the third-order nonlinearity of the simplest $\mathcal{PS}_{ap}$ bent function $f(x,y)=\TRACE(\frac{\lambda x}{y})$ with $x,y\in\F_{2^k}$ and $\lambda\in\F_{2^k}^*$, we have:
        \[nl_3(f)\ge 2^{n-1}-\frac{1}{2}\sqrt{A},\]
        where
        \begin{align*}
            A=2^n+&(2^{\frac{n}{2}}-1)\sqrt{(2^{\frac{3n}{2}+1}-2^{n+1})\left\lfloor 2^{\frac{n}{4}+1}\right\rfloor+5\cdot 2^{\frac{3n}{2}}-2^{n+2}}\\
            +&(2^n-2^{\frac{n}{2}})\sqrt{2^{\frac{3n}{2}+2}-15\cdot 2^n-2^{\frac{n}{2}+2}+(2^{n+2}-2^{\frac{n}{2}+1})\left\lfloor 2^{\frac{n}{4}+1}\right\rfloor+2^{n+2}L},
        \end{align*}
       in which $L$ is given in Lemma \ref{L:SumInv00}.
    \end{theorem}
    \begin{proof}
        We have
        \begin{align*}
            nl_3(f)&\ge 2^{n-1}-\frac{1}{2}\sqrt{\sum_{\alpha\in\F_{2^n}}\sqrt{2^{2n}-2\sum_{\beta\in\F_{2^n}} nl(D_{\beta}D_{\alpha}f)}}\\
            &=2^{n-1}-\frac{1}{2}\sqrt{\left( \sum_{\alpha=(0,0)}+\sum_{\alpha=(\alpha_1,0)\in\F_{2^k}^*\times\{0\}}+\sum_{\substack{\alpha=(\alpha_1,\alpha_2)\in\F_{2^k}\times\F_{2^k}^*}} \right)\sqrt{2^{2n}-2\sum_{\beta\in\F_{2^n}} nl(D_{\beta}D_{\alpha}f)}}\\
            &\ge 2^{n-1}-\frac{1}{2}\left[2^n+(2^{\frac{n}{2}}-1)\sqrt{2^{2n}-2(2^n-2^{\frac{n}{2}})(2^{n-1}-2^{\frac{n}{2}}\left\lfloor 2^{\frac{n}{4}+1}\right\rfloor-2^{\frac{n}{2}+1})}\right.\\
            &\qquad\qquad\quad\left.+(2^n-2^{\frac{n}{2}})\sqrt{2^{2n}-2\left( (2^{n-1}-2^{\frac{n}{2}+1})(2^n-1)-(2^{n+1}-2^{\frac{n}{2}})\left\lfloor 2^{\frac{n}{4}+1}\right\rfloor+2^{n+3}-2^{n+1}L \right)}\right]^{\frac{1}{2}}\\
            &=2^{n-1}-\frac{1}{2}\sqrt{A},
        \end{align*}
        where the last but one equation comes from Theorem \ref{thm:nl_DaDbf}.
        This completes the proof.
    \end{proof}
    \begin{corollary}
        Let $n\ge 6$ be an arbitrary even integer. The lower bound on the third-order nonlinearity of the simplest $\mathcal{PS}_{ap}$ bent functions in $n$ variables is
        approximately equal to $2^{n-1}-2^{\frac{7n}{8}-\frac{1}{2}}$.
    \end{corollary}
    \begin{corollary}
        Let $n=2k\ge 6$ be an arbitrary even integer and $4\le r\le n-2$ be an arbitrary integer.
        Let $f$ be an $n$-variable  simplest $\mathcal{PS}_{ap}$ bent function. Then we have
        \[nl_r(f)\ge 2^{n-1}-\frac{1}{2}\sqrt{2^{2n+1}-2^{\frac{n}{2}+2}(2^{\frac{n}{2}}-1)nl_{r-1}(f)+2^{n+1}l_{r-1}},\]
        where $l_c$ is defined by $l_1=2^{\frac{n}{4}}$ and $l_c=\sqrt{(2^{\frac{n}{2}}-1)(l_{c-1}+1)+2^{\frac{n}{2}-2 }}$ for $c\geq 2$.
    \end{corollary}
    \begin{proof}
        In \cite{Carlet2008lowbound_NL_profile}, Carlet proved that
        \[\forall\alpha=(\alpha_1,\alpha_2)\in\F_{2^k}\times\F_{2^k}^*, nl_r(D_{\alpha}f)\ge 2nl_r(f)-2^{n-1}-2^{\frac{n}{2}},\]
        and
        \[\forall\alpha=(\alpha_1,0)\in\F_{2^k}\times\{0\}, nl_r(D_{\alpha}f)\ge 2^{n-1}-2^{\frac{n}{2}}(l_r+1),\]
        where $l_c$ is defined by $l_1=2^{\frac{n}{4}}$ and $l_c=\sqrt{(2^{\frac{n}{2}}-1)(l_{c-1}+1)+2^{\frac{n}{2}-2 }}$ for $c\ge 2$.
        Therefore, by Lemma \ref{thm:High_order_nl_bound1}, we have
        \begin{align*}
            nl_r(f)
            \ge& 2^{n-1}-\frac{1}{2}\sqrt{2^{2n}-2\sum_{\alpha\in\F_{2^n}} nl_{r-1}(D_{\alpha}f)}\\
            =& 2^{n-1}-\frac{1}{2}\sqrt{2^{2n}-2\sum_{\alpha=(\alpha_1,\alpha_2)\in\F_{2^k}\times\F_{2^k}}nl_{r-1}(D_{\alpha}f)}\\
            =& 2^{n-1}-\frac{1}{2}\sqrt{2^{2n}-2\sum_{\alpha=(\alpha_1,\alpha_2)\in\F_{2^k}\times\F_{2^k}^*}nl_{r-1}(D_{\alpha}f)-2\sum_{\alpha=(\alpha_1,0)\in\F_{2^k}\times\{0\}}nl_{r-1}(D_{\alpha}f)} \\
            \ge& 2^{n-1}-\frac{1}{2}\sqrt{2^{2n}-2\sum_{\alpha=(\alpha_1,\alpha_2)\in\F_{2^k}\times\F_{2^k}^*}\left(2nl_{r-1}(f)-2^{n-1}-2^{\frac{n}{2}}\right)-2\sum_{\alpha=(\alpha_1,0)\in\F_{2^k}\times\{0\}}\left(2^{n-1}-2^{\frac{n}{2}}(l_{r-1}+1)\right)}\\
            =& 2^{n-1}-\frac{1}{2}\sqrt{2^{2n}-2^{k+1}(2^k-1)\left(2nl_{r-1}(f)-2^{n-1}-2^{\frac{n}{2}}\right)-2^{k+1}\left(2^{n-1}-2^{\frac{n}{2}}(l_{r-1}+1)\right)} \\
            =& 2^{n-1}-\frac{1}{2}\sqrt{2^{2n+1}-2^{\frac{n}{2}+2}(2^{\frac{n}{2}}-1)nl_{r-1}(f)+2^{n+1}l_{r-1}}.
        \end{align*}
        This completes the proof.
    \end{proof}

    \subsection{Comparison}
    In this subsection, we will compare our lower bound on the third-order nonlinearity of the simplest $\mathcal{PS}_{ap}$ bent functions 
    with the two lower bounds given in \cite{Carlet2011NL_Profile_Dillon,TangCT2013NL_2bent}.
    
    
    It should be noted that these three lower bounds (the lower bound present in this paper and 
    the two ones given in \cite{Carlet2011NL_Profile_Dillon,TangCT2013NL_2bent} respectively) on the third-order nonlinearity of the simplest $\mathcal{PS}_{ap}$ bent functions
    are all derived from Lemma \ref{thm:High_order_nl_bound1}.
    This means that these three lower bounds are heavily dependent on the nonlinearities of all second-order derivatives.
    It can be easily seen that our lower bound on the nonlinearity of every second-order derivative is greater than or equal to
    that bounds given in \cite{Carlet2011NL_Profile_Dillon,TangCT2013NL_2bent}.
    Particularly,  it follows from \hyperref[case_2]{Case 2} and \hyperref[case_3]{Case 3} in the proof of Theorem~\ref{thm:nl_DaDbf}
    that 
    $nl(D_{\beta}D_{\alpha}f)=2^{n-1}-2^{\frac{n}{2}+2}$
    and $nl(D_{\beta}D_{\alpha}f)=2^{n-1}-2^{\frac{n}{2}+1}$ respectively,
    whereas such values  given in \cite{Carlet2011NL_Profile_Dillon} and \cite{TangCT2013NL_2bent}
    are both $2^{n-1}-\sqrt{2^n+(2^{n+2}+2^{\frac{3n}{4}+1}+2^{\frac{n}{2}+1})(2^{\frac{n}{2}}-1)}-2^{\frac{n}{2}}$.
    Then by Lemma \ref{thm:High_order_nl_bound1} we can easily see that our lower bound is 
    much better than that the two ones given in \cite{Carlet2011NL_Profile_Dillon,TangCT2013NL_2bent}.

    % We compare in Table \ref{table:MyTableLabel} our lower bound on the third-order nonlinearity of the simplest $\mathcal{PS}_{ap}$ bent functions with the two ones in \cite{Carlet2011NL_Profile_Dillon,TangCT2013NL_2bent} for even $n$ ranging from $6$ to $26$.

    Table \ref{table:MyTableLabel} presents a comparison of our lower bound on the third-order nonlinearity of the simplest $\mathcal{PS}_{ap}$ bent functions with the two ones in \cite{Carlet2011NL_Profile_Dillon,TangCT2013NL_2bent}, for even $n$ ranging from $6$ to $26$.
    It can be seen from Table \ref{table:MyTableLabel} that our lower bound on the third-order nonlinearity is always much better than the other two ones.
 \newcommand{\rb}[1]{\raisebox{1.5ex}[0pt]{#1}}
    \begin{table}
        \centering
        \caption{Comparison of lower bounds on the third-order nonlinearity of the simplest $\mathcal{PS}_{ap}$ bent functions}
        \begin{threeparttable}
            \begin{tabular}{|c|c|c|c|c|}
                \hline
                        & Tang-Carlet-Tang bound      & Carlet bound                            & Our bound& \\
                \rb{$n$}& in \cite{TangCT2013NL_2bent}& in \cite{Carlet2011NL_Profile_Dillon} & in Theorem \ref{th:our_lower_bound}     &\rb{Difference\tnote{1}}   \\
                \hline
                $6  $ &  $ -           $       & $ -        $     & $ 1        $     & $  -       $ \\  \hline
                $8  $ &  $ -           $       & $ -        $     & $ 26       $     & $  -       $ \\  \hline
                $10 $ &  $ 111         $       & $ 63       $     & $ 186      $     & $  75      $ \\  \hline
                $12 $ &  $ 533         $       & $ 553      $     & $ 932      $     & $  379     $ \\  \hline
                $14 $ &  $ 3877        $       & $ 3205     $     & $ 4372     $     & $  495     $ \\  \hline
                $16 $ &  $ 15985       $       & $ 16103    $     & $ 19986    $     & $  3883    $ \\  \hline
                $18 $ &  $ 83435       $       & $ 75292    $     & $ 89028    $     & $  5593    $ \\  \hline
                $20 $ &  $ 336634      $       & $ 337330   $     & $ 384926   $     & $  47596   $ \\  \hline
                $22 $ &  $ 1565563     $       & $ 1469893  $     & $ 1628771  $     & $  63208   $ \\  \hline
                $24 $ &  $ 6278535     $       & $ 6282551  $     & $ 6807315  $     & $  524764  $ \\  \hline
                $26 $ &  $ 27581520    $       & $ 26479472 $     & $ 28238083 $     & $  656563  $ \\  \hline
            \end{tabular}
            \begin{tablenotes}
                \footnotesize
                \item[$\ast$] All concrete values of lower bounds in this table have been rounded up to the nearest integer.
                \item[1] The values in the table represent  the difference between our lower bound and the greater value obtained from Tang-Carlet-Tang's and Carlet's bounds.
            \end{tablenotes}
        \end{threeparttable}
        \label{table:MyTableLabel}
    \end{table}


\section{Conclusion}
    In this paper, we derived a lower bound on the third-order nonlinearity of the simplest $\mathcal{PS}_{ap}$ bent functions.
    Our lower bound improved the previous two lower bounds in earlier works \cite{Carlet2011NL_Profile_Dillon,TangCT2013NL_2bent}.
    We hope that the result obtained in this paper can give  more insight into the simplest $\mathcal{PS}_{ap}$ bent functions.

\bibliographystyle{model1b-num-names}
\bibliography{mybib}

\end{document}

% Boolean functions used in cryptosystems should have high higher-order nonlinearity to resist several known attacks, such as algebraic attack and low-degree approximation. The higher-order nonlinearity also plays an important role in coding theory and theoretical computer science. It is well-known that bent functions have the highest nonlinearity in even number of variables and thus they possess the best resistance against fast correlation attack and best affine approximation. However, there is limited knowledge regarding the higher-order nonlinearity of bent functions because computing the higher-order nonlinearity, or even providing tight lower bounds, is a hard task. In 1974, Dillon proposed two well-known classes of bent functions based on partial spread (in brief, $\mathcal{PS}$), called $\mathcal{PS}^-$ and $\mathcal{PS}^+$, respectively. He also exhibited a subclass of $\mathcal{PS}^-$, known as partial spread affine plane ($\mathcal{PS}_{ap}$ for short). In this paper, we provide a lower bound on the third-order nonlinearity of the simplest $\mathcal{PS}_{ap}$ bent functions in $n$ variables, where $n\ge 6$ is even, by calculating nonlinearities of all second-order derivatives of this kind of bent functions. Compared to the two known lower bounds on the third-order nonlinearity given by Carlet and Tang et al. respectively, our lower bound is much better than these two ones.

k:=12;
Z:=Integers();
F<w>:=GF(2,k);
Fstar:=[w^i:i in [0..2^k-2]];
max_element:=0;
for b in [w^i: i in [16..2^k-2]] do 
    for v in Fstar do
        tmp:= Abs(&+[1-2*Z!Trace((x+b)^(2^k-2)+x^(2^k-2)+v*x):x in F]);
        if max_element le tmp then 
            max_element:= tmp;
            print "now k =",k," ,beta=",b, " and v=",v;
            print "temp_max_element is", max_element;
        end if;
    end for;
end for;
max_element;

