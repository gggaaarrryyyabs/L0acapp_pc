% !TeX TS-program = xelatex
% This is SJTUBeamermin v1.0
% For more detailed desciption, see
% https://github.com/LogCreative/SJTUBeamermin/blob/main/doc/sjtubeamermintheme.pdf
%
\documentclass[
    % draft,                             % 草稿模式
    aspectratio=169,                   % 使用 16:9 比例
]{beamer}
\mode<presentation>
\usetheme[
    navigation=subsections,            % 使用子章节进度显示
     lang=en,                           % 使用英文
    % cjk=true,                          % 使用CJK而不是ctex
    color=blue,                         % 使用红色主题
    % pattern=all,                        % 使用全图案装饰
    % gbt=bibtex,                        % 使用 gbt (使用 bibtex 编译)
]{sjtubeamermin}
% \usecolortheme[]{beaver}                 % 使用其他颜色主题
% \addbibresource{ref.bib}               % gbt!=bibtex

\usepackage{amsmath,amsthm}
\usepackage{bm}
\usepackage{tikz-cd}
\usepackage{tikz}
\usetikzlibrary{trees,circuits.logic.US,circuits.ee.IEC}
\usepackage{empheq}
\usepackage{tabularx}
\usepackage{pifont}
\usepackage{float}
\usepackage{multicol}
\usepackage[boxruled]{algorithm2e}
\usepackage{hyperref}
\usepackage{amsfonts}
\usepackage{tasks}
\usepackage{fontspec}
\usepackage{mdframed} % Add easy frames to paragraphs
\usepackage{color,soul}
\usepackage{xparse} % Add support for \NewDocumentEnvironment
\usepackage{biblatex}
\addbibresource{refs.bib}


\usepackage[customcolors]{hf-tikz}
\tikzset{style green/.style={
    set fill color=green!50!lime!60,
    set border color=white,
  },
  style cyan/.style={
    set fill color=cyan!90!blue!60,
  },
  style orange/.style={
    set fill color=orange!80!red!60,
  },
  hor/.style={
    above left offset={-0.31,0.31},
    below right offset={0.31,-0.2},
    #1
  },
  ver/.style={
    above left offset={-0.1,0.3},
    below right offset={0.15,-0.15},
    #1
  }
}

\renewcommand{\Bbb}{\mathbb}
\newcommand{\Z}{\mathbb{Z}}
\newcommand{\GR}{\mathbb{GR}}
\newcommand{\F}{\mathbb{F}}
\newcommand{\R}{\mathbb{R}}
\newcommand{\Fns}{\mathbb{F}_{2^n}^*}
\newcommand{\Fn}{\mathbb{F}_{2^n}}
\newcommand{\Fks}{\mathbb{F}_{2^k}^*}
\newcommand{\Fk}{\mathbb{F}_{2^k}}
\newcommand{\df}{\delta_F}
\newcommand{\tr}{\operatorname{tr}_1^k}
\newcommand{\gtr}{\operatorname{T}}
\newcommand{\Bn}{\mathcal{B}_n}
% The symbol of sharing
\newcommand{\Sh}{\operatorname{Sh}}


\newcommand{\dis}{\operatorname{\widetilde{d}}}
\newcommand{\wt}{\operatorname{\widetilde{wt}}}
\newcommand{\nnl}{\operatorname{d}_{\mathcal{A}}}
\newcommand{\disl}{\operatorname{d}_{\mathcal{L}}}

% The symbol of Teichmuller sets
\newcommand{\teich}{\textit{Teichm$\ddot{u}$ller sets}}

% \newtheorem*{definition}{Definition}
\newtheorem{thm}{Theorem}
\newtheorem{lem}[thm]{Lemma}
\newtheorem{proposition}{Proposition}
\definecolor{graylight}{cmyk}{.30,0,0,.67} % define color using xcolor syntax
\setbeamertemplate{itemize items}[default]

% modify the footnote symbol 
% \makeatother
% \renewcommand{\thefootnote}{\ifcase\value{footnote}\or\dagger\or(\#\#)\or(\#\#\#)\or(\#\#\#\#)\or(\#\#\#\#\#)\fi}
% \makeatletter



\newmdenv[ % Define mdframe settings and store as leftrule
  linecolor=red,
  topline=false,
  bottomline=false,
  rightline=false,
  skipabove=\topsep,
  skipbelow=\topsep
]{leftrule}

% \NewDocumentEnvironment{example}{O{\textbf{Example:}}} % Define example environment
% {\begin{leftrule}\noindent\textcolor{blue}{#1}\par}
% {\end{leftrule}}
\setbeamertemplate{footline}[frame number]

\NewDocumentEnvironment{question}{O{\textbf{Something:}}} % Define something environment
{\begin{leftrule}\noindent\textcolor{graylight}{#1}\par}
{\end{leftrule}}

\NewDocumentEnvironment{remark}{O{\textbf{Remark:}}} % Define remark environment
{\begin{leftrule}\noindent\textcolor{blue}{#1}\par}
{\end{leftrule}}

\begin{document}
    \institute[School of Electronic Information and Electrical Engineering]{电子信息与电气工程学院}   % 组织
    % \logo{
    %     \includegraphics{support/cnlogored.pdf}  % 重定义 logo
    % }
    \titlegraphic{                         % 标题图像
        \begin{stampbox}[white]
            \includegraphics[width=0.3\textwidth]{support/head.png}
        \end{stampbox}
    }
    \title{Estimate the Nonlinearity of Boolean\\ Functions on Probabilistic Affine Tests}  % 标题
 %   \subtitle{SJTUBeamer \fbox{\textsc{min}} Template}         % 副标题
    \author{Zhaole Li}                  % 作者
    \date{\textcolor{red}{2023/5/24}}                          % 日期
    \maketitle                             % 创建标题页


\AtBeginSection[]{
   \begin{frame}
       % \tableofcontents[currentsection]           % 传统节目录             
       \sectionpage                   % 节页
   \end{frame}
}

% 使用小节目录
%\AtBeginSubsection[]{                  % 在每小节开始
 %   \begin{frame}
        % \tableofcontents[currentsection,currentsubsection]             % 传统小节目录             
  %      \subsectionpage                % 小节页
 %   \end{frame}
%}

\section{Introduction}
%\subsection{第 1 小节}
    \begin{frame}
        \frametitle{The Nonlinearity of a Boolean Function}
        \begin{itemize}
            \item Boolean functions are central objects for the design and the security of symmetric cryptosystems.
            \item It is important that Boolean functions used in many symmetric cryptosystems are not affine, and not even close (with respect to Hamming distance) to being affine.
            \item The nonlinearity of an $n$-variable Boolean function $f$, denoted $\nnl(f)$, defined as the minimum Hamming distance to
            any affine function in $n$ variables, is therefore an important cryptographic property. 
            \item In theory, the nonlinearity of an $n$-variable Boolean function $f$ can be efficiently computed by the Walsh transform in time $O(n\log n)$.
            \item In practice, when $n$ is large (e.g., $n\ge 70$), computing the Walsh transform is infeasible.
        \end{itemize}
        
    \end{frame}
    
    \begin{frame}
        \frametitle{The Estimation of the Nonlinearity of a Boolean Function}
        \begin{itemize}
            \item Consider a probabilistic test which has a outcome based on the values of $f$ at some fixed number of points in $\F_2^n$ and denote by $P(f)$ the probability of failing the test.
            \item Assume $P(f)$ is related with the $\nnl(f)$ and can be bounded by some functions in $\nnl(f)$, namely $\operatorname{LowerBound}(\nnl(f))\le P(f)\le \operatorname{UpperBound}(\nnl(f))$.
            \item We can estimate $\nnl(f)$ by 
            \[\nnl(f)\in\left[ \min(\operatorname{UpperBound}^{-1}(P(f))),\max(\operatorname{LowerBound}^{-1}(P(f))) \right],\]
            where $F^{-1}(x)$ means the preimage of $x$ under the function $F$.
        \end{itemize}
    \end{frame}
        
    \begin{frame}
        \frametitle{Preliminaries}
        
        \begin{itemize}
            \item Boolean functions in $n$ variables are functions $f:\F_2^n\rightarrow\F_2$, where $\F_2$ is the binary finite field, and $\F_2^n$ is the $n$ dimensional vector space over $\F_2$.
            \item Define the (normalized) Hamming weight and Hamming distance for any two vectors $a=(a_1,a_2,...,a_n)$ and $b=(b_1,b_2,...,b_n)\in\F_2^n$ as:
            \begin{align*}
                &\wt(a) =\frac{1}{n}\# \left\{ 1\le i\le n\mid a_i\ne 0 \right\},\\
                &\dis(a,b)=\frac{1}{n}\#\left\{ 1\le i\le n\mid a_i\ne b_i \right\}.
            \end{align*}
            \item The truth table (TT) is the most well-known way to uniquely represent Boolean functions, that is, for an $n$-variable Boolean function $f$, we have 
            \[\operatorname{TT}_f=\left[ f(0,0,\dots,0),f(1,0,\dots,0),\dots,f(0,1,\dots,1),f(1,1,\dots,1) \right].\]
            
        \end{itemize} 
    \end{frame}

    \begin{frame}
        \frametitle{Preliminaries (cont.)}
        \begin{itemize}
            \item Any Boolean function $f$ in variables $x_1,x_2,\dots,x_n$ can also be uniquely represented in its ANF (Algebraic Normal Form):
            \[f(x_1,x_2,\dots,x_n) = \bigoplus_{u\in\F_2^n}a_ux^u,\]
            where $x=(x_1,x_2,\dots,x_n)\in\F_2^n$, $u=(u_1,u_2,\dots,u_n)\in\F_2^n$, $a_u\in\F_2$.
            \item The algebraic degree of $f$ is denoted by $\operatorname{deg}(f)=\max\left\{ n\wt(u):u\in\F_2^n, a_u\ne 0 \right\}$.
            \item Functions of algebraic degree at most one are called affine; affine functions with zero constant are called linear. 
            \item Denote by $\mathcal{A}_n$ (resp. $\mathcal{L}_n$) the set of affine (resp. linear) functions in $n$ variables over $\F_2$. We will omit the subscript $n$ if the content is clear.
        \end{itemize}
        
    \end{frame}
    
    \begin{frame}
        \frametitle{Preliminaries (cont.)}
        \begin{itemize}
            \item The (normalized) Hamming weight of a Boolean function $f$, denoted by $\wt(f)$, is $\wt(\operatorname{TT}_f)$; The (normalized) Hamming distance between two Boolean function $f$ and $g$, denoted by $\dis(f,g)$, is $\dis(\operatorname{TT}_f,\operatorname{TT}_g)$. 
            \item We denote by $\disl(f)$ the minimum distance between $f$ and linear functions in $\mathcal{L}$.
            \item  The minimum distance between $f$ and affine functions in $\mathcal{A}$ is called the (normalized) nonlinearity of $f$, and we denote it $\nnl(f)$.
            \item From definition we have $\nnl(f)=\min\left\{ \disl(f),\disl(f+1) \right\}$.
        \end{itemize}
    \end{frame}

    \begin{frame}
        \frametitle{Walsh Transform}
        
        \begin{definition}[Walsh Transform]
            The Walsh transform of a function $f:\F_2^n\rightarrow\F_2$ is the function $\widetilde{W_f}:\F_2^n\rightarrow\R$ defined as 
            \[\widetilde{W_f}(u)=\frac{1}{2^n}\sum_{x\in\F_2^n}(-1)^{f(x)+u\cdot x},\]
            where the operator ``$\cdot$'' is the usual inner product and defined as $u\cdot x=\sum_{i=1}^{n}u_ix_i$. 
        \end{definition}
        There exists the so-called Parseval's identity 
        \[\sum_{u\in\F_2^n}\left( \widetilde{W_f}(u) \right)^2=1,\]
        which holds for any $n$-variable Boolean function $f$.
    
    \end{frame}

    \begin{frame}
        \frametitle{The Walsh Transform and Hamming Distance}
        It is easy to see that the Walsh transform of a Boolean function $f$ expresses its distance to $\mathcal{L}$, and consequently the distance between $f$ to $\mathcal{A}$.

        Denoting $l_{a}(u)=a\cdot u$, we have 
        \begin{align*}
            \dis(f,l_a)  &=\frac{1}{2}\left( 1-\widetilde{W_f}(a) \right),\\
            \dis(f,l_a+1)&=\frac{1}{2}\left( 1+\widetilde{W_f}(a) \right),\\
            \disl(f)     &=\frac{1}{2}\left( 1-\max_{v\in\F_2^n}\widetilde{W_f}(a) \right),\\
            \nnl(f)      &=\frac{1}{2}\left( 1-\max_{v\in\F_2^n}\left|\widetilde{W_f}(a)\right| \right).
        \end{align*}
    \end{frame}
    \begin{frame}
        \frametitle{BLR Linearity Testing}
        \begin{itemize}
            \item The linearity test most widely used, which is often called the BLR (Blum-Luby-Rubinfeld) Linearity Testing, is based on the definition of a linear function, \emph{i.e.}, $f(u+v)=f(u)+f(v)$.
            \begin{algorithm}[H]
                \caption{BLR Linearity Testing}
                 Sample $u,v\sim \mathbb{U}_{\F_2^n}$\; 
                 Evaluate $x = f(u), y=f(v)$ and $z = f(u + v)$\; 
                 Return ($x+y==z$)\; 
            \end{algorithm}
            \item If $f$ passes this test for many pairs $(u, v)$, then $f$ is probably linear.
            \item If $f$ fails the test for at least one pair $(u, v)$, then $f$ is certainly not linear. 
        \end{itemize}
    \end{frame}

    \begin{frame}
        \frametitle{BLR Linearity Testing (cont.)}
        \begin{itemize}
            \item Sometimes we are not so much interested in whether the function $f$ is linear, but rather whether it is affine.     
            \item In this case, we can use the BLR Linearity Testing to check whether $f(x)-f(0)$ is linear, \emph{i.e.}, check whether $f (u + v) + f (u) + f (v) + f (0) = 0$ holds, where $u,v$ chosen uniformly at random.
            \item Another test for deciding whether a Boolean function $f$ is affine is to 
            check whether the equation $f (u + v + w) + f (u) + f (v) + f (w) = 0$ holds, for some $u, v, w\in\F_2^n$ chosen uniformly at random:
            \begin{itemize}
                \item If $f$ passes the test for many triples $(u, v, w)$, then $f$ is probably affine.
                \item If $f$ fails the test for at least one pair $(u, v, w)$, then $f$ is certainly not affine. 
            \end{itemize}
        \end{itemize}
            
    \end{frame}

    \begin{frame}
        \frametitle{A Generalization of BLR Linearity Testing}
    
        A generalization of these tests above was proposed by Zhang and Zheng\footnote{Zhang, X.-M., Zheng, Y.: The nonhomomorphicity of Boolean functions. In: Tavares, S., Meijer,
        H. (eds.) Selected Areas in Cryptography, SAC, pp. 280-295. Springer (1999)}.

        To decide whether a Boolean function $f$ is affine (resp. linear), we choose $u_1,u_2,...,u_k\in\F_2^n$ uniformly at random to check whether the equation $f (u_1 +u_2+\cdots + u_k) + f (u_1) + f(u_2)+\cdots + f (u_k) = 0$ holds:
        \begin{itemize}
            \item When $k$ is even (resp. odd), if $f$ passes the test for many tuples $(u_1,u_2,...,u_k)$, then f is probably linear (resp. affine).
            \item When $k$ is even (resp. odd), if $f$ fails the test for at least one pair $(u_1,u_2,...,u_k)$, then $f$ is certainly not linear (resp. affine).
        \end{itemize}
            
    \end{frame}
    
    \begin{frame}
        \frametitle{The Defintition of the Nonhomomorphicity}

        Above test tells us that, if an $n$-variable Boolean function $f$ satisfies $f (u_1 +u_2+\cdots + u_k) + f (u_1) + f(u_2)+\cdots + f (u_k) = 0$ for a large number of tuples $(u_1,u_2,...,u_k)$, then the function behaves more like an affine function. 
        This leads to the definition of the nonhomomorphicity:
        \begin{definition}[$(k + 1)$-th order nonhomomorphicity]
            Let $f : \F_2^n\rightarrow \F_2$ be a Boolean function in $n$ variables and let $k\ge 2$ be an integer.
            The $(k + 1)$-th order nonhomomorphicity  is defined as the 
            probability that the equation $f (\sum_{i=1}^{k}u_i) + f (u_1) + f(u_2)+\cdots + f (u_k) = 0$ does not hold, \emph{i.e.},
            \begin{align*}
                &P_k(f)\\
                =&\frac{\#\left\{ (u_1,u_2,\dots,u_k)\in\F_2^{nk} :f (\sum_{i=1}^{k}u_i) + f (u_1) + f(u_2)+\cdots + f (u_k) = 0 \right\}}{2^{nk}}.
            \end{align*}
        \end{definition}
    
    \end{frame}
    \begin{frame}
        \frametitle{The Nonhomomorphicity}
    
        \begin{itemize}
            \item $f$ is affine iff $P_k(f)\in\{0,1\}$. Indeed, for odd $k$, $f$ is affine iff $P_k(f)=0$; for even $k$, $f$ is linear iff $P_k(f)=0$ and $f+1$ is linear iff $P_k(f)=1$.
            \item The nonhomomorphicity is concerned over all absolute values of the Walsh transform of $f$ as shown in the sequel, the nonlinearity focuses on the maximum absolute value of the Walsh transform of $f$.
            \item BLR test corresponds to the particular case of $k = 2$.
        \end{itemize}
    
    \end{frame}
\section{Estimate the Nonlinearity of Boolean Functions\\ on Probabilistic Affine Tests} 

    \begin{frame}
        \frametitle{The Walsh Transform and the Nonhomomorphicity}
        
        Assume $k\ge 2$, for any $n$-variable Boolean function $f$, we have 
        \begin{align*}
            &\sum_{u\in\F_2^n}^{}\left( \widetilde{W_f}(u) \right)^{k+1}=\sum_{u\in\F_2^n}\prod_{i=1}^{k+1}\left( \frac{1}{2^n}\sum_{x_i\in\F_2^n}(-1)^{f(x_i)+u\cdot x_i} \right)\\
            =&\frac{1}{2^{n(k+1)}}\sum_{(x_1,x_2,...,x_{k+1})\in\F_2^{n(k+1)}}^{}(-1)^{f(x_1)+f(x_2)+\cdots+f(x_{k+1})}\sum_{u\in\F_2^n}^{}(-1)^{u\cdot(x_1+x_2+\cdots+x_{k+1})}\\ 
            =&\frac{2^n}{2^{n(k+1)}}\sum_{(x_1,x_2,...,x_{k})\in\F_2^{nk}}^{}(-1)^{f(x_1)+f(x_2)+\cdots+f(x_{k})+f(x_1+x_2+\cdots+x_k)}\\
            =&\frac{1}{2^{nk}}\sum_{(x_1,x_2,...,x_{k})\in\F_2^{nk}}^{}\left\{ 1-2(f(x_1)+f(x_2)+\cdots+f(x_{k})+f(x_1+x_2+\cdots+x_k)) \right\}\\
            =&1-2P_{k}(f).
        \end{align*}
    \end{frame}
    


    \begin{frame}
        \frametitle{The Lower and Upper Bounds on $P_k(f)$}
        For $\sum_{u\in\F_2^n}^{}\left( \widetilde{W_f}(u) \right)^{k+1}=1-2P_{k}(f)$, we have 
        \begin{itemize}
            \item Lower bound:
            \[\sum_{u\in\F_2^n}^{}\left( \widetilde{W_f}(u) \right)^{k+1}\le\max_{u\in\F_2^n}\left( \widetilde{W_f}(u) \right)^{k-1}\sum_{u\in\F_2^n}^{}\left( \widetilde{W_f}(u) \right)^2=\max_{u\in\F_2^n}\left( \widetilde{W_f}(u) \right)^{k-1}. \]
            \item Upper bound for odd $k$:
            \begin{itemize}
                \item In general, $\sum_{u\in\F_2^n}^{}\left( \widetilde{W_f}(u) \right)^{k+1}\ge \max_{u\in\F_2^n}\left( \widetilde{W_f}(u) \right)^{k+1}$.
                \item A complicated bound related with $n$ and $k$:
                \begin{align*}
                    &\sum_{u\in\F_2^n}^{}\left( \widetilde{W_f}(u) \right)^{k+1}=\left| \widetilde{W_f}(u_0) \right|^{k+1}+\sum_{u\in\F_2^n\setminus\{u_0\}}\left( \left( \widetilde{W_f}(u) \right)^2 \right)^{\frac{k+1}{2}}\\
                    \ge&\left| \widetilde{W_f}(u_0) \right|^{k+1}+(2^n-1)\left( \frac{1}{(2^n-1)}\sum_{u\in\F_2^n\setminus\{u_0\}}\left( \widetilde{W_f}(u) \right)^2 \right)^{\frac{k+1}{2}}
                \end{align*}
            \end{itemize}
        \end{itemize}
    \end{frame}

    \begin{frame}
        \frametitle{The Lower and Upper Bounds on $P_k(f)$ (cont.)}
        \[\ge\left| \widetilde{W_f}(u_0) \right|^{k+1}+(2^n-1)\left( \frac{1}{(2^n-1)}\left( 1- \widetilde{W_f}(u_0)^2\right) \right)^{\frac{k+1}{2}},\]
        where $u_0$ is the element of $\F_2^n$ such that $|\widetilde{W_f}(u_0)|=\max_{u\in\F_2^n}|\widetilde{W_f}(u)|$.
        \begin{itemize}
            \item From the relation between $P_k(f)$ and the Walsh transform, a lower bound and an upper bound can be derived, where the upper bound holds for odd $k$.
            \item However, there exists lower and upper bounds on $P_k(f)$ depending on $\disl(f)$, which contribute to no knowledge about results regarding the lower and upper bounds on $P_k(f)$ in terms of the Walsh transform for even $k$. 
        \end{itemize}
    \end{frame}

    \begin{frame}
        \frametitle{The Equation from the Defintition of $P_k(f)$}
    
        \begin{itemize}
            \item $P_k(f)$ is the probability that the equation $f (u_1 +u_2+\cdots + u_k) + f (u_1) + f(u_2)+\cdots + f (u_k) = 0$ fails for $(u_1,u_2,...,u_k)\in\F_2^{nk}$, where $f$ is an $n$-variable Boolean function.
            \item Given an arbitrary linear (resp. affine) function $l$ in $n$ variables for $k$ even (resp. odd), it must pass the test.
            \item Clearly the failure means there are odd number of values $g(u_1),g(u_2),...,g(u_k),g(u_{k+1})$ equals to $1$, where $g(u_i)=f(u_i)-l(u_i)$ and $u_{k+1}=u_1+u_2+\cdots+u_k$. 
            \item Denote by $A_j$ the probability that the first $j$ of these $k + 1$ values are equal to $1$ and the left are equal to $0$, \emph{i.e.}, 
            \[A_j=P(g(u_1)=1,g(u_2)=1,...,g(u_j)=1,g(u_{j+1})=0,g(u_{j+2})=0,...,g(u_{k+1})=0),\]
            where the probability is taken over all $(u_1,u_2,...,u_{k+1})\in\F_2^{n(k+1)}$ and $u_{k+1}=u_1+u_2+\cdots+u_k$.
        \end{itemize}
    \end{frame}

    \begin{frame}
        \frametitle{The Equation from the Defintition of $P_k(f)$ (cont.)}
    
        \begin{itemize}
            \item It can be easily seen that there are $\binom{k+1}{j}$ subsets of $\{1,2,...,k+1\}$ of cardinality $j$.
            \item Thus we have \[P_k(f)=\sum_{\substack{1\le j\le k+1\\ j\text{ odd}}} \binom{k+1}{j}A_j.\]
            \item Additionally,  $P(g(u_i)=1)=P(f(u_i)-l(u_i)=1)=\dis(f,l)$ as $u_i\in\F_2^n$, 
            therefore, we have 
            \begin{align*}
                A_j=&P(g(u_1)=1,...,g(u_j)=1,g(u_{j+1})=0,...,g(u_{k})=0)\\
                &\quad-P(g(u_1)=1,...,g(u_j)=1,g(u_{j+1})=0,...,g(u_{k+1})=1)\\
                =&x^j(1-x)^{k-j}-A_{j+1}=\cdots=\sum_{i=j}^{k}(-1)^{i-j}x^i(1-x)^{k-i}+(-1)^{k-j+1}A_{k+1},
            \end{align*}
            where $x=\dis(f,l)$.
        \end{itemize}
    \end{frame}

    \begin{frame}
        \frametitle{$P_k(f), A_{k+1}$ and $\dis(f,l)$}
    
        We obtain 
        \begin{align*}
            P_k(f)=& \sum_{\substack{1\le j\le k+1\\j\text{ odd}}}\binom{k+1}{j}\left( \sum_{i=j}^{k}(-1)^{i-j}x^i(1-x)^{k-i}+(-1)^{k-j+1}A_{k+1} \right)\\
            =&\frac{1}{2}\left( 1-(1-2x)^{k+1}+(-2)^{k+1}(x^{k+1}-A_{k+1}) \right).
        \end{align*}
        where $x=\dis(f,l)$ and $A_{k+1}=P(g(u_1)=1,...,g(\sum_{i=1}^{k}u_i)=1)\ge 0$ with the probability taken over all $(u_1,u_2,...,u_k)\in\F_2^{nk}$.
        \begin{itemize}
            \item When $k$ even, $P_k(f)$ is positively correlated with $x=\dis(f,l)$ and $A_{k+1}$. So we choose $l$ be the linear function with the minimum distance to $f$ and assume $A_{k+1}=0$, \emph{i.e.}, 
            \[P_k(f)\ge\frac{1}{2}\left( 1-(1-2x)^{k+1}-(2x)^{k+1} \right).\]
        \end{itemize}
    \end{frame}

    \begin{frame}
        \frametitle{$P_k(f),A_{k+1}$ and $\dis(f,l)$ (cont.)}
    
        \begin{itemize}
            \item When $k$ is odd, we have $A_{k+1}\le P(g(u_1)=1,...,g(u_k)=1)=\dis(f,l)^k$, where $l$ is any affine function. So we have 
            \[P_k(f)\ge\frac{1}{2}\left( 1-(1-2x)^{k+1}+2^{k+1}(x^{k+1}-x^k)\right),\]
            where $x=\dis(f,l_0)$ and $l_0$ is an affine function with the minimum distance to $f$.
            \item When $k$ is odd, we have obtained a lower bound on $P_k(f)$ by the equation with $\nnl(f)$; when $k$ is even, a lower bound on $P_k(f)$ involved $\disl(f)$ can be derived.
        \end{itemize}
    \end{frame}
\begin{frame}
        \frametitle{The Lower and Upper Bounds on $\overline{P_2}(f)$}
        \begin{itemize}
            \item Bellare et al.\footnote{Bellare, M., Coppersmith, D., H$\mathring{\text{a}}$stad, J., Kiwi, M., Sudan, M.: Linearity testing in characteristic two. IEEE Trans. Inf. Theory 42(6), 1781-1795 (1996)} showed a connection between the linearity testing problem and the Walsh transform.
            Namely, \[\operatorname{Lower}_2(\disl(f))\le P_2(f)\le \operatorname{Upper}_2(\disl(f)).\]
            \item Since affine test is equivalent to testing both $f$ and $f + 1$ for linearity, we consider $\overline{P_2}(f)=\min\{P_2(f),P_2(f+1)\}$.
            \item Then $\overline{P_2}(f)$ can be bounded in terms of $\nnl(f)$:
            \[\operatorname{Lower}_2(\nnl(f))\le \overline{P_2}(f)\le\min\left\{ \frac{1}{2},  \operatorname{Upper}_2(\nnl(f))\right\}.\]
        \end{itemize}
    \end{frame}
    \begin{frame}
        \frametitle{Estimate the Nonhomomorphicity}
        \begin{itemize}
            \item The nonhomomorphicity of a function $f$ can be determined precisely by its definition, but the calculation needs us to search through the entire $\F_2^{nk}$.
            \item This calculation is also infeasible when $n$ and $k$ large. 
            \item We introduce a statistical method to estimate the nonhomomorphicity\footnote{Zhang, X.-M., Zheng, Y.: The nonhomomorphicity of Boolean functions. In: Tavares, S., Meijer,
            H. (eds.) Selected Areas in Cryptography, SAC, pp. 280-295. Springer (1999)}.
            \item The nonhomomorphicity can be fast estimated and the result is reliable if the parameters of this estimation are well chosen.
        \end{itemize}
    
    \end{frame}
    
    \begin{frame}
        \frametitle{Estimate the Nonhomomorphicity}
    
        \begin{itemize}
            \item Denote the function $t$ on $\F_2^{n(k+1)}$ as follows:
            \begin{equation*}
                t(u_1,u_2,\dots,u_{k+1})=\begin{cases}
                    1,\text{if }f(u_1)+f(u_2)+\cdots+f(u_{k+1})=1;\\
                    0,\text{otherwise}.
                \end{cases}
            \end{equation*}
            \item Then from the definition of nonhomomorphicity we have $P_k(f)=\sum_{(u_1,u_2,...,u_k)\in\F_2^{nk}}^{}t(u_1,u_2,...,u_{k+1})/2^{nk}$.
            \item Let $\Omega$ be a random subset of $\F_2^{n(k+1)}$, then the sample mean is 
            \[\overline{t}=\frac{1}{\#\Omega}\sum_{(u_1,u_2,...,u_k)\in\Omega}^{}t(u_1,u_2,...,u_{k+1}).\]
        \end{itemize}
    
    \end{frame}

    \begin{frame}
        \frametitle{Estimate the Nonhomomorphicity}
    
        \begin{itemize}
            \item Since  
            \begin{align*}
                &\sum_{(u_1,u_2,...,u_k)\in\Omega}^{}(t(u_1,u_2,...,u_{k+1})-\overline{t})^2\\
                =&\sum_{(u_1,u_2,...,u_k)\in\Omega}^{}t^2(u_1,u_2,...,u_{k+1})-2\overline{t}\cdot\sum_{(u_1,u_2,...,u_k)\in\Omega}^{}t(u_1,u_2,...,u_{k+1})+\#\Omega\cdot\overline{t}^2\\
                =&\sum_{(u_1,u_2,...,u_k)\in\Omega}^{}t(u_1,u_2,...,u_{k+1})-2\overline{t}\cdot\sum_{(u_1,u_2,...,u_k)\in\Omega}^{}t(u_1,u_2,...,u_{k+1})+\#\Omega\cdot\overline{t}^2\\
                =&\#\Omega\cdot\overline{t}(1-\overline{t}).
            \end{align*}
            \item So we have the sample standard deviation $s$ as follows 
            \[s=\sqrt{\frac{1}{\#\Omega-1}\sum_{(u_1,u_2,...,u_k)\in\Omega}^{}(t(u_1,u_2,...,u_{k+1})-\overline{t})^2}=\sqrt{\frac{\#\Omega\cdot\overline{t}(1-\overline{t})}{\#\Omega-1}}.\]
        \end{itemize}
    
    \end{frame}

    \begin{frame}
        \frametitle{Estimate the Nonhomomorphicity}
    
        \begin{itemize}
            \item Therefore $P\left( \left| \frac{P_k(f)-\overline{t}}{s/\sqrt{\#\Omega}}\right|<Z_{\alpha/2} \right)=1-\alpha$, where $Z_{\alpha/2}$ denotes the value of a standard normal distribution which to
            its right a fraction $\alpha/2$ of the data.
            \item We have $\overline{t}(1-\overline{t})\le 1/4$ since $0\le\overline{t}\le 1$, so $s\le\sqrt{\frac{\#\Omega/4}{\#\Omega-1}} $. Thus we obtain that 
            \[\overline{t}-\frac{Z_{\alpha/2}}{2\sqrt{\#\Omega-1}}< P_k(f) <\overline{t}+\frac{Z_{\alpha/2}}{2\sqrt{\#\Omega-1}} \]
            holds with the probability $1-\alpha$.
            \item If $\#\Omega$ is large (e.g. $\#\Omega>Z_{\alpha/2}^2 10^{2p}$, where $p\ge 1$), the lower bound and upper bound are closer to each other, which means we have more information about $P_k(f)$.
        \end{itemize}
    
    \end{frame}

    \begin{frame}
        \frametitle{Examples of estimating the nonlinearity}
        \begin{table}
            \small
            \begin{tabular}{|ccccc|}
            \hline 
            Function & $\mathrm{d}_{\mathcal{A}}$ & $k=3$, estimated $\mathrm{d}_{\mathcal{A}}$ & $k=5$, estimated $\mathrm{d}_{\mathcal{A}}$ & $k=7$, estimated $\mathrm{d}_{\mathcal{A}}$ \\
            \hline
            $x_1 x_2$ & $\frac{1}{4}$ & $\left[\frac{1}{4}-0.10355, \frac{1}{4}\right]$ & $\left[ \frac{1}{4}-0.06498, \frac{1}{4} \right]$ & $\left[\frac{1}{4}-0.0473, \frac{1}{4}\right]$ \\
            $x_1 x_2+x_3 x_4$ & $\frac{3}{8}$ & $\left[\frac{3}{8}-0.125, \frac{3}{8}\right]$ & $\left[\frac{3}{8}-0.07343, \frac{3}{8}\right]$ & $\left[\frac{3}{8}-0.05178, \frac{3}{8}\right]$ \\
            $\begin{array}{c}x_1 x_2+x_3 x_4 \\ +x_5 x_6\end{array}$ & $\frac{7}{16}$ & $\left[\frac{7}{16}-0.11428, \frac{7}{16}\right]$ & $\left[\frac{7}{16}-0.0625, \frac{7}{16}\right]$ & $\left[\frac{7}{16}-0.04261, \frac{7}{16}\right]$ \\
            $\begin{array}{c}x_1 x_2+x_3 x_4\\+x_5 x_6+x_7 x_8\end{array}$ & $\frac{31}{32}$ & $\left[\frac{31}{32}-0.09375, \frac{31}{32}\right]$ & $\left[\frac{31}{32}-0.0475, \frac{31}{32}\right]$ & $\left[\frac{31}{32}-0.03125, \frac{31}{32}\right]$ \\
            $x_1 x_2 x_3$ & $\frac{1}{8}$ & $\left[ \frac{1}{8}-7.85 \cdot 10^{-3}, \frac{1}{8} \right]$ & $\left[ \frac{1}{8}-5.98 \cdot 10^{-4}, \frac{1}{8} \right]$ & {$\left[\frac{1}{8}-5.00 \cdot 10^{-5}, \frac{1}{8}\right.$} \\
            $x_1 x_2 x_3 x_4$ & $\frac{1}{16}$ & $\left[ \frac{1}{16}-6.82 \cdot 10^{-4}, \frac{1}{16} \right]$ & $\left[ \frac{1}{16}-9.30 \cdot 10^{-6}, \frac{1}{16} \right]$ & $\left[ \frac{1}{16}-1.42 \cdot 10^{-7}, \frac{1}{16} \right]$ \\
            $x_1 \cdots x_5$ & $\frac{1}{32}$ & $\left[ \frac{1}{32}-7.17 \cdot 10^{-5}, \frac{1}{32} \right]$ & $\left[ \frac{1}{32}-2.13 \cdot 10^{-7}, \frac{1}{32} \right]$ & $\left[ \frac{1}{32}-7.09 \cdot 10^{-10},\frac{1}{32} \right]$ \\
            $x_1 \cdots x_6$ & $\frac{1}{64}$ & $\left[ \frac{1}{64}-8.26 \cdot 10^{-6}, \frac{1}{64} \right]$ & {$\left[\frac{1}{64}-5.73 \cdot 10^{-9}, \frac{1}{64}\right]$} & $\left[ \frac{1}{64}-4.47 \cdot 10^{-12},\frac{1}{64} \right]$ \\
            \hline
            \end{tabular}
            \caption{Examples of estimating the nonlinearity}

        \end{table}
    
    \end{frame}

    \makebottom     % 创建尾页  % 非标准命令

    
    % \begin{frame}[t,allowframebreaks]
    %     \nocite{*}
    %     \frametitle{References}
    %     \printbibliography
    % \end{frame}


\end{document}