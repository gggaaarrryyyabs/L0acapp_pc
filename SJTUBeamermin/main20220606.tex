% !TeX TS-program = xelatex
% This is SJTUBeamermin v1.0
% For more detailed desciption, see
% https://github.com/LogCreative/SJTUBeamermin/blob/main/doc/sjtubeamermintheme.pdf
%
\documentclass[
    % draft,                             % 草稿模式
    aspectratio=169,                   % 使用 16:9 比例
]{beamer}
\mode<presentation>
\usetheme[
    navigation=subsections,            % 使用子章节进度显示
     lang=en,                           % 使用英文
    % cjk=true,                          % 使用CJK而不是ctex
    color=blue,                         % 使用红色主题
    % pattern=all,                        % 使用全图案装饰
    % gbt=bibtex,                        % 使用 gbt (使用 bibtex 编译)
]{sjtubeamermin}
% \usecolortheme[]{beaver}                 % 使用其他颜色主题
% \addbibresource{ref.bib}               % gbt!=bibtex

\usepackage{amsmath,amssymb,amsthm}
\usepackage{bm}
\usepackage{xcolor}
\usepackage{tikz-cd}
\usepackage{tikz}
\usetikzlibrary{trees}
\usepackage{empheq}
\usepackage{tabularx}
\usepackage{multicol}
\usepackage[linesnumbered,ruled]{algorithm2e}
\usepackage{hyperref}
\usepackage{amsfonts}
\usepackage{fontspec}
\usepackage{mdframed} % Add easy frames to paragraphs
\usepackage{color,soul}
\usepackage{xparse} % Add support for \NewDocumentEnvironment
\renewcommand{\Bbb}{\mathbb}
\newcommand{\Z}{\mathbb{Z}}
\newcommand{\GR}{\mathbb{GR}}
\newcommand{\F}{\mathbb{F}}
\newcommand{\R}{\mathbb{R}}
\newcommand{\Fn}{\mathbb{F}_2^n}
\newcommand{\Fm}{\mathbb{F}_2^m}
\newcommand{\df}{\delta_F}
\newcommand{\tr}{\operatorname{tr}}
\newcommand{\gtr}{\operatorname{T}}
\newcommand{\teich}{\textit{Teichm$\ddot{u}$ller sets}}
% \newtheorem*{definition}{Definition}
\newtheorem{thm}{Theorem}
\newtheorem{lem}[thm]{Lemma}
\newtheorem{proposition}{Proposition}
\definecolor{graylight}{cmyk}{.30,0,0,.67} % define color using xcolor syntax
\setbeamertemplate{itemize items}[default]

\newmdenv[ % Define mdframe settings and store as leftrule
  linecolor=red,
  topline=false,
  bottomline=false,
  rightline=false,
  skipabove=\topsep,
  skipbelow=\topsep
]{leftrule}

% \NewDocumentEnvironment{example}{O{\textbf{Example:}}} % Define example environment
% {\begin{leftrule}\noindent\textcolor{blue}{#1}\par}
% {\end{leftrule}}
\setbeamertemplate{footline}[frame number]

\NewDocumentEnvironment{question}{O{\textbf{Something:}}} % Define something environment
{\begin{leftrule}\noindent\textcolor{graylight}{#1}\par}
{\end{leftrule}}

\NewDocumentEnvironment{remark}{O{\textbf{Remark:}}} % Define remark environment
{\begin{leftrule}\noindent\textcolor{blue}{#1}\par}
{\end{leftrule}}

\begin{document}
    \institute[School of Electronic Information and Electrical Engineering]{电子信息与电气工程学院}   % 组织
    % \logo{
    %     \includegraphics{support/cnlogored.pdf}  % 重定义 logo
    % }
    \titlegraphic{                         % 标题图像
        \begin{stampbox}[white]
            \includegraphics[width=0.3\textwidth]{support/head.png}
        \end{stampbox}
    }
    \title{A new combinational logic \\minimization technique}  % 标题
 %   \subtitle{SJTUBeamer \fbox{\textsc{min}} Template}         % 副标题
    \author{Zhaole Li}                  % 作者
    \date{\textcolor{red}{Workshop of AES, 2022}}                          % 日期
    \maketitle                             % 创建标题页


\AtBeginSection[]{
    \begin{frame}
        % \tableofcontents[currentsection]           % 传统节目录             
        \sectionpage                   % 节页
    \end{frame}
}

% 使用小节目录
%\AtBeginSubsection[]{                  % 在每小节开始
 %   \begin{frame}
        % \tableofcontents[currentsection,currentsubsection]             % 传统小节目录             
  %      \subsectionpage                % 小节页
 %   \end{frame}
%}

\section{Introduction}
%\subsection{第 1 小节}

    \begin{frame}
        \frametitle{Optimal combinational circuits}
        
       The meaningful metrics for constructing optimal combinational circuits are gate count, depth, energy consumption, etc.

       The number of $ n $-variable Boolean functions is $ 2^{2^n} $, so no known techniques can even find the optimal circuits for $ 8 $-variable Boolean functions. 
       Thus we build the implementations using some heuristics. 

    \end{frame}
    \begin{frame}
        \frametitle{Optimal combinational circuits}
        
        \begin{enumerate}
            \item This work presented a new technique for circuit implementations with two steps:
            \begin{enumerate}
                \item Reducing multiplicative complexity for the non-linear components;
                \item Then optimizing the linear components.
            \end{enumerate}
            \item The metric is gate count with AND, XOR and 1;
        \end{enumerate}
        
    \end{frame}

    \begin{frame}
        \frametitle{First Step}
    
        \begin{definition}
            The multiplicative complexity of a function is the number of $ \F_2 $ multiplications necessary and sufficient to compute it.
        \end{definition}

        \begin{example}
            The multiplicative complexity of 
            $ f(x_1,x_2,x_3,x_4)=x_{1} x_{2} x_{3} x_{4}+x_{1} x_{2} x_{3}+x_{1} x_{2} x_{4}+x_{2} x_{3} x_{4}+x_{1} x_{2}+x_{1} x_{3}+x_{1} x_{4}+x_{2} x_{3}+x_{3} x_{4} $ 
            is no greater than $ 3 $ since $ f=(x_1+1)(x_2+1)(x_3+1)(x_4+1)+x_1+x_2+x_3+x_4+1 $
            and is $ 3 $ due to $ \operatorname{deg}(f)=4 $.
        \end{example}
    
    \end{frame}

    \begin{frame}
        \frametitle{Second Step}
    
        The second step is composed by finding the maximal linear components of the circuit and minimizing the number of XOR gates needed. 
        A new heuristic for the second step is proposed. 
    \end{frame}

\section{Non-linear components of Sbox in AES\footnote{Boyar J, Peralta R. A new combinational logic minimization technique with applications to cryptology[C]//International Symposium on Experimental Algorithms. Springer, Berlin, Heidelberg, 2010: 178-189.}}

    \begin{frame}
        \frametitle{$ \F_{2^4} $ inversion}
        The only non-linear component in AES's Sbox is to compute the inverse in the finite field field $ \F_{2^8} $.
        Canright built a circuit for inverses in $ \F_{2^8} $ by giving a circuit for inverses in $ \F_{2^4} $. 
        Using the same general technique but in different bases\footnote{$ W $ is a root of $ x^2+x+1 $ over $ \F_2 $, $ Z $ is a root of $ x^2+x+W $ over $ \F_{2^2} $.} $ \{W,W^2,Z^2,Z^8\} $ we can represent an element
         $ \Delta=\left(x_{0} W+x_{1} W^{2}\right) Z^{2}+\left(x_{2} W+x_{3} W^{2}\right) Z^{8} $ of $ \F_{2^4} $,
        and the inverse of this element $ \Delta^{\prime}=\left(y_{0} W+y_{1} W^{2}\right) Z^{2}+\left(y_{2} W+y_{3} W^{2}\right) Z^{8} $ can be calculated as the following:
    \end{frame}
    \begin{frame}
        \frametitle{$ \F_{2^4} $ inversion}
        \begin{empheq}[left=\empheqlbrace]{align*}
            -y_0&=x_1 x_2 x_3+x_0 x_2+x_1 x_2+x_2+x_3\\
            -y_1&=x_0 x_2 x_3+x_0 x_2+x_1 x_2+x_1 x_3+x_3\\
            -y_2&=x_1 x_0 x_3+x_0 x_2+x_0 x_3+x_0+x_1\\
            -y_3&=x_1 x_2 x_0+x_0 x_2+x_0 x_3+x_1 x_3+x_1
        \end{empheq}
    \end{frame}


    \begin{frame}
        \frametitle{Optimal non-linear components}
        For $ \F_{2^4} $ inversion, we take the method:
        \begin{enumerate}
            \item pick an equation and build an efficient circuit for it;
            \item store the intermediate functions used in above for possible usage in the other equations;
            \item iterate until all equations have been computed.
        \end{enumerate}
        \begin{remark}
            It turns out that $3$ multiplications are enough to compute any functions on four variables. 
        \end{remark}
    \end{frame}

    \begin{frame}
        \frametitle{Optimal non-linear components}
        
        \begin{empheq}[left=\empheqlbrace]{align*}
            -\mathbin{\color{blue}y_1}&=(\mathbin{\color{red}x_0 x_2} +x_1)(x_2 +x_3) +x_3\\
            -\mathbin{\color{green}y_3}&=(\mathbin{\color{red}x_0 x_2} +x_3)(x_0 +x_1) +x_1\\
            -y_0&=(\mathbin{\color{red}x_0 x_2} +\mathbin{\color{blue}y_1})x_3 +\mathbin{\color{blue}y_1} +x_2+x_3\\
            -y_2&=(\mathbin{\color{red}x_0 x_2} +\mathbin{\color{green}y_3})x_3 +\mathbin{\color{green}y_3} +x_0+x_1
        \end{empheq}
        This circuit needs $ 5 $ AND gates and $ 11 $ XOR gates. 

    \end{frame}

\section{Minimizing linear components\footnote{Boyar J, Peralta R. A new combinational logic minimization technique with applications to cryptology[C]//International Symposium on Experimental Algorithms. Springer, Berlin, Heidelberg, 2010: 178-189.}}

    \begin{frame}
            \frametitle{An example of linear component}
        
            \begin{example}
                \[\begin{pmatrix}
                    1 &1 &0 &0\\
                    1 &1 &1 &0\\
                    1 &1 &1 &1\\
                    0 &1 &1 &1
                \end{pmatrix}\begin{pmatrix}
                    x_1\\x_2\\x_3\\x_4
                \end{pmatrix}\]
                Actually written in the equations  $ x_1+x_2;x_1+x_2+x_3;x_1+x_2+x_3+x_4;x_2+x_3+x_4. $
                It's easy to see that we only need $ 4 $ XOR to compute the linear component $ v_1=x_1+x_2;v_2=v_1+x_3;v_3=v_2+x_4;v_4=v_3+x_1. $
            \end{example}
        \end{frame}
    
    \begin{frame}
        \frametitle{A new heuristic}
    
        Let $S$ be a set of linear functions. For any linear predicate $f$, we define the distance $\delta(S,f)$ 
        as the minimum number of additions of elements from $S$ necessary to obtain $f$.
        For the linear Boolean function, initially $ S $ is just the set of all variables $ x_1,x_2,...,x_n $,
        then a new base element is the form of two old base elements, update the $ \delta(S,f) $ until $ \delta(S,f)=0 $.
        \begin{enumerate}
            \item For the $ (n,m) $-linear Boolean functions, we use the $ m\times n $ matrix over $ \F_2 $ such as $ f(\textbf{x})=M\textbf{x} $.
            $ S $ still be just the set of all variables $ x_1,x_2,...,x_n $;
            \item Denote $ Dist[] $ the distance from $ S $ to the linear function given by rows of $ M $, 
            in fact, $ Dist[i]=\delta(S,f_i) $ where $ f_i $ is the $ i^{th} $ linear function given by $ M $;
            \item Pick a new base element by adding two old base elements and then update $ Dist[] $;
            \item Iterate the last step until $ Dist[]=(0,0,...,0) $.
        \end{enumerate}
    
    \end{frame}

    \begin{frame}
        \frametitle{How to pick the new base element}
    
        \begin{enumerate}
            \item pick those that minimize the sum of new distances;
            \item pick one that maximizing the Euclidean norm of the vector of new distances;
        \end{enumerate}
        This criterion seems strange for maximizing. But we want a distance $ (0,2,1) $ rather than $ (1,1,1) $. 
    \end{frame}


    \begin{frame}
        \frametitle{Example}
        We  bulid the circuit of the following equation system:
        \[\begin{aligned}
            &y_{0}=x_{0}+x_{1}+x_{2} \\
            &y_{1}=x_{1}+x_{3}+x_{4} \\
            &y_{2}=x_{0}+x_{2}+x_{3}+x_{4} \\
            &y_{3}=x_{1}+x_{2}+x_{3} \\
            &y_{4}=x_{0}+x_{1}+x_{3} \\
            &y_{5}=x_{1}+x_{2}+x_{3}+x_{4}
        \end{aligned}\]
        so the matrix $ M $ is $\begin{pmatrix}
                1 & 1 & 1 & 0 & 0 \\
                0 & 1 & 0 & 1 & 1 \\
                1 & 0 & 1 & 1 & 1 \\
                0 & 1 & 1 & 1 & 0 \\
                1 & 1 & 0 & 1 & 0 \\
                0 & 1 & 1 & 1 & 1
            \end{pmatrix}.$
    \end{frame}

    \begin{frame}
        \frametitle{Example}
    
        \begin{enumerate}
            \item In above situation, the initial $ S $ is the set $ \{[0 0 0 0 1],[00010],[00100],[01000],[10000]\} $;
            \item The initial distance is $ Dist=[2,2,3,2,2,3] $;
            \item First choose the two coloumn which have the most $ 1 $ in the same row;
            \item \dots 
        \end{enumerate}

    \end{frame}

    

    \makebottom     % 创建尾页  % 非标准命令

\end{document}