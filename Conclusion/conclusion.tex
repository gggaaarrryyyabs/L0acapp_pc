\documentclass[8pt,oneside]{article}
\usepackage{amsmath,amssymb}
\usepackage{bm}
\usepackage{xcolor}
\usepackage[noadjust]{cite}
\usepackage{tikz-cd}
\usepackage{forest}
\usepackage{soul}
\usepackage[colorlinks=true,citecolor=blue]{hyperref}
\usepackage{enumitem,empheq}
\usepackage{booktabs}
\usepackage{rotating,tabularx}
\usepackage{graphicx}
\usepackage{siunitx}
\usepackage{pifont}
% \newtheorem{theorem}{Theorem}
\newcommand{\0}{\textbf{0}}
\newcommand{\1}{\textbf{1}}
\newcommand{\E}{\mathcal{E}}
\newcommand{\B}{\mathcal{B}}
\newcommand{\nl}{\mathrm{nl}}
\newcommand{\Tr}{\mathrm{Tr}_1^n}
\newcommand{\tr}{\mathrm{tr}_1^k}
\newcommand{\Z}{\mathbb{Z}}
\newcommand{\F}{\mathbb{F}}
\newcommand{\Com}{\mathbb{C}}
\newcommand{\ord}{\operatorname{ord}}
\newcommand{\Q}{\mathbb{Q}}
\newcommand{\R}{\mathbb{R}}
\usepackage{bbding}%characteristic symbol
\newcommand{\cmark}{\ding{51}}%checkmark 对号
\newcommand{\xmark}{\ding{55}}%crossmark 错号
\usetikzlibrary{graphs,quotes}
\usetikzlibrary{arrows,chains,matrix,positioning,scopes}

\renewcommand{\thesection}{}
\renewcommand{\thesubsection}{\arabic{section}.\arabic{subsection}}
\makeatletter
\def\@seccntformat#1{\csname #1ignore\expandafter\endcsname\csname the#1\endcsname\quad}
\let\sectionignore\@gobbletwo
\let\latex@numberline\numberline
\def\numberline#1{\if\relax#1\relax\else\latex@numberline{#1}\fi}
\makeatother

\newtheorem{theorem}{Theorem}
\newtheorem{corollary}{Corollary}[theorem]
% \newtheorem{example}{例}
\newtheorem{innercustomthm}{Example}
\newtheorem{example}{Example}[subsection]
\newtheorem{lemma}{Lemma}
\newtheorem{definition}{Definition}
\newtheorem{remark}{Remark}
\newtheorem{proof}{Proof}
\renewcommand{\proof}{\textbf{Proof:}}


\begin{document}

Also, Additionally, Furthermore, Moreover, In addition to, as well as, 
Particularly, Especially, For instance, 
To this end, Regarding, As for, With regards to, 
similarly, Equally, In the same way,
Specifically, namely, That is, In other words, To put it another way, 
Conversely, Whereas, Indeed, Significantly, 

\begin{proof}
    no 
\end{proof}


\section{Finite Field}
    \textbf{Hilbert 90} We have $ \tr(x)=0 $ for $ x\in\F_{2^k} $ iff there exist some $ y\in\F_{2^k} $ with 
    $ x=y^2+y $.

    \textbf{Theorem}\cite{BerlekampRS1967roots_quad_tri} 
    The cubic equation $ x^3+x+a=0 $, $ a\in\F_{2^k}^* $ has a unique solution iff $ \tr(a^{-1})\ne\tr(1) $. 
    
    \textbf{Theorem}\cite{BerlekampRS1967roots_quad_tri} 
    A necessary and suffcient condition that all three roots of the cubic equation $ x^3+x+a=0 $ lie in 
    $ \F_{2^k} $ is that $ P_k(a)=0 $ where the polynomials $ P_k(x) $ may be defined recursively by the equations
    \begin{align*}
        &P_1(x)=P_2(x)=x\\
        &P_k(x)=P_{k-1}(x)+x^{2^{k-3}}P_{k-2}(x).
    \end{align*}

\section{Number Theory\cite{mceliece2012finite}}
    \begin{lemma}
        For $ 1\leq e\leq m $,
        \begin{empheq}[left={\gcd (2^e+1,2^m-1)=\empheqlbrace}]{align*}
            &1, &\text{if}~ \gcd (2e,m)=\gcd (e,m)\\
            &2^{\gcd (e,m)}+1, &\text{if}~ \gcd (2e,m)=2\gcd (e,m)
        \end{empheq}
    \end{lemma}
    \noindent\rule{\linewidth}{0.4pt}

\section{AES Acceleration}
    Now we give the process of inversion of $ A=a_0 Y+a_1 Y^{16}\in\F_{2^8} $, where $ a_0,a_1\in\F_{2^4} $: 
    \begin{enumerate}
        \item Assume $ (W,W^2) $ to be the basis of $ \F_{2^2} $, $(Z^2,Z^8) $ to be the basis of $ \F_{2^4} $, and 
        $ (Y,Y^{16} ) $ the basis of $ \F_{2^8} $.
        \item The method to calculate the inversion of $ A $ is $ A^{-1}=(AA^{16} )^{-1}A^{16}=((a_0+a_1 )^2 WZ+a_0 a_1 )^{-1}(a_1 Y+a_0 Y^{16}) $.
    \end{enumerate}

    The idea in the method: We need to obtain the results of $ T_1=(a_0+a_1);T_2=(WZ) (T_1 )^2;T_3=a_0 a_1;T_4=T_2+T_3;T_5=(T_4 )^{-1};T_6=T_5 a_1;T_7=T_5 a_0 $. 
    Note that all the calculation is based on $ \F_{2^4} $, which implies we acts on the vectors of length $ 4 $.
    \begin{enumerate}
        \item $ T_1 $ and $ T_4 $ are vectorial addition.
        \item $ T_2 $ is the scalar  multiplication, using the matrix $ P=\begin{bmatrix}
            0 &1 &0 &1\\
            1 &0 &1 &0\\
            1 &1 &0 &0\\
            0 &1 &0 &0
        \end{bmatrix} $.
        \item $ T_3, T_6 $ and $ T_7 $ are vectorial multiplication, 
        and we need to define vectorial multiplication $ z=x\star y $ in the following. 
        \item $ T_5 $ is the inversion over finite field in hardware, and the inversion is showed in below.
    \end{enumerate}

    Vectorial multiplication: $ z=(z_0,z_1,z_2,z_3 )=x\star y=(x_0,x_1,x_2,x_3 )\star (y_0,y_1,y_2,y_3 ) $: 
    \begin{align*}
        z_0&=x_1 y_1+(x_0+x_2 )(y_0+y_2 )+x_3 y_2+x_2 y_3+x_0 y_3+x_3 y_0+x_1 y_2+x_2 y_1\\
        z_1&=x_0 y_0+(x_0+x_2 )(y_0+y_2 )+x_0 y_1+x_1 y_0+(x_1+x_3 )(y_1+y_3 )\\
        z_2&=x_3 y_3+(x_0+x_2 )(y_0+y_2 )+x_0 y_1+x_1 y_0+x_0 y_3+x_3 y_0+x_1 y_2+x_2 y_1\\
        z_3&=x_2 y_2+(x_0+x_2 )(y_0+y_2 )+x_3 y_2+x_2 y_3+(x_1+x_3 )(y_1+y_3 ),\\
    \end{align*}
    after some calculation, we have another form of $ 10 $ multiplication, 
    \begin{empheq}{align*}
        z_0&=(x_1 +x_2)(y_1+y_2) +(x_0+x_2)(y_0+y_2) +(x_2 +x_3)(y_2+y_3) +(x_0 +x_3)(y_0 +y_3) +x_0 y_0\\
        z_1&=(x_1 +x_3)(y_1+y_3) +(x_0+x_2)(y_0+y_2) +(x_0 +x_1)(y_0+y_1) +x_1 y_1\\
        z_2&=(x_0 +x_3)(y_0+y_3) +(x_0+x_2)(y_0+y_2) +(x_0 +x_1)(y_0+y_1) +(x_1 +x_2)(y_1 +y_2) +x_2 y_2\\
        z_3&=(x_1 +x_3)(y_1+y_3) +(x_0+x_2)(y_0+y_2) +(x_2 +x_3)(y_2+y_3) +x_3 y_3.
    \end{empheq}
    Inversion: $ y=(y_0,y_1,y_2,y_3 )=x^{-1}= (x_0,x_1,x_2,x_3 )^{-1}$ has $ 8 $ multiplication: 
    \begin{empheq}[left=\empheqlbrace]{align*}
        -y_0&=x_1 x_2 x_3+x_0 x_2+x_1 x_2+x_2+x_3\\
        -y_1&=x_0 x_2 x_3+x_0 x_2+x_1 x_2+x_1 x_3+x_3\\
        -y_2&=x_1 x_0 x_3+x_0 x_2+x_0 x_3+x_0+x_1\\
        -y_3&=x_1 x_2 x_0+x_0 x_2+x_0 x_3+x_1 x_3+x_1
    \end{empheq}
    The reuseness of some intermediate variables can decrease the times of multiplication ($ 5 $ multiplication): 
    \begin{empheq}[left=\empheqlbrace]{align*}
        -\mathbin{\color{blue}y_1}&=(\mathbin{\color{red}x_0 x_2} +x_1)(x_2 +x_3) +x_3\\
        -\mathbin{\color{green}y_3}&=(\mathbin{\color{red}x_0 x_2} +x_3)(x_0 +x_1) +x_1\\
        -y_0&=(\mathbin{\color{red}x_0 x_2} +\mathbin{\color{blue}y_1})x_3 +\mathbin{\color{blue}y_1} +x_2+x_3\\
        -y_2&=(\mathbin{\color{red}x_0 x_2} +\mathbin{\color{green}y_3})x_3 +\mathbin{\color{green}y_3} +x_0+x_1.
    \end{empheq}

    List all $ T_i $: 
    \[a_0=\begin{pmatrix}
        x_0\\x_1\\x_2\\x_3
    \end{pmatrix},a_1=\begin{pmatrix}
        y_0\\y_1\\y_2\\y_3
    \end{pmatrix},
    T_1=x+y=\begin{pmatrix}
        x_0+y_0\\x_1+y_1\\x_2+y_2\\x_3+y_3
    \end{pmatrix}. \]
    \[ T_2=\begin{pmatrix}
        x_1+y_1+x_3+y_3\\x_0+y_0+x_2+y_2\\x_0+y_0+x_1+y_1\\x_1+y_1
    \end{pmatrix}.\]
    $ T_3= $\[\begin{pmatrix}
        (x_1 +x_2)(y_1+y_2) +(x_0+x_2)(y_0+y_2) +(x_2 +x_3)(y_2+y_3) +(x_0 +x_3)(y_0 +y_3) +x_0 y_0\\
        (x_1 +x_3)(y_1+y_3) +(x_0+x_2)(y_0+y_2) +(x_0 +x_1)(y_0+y_1) + x_1  y_1\\
        (x_0 +x_3)(y_0+y_3) +(x_0+x_2)(y_0+y_2) +(x_0 +x_1)(y_0+y_1) +(x_1 +x_2)(y_1 +y_2) +x_2 y_2\\
        (x_1 +x_3)(y_1+y_3) +(x_0+x_2)(y_0+y_2) +(x_2 +x_3)(y_2+y_3) + x_3  y_3
    \end{pmatrix}\]
    $ T_4=T_2+T_3= $ \[\begin{pmatrix}
        (x_1 +x_2)(y_1+y_2) +(x_0+x_2)(y_0+y_2) +(x_2 +x_3)(y_2+y_3) +(x_0 +x_3)(y_0 +y_3) +x_0 y_0+x_1+y_1+x_3+y_3\\
        (x_1 +x_3)(y_1+y_3) +(x_0+x_2+1)(y_0+y_2+1) +(x_0 +x_1)(y_0+y_1) + x_1  y_1+1\\
        (x_0 +x_3)(y_0+y_3) +(x_0+x_2)(y_0+y_2) +(x_0 +x_1+1)(y_0+y_1+1) +(x_1 +x_2)(y_1 +y_2) +x_2 y_2+1\\
        (x_1 +x_3)(y_1+y_3) +(x_0+x_2)(y_0+y_2) +(x_2 +x_3)(y_2+y_3) + x_3  y_3+x_1+y_1
    \end{pmatrix}\]
    Difficult
    \noindent\rule{\linewidth}{0.4pt}

    Lylia APN function\cite{Budaghyan2005}
    \[\left(x+Tr^n_1\left(x^{2^i+1}\right)\right)^{2^i+1}\]
    has covered all APN function(when $ n=8 $)
    \[\left(x+Tr^n_1\left(x^{2^i+1}\right)\right)^{2^j+1}\]
    where $ 1\leq i\neq j\leq n-1 $ and $ n $ is even.
    So we guess all functions like this form have been covered.
    
    
    \[\left(x+Tr^n_1\left(x^{2^i+1}\right)\right)^{2^{2j}-2^j+1}\]
    also are CCZ-equivalent to Kasami APN function.

    
    %     //result, when n=8 the results are listed below:
% [i j]
% [1 1] 
% [1 7] ccz eq [1 1]
% [3 3] 
% [3 5] ccz eq [3 3]
% =======
% [4 1] ccz eq [4 7]
% [4 3] ccz eq [4 5]
% [4 5] ccz eq [3 3]
% [4 7] ccz eq [1 1] 
% =======
% [5 3] ccz eq [5 5]
% [5 5]
% [7 1] ccz eq [7 7]
% [7 7]
    \begin{enumerate}
        \item EA-equivalent functions are CCZ-equivalent
        \item if a function $F$ is a permutation then $F$ is CCZ-equivalent to $F^{-1}$\cite{carlet1998codes}
        \item CCZ-equivalence coincides with\begin{enumerate}
            \item EA-equivalence for planar functions [36, 38];
            \item linear equivalence for DO planar functions [36, 38];
            \item EA-equivalence for all functions whose derivatives are surjective [36];
            \item EA-equivalence for all Boolean functions [24];
            \item EA-equivalence for all vectorial bent Boolean functions [25];
            \item EA-equivalence for two quadratic APN functions (conjectured by Edel,
            proven by Yoshiara [145]).
        \end{enumerate}
    \end{enumerate}
    \noindent\rule{\linewidth}{0.4pt}

    \begin{theorem}[Carlet, Charpin, Zinoviev 1998]
        Let $ F:\F_{2^n}\rightarrow\F_{2^n} $ with $ F(0)=0 $ and $ u $ be a primitive element of 
        $ \F_{2^n} $. Then $ F $ is APN iff the binary linear code $ C_F $ defined by the 
        parity check matrix
        \[H_F=\begin{bmatrix}
            u&u^2&\cdots&u^{2^n-1}\\
            F(u)&F(u^2)&\cdots&F(u^{2^n-1})\\
        \end{bmatrix}\]
        has minimum distance $ 5 $.
    \end{theorem}

    Two functions $ F,G:\F_{2^n}\rightarrow\F_{2^n} $ are CCZ equivalent iff
    $ G_F $ and $ G_G $ are affine-equivalent,

    i.e. if the extended codes with parity check matrices

    $ \begin{bmatrix}
        1 & 1 &\cdots &1\\
        0& u& \cdots &u^{2^n-1}\\
        F(0)& F(u)& \cdots &F(u^{2^n-1})
    \end{bmatrix} $ 
    and 
    $\begin{bmatrix}
        1 & 1 &\cdots &1\\
        0& u& \cdots &u^{2^n-1}\\
        G(0)& G(u)& \cdots &G(u^{2^n-1})
    \end{bmatrix}$
    are equivalent.


    \noindent\rule{\linewidth}{0.4pt}

    \begin{theorem}\label{gowers}
        Let $ k\in\Z^+ $, $ \epsilon>0 $. Let $ P:\F_2^n\rightarrow\F_2 $ be a polynomial of degree
        at most $ k $, and $ f:\F_2^n\rightarrow\R $. Suppose $ \lvert \mathbb{E}_x\left[f(x)(-1)^{P(x)}\right]\rvert\geq\epsilon  $, then $ \left\lVert f\right\rVert _{U_{k+1}}\geq\epsilon $.
    \end{theorem}
    The converse of Theorem \ref{gowers} is also true for $k = 1, 2$.



    
    \noindent\rule{\linewidth}{0.4pt}

    In 2003, algebraic attacks to LFSRs based on stream ciphers, by finding a way of solving 
    the over defined system of multivariate equations whose unknowns are the secret key bits,
    were proposed by Courtois and Meier\footnote{Courtois N., Meier W.: Algebraic Attacks on Stream Ciphers with Linear Feedback EUROCRYPT 2003,
    LNCS, vol. 2656, pp. 345-359. Springer, Heidelberg (2003).}. In 2004, the algebraic immunity of a Boolean
    function, representing its ability to resist this type of attacks, was introduced by Meier\footnote{Meier W., Pasalic E., Carlet C.: Algebraic attacks and decomposition of Boolean functions. In: Advances
    in Cryptology-EUROCRYPT 2004, LNCS, vol. 3027, pp. 474-491. Springer, Heidelberg (2004)}.

    \noindent\rule{\linewidth}{0.4pt}



    Let $n=2k$ and $\F_{2^n}=\F_{2^k}^2$. For any $\beta\in\F_{2^k}$ with $\mathrm{Tr}_1^k(\beta)=1$,
    then any element $X$ of $\F_{2^n}$ can be written as $X=x+\mu y$ where $x,y\in\F_{2^k}$ and $\mu$ is a root of the equation
    $\mu^2+\mu+\beta=0$ over $\F_{2^n}$. Thus, the inverse function $X^{2^n-2}$ can be decomposed
    (using $(x+\mu y)(x'+\mu y')=1$ and $0+0y=0\in\F_{2^n}$ or computing $(x+\mu y)^{2^n-2}$, see for examples \cite{Dumas2014arXiv} and \cite[Theorem 5]{PerrinUBCRYPTO16}) as
    $\big(x,y\big)\mapsto \big(x(y^2+xy+\beta x^2)^{d}, (x+y)(y^2+xy+\beta x^2)^{d}\big)$ ($(x,y)$ should be $(y,x)$),
    where $d={2^k-2}$ (clearly such mapping is bijective and is CCZ-equivalent to the inverse function over $\F_{2^n}$).
    Experiments show that when $d$ has the form $2^i$ this mapping is a differentially 4-uniform bijection
    for some integers $n$ and $i$. We now express this mapping with the univariate representation.
    Assume that $\mu$ is a root of the $\mu^2+\mu+\beta=0$ over $\F_{2^n}$.
    Then the mapping \noindent {\color{red}$\big(x,y\big)\mapsto \big(x(y^2+xy+\beta x^2)^{d}, (x+y)(y^2+xy+\beta x^2)^{d}\big)$}
    can be written as $x+\mu y\mapsto x(y^2+xy+\beta x^2)^{d}+\mu(x+y)(y^2+xy+\beta x^2)^{d}$.
    We have $\mu^2=\mu+\beta,\mu^4=\mu+\beta+\beta^2,\mu^8=\mu+\beta+\beta^2+\beta^4,\cdots,\mu^{2^k}=\mu+\mathrm{Tr}_1^k(\beta)$.
    Let $X=x+\mu y$. We have $X^{2^k}=x^{2^k}+\mu^{2^k}y^{2^k}=x+(\mu+1)y=X+y$ and so $y=X+X^{2^k}$.
    We have $x=X+\mu y=X+\mu(X+X^{2^k})=(\mu+1)X+\mu X^{2^k}$. By taking $d=2^i$, we could obtain the
    function $F$ defined over $\F_{2^n}$ with the univariate representation, which is given in \eqref{E:diff4uniformsixterms}.
    
    
    Let $n=2k$.
    For any $\beta\in\F_{2^k}$ with $\mathrm{Tr}_1^k(\beta)=1$ (so $\mathrm{Tr}_1^n(\beta)=0$),
    $\mu$ is a root of the equation $\mu^2+\mu+\beta=0$ over $\F_{2^n}$.

    \[y^2+xy+\beta x^2=(\beta+1)(\beta+\mu)x^2+(\beta+\mu+1)(\beta+1)x^{2^{k+1}}+x^{2^k+1}.\]
    \[x+\mu(x+y)=(\beta+1)x+(\beta+\mu)x^{2^k}\]
    Then we define a polynomial over $\F_{2^n}$ as follows:
    \begin{eqnarray}\label{E:diff4uniformsixterms}
    F(x)&=&(1+\beta)^2 x^{2^{k+i+1}+1}+(1+\beta)^2 x^{2^{i+1}+1}+(1+\beta) x^{2^{k+i}+2^i+1}\\
    &&+(\beta+\mu)(\beta+1) x^{2^{k+i+1}+2^k}+(\beta+\mu)(\beta+1) x^{2^{i+1}+2^k}+(\beta+\mu) x^{2^{k+i}+2^i+2^k}\nonumber.
    \end{eqnarray}
    
    We now consider the equation $\mu^2+\mu+\beta=0$. Note that
    $\mathrm{Tr}_1^k(\beta)=\mathrm{Tr}_1^k(\mu^2+\mu)=\mu+\mu^{2^k}$.
    If we want to get $\mathrm{Tr}_1^k(\beta)=1$ with $\beta\in\F_{2^k}$ then we
    only need to find an element $\mu\in\F_{2^n}\setminus\F_{2^k}$ such that $\mu+\mu^{2^k}=1$.
    Assume that $\mu=x+\alpha y$ ($\alpha$ is a primitive element of $\F_{2^n}$ and clearly we
    have $\alpha\in\F_{2^n}\setminus\F_{2^k}$; indeed, primitive element $\alpha$ can replaced by any element
    in $\alpha\in\F_{2^n}\setminus\F_{2^k}$),
    we have $\mu+\mu^{2^k}=y(\alpha+\alpha^{2^k})=1$ and thus $y=(\alpha+\alpha^{2^k})^{2^n-2}\in\F_{2^k}$
    since $(\alpha+\alpha^{2^k})^{2^k}=(\alpha+\alpha^{2^k})$. Thus $\mu$ can take $\alpha (\alpha+\alpha^{2^k})^{2^n-2}=\frac{\alpha}{\alpha+\alpha^{2^k}}$.
    Thus we have $\beta=\frac{\alpha}{\alpha+\alpha^{2^k}}+\frac{\alpha^2}{(\alpha+\alpha^{2^k})^2}
    =\frac{\alpha^{2^k+1}}{(\alpha+\alpha^{2^k})^2}\in\F_{2^k}$ (we also need to assume that $\beta\not=1$
    since \eqref{E:diff4uniformsixterms}, for doing this we only need to check that $\beta$ is a generator of $\F_{2^k}^*$).
    So the conditions become: 1) any $\alpha\in\F_{2^n}$ such that $\alpha+\alpha^{2^k}\not=1$ and $\frac{\alpha^{2^k+1}}{(\alpha+\alpha^{2^k})^2}\not=1$;
    2) $\mu=\alpha (\alpha+\alpha^{2^k})^{2^n-2}=\frac{\alpha}{\alpha+\alpha^{2^k}}$ in \eqref{E:diff4uniformsixterms};
    3) $\beta=\frac{\alpha^{2^k+1}}{(\alpha+\alpha^{2^k})^2}$.
    How to choose $i$ to ensure $F$ is a differentially 4-uniform bijection?
    
    
    Another way to rewrite \eqref{E:diff4uniformsixterms} is as follows:
    For $n=2k$, $\mu\in\F_{2^n}$ is such that $\mu+\mu^{2^k}=1$, $\mu+\mu^2\not=1$
    ($\mu+\mu^2\not=1$ is equivalent to $\mu\not\in\F_4$;
    we have $\mu+\mu^{2^k}=1$ implies that $\mu\not\in\F_{2^k}$), and $\mu+\mu^2\in\F_{2^k}$.
    Let $\beta=\mu+\mu^2$ (this implies that $\mathrm{Tr}_1^k(\beta)=\mu+\mu^{2^k}=1$).
    Then we define a polynomial over $\F_{2^n}$ as follows:
    \begin{eqnarray}\label{E:diff4uniformsixtermsmu}
    F(x)&=&(1+\beta)^2 x^{2^{k+i+1}+1}+(1+\beta)^2 x^{2^{i+1}+1}+(1+\beta) x^{2^{k+i}+2^i+1}\\
    &&+(\beta+\mu)(\beta+1) x^{2^{k+i+1}+2^k}+(\beta+\mu)(\beta+1) x^{2^{i+1}+2^k}+(\beta+\mu) x^{2^{k+i}+2^i+2^k}\nonumber.
    \end{eqnarray}
    How to choose $i$?

    \textbf{Simulations for \eqref{E:diff4uniformsixterms}:}
    \begin{enumerate}
      \item [-] For $n=6$, we take $\beta=1$ and $i=2$ have $F(x)$ is CCZ-equivalent to $x^{11}$ and $x^{23}$.
      \item [-] For $n=10$, we take $\beta$ such that $\mathrm{Tr}_1^5(\beta)=1$ and $i=0$ (then $F$ is quadratic) have $F(x)$ is
    differentially 4-uniform bijection. $F(x)$ is CCZ-inequivalent to $x^{3}$.
    {\color{red}We must consider if this function is CCZ-equivalent to $x^{2^k+2}$ since all terms include $z^3$ ($z\in\F_{2^k}$) when decomposing this
    function in to $\F_{2^k}^2$ ($2^k+2=3 \pmod {2^k-1}$).}
      \item [-] For $n=12$, by taking $\beta$ such that $\mathrm{Tr}_1^6(\beta)=1$ and $d=2^i=8$, then $F(x)$ is a differentially 4-uniform bijection.
    \end{enumerate}
    
    \vspace{5mm}
    \noindent {\color{red}The quadratic case with four terms (i.e. $i=0$):}
    
    For $n=2k$ ($k$ odd), $\mu\in\F_{2^n}\setminus \F_4$ is such that $\mu+\mu^{2^k}=1$.
    Let $i=0$, we have
    \begin{eqnarray}\label{E:diff4uniformQuadratic}
    F(x)&=&(1+\beta+\beta^2+\mu) x^{2^{k+1}+1}+(1+\beta^2) x^{3}+(1+\beta^2+\mu\beta+\mu) x^{2^k+2}\nonumber\\
    &&+(\beta^2+\beta+\mu\beta+\mu) x^{3\cdot 2^k}\nonumber\\
    %&=&(1+\mu^4) x^{2^{k+1}+1}+(1+\mu^2+\mu^4) x^3+(1+\mu+\mu^3+\mu^4) x^{2^k+2}+(\mu^2+\mu^3+\mu^4) x^{3\cdot 2^k}\nonumber\\
    &=&(1+\mu^4) x^{2^{k+1}+1}+(1+\mu+\mu^3+\mu^4) x^{2^k+2}+(\mu^2+\mu^3+\mu^4) x^{3\cdot 2^k}+(1+\mu^2+\mu^4) x^3\nonumber
    \end{eqnarray}
    
    $F$ is a quadratic bijection over $\F_{2^n}$, we checked by $n=6,10,14,18,22$ and this function may be CCZ-equivalent to $x^{3}$.
    \begin{remark}\label{R:TuPermutation}
    {\color{red}Indeed, the function $(x+ax+bx^{2^k})^3$ includes similar polynomials and the bijections in \cite{TuFFA19,TuFFA18}. and the functions in \cite{TuBCTIT20}.
    The bijections in \cite[Theorem 1-(2)]{TuFFA18} are included in class $\Gamma_1$ in \cite{TuBCTIT20} which have boomerang uniformity four.
    It seems that these bijections can be given by the function $(x+ax+bx^{2^k})^3$.
    Some recent results in~\cite{QuButterflyDCC21,KimTypeCCDS21}-\cite{NoteButterflyDCC2022}.
    }
    \end{remark}

\section{APN Functions}
    Until now, only a single instance of an APN permutation in even dimension is known, namely for $ n = 6 $\cite{Browning2010sixbitAPN}: 
    \[K:\F_{2^6}\rightarrow\F_{2^6},\quad x\mapsto x^3+x^{10}+gx^{24},\]
    where $ g $ is an element in $ \F_{2^6}^* $ with minimal polynomial $ X^6+X^4+X^3 + X + 1\in \F_2[X] $. 

    We know that two quadratic APN functions are CCZ-equivalent if and only if they are EA-equivalent\cite{Yoshiara2012EAeqCCZ_quadAPN}.

    \begin{table}[!t]
        \caption{All Known APN monomials over $\F_{2^n}$} \label{APN monomials}
        \centering
        \small
        \begin{tabular}{c c cc }	
            \toprule
            Family   &	Function                       & Conditions        & Ref. \\
            \midrule
            Gold     &$ z^{2^i+1}$                     & $ \gcd(i,n)=1$    &  \cite{Gold1968goldfunction_APN_sequences} \\
            Kasami   &$ z^{2^{2i}-2^i+1}$              & $ \gcd(i,n)=1$    &  \cite{Kasami1971kasamifunction_APN_sequences} \\
            Welch    &$ z^{2^t+3}$                     & $ n=2t+1$         &  \cite{Dobbertin1999Welchfunction_APN_sequences} \\
            Niho-1   &$ z^{2^t+2^{t/2}-1}$             & $ n=2t+1, t$ even &  \cite{Dobbertin1999Nihofunction_APN_sequences} \\
            Niho-2   &$ z^{2^t+2^{(3t+1)/2}-1}$        & $ n=2t+1, t$ odd  &  \cite{Dobbertin1999Nihofunction_APN_sequences} \\
            Inverse  &$ z^{2^{2t}-1} $                 & $ n=2t+1$         &  \cite{Nyberg1993differential_and_inversefunction} \\
            Dobbertin&$ z^{2^{4i}+2^{3i}+2^{2i}+2^i-1}$& $ n=5i$           & \cite{Dobbertin2001dobbbertinAPN} \\
            \bottomrule
        \end{tabular}
    \end{table}

    \begin{sidewaystable}[!htbp]
        \caption{All Known APN infinite families with univariate forms (non-monomials) over $\F_{2^n}$} \label{APN polynomials}
        \centering
        \small
        \begin{tabular}{cccc}	
            \toprule
            No.&	Function  & Conditions & Ref. \\
            \midrule
            \begin{tabular}[c]{@{}l@{}} F1- F2\end{tabular} & $z^{2^s+1}+u^{2^k-1}z^{2^{ik}+2^{mk+s}}$ & \begin{tabular}[c]{@{}l@{}} $n=pk, \gcd(k,3)=\gcd(s,3k)=1,$\\ $p\in\{3,4\}, i=sk\pmod p, m=p-i,$\\ $n\ge 12$, $u$ primitive in $\F_{2^n}^*$\end{tabular} & \cite{Budaghyan2008twobinomialapn} \\
            \hline
            F3                                              & \begin{tabular}[c]{@{}l@{}} $sz^{q+1}+z^{2^i+1}+z^{q(2^i+1)}$\\ $+cz^{2^iq+1}+c^qz^{2^i+q}$\end{tabular} & \begin{tabular}[c]{@{}l@{}} $q=2^m$, $n=2m$, $\gcd(i,m)=1,$\\ $c\in\F_{2^n}, s\in\F_{2^n}\setminus\F_q, z^{2^i+1}+cz^{2^i}+$\\ $c^qz+1$ has no solution $z$ with $z^{q+1}=1$\end{tabular}  & \cite{BudaghyanC2008tri_hexanomialAPN}  \\ 
            \hline  \specialrule{0em}{1pt}{1pt}
            F4& $z^3+a^{-1}\operatorname{Tr}_1^n(a^3z^9)$  & $a\ne 0$ & \cite{BudaghyanCL2009newAPN_fromknownone} \\
            \hline \specialrule{0em}{1pt}{1pt}
            F5&  $z^3+a^{-1}\operatorname{Tr}_1^n(a^3z^9+a^6z^{18})$& $3 \mid n, a\ne 0$&  \cite{BudaghyanCL2009newquadAPN} \\
            \hline \specialrule{0em}{1pt}{1pt}
            F6& $z^3+a^{-1}\operatorname{Tr}_1^n(a^6z^{18}+a^{12}z^{36})$ & $3 \mid n, a\ne 0 $ &  \cite{BudaghyanCL2009newquadAPN} \\
            \hline
            \begin{tabular}[c]{@{}l@{}}F7-F9\end{tabular}   & \begin{tabular}[c]{@{}l@{}} $uz^{2^s+1}+u^{2^m}z^{2^{-m}+2^{m+s}}+$\\ $vz^{2^{-m}+1}+wu^{2^m+1}z^{2^s+2^{m+s}}$\end{tabular} & \begin{tabular}[c]{@{}l@{}} $n=3m, \gcd(m,3)=\gcd(s,3m)=1, v,w\in\F_{2^m}$ \\ $vw\neq1, 3 \mid {m+s}, u$ primitive in $\F_{2^n}^{*}$ \end{tabular}               &  \cite{BrackenBMM2011quadAPN} \\
            \hline
            F10 & \begin{tabular}[c]{@{}l@{}} $a^2z^{2^{2m+1}+1}+b^2z^{2^{m+1}+1}+$ \\ $az^{2^{2m}+2}+bz^{2^m+2}+(c^2+c)z^3$ \end{tabular} & \begin{tabular}[c]{@{}l@{}} $n=3m, m$ odd, $L(z)=az^{2^{2m}}+bz^{2^m}+cz$\\ satisfies the conditions of Theorem 2 \end{tabular}                   & \cite{BudaghyanCCCV2020APN_isotopicshifts}  \\
            \hline
            F11 & \begin{tabular}[c]{@{}l@{}} $z^3+wz^{2^{i+1}}+w^2z^{3\cdot 2^m}$\\ $+z^{2^{i+m}+2^m}$ \end{tabular}  & \begin{tabular}[c]{@{}l@{}} $n=2m, m$ odd, $3\nmid m$, $w$ primitive  \\ in $\F_{2^2}, s = m-2, (m-2)^{-1} \pmod n$ \end{tabular}        &  \cite{BudaghyanHK2020quadrinomialAPN}\\
            \hline
            F12 & \begin{tabular}[c]{@{}l@{}} $a\operatorname{Tr}_1^n(bz^3)+a^q\operatorname{Tr}_1^n(b^3z^9)$\\ \end{tabular}   & \begin{tabular}[c]{@{}l@{}}$n=2m, m $ odd, $q=2^m$, $a\notin \F_q,$\\ $b$ not a cube\end{tabular}                   & \cite{Zheng2022newAPN_trace} \\
            \hline
            F13 & $ L(z)^{2^m+1} + vz^{2^m+1} $ & \begin{tabular}[c]{@{}l@{}} $ \gcd(s,m) = 1,v\in\F_{2^m}^*,\mu\in\F_{2^{3m}}^* $ \\ $ L(z)=z^{2^{m+s}} + µz^{2^s} + z $ permutes $ \F_{2^{3m}} $ \end{tabular} & \cite{LiZLQ2022quadraticAPN}\\
            \hline
            F14 & $\begin{aligned} 
                &u\left[\left(u^q x+x^q u\right)^{2^i+1}+\left(u^q x+x^q u\right)\left(x^q+x\right)^{2^i}+\left(x^q+x\right)^{2^i+1}\right]\\
                &+\left(u^q x+x^q u\right)^{2^{2 i}+1}+\left(u^q x+x^q u\right)^{2^{2 i}}\left(x^q+x\right)+\left(x^q+x\right)^{2^{2 i}+1}\end{aligned}$ & $q=2^m, n=2 m, \operatorname{gcd}(3 i, m)=1, u \text { primitive in } \mathbb{F}_{2^*}^*$ & ~\\
            \hline
            15 & $\begin{aligned}
                 &u\left[\left(u^q x+x^q u\right)^{2^i+1}+\left(u^q x+x^q u\right)\left(x^q+x\right)^{2^i}+\left(x^q+x\right)^{2^i+1}\right]\\
                 &+\left(u^q x+x^q u\right)^{2^{3 i}}\left(x^q+x\right)+\left(u^q x+x^q u\right)\left(x^q+x\right)^{2^{3 i}}
            \end{aligned}$ & $m \text { odd, } q=2^m, n=2 m, \operatorname{gcd}(3 i, m)=1, u \text { primitive in } \mathbb{F}_{2^n}^*$& ~\\

            \bottomrule
        \end{tabular}
    \end{sidewaystable}
    %     \caption{All known APN families with bivariate forms over $\F_{2^m}^2$} \label{APN bivariate}
    %     \centering
    %     \small
    
    \begin{sidewaystable}
        \caption{\label{demo-table}Your caption.}
        \begin{tabular}{cccc}	
            \toprule
            No. &	Function  & Conditions & Ref. \\
            \midrule
            F14 & $(xy, x^{2^k+1} + \alpha x^{(2^k+1)2^i})$ & $\gcd(k,m)=1$, $m$ even, $\alpha$ non-cubic &  \cite{ZhouP2012semifieldsAPN} \\
            \hline
            F15 & $(xy, x^{2^{3k}+2^{2k}}+ax^{2^{2k}}y^{2^k}+by^{2^k+1})$ & $\gcd(k,m)=1$, $P_1$ no root in $\F_{2^{m}}$ &  \cite{Taniguchi2019quadraticAPN} \\
            \hline
            F16 & $(xy, x^{2^i+1}+x^{2^{i+m/2}}y^{2^{m/2}}+bxy^{2^i}+cy^{2^i+1})$ & $m$ even, $\gcd(i,m)=1$, $P_2$ no root in $\F_{2^{m}}$ &  \cite{CalderiniBC2020APN_ABfunction} \\
            \hline
            F17 & $(x^{2^i+1}+xy^{2^i}+y^{2^i+1}, x^{2^{2i}+1}+x^{2^{2i}}y+y^{2^{2i}+1})$ & $\gcd(3i,m)=1$ &\cite{Gologlu2020APN_biprojective}\\
            \hline
            F18 & $(x^{2^i+1}+xy^{2^i}+y^{2^i+1}, x^{2^{3i}}y+xy^{2^{3i}})$ & $\gcd(3i,m)=1$, $m$ odd & \cite{Gologlu2020APN_biprojective} \\
            \hline
            F19 & $(x^3+xy^2+y^3+xy, x^5+x^4y+y^5+xy+x^2y^2)$ & $\gcd(3,m)=1$ & \cite{LiZLQ2022quadraticAPN} \\
            \hline
            F20 & $(x^{q+1}+By^{q+1},x^ry+\frac{a}{B}xy^r)$ & \begin{tabular}[c]{@{}l@{}}$0<k<m,q=2^k,r=2^{k+m/2},$\\$m\equiv 2\pmod{4},\gcd(k,m)=1,a\in\F_{2^{m/2}}^*$\\$,B\in\F_{2^m}, B $ note a cube, $ B^{q+r}\ne a^{q+1} $ \end{tabular} & \cite{GolougluK2021biprojectiveAPN} \\
            \hline
            F21 & \begin{tabular}[c]{@{}l@{}} $(x^{q+1}+xy^q+\alpha y^{q+1},$ \\\quad $ x^{q^2+1}+\alpha x^{q^2}y+(1+\alpha)^qxy^{q^2}+\alpha y^{q^2+1})$ \end{tabular}& \begin{tabular}[c]{@{}l@{}} $k,m>0,\gcd(k,m=1),q=2^k,$\\$\alpha\in\F_{2^m},x^{q+1}+x+\alpha $ has no roots in $ \F_{2^m}$  \end{tabular} & \cite{CalderiniLV2022twobivariateAPN} \\
            \hline
            F22 & \begin{tabular}[c]{@{}l@{}} $(x^3+xy+xy^2+\alpha y^3,$\\$x^5+xy+\alpha x^2y^2+\alpha x^4y+(1+\alpha)^2xy^4+\alpha y^5)$\end{tabular} & $\alpha\in\F_{2^m},x^3+x+\alpha$ has no roots in $ \F_{2^m} $ & \cite{CalderiniLV2022twobivariateAPN} \\
            \bottomrule
        \end{tabular}
    \end{sidewaystable}
    % \end{table}

    In \cite{Beierle2022search_quadAPN}, compared to the $ 8192 $ previously-known APN instances, 
    authors found $ 12733 $ new CCZ-inequivalent quadratic APN functions in dimension $ n=8 $.
    They also presented $ 35 $ and $ 5 $ new APN instances in dimension $ n=9 $ and $ n=10 $, respectively.
    Remarkably, two new $ 9 $-bit APN permutations are given in univariate representation over $ \F_{2^9} $ by 
    \begin{align*}
        x&\mapsto x^3+ux^{10}+u^2x^{17}+u^4x^{80}+u^5x^{192},\\
        x&\mapsto x^3+u^2x^{10}+ux^{24}+u^4x^{80}+u^5x^{136},
    \end{align*}
    where $ u\in\F_{2^9}^* $ is an element with minimal Ploynoimal $ X^3+X+1\in\F_2[X] $. 
    \begin{remark}
        Fortunately, Kangquan Li and Nikolay Kaleyski in \cite{Li2022trivariateAPN} 
        have generalized above two new $ 9 $-bit APN permutations into two infinite families:
        
        Assume $ \gcd(i,m)=1 $ and $ q=2^i $, for $ (x,y,z)\in\F_{2^m}^3 $, we have 
        \begin{align*}
            F(x, y, z) =(x^{q+1} + x^qz + yz^q, x^qz + y^{q+1}, xy^q + y^qz + z^{q+1}),\\
            F(x, y, z) =(x^{q+1} + xy^q + yz^q, xy^q + z^{q+1}, x^qz + y^{q+1} + y^qz)
        \end{align*}
        are two APN over $ \F_{2^m}^3 $ with some complex restriction. They are semi-bent functions.
    \end{remark}
    Yuyin Yu and Léo Perrin in \cite{Yu2022search_quadAPN} present another $ 5412 $ new quadratic APN functions. 
    Thus, the number of CCZ-inequivalent quadratic APN functions in dimension $ 8 $ increases to $ 26525 $ (up to April 30, 2021).
    
    \begin{enumerate}
        \item $ x^3+\omega x^{36} $, $ \omega\in \{u\F_{2^5}^*\}\cup\{u^2\F_{2^5}^*\} $ where $ u\in\F_{2^5}^* $ of order $ 3 $ in theorem 2 of \cite{edel2006newapn}
        \item Let $s$ and $k$ be positive integers with $\gcd(s, 3k) = 1$ and let $t \in\{1, 2\}$, $i = 3 - t$. Let further $a = 2s + 1$ and $b = 2^{ik} + 2^{tk+s}$ and let $\omega = \alpha^{2k}-1$ for a primitive element $\alpha \in\F_{2^{3k}}^* $. If $\gcd(2^{3k}-1, (b-a)/(2^k-1)) \ne \gcd(2^k-1, (b-a)/(2^k-1))$,
        the function $F : \F_{2^{3k}}\rightarrow \F_{2^{3k}}, x \mapsto x^a + \omega x^b$ is APN in theorem 1 of 
        \cite{Budaghyan2008twobinomialapn}.
        \item Let $s$ and $k$ be positive integers such that $s \leq 4 k-1$, $\gcd(k, 2)=\gcd(s, 2 k)=1$, and $i=s k \bmod 4, t=4-i$. Let further $a=2^{s}+1$ and $b=2^{i k}+2^{t k+s}$ and let $\omega=$ $\alpha^{2^{k}-1}$ for a primitive element $\alpha \in \mathbb{F}_{2^{4 k}}^{*}$. Then, the function $F: \mathbb{F}_{2^{4 k}} \rightarrow \mathbb{F}_{2^{4 k}}, x \mapsto x^{a}+\omega x^{b}$ is APN in theorem 2 of 
        \cite{Budaghyan2008twobinomialapn}.
        \item Let $k$ and $s$ be odd integers with $\gcd(k, s) = 1$. Let $ b\in\F_{2^{2k}} $ which is not a cube, 
        $ c\in\F_{2^{2k}}\setminus\F_{2^k} $ and $ r_i\in\F_{2^k} $ for all $ i\in\{1,...,k-1\} $, then the function
        $ F:\F_{2^{2k}}\rightarrow\F_{2^{2k}},x\mapsto bx^{2^s+1}+b^{2^k}x^{2^{k+s}+x^k+cx^{2^k+1}+\sum_{i=1}^{k-1}r_ix^{2^{i+k}+2^i}} $ is APN in Theorem 1 of \cite{Bracken2008apn}
        \item Let $k$ and $s$ be positive integers such that $k+s=0 \bmod 3$ and $\gcd(s, 3 k)=\gcd(3, k)=1$. Let further $u \in \mathbb{F}_{2^{3 k}}^{*}$ be primitive and let $v, w \in \mathbb{F}_{2^{k}}$ with $v w \neq 1$. Then, the function
        $$
        \begin{aligned}
        &F: \mathbb{F}_{2^{3 k}} \rightarrow \mathbb{F}_{2^{3 k}} \\
        &x \mapsto u x^{2^{s}+1}+u^{2^{k}} x^{2^{2 k}+2^{k+s}}+v x^{2^{2 k}+1}+w u^{2^{k}+1} x^{2^{k+s}+2^{s}}
        \end{aligned}
        $$
        is APN in Theorem 2.1 of \cite{BrackenBMM2011quadAPN}
        \item 
    \end{enumerate}

\subsection{YU yuying QAM APN}    
    Suppose $\{\alpha_1, \alpha_2, \dots , \alpha_n\}$ is a basis of $\F_{2^n}$ over $\F_2$, let $M_{\alpha}\in\F_{2^n}^{n\times n}$ be a matrix with $M_{\alpha}(i,u)=\alpha_u^{2^{i-1}}$, i.e.,
    \[M_{\alpha}=\begin{pmatrix}
        \alpha_1           & \alpha_2           &\cdots & \alpha_n          \\
        \alpha_1^{2^1}     & \alpha_2^{2^1}     &\cdots & \alpha_n^{2^1}    \\
        \alpha_1^{2^2}     & \alpha_2^{2^2}     &\cdots & \alpha_n^{2^2}    \\
        \vdots             & \vdots             &\ddots & \vdots            \\
        \alpha_1^{2^{n-1}} & \alpha_2^{2^{n-1}} &\cdots & \alpha_n^{2^{n-1}}\\
    \end{pmatrix}.\]
    \begin{definition}
        Let $H$ be an $n\times n$ matrix over $\F_{2^n}$. $H$ is called a quadratic APN matrix (QAM) if 
        \begin{enumerate}[label=(\arabic{*})]
            \item $H$ is symmetric and the elements in its main diagonal are zero.
            \item Every nonzero linear combination of the $n$ rows (or ``columns'' because of $H$ being symmetric) of $H$ has rank $\operatorname{Rank}_{\F_2}=n-1$.
        \end{enumerate}
    \end{definition}
    \begin{theorem}
        Let $F(x)=\sum_{1\le j<i\le n}c_{i,j}x^{2^{i-1}+2^{j-1}}\in\F_{2^n}\left[ x \right]$ be a quadratic homogeneous
        functions and $H=M_{\alpha}^TC_FM_{\alpha}$, where 
        \[C_F(i,j)=\begin{cases}
            c_{i,j}, & i>j\\
            0, & i =j
        \end{cases} \] 
        is a symmetric matrix and $M_{\alpha}$ is defined above.
        Then $F(x)$ is APN iff $H$ is a QAM. 
    \end{theorem}

    \begin{remark}
        \begin{enumerate}[label=(\arabic{*})]
            \item Let $P\in\F_2^{n\times n}$ be an invertible matrix and $H'=P^THP$. Then $H$ is a QAM iff $H'$ is a QAM.
            \item Let $L$ be a linear permutation over $\F_{2^n}$ and $H'=(h_{i,j}')\in\F_{2^n}^{n\times n}$ such that 
            $h_{i,j}'=L(h(i,j))$. Then $H$ is a QAM iff $H'$ is a QAM.
        \end{enumerate}
    \end{remark}

\section{Decomposition of $ \F_{2^{2m}} $} 
    The field extension $ \F_{2^{2m}} $ over $ \F_{2^m} $ has similarities with the extension $ \Com $ over $ \R $.
    The unit circle of $ \F_{2^{2m}} $ is the set 
    \[U=\left\{ u\in\F_{2^{2m}}:u^{2^m+1}=1 \right\}\]
    of all elements with norm $ 1 $.
    Note that $ U\cap\F_{2^m}=\left\{ 1 \right\} $.
    \begin{remark}
        The range set of $ c=u+u^{2^m} $ for $ u\in U\setminus\left\{ 1 \right\} $ can be characterized. Define
        \[\mathcal{H}_1=\left\{ x\in\F_{2^m}:\tr(\frac{1}{x})=1 \right\}.\]
        Then we have 
        \[\mathcal{H}_1=\left\{ u+u^{2^m}:u\in U\setminus\left\{ 1 \right\} \right\}.\]
        More precisely, the mapping $ u\rightarrow\tr(u) $ from $ U\setminus\left\{ 1 \right\} $ to $ \mathcal{H}_1 $ 
        is onto, and $ \tr(u)=\tr(v) $ iff $ u=v $ or $ u=v^{2^m} $ for $ u,v\in U\setminus\left\{ 1 \right\} $. 
    \end{remark}
    Denote $ \mathcal{T}_1=\left\{ x\in\F_{2^{2m}}\mid \operatorname{tr}_m^{2m}(x)=x^{2^m}+x=1 \right\} $. 
    We have, for $ g\in\mathcal{T}_1 $, $ g+g^2\in\F_{2^m} $ and $ \operatorname{tr}_1^m(g+g^2)=1 $, 
    since $ (g+g^2)^{2^m}=g^{2^m}(g+1)^{2^m}=(g+1)g=g+g^2 $ and $ \operatorname{tr}_1^m(g+g^2)=g+g^{2^m}=1 $.

    We can decompose the finite field $ \F_{2^{2m}} $ as follows:
    \begin{enumerate}[label=(\arabic{*})]
        \item $ \F_{2^{2m}}=\F_{2^m}\times\F_{2^m} $, the simplest form.
        \item Polar-Coordinate Decomposition: $ \F_{2^{2m}}=\F_{2^m}\times U $, 
        where $ U=\left\{ x^{2^m-1}\mid x\in\F_{2^{2m}}\right\} $ 
        or $ U=\left\{ x\in\F_{2^{2m}}\mid x^{2^m+1}=1 \right\} $: 
        since $ 2^{2m}-1=(2^m-1)(2^m+1) $ and $ \gcd(2^m-1,2^m+1)=1 $, 
        we decompose $ x\in\F_{2^{2m}} $ into the multiplication of two elements $ \lambda\in\F_{2^m},\varepsilon\in U $ 
        with $ \ord(\lambda)\mid 2^m-1 $ and $ \ord(\varepsilon)\mid 2^m+1 $, respectively. 
        This decomposition is frequently used in the proof of properties of Niho type exponents, 
        since Niho type exponents $ \operatorname{tr}_1^{2m}(x^d) $ are defined to be linear in $ \F_{2^m} $, 
        which means $ d\equiv (2^m-1)s+2^i\pmod{2^{2m}-1} $, and we always assume $ i=0 $. 
        \item Trace-$ 0 $/Trace-$ 1 $ Decomposition: $ \F_{2^{2m}}=\F_{2^m}\times\mathcal{T} $, 
        where Trace-$ 0 $ is $ \F_{2^m} $ due to $ \operatorname{tr}_m^{2m}(x)=0 $ for all $ x\in\F_{2^m} $ and 
        Trace-$ 1 $ means $ \mathcal{T}=\mathcal{T}_1\cup\left\{ 1 \right\} $ in \cite{gologlu2019proofconjecture}. 
        If $ X\in\F_{2^{2m}}^* $ has two decomposition $ xg,yh $, where $ x,y\in\F_{2^m}^* $ and $ g,h\in\mathcal{T} $, 
        then $ \operatorname{tr}_m^{2m}(xg)=\operatorname{tr}_m^{2m}(yh) $. And $ \operatorname{tr}_m^{2m}(xg)=0 $ means $ g=h=1 $, implying $ x=y $, while $ \operatorname{tr}_m^{2m}(xg)\ne 0 $ means $ x=y $, implying $ h=g $. 
    \end{enumerate} 
    Also in \cite{gologlu2019proofconjecture},  with the help of Trace-$ 0 $/Trace-$ 1 $ Decomposition, 
    we can derive a decomposition for Trace-$ 0 $ hyperplane $ \mathcal{H}_0 $ of $ \F_{2^{2m}} $: 
    $ \mathcal{H}_0=\left\{ xg:x\in\mathcal{H}_0^{\F_{2^m}},g\in\mathcal{T} \right\}\cup\F_{2^m} $, 
    where $ \mathcal{H}_0^{\F_{2^m}} $ is the Trace-$ 0 $ hyperplane of $ \F_{2^m} $. 
    The idea is natural, that is, the Trace-$ 0 $ hyperplane of $ \F_{2^{2m}} $ must contain $ \F_{2^m} $, 
    besides, others must satisfy $ \operatorname{tr}_1^m(x)=0 $ with $ x=\operatorname{tr}_m^{2m}(y)=y+y^{2^m}\in\F_{2^m}^* $. 
    Note that $ y\in\F_{2^{2m}} $ can be decomposed by Trace-$ 0 $/Trace-$ 1 $ Decomposition, then $ y=xg $,  
    where $ g\in\mathcal{T} $, can hold this property.

\section{Nonlinearity Profile}
    Only Iwata and Kurosawa gave a class of Boolean functions with higher order nonlinearities in \cite{IwataK1999highorderbentfunction} before 2008. 
    In 2008, Carlet\cite{Carlet2008lowbound_NL_profile} gives two lower bounds for the nonlinearity profile of a Boolean function 
    by the nonlinearity profiles of its derivatives, along with nonlinearity profile of Maiorana-McFarland, Welch, Kasami and Inverse functions.
    In 2009, Sun and Wu deduce the lower bounds of the second-order nonlinearity of three classes of Boolean functions, 
    all of three have high nonlinearity\cite{SunW2009NL_2}. 
    Meanwhile, Sarkar and Gangopadhyay determine a lower bound 
    of the second-order nonlinearity of a new class of cubic Maiorana-McFarland bent functions\cite{SarkarG2009NL_2MM}. 
    Next year, Gangopadhyay et al. deduce the lower bounds of the second order nonlinearity of two types of Boolean functions\cite{GangopadhyayST2010NL_2}. 
    In the same year, Gode and Gangopadhyay obtain a lower bound of the third-order nonlinearities of 
    Kasami functions $ \operatorname{tr}_1^n(\mu x^{57}) $\cite{GodeG2010NL_3Kasami}, the lower bound is sharper than 
    those obtained by Carlet in \cite{Carlet2008lowbound_NL_profile}. Then in 2012, Garg and Khalyavin give tighten 
    bounds on the nonlinearity profile of Kasami functions\cite{GargK2012NLr_Kasami}. 
    Then in \cite{SunW2011NL_2}, Sun and Wu give a lower bound of the second-order nonlinearity of a class of Boolean functions. 
    And in 2011, Singh presents lower bounds of two classes of Boolean functions with high nonlinearities\cite{Singh2011NL_2}. 
    At the same time, Carlet\cite{Carlet2011NL_Profile_Dillon} deduces the nonlinearity profile of the simplest 
    Dillon's Partial Spread bent functions. 
    Tang, Carlet and Tang\cite{TangCT2013NL_2bent} improve the lower bounds in \cite{Carlet2011NL_Profile_Dillon} 
    and obtain nonlinearity profile of a class of Maiorana-McFarland bent functions in 2013. 
    And Li, Hu and Gao obtain a lower bounds on the second order nonlinearity of cubic monomial Boolean functions\cite{LiHG11BNL_2}, which are better than the existing ones\cite{GodeG2009NL_2_cubicmonomial}.
    In the next year, Singh deduce the lower bounds on the third-order nonlinearities of two classes of 
    biquadratic monomial Boolean functions over finite fields\cite{Singh2014NL_3_biquadratic}. 
    Note that Sun and Wu firstly give the higher-order nonlinearities of Niho type Boolean functions\cite{SunW2015NLr_Niho} in 2015. 
    In 2016, Wang and Tan deduce a lower bounds on the nonlinearities of a special class of Boolean functions\cite{WangT2016NL_2}. 
    In 2017, Singh and Paul deduce the lower bounds on the second-order nonlinearities of a class of 
    Boolean functions, and obtain lower bounds on 4th order nonlinearity of $ 10 $-variable monomial Partial Spreads 
    $ \phi(x)=\operatorname{tr}_1^{10}(\lambda x^{2^5-1}) $ where $ \lambda\in\F_{2^{10}}^* $\cite{SinghP2017NL_2_4}. 
    At the same year, Gao and Tang propose a systematic approach for the lower bounds on the second-order nonlinearity of 
    Maiorana-McFarland bent functions\cite{GaoT2017NL_2_MM}. 
    In 2019, Tang et al.\cite{TangYZZ2020NL_2bent} completely determine the distributions of the nonlinearities of 
    the derivatives of a class of bent functions, and present a new lower bound on the second-order nonlinearity of 
    this class of bent functions, which is better than the previous one. 
    In 2020, Tang, Mandal and Maitra\cite{TangMM2022inversefunction} derive cryptographic properties of 
    the multiplicative inverse functions, and give the bounds related to their nonlinearity profile. 
    At the same year, Yan and Tang improve the lower bounds on the second-order nonlinearity of three classes of Boolean functions\cite{YanT2020NL_2}. 
    And Liu\cite{Liu2023NL_2} provide the tight lower bounds on the second-order nonlinearity of 
    three classes of Boolean functions with high nonlinearities. 
    Meanwhile, Sihem, Kwang and Myong\cite{MesnagerKJ2020NL_2cubic} investigate an upper bound 
    on the number of the rational zeros of any linearized polynomial over arbitary finite field, 
    and get tighter estimations of the lower bounds on the second-order nonlinearities of general cubic Boolean functions. 
    In 2022, Singh et al. provide the lower bounds on 4th order nonlinearity of two classes of Boolean functions 
    of degree $ 5 $\cite{SinghPKSV2022res_NL_profile}. 
    Meanwhile, Saini and Garg\cite{SainiG2022NL_r_Niho} present a lower bound on the $ \frac{m}{2} $th-order nonlinearity of a class of 
    bent Boolean function constructed by Dobbertin et al.\cite{DobbertinLCCFG2006bent_Niho}, and the lower bounds 
    are better than the results in \cite{GodeG2009NL_2_cubicmonomial,GodeG2010NL_3Kasami,Singh2014NL_3_biquadratic}. 
    While, Gode, Faruqi and Mishra\cite{GodeFM2022NL_2cubic} obtain improved lower bounds on the second-order 
    nonlinearities of few classes of degree $ 3 $ monomial Boolean functions for $ 7\le n\le 13 $, compared with 
    the lower bounds in \cite{GodeG2009NL_2_cubicmonomial,LiHG11BNL_2,MesnagerKJ2020NL_2cubic}.

    \begin{theorem}\label{TH:NL_onederivate}
        Let $f$ be any $ n $-variable function and $ r $ a positive integer smaller than $ n $. We have:
        \[nl_r(f)\ge\frac{1}{2}\max_{a\in\F_{2^n}}nl_{r-1}(D_af).\]
        Obviously, above inequality can be repeatedly applied: 
        \[nl_r(f)\ge \frac{1}{2^i}\max_{a_1,\dots,a_i\in\F_{2^n}}nl_{r-i}(D_{a_1}\cdots D_{a_i}f).\]
    \end{theorem}
    Another potentially stronger lower bound, given a lower bound on the $ (r-1) $-th order nonlinearity is 
    known for all the derivatives (in nonzero directions) of the function:
    \begin{theorem}\label{TH:NL_allderivates}
        Let $f$ be any $ n $-variable function and $ r $ a positive integer smaller than $ n $. We have:
        \[nl_r(f) \ge 2^{n-1} -\frac{1}{2}\sqrt{2^{2n}-2\sum_{a\in\F_{2^n}}nl_{r-1}(D_af)}.\]
    \end{theorem}
    \begin{remark}
        {\color{red}Theorem \ref{TH:NL_allderivates} is not always better than Theorem \ref{TH:NL_onederivate}}: 
        In \cite{SihemGJDK2017lowerbound_nl_2_notalwaysbetter}, Sihem et al. construct a class of Boolean functions where the bound on the second order nonlinearity in Theorem \ref{TH:NL_onederivate} is tight but the bound in Theorem \ref{TH:NL_allderivates} is strictly worse than the former. 
    \end{remark}
    
    In \cite{dobbertin2006niho_dickson_kloosterman}, Dobbertin et al. studied the cross-correlation function between two 
    $ m $-sequences for Niho type decimation $ d=(2^k-1)s+1 $. 
    They derived the distribution of six-valued cross-correlation function for $ s=3 $ and odd $ k $. 
    
\section{APN decomposition}
    The decomposition \cite{BeierleCLP2022ninebitAPNpermutation_error} we introduced here
    is based on the TU-decomposition, but before investigating this pattern, we first need the notion of Walsh zeros.
    \begin{definition}
        Let $ \mathcal{V} $ and $ \mathcal{W} $  be finite-dimensional $\F_2 $-vector spaces, 
        with $ \dim_{\mathcal{V}}=n,\dim_{\mathcal{V}}=m $, respectively. 
        The Walsh zeroes of a function $ F : \mathcal{V}\rightarrow \mathcal{W} $, denoted by $ \mathcal{Z}_F $, 
        is the set of the coordinates of the zeroes in its Walsh spectrum together with $ (0, 0) $, i.e.,
        \[\mathcal{Z}_F=\left\{ (a,b)\in\mathcal{V}\times\mathcal{W}\mid W_F(a,b)=0 \right\}\cup\left\{ (0,0) \right\}.\]
    \end{definition}
    Note that Walsh zeros of a vBF must contain $ \left\{ (0,x):x\in\mathcal{V} \right\} $, so discussing Walsh zeros 
    is valid for all vBF. 
    
    For two functions $ F $ and $ G $ mapping from $ \F_2^n $ to $ \F_2^m $, 
    if $ \Gamma_G=\mathcal{A}(\Gamma_F) $ for some affine permutation $ \mathcal{A} $ of $ \F_2^{n+m} $, 
    then $ \mathcal{Z}_G=(\mathcal{L}^T)^{-1}(\mathcal{Z}_F) $, where $ \mathcal{L} $ is the linear part of $ \mathcal{A} $. 
    Thus, there exists a linear subspace $ V $ of $ \mathcal{Z}_G $ with dimension $ n $, 
    such that $ \mathcal{L}^T(V)=\left\{ (0,x):x\in\mathcal{V} \right\} $. 
    Therefore, we have 


\section{An APN instance with the worst linearity $ 2^7 $}
    \begin{align*}
        x \mapsto & x^3+g^{60} x^5+g^{191} x^6+g^{198} x^9+g^{232} x^{10}+g^{120} x^{12}+ \\
        & g^{54} x^{17}+g^{64} x^{18}+g^{159} x^{20}+g^{144} x^{24}+g^{248} x^{33}+\\
        & g^{203} x^{34}+g^{32} x^{36}+g^{18} x^{40}+g^{216} x^{48}+g^{78} x^{65}+ \\
        & g^{46} x^{66}+g^{91} x^{68}+g^{27} x^{72}+g^{70} x^{80}+g^{52} x^{96}+ \\
        & g^{224} x^{129}+g^{18} x^{130}+g^{197} x^{136}+g^{253} x^{144}+x^{160},
    \end{align*}
    where $ g\in\F_{2^8}^* $ is an element with minimal polynomial $ X^8+X^4+X^3+X^2+1\in\F_2[X] $.


\section{Algebraic Geometry} 
    The main goal of algebraic geometry is to study solution sets of polynomial equations in several variables. 
    So, in its easiest form, if $f_1,\dots, f_k \in K\left[ x_1,..., x_n \right] $ are polynomials in $ n $ variables over a given ground field $ K $ we want to consider the set 
    \[X=\left\{ x\in K^n:f_1(x)=\cdots=f_k(x)=0 \right\},\]
    which is called an (affine) variety.

    Genera is the plural form of genus.

    Let $ q $ be a prime power. For the finite field $ \F_q $, let $ F $ be an algebraic function field with the full constant field $ \F_q $.
    
    An algebraic function field of $ n $ variables over a field $ K $ is finitely generated field extension $ K/k $ which has transcendence degree $ n $ over $ k $. 
    As an example, in the polynomial ring $ k\left[ X,Y \right] $, consider the idea l generated by the irreducible polynomial $ Y^2-X^3 $ and form the field of fractions of the quotient ring $ k\left[ X,Y \right]/\left( Y^2-X^3 \right) $. 

    $ (0:0:0) $ is not a point at infinity. 
    

    the APN functions $x^{81}$ over $\F_{2^9}$ and $x^5$ over $\F_{2^7}$ have been respectively used in MISTY and KASUMI block ciphers

\section{Regular Expression}
    We can use the regular expression to find some gibberishes, 
    % [^ -~\u2E80-\u2FDF\u3040-\u318F\u31A0-\u31BF\u31F0-\u31FF\u3400-\u4DB5\u4E00-\u9FFF\uA960-\uA97F\uAC00-\uD7FF\u3002\u00a5\uff1f\uff01\uff0c\u3001\uff1b\uff1a\u201c\u201d\u2018\u2019\uff08\uff09\u300a\u300b\u3008\u3009\u3010\u3011\u300e\u300f\u300c\u300d\ufe43\ufe44\u3014\u3015\u2026\u2014\uff5e\ufe4f\uffe5\u00a5]+

\bibliographystyle{plain}
\bibliography{ref.bib}
\end{document}

n:=8;
F<v>:=GF(2,n);
f:=func<x|(x eq 0) select 0 else x^-1>;
for beta in F do
    list_rangev:=[];
    for v in F do
        count:=0;
        for x in F do
            for y in F do
                if y^2+y eq f(x+beta)+f(x)+v*x then 
                    count+:=1;
                    break;
                end if;
            end for;
        end for;
    Append(~list_rangev,Abs(count-2^(n-1)));
    end for;    
    print "beta=",beta;
    print "list_rangev=",list_rangev;
    print "max_list_rangev=",Max(list_rangev);
end for;


n=4, tr=0的x的数量-2^(n-1)=8
n=6, tr=0:16
n=8, tr=0:32
n=10, tr=0:60, 计算量比较大, 没看到64, 但应该也是64. 


k<w> := GF(2,6);
kt<t> := PolynomialRing(k);
kty<y> := PolynomialRing(kt);

for beta in F do
    if beta eq 0 then continue; end if;
    Genus_v_list:=[];
    for v in F do 
        f:=FunctionField(beta+v*beta*t^2*(t+beta)+(y^2+y)*(t+beta)*t);
        Append(~Genus_v_list,Genus(f));
    end for;
    Genus_v_list;
end for;



for beta in F do
Max([Abs(&+[(-1)^(Integers()!Trace(f(x+beta)+f(x)+v*x)):x in F]):v in F]);
end for;


for beta in GF(2,3) do
    // if beta in GF(2,3) then continue; end if;
    count_v:=[];
    for v in F do
        count:=0;
        for x in GF(2,3) do
            if Trace(f(beta+x)+f(x)+v*x) eq 0 then 
                count+:=1;
            end if;
        end for;
        Append(~count_v,count);
    end for;
    count_v;
end for;