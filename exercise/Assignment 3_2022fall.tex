\documentclass[a4paper,12pt]{ctexart}
\usepackage{fullpage,enumitem,amsmath,amssymb,graphicx}
\newcommand{\Z}{\mathbf{Z}}
\newcommand{\F}{\mathbf{F}}
\newcommand{\Com}{\mathbf{C}}
\newcommand{\ord}{\operatorname{ord}}
\newcommand{\Q}{\mathbf{Q}}
\newcommand{\R}{\mathbf{R}}


\title{NIS2312-1 2022-2023 Fall Homework~1}
\author{唐灯}



\begin{document}
%   \maketitle
  \begin{center}

  \vspace{-0.3in}
  \begin{tabular}{c}
    \textbf{\Large NIS2312-1 2022-2023 Fall} \\
    \textbf{\Large  } \\
    \textbf{\Large  信息安全的数学基础(1)} \\
    \textbf{\Large  } \\
    \textbf{\Large  Assignment~3} \\
    \textbf{\Large  } \\
    \textbf{\Large 2022年9月21日} \\
  \end{tabular}
  \end{center}
  \noindent
  \rule{\linewidth}{0.4pt}
  
%   可以使用计算机求模的运算.
  
  \subsubsection*{Problem 1} 
      设 $ f: A\rightarrow B, ~g: B\rightarrow C $ 和 $ h: C\rightarrow D $为集合之间的映射, 证明: 
      \begin{align*}
          (h\circ g)\circ f=h\circ (g \circ f).
      \end{align*}

      解: 只需证明 $ \forall x\in A $, 都有 $ (h\circ g)\circ f(x)=h\circ (g \circ f)(x) $ 成立即可.
      因此任意选取 $ x\in A $, 则 $ h\circ(g\circ f)(x)=h(g\circ f(x))=h(g(f(x))=h\circ g(f(x))=(h\circ g)\circ f(x) $,
      又因为 $ x $是集合 $ A $中的任意元素, 所以得到结论
      $ \forall x\in A $, 都有 $ (h\circ g)\circ f(x)=h\circ (g \circ f)(x) $ 成立. Q.E.D.
\subsubsection*{Problem 2}
    证明: 映射 $ f: A\rightarrow B $ 是一一映射的充分必要条件是 $ f $是可逆映射.
      
    解: $ \Leftarrow $: $ f $是可逆映射, 假设其逆映射为 $ f^{-1} $, 下证 $ f $ 为一一映射:
    假设 $ f(a)=f(a') $, 
    那么 $ a=1_A(a)=(f^{-1}\circ f)(a)=f^{-1}(f(a))=f^{-1}(f(a'))=(f^{-1}\circ f)(a')=1_A(a')=a' $, 故 $ f $
    是单射; 假设 $ b\in B $, 则 $ f(f^{-1}(b))=(f\circ f^{-1})(b)=1_B(b)=b $, 也就是说 $ b\in f(A) $, 因此 $ f $
    是满射; 综上, $ f $是一一映射.

    $ \Rightarrow $: 设 $ f $是一一映射, 则定义一个新映射 $ g: B\rightarrow A $, 
    其中 $ g(b)=w $且 $ f(w)=f(g(b))=b $. 因此有 $ f\circ g=1_B $; 
    同时根据一一映射, 可知$ (g\circ f)(a)=g(f(a))=a $, 即 $ g\circ f=1_A $; 综上, $ f $是可逆映射. Q.E.D. 
% 
\subsubsection*{Problem 3}
    设 $ f: A\rightarrow B $ 和 $ g: B\rightarrow C $ 都是一一映射, 证明: $ g \circ f: A\rightarrow C $也
    是一一映射, 并且 $ ( g \circ f)^{−1} = f^{−1}\circ g^{−1}$. 
    
    解: 根据引理54可知 $ f,~g $均为可逆映射, 则
    \begin{align*}
        (f^{-1} \circ g^{-1})\circ(g\circ f)& = f^{-1} \circ (g^{-1}\circ(g\circ f))\\
         &= f^{-1} \circ ((g^{-1}\circ g)\circ f)\\
         & = f^{-1} \circ (1_B\circ f)\\
         & = f^{-1} \circ f = 1_A. 
    \end{align*}
    同理, 可以得到 $ (g\circ f)\circ(f^{-1} \circ g^{-1})=1_B $. 因此证明了 $ g \circ f $ 是可逆映射, 
    根据引理54可知 $ g\circ f $为一一映射, 同时得到 $ (g \circ f)^{−1} = f^{−1}\circ g^{−1} $. Q.E.D.
\end{document}