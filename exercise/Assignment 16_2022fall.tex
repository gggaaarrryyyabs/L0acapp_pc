\documentclass[a4paper,12pt]{ctexart}
\usepackage{fullpage,enumitem,amsmath,amssymb,graphicx}
\newcommand{\Z}{\mathbb{Z}}
\newcommand{\F}{\mathbb{F}}
\newcommand{\Com}{\mathbf{C}}
\newcommand{\ord}{\operatorname{ord}}
\newcommand{\Q}{\mathbb{Q}}
\newcommand{\R}{\mathbb{R}}


\title{NIS2312-1 2022-2023 Fall Homework~1}
\author{唐灯}



\begin{document}
%   \maketitle
  \begin{center}

  \vspace{-0.3in}
  \begin{tabular}{c}
    \textbf{\Large NIS2312-1 2022-2023 Fall} \\
    \textbf{\Large  } \\
    \textbf{\Large  信息安全的数学基础(1)} \\
    \textbf{\Large  } \\
    \textbf{\Large  Answer~16} \\
    \textbf{\Large  } \\
    \textbf{\Large 2022年11月21日} \\
  \end{tabular}
  \end{center}
  \noindent
  \rule{\linewidth}{0.4pt}
  
%   可以使用计算机求模的运算.

\subsubsection*{Problem 1}
    在特征是素数 $ p $的交换环 $ R $中, $ k\ge 1 $为正整数. 证明如下公式:
    \[\left( a+b \right)^{p^k}=a^{p^k}+b^{p^k}.\]

    解: 

    解: 使用 Lucas's Theorem, 可以确定 
    \[ \binom{p^k}{n}\equiv\prod_{i=0}^k\binom{m_i}{n_i}\pmod{p}, \]
    其中 
    \[p^k=m_kp^k+m_{k-1}p^{k-1}+\cdots+m_1p+m_0,\]
    和 
    \[n=n_kp^k+n_{k-1}p^{k-1}+\cdots+n_1p+n_0.\]
    显然 $ m_k=1$且$ m_i=0 $, 其中 $ i=0,1,...,k-1 $. 
    
    因此对任意 $ 1\le n<p^k $, 上式可以化简为  
    \[ \binom{p^k}{n}=\binom{1}{0}\equiv 0\pmod{p}.  \]
    因此在特征为 $ p $的交换环中, 等式 $ \left( a+b \right)^{p^k}=a^{p^k}+b^{p^k} $成立.


\subsubsection*{Problem 2} 
    假设 $ F $是一个四个元素的域. 证明:
    \begin{enumerate}[label=(\arabic{*})]
      \item $ F $的特征是 $ 2 $;
      \item $ F $的不等于 $ 0 $和单位元的两个元素均满足方程 $ x^2+x+1=0 $.
    \end{enumerate}  

    解:\begin{enumerate}[label=(\arabic{*})]
      \item $ 4 $的素因子只有 $ 2 $, 因此 $ Char(F)=2 $;
      \item 显然 $ F=\left\{ 0,1,x,x+1 \right\} $, 因为 $ x^2\in F $, 且 $ x^2\ne 0,1 $, 故 $ x^2=x+1 $.
    \end{enumerate}

\subsubsection*{Problem 3}
    证明: $ \langle x^3+x+1\rangle $ 是 $ \mathbf{Z}_2[x] $的极大理想.
    因此 $ \mathbf{Z}_2[x]/\langle x^3+x+1\rangle $是一个域, 给出这个域中非零元素的乘法表.

    解: 极大理想: 
    
    可以证明, 任意 $ ax^2+bx+c+\langle x^3+x+1\rangle $ 有逆元. 因为 
    \[ (ax^2+bx+c)(dx^2+ex+f)\equiv(a(d+f)+be+cd)x^2+(a(d+e)+b(d+f)+ce)x+(ae+bd+cf)\pmod{x^3+x+1} .\]
    注意到对非全零的三个变量 $ a,b,c\in\mathbf{Z}_2 $, 方程组 
    \begin{align*}
      ae+bd+cf&=1\\
      a(d+f)+be+cd&=0\\
      a(d+e)+b(d+f)+ce&=0
    \end{align*} 
    都有解, 则 $ \mathbf{Z}_2[x]/\langle x^3+x+1\rangle $是一个域, 故 $ \langle x^3+x+1\rangle $ 是 $ \mathbf{Z}_2[x] $的极大理想.

    另解: 设 $ f(x)\in\Z_2[x] $且 $ f(x)\in J $, 其中 $ I\subseteq J\subseteq\Z_2[x] $, 有 $ f(x)=ax^2+bx+c+q(x^3+x+1) $, 
    因此 $ ax^2+bx+c\in J $.

    考虑 $ a,b,c\in\F_2 $但不全为 $ 0 $的情况:
    \begin{enumerate}
      \item[$ a=b=c=1 $] 此时有 $ x^2+x+1\in J $, 故 $ (x^2+x+1)x+(x^3+x+1)=x^2+1\in J $, 进一步有 $ (x^2+1)+(x^2+x+1)=x\in J $, 
      则有 $ xx+(x^2+1)=1\in J $, 即 $ J=\Z_{2}[x] $.
      \item[$ a=b=1,c=0 $] 此时有 $ x^2+x\in J $, 故 $ (x^2+x)x+(x^3+x+1)=x^2+x+1\in J $, 其余的步骤同上.
      \item[$ a=c=1,b=0 $] 此时有 $ x^2+1\in J $, 其余的步骤同第一种情况.
      \item[$ a=1,b=c=0 $] 此时有 $ x^2\in J $, 故 $ x^2+(x^2+x+1)=x+1\in J $, 故 $ (x+1)x=x^2+x\in J $, 其余步骤同第二种情况.
      \item[$ a=0,b=c=1 $] 此时有 $ x+1\in J $, 其余的步骤同第四种情况. 
      \item[$ a=c=0,b=1 $] 此时有 $ x\in J $, 其余的步骤同第一种情况.
      \item[$ a=b=0,c=1 $] 显然 $ 1\in J $ 可以得到 $ J=\Z_2[x] $.
    \end{enumerate}
    综上所述, $ \langle x^3+x+1\rangle $ 是 $ \mathbf{Z}_2[x] $的极大理想.
    % 设 $ I $为 $ \mathbf{Z}_2[x] $的任一理想且满足 $ \langle x^3+x+1\rangle\subsetneq I $. 
    % 故设 $ f(x)\in I\setminus\langle x^3+x+1\rangle $, 因此有 $ q(x),ax^2+bx+c\in\mathbf{Z}_3[x] $ 满足 
    % \[f(x)=(x^3+x+1)q(x)+ax^2+bx+c.\]
    % 从而 
    % \[ax^2+bx+c=f(x)+(x^3+x+1)f(x)\in I.\]
    % 因为 $ f(x)\notin I $, 从而 $ ax^2+bx+c\notin\langle x^3+x+1\rangle $, 所以 $ a,b,c $不全为 $ 0 $.
    % \begin{enumerate}[label=(\arabic{*})]
    %   \item  在 $ \mathbf{Z}_2[x] $中, 如果 $ a\ne 0 $, 则 $ a^2+b^2+c^2\ne 0 $, 且 
    %   \[a^2+b^2+c^2=\]
    % \end{enumerate}

    下面是 $\mathbf{Z}_2[x] /\langle x^3+x+1\rangle$的乘法表
    \[\begin{array}{r|ccccccc}
      \hline
       & 1 &\alpha& 1+\alpha & \alpha^2 & 1+\alpha^2 & \alpha+\alpha^2 & 1+\alpha+\alpha^2\\
       \hline
      1 & 1 &\alpha &1+\alpha & \alpha^2 & 1+\alpha^2 & \alpha+\alpha^2 & 1+\alpha+\alpha^2\\
      \alpha & \alpha & \alpha^2 & \alpha+\alpha^2 & 1+\alpha  & 1 &1+\alpha+\alpha^2 & 1+ \alpha^2\\
      1+\alpha & 1+\alpha & \alpha+\alpha^2 & 1+\alpha^2 & 1+\alpha+\alpha^2  & \alpha^2 &1& 1+ \alpha^2\\
      \alpha^2 & \alpha^2 & 1+\alpha & 1+\alpha+\alpha^2 & \alpha+\alpha^2 & \alpha &1+\alpha^2 & 1\\
      1+\alpha^2 & 1+\alpha^2 & 1 &  \alpha^2 &  \alpha & 1+\alpha+\alpha^2 &1+\alpha& \alpha+ \alpha^2\\
      \alpha+\alpha^2 & \alpha+\alpha^2 & 1+\alpha+\alpha^2 &  1&1+\alpha^2 & 1+\alpha & \alpha&\alpha^2\\
      1+\alpha+\alpha^2 & 1+\alpha+\alpha^2 & 1+\alpha^2 &\alpha&1 & \alpha+\alpha^2 & \alpha^2&1+\alpha\\
      \hline
    \end{array}\]

\end{document}

