\documentclass[a4paper,12pt]{ctexart}
\usepackage{fullpage,enumitem,amsmath,amssymb,graphicx}
\usepackage{tikz-cd}
\newcommand{\Z}{\mathbf{Z}}
\newcommand{\F}{\mathbf{F}}
\newcommand{\Com}{\mathbf{C}}
\newcommand{\ord}{\operatorname{ord}}
\newcommand{\Q}{\mathbf{Q}}
\newcommand{\R}{\mathbf{R}}


% \title{NIS2312-2 Spring 2022 Homework~1}
% \author{唐灯}



\begin{document}
%   \maketitle
  \begin{center}

  \vspace{-0.3in}
  \begin{tabular}{c}
    \textbf{\Large NIS2312-1 Spring 2021-2022} \\
    \textbf{\Large  } \\
    \textbf{\Large  信息安全的数学基础(1)} \\
    \textbf{\Large  } \\
    \textbf{\Large  Answer 5} \\
    \textbf{\Large  } \\
    \textbf{\Large 2022年3月31日} \\
  \end{tabular}
  \end{center}
  \noindent
  \rule{\linewidth}{0.4pt}

  \begin{enumerate}
    \item 如果 $ H_1,H_2\triangleleft G $ 且 $ H_1\cong H_2 $, 那么 $ G/H_1\cong G/H_2 $. (不成立)
    \item[反例:] $ G=\Z $ 且 $ H_1=2\Z,H_2=3\Z $, 但是 $ G/H_1 $和 $ G/H_2 $不同构.
    \item 如果 $ H_1,H_2\triangleleft G $ 且 $ G/H_1\cong G/H_2 $, 那么 $ H_1\cong H_2 $. (不成立)
    \item[反例:] $ G=\Z_4\times\Z_2 $, $ H_1=\Z_4\times\{0\},H_2=\Z_2\times\Z_2 $, 但是 $ H_1 $和 $ H_2 $不同构.
  \end{enumerate}


  在这个练习中, 默认 $G$ 是一个群, $H$ 和 $K$ 是 $ G $的子群.
\subsubsection*{Problem 1}
    对任意 $ n\in\Z^+ $, 证明 $ \Z/n\Z\cong\Z_n $.

    证明: 对任意 $ n\in\Z^+ $ 构造一个映射
    \[ \begin{array}{cccc}
      \phi:& \Z&\rightarrow&\Z_n\\
      &a&\mapsto&\bar{a}
    \end{array} \]
    \begin{enumerate}
      \item 显然 $ \phi $是一个映射;
      \item 对任意 $ \bar{a}\in\Z_n $, 我们都有 $ a\in\Z $使得 $ \phi(a)=\bar{a} $, 因此是个满映射;
      \item 有 $ \phi(a+b)=\overline{a+b}=\overline{a}+\overline{b}=\phi(a)+\phi(b),\forall a,b\in\Z $, 因此是个满同态;
      \item 显然 $ ker(\phi)=\{x\in\Z\mid \phi(x)=0\}=\{x\in\Z\mid n\mid x\}=n\Z $
    \end{enumerate}
    因此同态基本定理可以得到 $ \Z/n\Z\cong \Z_n $.
\subsubsection*{Problem 2}
    第二同构定理: 给定群 $ G $和其正规子群 $ N $, 子群 $ H $, 证明:
    \begin{enumerate}
      \item $ HN $是一个群且 $ N\triangleleft HN $
      \item $ N\cap H $是 $ H $的正规子群;
      \item 有群同构: $ H/(N\cap H)\cong HN/N  $. 
    \end{enumerate}
    \begin{tikzcd}
      &HN \arrow[dl,dash] \arrow[dr,dash]&\\
      N\arrow[dr,dash] & & H \arrow[dl,dash] \\
      &N\cap H &
    \end{tikzcd}

    注意:书上习题2-1的14题, 看起来形式和第二同构定理非常相似, 但是证明不能使用此定理, 至少用此定理无法覆盖全部的情况, 因为如果 $ N $不是正规子群的话 (或者其他条件), $ HN $不一定是个群.

    解:\begin{enumerate}
      \item $ \forall hn,h'n'\in HN $, 且 $ N\triangleleft G $,  有 $ hn(h'n')^{-1}=hnn'^{-1}h'^{-1}=hh'^{-1}h'nn'^{-1}h'^{-1}\in HN $, 因此 $ HN< G $, 则 $ HN $是个群. 同时 $ \forall hn\in HN<G $, 有 $ hnN(hn)^{-1}=N $ 因为 $ N\triangleleft G $.
      \item 2 和 3 一起证明.  对任意 $ n\in\Z^+ $ 构造一个映射
      \[ \begin{array}{cccc}
        \phi:& H&\rightarrow&HN/N\\
        &h&\mapsto&hN
      \end{array} \]
      \begin{enumerate}
        \item 显然 $ \phi $是 $ H $到 $ HN $的映射;
        \item $ \forall hn\in HN $, 有 $ h\in H $使得 $ \phi(h)=hN $, 因此是满映射;
        \item $ \forall h_1,h_2\in H $, 有 $ \phi(h_1h_2)=h_1h_2N=h_1Nh_2N=\phi(h_1)\phi(h_2) $, 因此是满同态;
        \item $ ker(\phi)=\{h\in H\mid \phi(h)=N\}=\{h\in H\mid hN=N\}=\{h\in H\mid h\in N\}=H\cap N $.
      \end{enumerate}
      则利用同态基本定理得到 $ H/(H\cap N)\cong HN/N $, 同时得到 $ H\cap N\triangleleft H $.
    \end{enumerate}
\subsubsection*{Problem 3}
    假设 $ C\triangleleft A $并且 $ D\triangleleft B $, 证明: $ C\times D\triangleleft A\times B $和 $ (A\times B)/(C\times D)\cong(A/C)\times(B/D) $ (直积保持正规性).
    
    证明: 构造映射:\[ \begin{array}{cccc}
      \phi:& A\times B&\rightarrow&A/C\times B/D\\
      &(a,b)&\mapsto&(aC,bD)
    \end{array} \]
    \begin{enumerate}
      \item 显然 $ \phi $是一个映射;
      \item $ \forall (aC,bD)\in A/C\times B/D $, 有 $ (a,b)\in A\times B $使得 $ \phi((a,b))=(aC,bD) $, 即 $ \phi $是一个满映射;
      \item $ \forall (a,b),(a',b')\in A\times B $, 有 $ \phi((a,b)(a',b'))=\phi((aa',bb'))=(aa'C,bb'D)=(aCa'C,bDb'D)=(aC,bD)(a'C,b'D)=\phi((a,b))\phi((a',b')) $, 因此是个满同态;
      \item $ ker(\phi)=\{(a,b)\in A\times B\mid \phi((a,b))=(C,D)\}=\{(a,b)\in A\times B\mid aC=C~and~bD=D\}=C\times D $.
    \end{enumerate}
    则利用同态基本定理得到 $ A\times B/(C\times D)\cong A/C\times B/D $并且 $ C\times D\triangleleft A\times B $.
\subsubsection*{Problem 4}
    假设 $ G=H_1H_2\cdots H_n $, 且对任意 $ i\in\{1,2,\dots,n\} $都有 $ H_i\triangleleft G $. 证明下面的条件是等价的:
    \begin{enumerate}
      \item $ G $是 $ H_i $的内直积, 其中 $ i=1,2,\dots,n $;
      \item $ H_1H_2\cdots H_{i-1}\cap H_i=\{e\},~\forall i=2,3,\dots,n $;
      \item $ H_1\cdots H_{i-1}H_{i+1}\cdots H_n\cap H_i=\{e\},~\forall i=1,2,\dots,n $.
    \end{enumerate}

    证明: (方法不少, 仅提供一种, 也可以3-2-1-3 这样证明, ) \begin{enumerate}
      \item[$1\rightarrow 2$] 这是定义, 直接得到结果
      \item[$2\rightarrow 3$] $ \forall h_i\in H_i,h_j\in H_j $, 有 $ h_ih_jh_i^{-1}h_j^{-1}=h_ih_jh_i^{-1}h_j^{-1}\in h_jH_j=H_j $, 
      同理可得 $ h_ih_jh_i^{-1}h_j^{-1}\in H_i $. 又因为 (2), 我们得到 
      $ H_i\cap H_j=\{e\},\forall i\neq j $. 所以 $ h_ih_jh_i^{-1}h_j^{-1}\in H_i\cap H_j=\{e\} $. 
      故 $ h_ih_j=h_jh_i $.
      
      所以用反证法: 假设 $ g=h_1h_2\cdots h_{i-1}h_{i+1}\cdots h_n=h_i\neq e $,
      因此 $ h_1h_2\cdots h_{i-1}h_i^{-1}h_{i+1}\cdots h_n=e $, 
      即 $ h_1h_2\cdots h_{i-1}h_i^{-1}h_{i+1}\cdots h_{n-1}=h_n^{-1} $, 在 (2)中, 取$ i=n $, 
      得到 $ h_n=e $, 以此类推, 得到 $ h_i=e,\forall i=1,2,3,\dots,n $, 即 $ g=e $与假设矛盾, 
      所以 (3) 成立.
      \item[$3\rightarrow 1$] 由 (3)得到 (2)是显然的. 同时类似 $ 2\rightarrow 3 $有 $ h_ih_j=h_jh_i $.
      假设 $ g $有两种表示方法 $ h_1h_2\cdots h_n $和 $ h_1'h_2'\cdots h_n' $, 
      因此 $ (h_1h_1'^{-1})(h_2h_2'^{-1})\cdots (h_{n-1}h_{n-1}'^{-1})=h_nh_n'^{-1} $, 由 (3)
      得到 $ h_ih_i'^{-1}=e $, 所以 $ g $是唯一表示的, (1)得证.  
    \end{enumerate}

\end{document}