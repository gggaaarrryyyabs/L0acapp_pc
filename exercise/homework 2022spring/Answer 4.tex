\documentclass[a4paper,12pt]{ctexart}
\usepackage{fullpage,enumitem,amsmath,amssymb,graphicx}
\newcommand{\Z}{\mathbf{Z}}
\newcommand{\F}{\mathbf{F}}
\newcommand{\Com}{\mathbf{C}}
\newcommand{\ord}{\operatorname{ord}}
\newcommand{\Q}{\mathbf{Q}}
\newcommand{\R}{\mathbf{R}}


% \title{NIS2312-2 Spring 2022 Homework~1}
% \author{唐灯}



\begin{document}
%   \maketitle
  \begin{center}

  \vspace{-0.3in}
  \begin{tabular}{c}
    \textbf{\Large NIS2312-1 Spring 2021-2022} \\
    \textbf{\Large  } \\
    \textbf{\Large  信息安全的数学基础(1)} \\
    \textbf{\Large  } \\
    \textbf{\Large  Answer 4 } \\
    \textbf{\Large  } \\
    \textbf{\Large 2022年3月17日} \\
  \end{tabular}
  \end{center}
  \noindent
  \rule{\linewidth}{0.4pt}

      % \begin{enumerate}
        %   \item 一般而言, 大写字母代表集合(比如$S$一般用于代表集合(set), $G$一般用于表示群(group), $ R $一般用于表示环(ring)), 小写字母代表集合中的元素(比如 $ p $一般用于表示素数(prime), $ a,g $一般用于表示群的元素等等)
        %   \item 群中的运算如果不作特别说明不一定满足交换律(commutative), 运算满足交换律的群叫交换群或者阿贝尔群
        %   \item 今后, 如果不作特别说明, 总假定群的运算是``乘法''. cf P8 例 4 上面那一段
        %   \item 群中的乘法不一定是我们熟知的乘法, 比如 $ SL_n(\R) $的乘法就是矩阵乘法, 四面体群(dihedral group)的乘法是函数的复合, 甚至可以把整数加法群中的整数加法运算叫做``乘法''
        %   \item 群的运算如果是``加法''的话, 加法单位元(或者叫零元)常用 $0$表示, 元素 $a$的加法逆元记作 $ -a $
        %   \item 群的运算如果是``乘法''的话, 乘法单位元常用 $e$表示, 元素 $a$的乘法逆元记作 $ a^{-1} $
        %   \item $ \left\langle a\right\rangle=\{a^r\mid r\in\Z\}  $由元素 $ a $生成的循环群
        % %   \item 如果 $ S=\{a_1,a_2,\dots,a_r\} $, 则记 $ \langle S\rangle=\langle a_1,a_2,\dots,a_r\rangle=\{a_1^{l_1}a_2^{l_2}\cdots a_k^{l_k}\mid a_i\in S,\} $
        %   \item 对于群 $ G $中的集合 $ S,T $和元素 $ x $, $ Sx=\{ax\mid a\in S\} $, $ xS=\{xa\mid a\in S\} $(陪集,cf P66), $ xSx^{-1}=\{xax^{-1}\mid a\in S\} $, $ ST=\{ab\mid a\in S,b\in T\} $
        %   \item lcm 最小公倍数, mod 求模运算, 
      %     \item 群的外直积(cf P90)可以类比笛卡尔坐标系: $ \Z_n\times\Z_m\times\Z_k=\{(a,b,c)\mid a\in\Z_n,b\in\Z_m,c\in\Z_k\} $, 运算是按位运算(bitwise). 当 $ m=n=k $时, 
      %     将$ \Z_n\times\Z_n\times\Z_n $简写为 $ \Z_n^3 $
      %     \item 为了简便起见,从现在开始, 在不致误解的情况下, 我们将把 $ \Z_n $中的元素 $ \overline{a} $简记为 $ a $. 
      %     在运算过程中, 读者必须首先分清, $ a $所表示的究竟是数 $ a $还是 剩余类 $ \overline{a} $.
      %     \item 函数的复合(composition)是从右向左的: $ f\circ g(x)=f(g(x)) $
      % \end{enumerate}

      % The product of commutators maybe are not a commutator:

      

\subsubsection*{Problem 1}
证明: 如果有 $ g\in G $使得 $ H $的左陪集 $ gH $ 等于 $ H $的某个右陪集, 那么这个右陪集一定是 $ Hg $.

      不同左 (右)陪集不相交, 还有就是子群里有单位元.

   解: $ H $中有单位元 $ e $, 因此 $ g\in gH $, 同时 $ g $也在 $ H $的一个右陪集中. 
    显然 $ g\in Hg $, 又因为不同陪集不相交, 故这个右陪集一定是 $ Hg $.   

\subsubsection*{Problem 2}
    假设 $ G $是有限群, 素数 $ p $整除$ \rvert G\lvert $并且集合 $ S=\{(x_1,x_2,\dots,x_p)\mid x_i\in G~and~x_1x_2\cdots x_p=1\} $, 其中 $ 1 $ 是单位元. 
    定义 $ S $上的关系: 如果 $ a\in S $是$ b\in S $的循环移位 (可以认为是被 $ p $ -轮换作置换, 比如 $ p=3 $时, $(1,2,3)$被置换后有 $ (2,3,1) $和 $ (3,1,2) $), 那么 $ a\sim b $.
    证明:
    \begin{enumerate}
      \item 集合 $ S $中的元素数量是 $ \rvert G\lvert^{p-1} $ (因此 $ p $ 整除集合中元素的数量).
      \item $ \sim $是一个等价关系, 且等价类中的元素数量不是 $ 1 $就是 $ p $, 且至少有一个等价类其元素数量为 $ 1 $.
      \item $ G $中必定有一个元素的阶为 $ p $.
    \end{enumerate}
    
    解:
    \begin{enumerate}
      \item 考虑 $ x_1x_2\cdots x_{p-1} $, 因为 $ x_1x_2\cdots x_{p-1}x_p=1 $, 故 $ x_1x_2\cdots x_{p-1}=x_p^{-1} $, 即 $ p-1 $个元素可以确定剩下的一个元素, 因此仅需要
      考虑 $ x_1,x_2,\dots,x_{p-1} $这 $ p-1 $个元素的取值范围, 显然是 $ |G|^{p-1} $.
      \item 反身性: $ \forall x\in S $, $ x $经过 $ 0 $次循环移位 (置换)仍是 $ x $; 对称性: 
      对任意 $ x,y\in S $, 如果 $ x $经过 $ i $次置换得到 $ y $, 那么 $ y $可以经过 $ p-i $次置换得到 $x$; 传递性: $ \forall x,y,z\in S $, 如果 $ x $经过 $ i $次置换得到 $ y $, 那么 $ y $可以经过 $ j $次置换得到 $z$, 显然 $ x $可以经过 $ i+j\mod{p} $次置换得到 $ z $. 因此 $ \sim $
      是一个等价关系.

      显然 $ x=(x_1,x_2,\dots,x_p)=(1,1,\dots,1) $是满足 $ x_1x_2\cdots x_p=1 $的, 此时 $ x $所在等价类仅有 $ 1 $个元素. 注意到如果 $ x=(x_0,x_0,\dots,x_0) $是一个元素数量为 $ 1 $的 等价类, 我们可以得到 $ x_0^p=1 $.

      如果一个 $ x $中有$2$个元素不同, 显然在这个等价类中有 $ p $个元素.
      
      % 一个等价类的元素集合可以写为 
      % $ \{x,\sigma (x),\sigma^2 (x),\dots,\sigma^{p-1}(x)\} $, 其中 $ \sigma $是 $p$-轮换. 
      % 如果存在相同的元素, 假设是 $ \sigma^i(x)=\sigma^j(x) $, 那么 $ \sigma^(i-j)(x)=x $, 
      % 显然仅有
      \item 由之前的结论可知, $ p\mid |S|=|G|^{p-1} $, 且 $ |S|= np+r $, 其中 $ n $是元素数量
      是 $ p $的等价类的数量, $ r $是等价类元素数量为 $1$ 的数量. 因此 $ p\mid r $, 所以 存在$r=p>1$个元素数量为$1$的等价类, 显然能得到 存在元素 $ x^p=1 $, i.e. $ \ord(x)=p $.
    \end{enumerate}

\subsubsection*{Problem 3}
\begin{enumerate}
  \item 假设$ S_k $作用在 $ n $个元素的集合 
  (比如 $ \sigma\in S_k $是对集合 $ \{1,2,\dots,n\} $的前$ k $个数进行置换操作). 
  证明 $ S_k $是$ S_n $的子群, 其中 $ 0\leq k<n $.
  \item 证明二项式系数为整数 (不要使用数学归纳法, 建议用群论中的拉格朗日定理), 
  其中二项式系数是多项式 $ (1+x)^n $展开后的 $ x^k $的系数, 
  $ 0\leq k\leq n $. ($ C_n^k $, 或者 $ \binom{n}{k} $, 写法多样, 
  就是从 $n$ 个不同元素中取出$k$个元素的方法的数量). 
\end{enumerate}

解:
\begin{enumerate}
  \item 封闭性, $ S_k $中置换的复合仍然是置换;逆元, $ S_k $中置换的逆仍在 $ S_k $中; 因此是子群.
  \item 同样的可以证明 $ S_n $中, 有一个子群, 子群内的置换分别独立地作用在集合 $ \{1,2,\dots,k\} $和 $ \{k+1,k+2,\dots,n\} $上, 此子群阶为 $ k!(n-k)! $. 再利用拉格朗日定理, 可以确定 $ k!(n-k)!\mid |S_n|=n! $, 所以二项式系数是整数. 
\end{enumerate}

\subsubsection*{Problem 4}
如果 $ \lvert G\rvert=30 $, 那么 $ G $ 最多有几个 $ 7 $阶子群? 最多有几个 $ 5 $阶子群?

  解:  因为 $ 7\nmid 30 $, 故 $ G $中没有 $7$ 阶子群.
    
    $ 5 $阶子群是循环群, 所以这些子群两两相交于单位元, 因此假设有$ n $个子群, 则有公式 $ n(5-1)+1\leq 30 $, 可以得到 $ n\leq 7 $.

    注意:这个是粗略的估计,得到小于7的证明可能是正确的,因为利用西罗定理 (本科不学习)可以证明实际上只有$1$个5阶子群.
\subsubsection*{Problem 5 (这个题的难度可能就是考试的, 但不是说考这道题啊)}

证明: 如果有阿贝尔群$ H $是 $ G $的正规子群, 那么对于子群 $ K<G $, 有$ H\cap K\triangleleft HK $.

    解: 先证明 $ HK $是一个群: 封闭性是 $ \forall hk,h'k'\in HK $, 有 $ hkh'k'=hh'kk'\in HK $, 剩下的结合,单位元,逆元同样类似在此不证明了.

     再证明 $ H\cap K<HK $: 显然的, 子群的交仍然是子群.

     现证明正规性: $ \forall hk\in HK $, 有 $ hkH\cap K(hk)^{-1}=hkH\cap Kk^{-1}h^{-1} $, 因为
    $ H\triangleleft G $,  故$ hkH\cap K(hk)^{-1}\in H $, 又因为 $ hkH\cap K(hk)^{-1}=hkH\cap Kk^{-1}h^{-1}=hh^{-1}kH\cap Kk^{-1}=kH\cap Kk^{-1}\in K $, 则$ hkH\cap K(hk)^{-1}\in H\cap K $
     
\subsubsection*{Problem 6 (这个难度也是考试的难度)}

假设有 $ g\in G $, 证明:
    \begin{enumerate}
      \item $ gHg^{-1}<G $, 且 $ \lvert gHg^{-1}\rvert=\lvert H\rvert $
      \item 如果有 $ n\in\Z^+ $并且 $ H $是群 $ G $唯一的 $ n $ 阶子群, 那么 $ H \triangleleft G$.
    \end{enumerate}
    
    解:
    \begin{enumerate}
      \item $ \forall h_0,h_1\in H $, 有 $ gh_0g^{-1}(gh_1g^{-1})^{-1}=gh_0g^{-1}gh_1^{-1}g^{-1}=gh_0h_1^{-1}g^{-1}\in gHg^{-1} $, 因此 $ gHg^{-1}<G $. 显然$ |gH|=|H| $, 否则有 $ x,y\in H $ s.t. $ gx=gy $, 即 $ x=y $. 同理得到 $ |gHg^{-1}|=|gH| $, 证毕.
      \item 由于 $ |gHg^{-1}|=|H|=n $且 $H$是唯一的$ n $阶子群, 所以 $ H=gHg^{-1} $, 即 $ H\triangleleft G $.
    \end{enumerate}
\subsubsection*{Problem 7}
\begin{enumerate}
  \item 定义 $ Z(G)=\{z\in G\mid \forall g\in G, zg=gz\} $, 证明: $ Z(G)\triangleleft G $
  \item 如果 $ G/Z(G) $是一个循环群, 证明: $ G $是阿贝尔群 (hint: $ G $中的元素是否可以用两部分组成 $ x^nz $, 其中 $ z\in Z(G) $, $ xZ(G) $是循环群生成元, $ n $是整数)
  \item 如果 $ \lvert G\rvert=pq $, 其中 $ p,q $是素数, 证明: $ \lvert Z(G)\rvert=pq $或 $ 1 $.
\end{enumerate}

解:
    \begin{enumerate}
      \item $ Z(G)<G $在之前作业证明过, 因此仅证明正规性: $ \forall g\in G $, 我们有 $ gZ(G)g^{-1}=Z(G)gg^{-1}=Z(G) $, 证毕.
      \item 循环群有 $ G/Z(G)=\langle xZ(G)\rangle $, 其中 $ x\in G $且 $ xZ(G) $是一个陪集. 因此
      对于 $ g,h\in G $有 $ g\in(xZ(G))^n=x^nZ(G),h\in x^mZ(G) $, 其中 $ n,m\in\Z $. 所以 $ gh=x^nZ(G)x^mZ(G)=x^mZ(G)x^nZ(G)=hg $. 故 $ G $是阿贝尔群.
      \item 如果 $ |G|=pq $, 那么 $ |G|/|Z(G)|=1,p,q,pq $中的一个, $ 1,pq $的情况过于简单不再叙述. 假设 $ |G|/|Z(G)|=p $, 那么 $ G/Z(G) $是一个循环群, 所以 $ G $是一个阿贝尔群, 因此 $ Z(G)=G $, i.e. $ |G|/|Z(G)|=1 $与假设矛盾. 另一个情况类似不再叙述. 
    \end{enumerate}
\subsubsection*{Problem 8}
证明下面的子群 $ H $ 是否是群 $ G $ 的正规子群, 如果是正规子群, 请写出对应的商群:
\begin{enumerate}
  \item $ H=SL_n(\R) $, $ G=GL_n(\R) $;
  \item $ H=\{(a,1)\mid a\in A\} $, $ G=A\times B $其中 $ A,B $是群;
  \item $ H=\{(a,a)\mid a\in A\} $, $ A $是阿贝尔群, $ G=A\times A $;
  \item $ H=\{|x|\mid x\in \R^*\} $, $ G=\R^* $, 其中  ``|~|'' 是绝对值符号;
  \item 如果 $ [G:H]=2 $, 那么 $ H $是否为 $ G $的正规子群 (不需要写出商群);
  \item 如果有子群 $ H,K<G $满足 $ H\subset K\subset G $且 $ K\triangleleft G $, $ H\triangleleft K $ (不需要写出商群);
  \item[选做$^*$] $ H=<[x,y]\mid x,y\in G> $, 这里是由所有的$ [x,y] $生成的最小的子群, 如果是正规子群的话, 这道题不需要写出对应的商群 (思考商群是否为阿贝尔群);
\end{enumerate}

解:
    \begin{enumerate}
      \item 显然是的, 对任意 $ m\in GL_n(\R) $, 有$ |mSL_n(\R)m^{-1}|=|m||m^{-1}|=1 $, 即 $ mSL_n(\R)m^{-1}=SL_n(\R) $, 是正规子群. 商群是 $ \{mSL_n(\R)\mid |m|\text{各不相同}\} $同构于 $ \R^* $
      \item 显然是的, $ \forall (x,y)\in A\times B $, 有 $ (x,y)(a,1)(x,y)^{-1}=(xax^{-1},1)\in H $. 商群 $ \{(1,b)H\mid b\in B\} $同构于 $ B $, 注意陪集首里面的 $ 1 $可以换成任意 $ A $中元素.
      \item 证明显然的. 商群直接看的话可能不好看出来, 当然写一写陪集就能看出陪集的问题了. 可以构造一个满同态, $ f:A\times A\rightarrow A $, 使得 $ f(x,y)=xy^{-1} $, 显然 $ ker(f)=H $. 这里商群 $ \{(a,1)H\mid a\in G\} $
      \item 同样是明显的. 商群 $ \{H,-H\}\cong\Z_2 $
      \item 如果 $ \left[G:H\right]=2 $, 可以将 $ G $写为 $ H\cup aH $和 $ H\cup Ha $, 其中 $ a\in G\setminus H $. 所以 $ Ha=aH $, i.e. $ H\triangleleft G $.
      \item 不满足传递性: $ H=\{(\sigma,\sigma)\mid \sigma\in A_3\}\triangleleft A_3\times A_3\triangleleft S_3\times S_3 $, 但是 $ H $不是 $ S_3\times S_3 $的正规子群.
      \item 注意第六题可以不用对称群, 但是对称群是比较简单的非阿贝尔群. $ H $不是 $ S_3\times S_3 $的正规子群是因为显然 $ (a,b)H(a,b)^{-1} $不能得到 $ H $中的元素, 毕竟如果 $ a\neq b\in S_3 $, 则$ axa^{-1}\neq bxb^{-1} $.
      \item $ \forall g\in G $, $ \left[x,y\right]\in H $ 有 $ g^{-1}x^{-1}y^{-1}xyg=g^{-1}x^{-1}gg^{-1}y^{-1}gg^{-1}xgg^{-1}yg $, 令 $ u=g^{-1}xg,v= g^{-1}yg $, 可以得到 $ g^{-1}x^{-1}y^{-1}xyg=u^{-1}v^{-1}uv\in H  $, 因此是正规子群.
    \end{enumerate}

\end{document}