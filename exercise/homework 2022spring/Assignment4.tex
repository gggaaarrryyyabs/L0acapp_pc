\documentclass[a4paper,12pt]{ctexart}
\usepackage{fullpage,enumitem,amsmath,amssymb,graphicx}
\newcommand{\Z}{\mathbf{Z}}
\newcommand{\F}{\mathbf{F}}
\newcommand{\Com}{\mathbf{C}}
\newcommand{\ord}{\operatorname{ord}}
\newcommand{\Q}{\mathbf{Q}}
\newcommand{\R}{\mathbf{R}}


% \title{NIS2312-2 Spring 2022 Homework~1}
% \author{唐灯}



\begin{document}
%   \maketitle
  \begin{center}

  \vspace{-0.3in}
  \begin{tabular}{c}
    \textbf{\Large NIS2312-1 Spring 2021-2022} \\
    \textbf{\Large  } \\
    \textbf{\Large  信息安全的数学基础(1)} \\
    \textbf{\Large  } \\
    \textbf{\Large  Assignment~4} \\
    \textbf{\Large  } \\
    \textbf{\Large 2022年3月14日} \\
  \end{tabular}
  \end{center}
  \noindent
  \rule{\linewidth}{0.4pt}

  在这个练习中, 默认 $G$ 是一个群, $H$ 和 $K$ 是 $ G $的子群.
\subsubsection*{Problem 1}
    证明: 如果有 $ g\in G $使得 $ H $的左陪集 $ gH $ 等于 $ H $的某个右陪集, 那么这个右陪集一定是 $ Hg $.
\subsubsection*{Problem 2}
    [柯西定理]假设 $ G $是有限群, 素数 $ p $整除$ \rvert G\lvert $并且集合 $ S=\{(x_1,x_2,\dots,x_p)\mid x_i\in G~and~x_1x_2\cdots x_p=1\} $, 其中 $ 1 $ 是单位元. 
    定义 $ S $上的关系: 如果 $ a\in S $是$ b\in S $的循环移位 (可以认为是被 $ p $ -轮换作置换, 比如 $ p=3 $时, $(1,2,3)$被置换后有 $ (2,3,1) $和 $ (3,1,2) $), 那么 $ a\sim b $.
    证明:
    \begin{enumerate}
      \item 集合 $ S $中的元素数量是 $ \rvert G\lvert^{p-1} $ (因此 $ p $ 整除集合中元素的数量).
      \item $ \sim $是一个等价关系, 且等价类中的元素数量不是 $ 1 $就是 $ p $, 且至少有一个等价类其元素数量为 $ 1 $.
      \item $ G $中必定有一个元素的阶为 $ p $.
    \end{enumerate}
\subsubsection*{Problem 3}
    \begin{enumerate}
      \item 假设$ S_k $作用在 $ n $个元素的集合 
      (比如 $ \sigma\in S_k $是对集合 $ \{1,2,\dots,n\} $的前$ k $个数进行置换操作). 
      证明 $ S_k $是$ S_n $的子群, 其中 $ 0\leq k<n $.
      \item 证明二项式系数为整数 (不要使用数学归纳法, 建议用群论中的拉格朗日定理), 
      其中二项式系数是多项式 $ (1+x)^n $展开后的 $ x^k $的系数, 
      $ 0\leq k\leq n $. ($ C_n^k $, 或者 $ \binom{n}{k} $, 写法多样, 
      就是从 $n$ 个不同元素中取出$k$个元素的方法的数量). 
    \end{enumerate}
\subsubsection*{Problem 4}
    如果 $ \lvert G\rvert=30 $, 那么 $ G $ 最多有几个 $ 7 $阶子群? 最多有几个 $ 5 $阶子群?
    
\subsubsection*{Problem 5}
    证明: 如果有阿贝尔群$ H $是 $ G $的正规子群, 那么对于子群 $ K<G $, 有$ H\cap K\triangleleft HK $.

\subsubsection*{Problem 6}
    假设有 $ g\in G $, 证明:
    \begin{enumerate}
      \item $ gHg^{-1}<G $, 且 $ \lvert gHg^{-1}\rvert=\lvert H\rvert $
      \item 如果有 $ n\in\Z^+ $并且 $ H $是群 $ G $唯一的 $ n $ 阶子群, 那么 $ H \triangleleft G$.
    \end{enumerate}
\subsubsection*{Problem 7}
    \begin{enumerate}
      \item 定义 $ Z(G)=\{z\in G\mid \forall g\in G, zg=gz\} $, 证明: $ Z(G)\triangleleft G $
      \item 如果 $ G/Z(G) $是一个循环群, 证明: $ G $是阿贝尔群 (hint: $ G $中的元素是否可以用两部分组成 $ x^nz $, 其中 $ z\in Z(G) $, $ xZ(G) $是循环群生成元, $ n $是整数)
      \item 如果 $ \lvert G\rvert=pq $, 其中 $ p,q $是素数, 证明: $ \lvert Z(G)\rvert=pq $或 $ 1 $.
    \end{enumerate}
    \subsubsection*{Problem 8}
    证明下面的子群 $ H $ 是否是群 $ G $ 的正规子群, 如果是正规子群, 请写出对应的商群:
    \begin{enumerate}
      \item $ H=SL_n(\R) $, $ G=GL_n(\R) $;
      \item $ H=\{(a,1)\mid a\in A\} $, $ G=A\times B $其中 $ A,B $是群;
      \item $ H=\{(a,a)\mid a\in A\} $, $ A $是阿贝尔群, $ G=A\times A $;
      \item $ H=\{|x|\mid x\in \R^*\} $, $ G=\R^* $, 其中  ``|~|'' 是绝对值符号;
      \item 如果 $ [G:H]=2 $, 那么 $ H $是否为 $ G $的正规子群 (不需要写出商群);
      \item 如果有子群 $ H,K<G $满足 $ H\subset K\subset G $且 $ K\triangleleft G $, $ H\triangleleft K $ (不需要写出商群);
      \item[选做$^*$] $ H=<[x,y]\mid x,y\in G> $, 这里是由所有的$ [x,y] $生成的最小的子群, 如果是正规子群的话, 这道题不需要写出对应的商群 (思考商群是否为阿贝尔群);
    \end{enumerate}

    %  $ G $ 一定有阶为$2$的元素. 见习题. 假如至少有两个 $ 2 $阶的 $ a,b $, 那么 $ \{1,a,b,ab\} $就是一个群了, 
  %  (因为 $ a^2=1,b^2=1 $, 并且 $ (ab)^2=abab=aabb=1 $)
  %  但是拉格朗日定理得知这是矛盾的. 所以只有一个 $ 2 $阶的.

  %   假设有6阶的,证毕.

  %  假设没有 6 阶的, 所以剩下的元素必然是2 或者3 或者1, 1阶的只能有一个, 2阶的只有一个, 那么剩下的就是3 阶的,
  %  所以一个二阶一个三阶的相乘, 就能组成一个 6阶的. 证毕.

  % 循环移位: $ (1,2,3,4,5) $循环右移 $1$ 位的话, 就是 

% \subsubsection*{Problem 8}
    % 首先 $ S_3 $不是阿贝尔群,所以 $ S_3 $不是循环群,也就是不同构于 $ \Z_6 $.
    % 假设有一个 $ 6 $阶非阿贝尔群 $ G $ ,下面证明这个群一定是同构于 $ S_3 $:

    % 假设有非单位元 $ a,b\in G $满足 $ ab\neq ba $ ,所以能够得到集合 $ \{1,a,b,ab,ba\} $.
    % 注意到 $ a^2\notin\{a,b,ab,ba\} $, 因此
    % \begin{enumerate}
    %   \item 如果 $ a^2\neq 1 $, 那么 $ G=\{1,a,b,ab,ba,a^2\} $. 同时也有 $ a^3\notin\{a,b,ab,ba,a^2\} $, 即
    %   $ a^3=1 $. 我们还有 $ b^2\notin\{a,b,ab,ba,b^2\} $. 因此 $ b^2=1 $. 故令  $ a=(123),b=(12) $即可.
    %   \item 如果$ a^2=1 $, 那么 $ b^2=1 $. 因此 $ aba\notin\{1,a,b,ab,ba\} $, 所以 $ aba=bab $, 故 $ (ab)^3=1 $.同上.
    % \end{enumerate}
    % 证毕.

\end{document}