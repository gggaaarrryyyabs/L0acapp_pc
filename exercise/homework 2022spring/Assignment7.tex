\documentclass[a4paper,12pt]{ctexart}
\usepackage{fullpage,enumitem,amsmath,amssymb,graphicx}
\usepackage{tikz-cd}
\newcommand{\Z}{\mathbf{Z}}
\newcommand{\F}{\mathbf{F}}
\newcommand{\Com}{\mathbf{C}}
\newcommand{\ord}{\operatorname{ord}}
\newcommand{\Q}{\mathbf{Q}}
\newcommand{\R}{\mathbf{R}}


% \title{NIS2312-2 Spring 2022 Homework~1}
% \author{唐灯}



\begin{document}
%   \maketitle
  \begin{center}

  \vspace{-0.3in}
  \begin{tabular}{c}
    \textbf{\Large NIS2312-1 Spring 2021-2022} \\
    \textbf{\Large  } \\
    \textbf{\Large  信息安全的数学基础(1)} \\
    \textbf{\Large  } \\
    \textbf{\Large  Assignment~7} \\
    \textbf{\Large  } \\
    \textbf{\Large 2022年4月14日} \\
  \end{tabular}
  \end{center}
  \noindent
  \rule{\linewidth}{0.4pt}

  % 在这个练习中, 带 * 的题目属于数论方面, 可以不做的 (考试不会单独考数论的题).  
\subsubsection*{Problem 1}
    % 判断下列代数结构在给定运算下是否构成环
    \begin{enumerate}
      \item 已知 $ 2\Z $和 $ 3\Z $是 $ \Z $的理想, 证明或证否 $ 2\Z\cap 3\Z $也是 $ \Z $的理想.
      \item 证明交换环 $ R $ 的幂零元组成的集合 $ \mathfrak{R} (R) $ 为 $ R $的理想.
      \item 假设 $ I $是环 $ R $的理想, $ S $是环 $ R $的子环, 证明: $ I\cap S $是环 $ S $的理想.
    \end{enumerate}
\subsubsection*{Problem 2}
   证明: 假设环 $ R $中有单位元, 那么
   \begin{enumerate}
      \item 如果 $ I $是环 $ R $的理想, $ I=R $当且仅当 $ I $中有单位;
      \item 交换环 $ R $是一个域当且仅当 $ R $只有平凡理想.
    \end{enumerate}
\subsubsection*{Problem 3}
    假设 $ R=\{\begin{pmatrix}a&b\\b&a\end{pmatrix}\mid a,b\in\Z\} $, 
    $ \phi:R\rightarrow\Z $是一个映射满足 $ \phi(\begin{pmatrix}a&b\\b&a\end{pmatrix})=a-b $,
    证明:
    \begin{enumerate}
      \item $ \phi $是一个同态;
      \item 计算 $ ker(\phi) $;
      \item $ R/ker(\phi)\cong\Z $. 
    \end{enumerate} 
\subsubsection*{Problem 4}
    假设 $ D\in\Z $是一个非完全平方数, 
    令 $ S=\{\begin{pmatrix}a &b\\Db &a\end{pmatrix}\mid a,b\in \Z\} $. 证明:
    \begin{enumerate}
      \item $ S $是 $ M_2(\Z) $的子环;
      \item 映射$ \begin{array}{rccc}
        \phi:&\Z[\sqrt{D}]&\rightarrow&S\\
        &a+b\sqrt{D}&\mapsto&\begin{pmatrix}a &b\\Db &a\end{pmatrix}
      \end{array} $是环的同构映射.
    \end{enumerate}
\subsubsection*{Problem 5}
    假设 $ f(x)\in\R[x] $, 如果 $ a+bi $是 $ f(x) $的复数根, 证明 $ a-bi $也是 $ f(x) $的根.
    注意:需要构造一个合理的同构映射. 
 
\end{document}