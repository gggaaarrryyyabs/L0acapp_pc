\documentclass[a4paper,12pt]{ctexart}
\usepackage{fullpage,enumitem,amsmath,amssymb,graphicx}
\newcommand{\Z}{\mathbf{Z}}
\newcommand{\F}{\mathbf{F}}
\newcommand{\Com}{\mathbf{C}}
\newcommand{\ord}{\operatorname{ord}}
\newcommand{\Q}{\mathbf{Q}}
\newcommand{\R}{\mathbf{R}}


\title{NIS2312-1 Spring 2022 Homework~1}
\author{唐灯}



\begin{document}
%   \maketitle
  \begin{center}

  \vspace{-0.3in}
  \begin{tabular}{c}
    \textbf{\Large NIS2312-1 Spring 2021-2022} \\
    \textbf{\Large  } \\
    \textbf{\Large  信息安全的数学基础(1)} \\
    \textbf{\Large  } \\
    \textbf{\Large  Assignment~1} \\
    \textbf{\Large  } \\
    \textbf{\Large 2022年2月17日} \\
  \end{tabular}
  \end{center}
  \noindent
  \rule{\linewidth}{0.4pt}
  
\subsubsection*{Problem 1}
    设集合 $ S=\{(a,b)|a,b\in\Z,b\neq 0\}$,
    在集合 $ S $ 中, 规定关系 "$ \sim $":\[(a,b)\sim(c,d)\Leftrightarrow  ad=bc.\]
    证明: "$\sim$" 是 $ S $ 的一个等价关系.
\subsubsection*{Problem 2} 
    试分别给出满足下列条件的关系:
    \begin{enumerate}
        \item 有对称性、传递性,但无反身性;
        \item 有对称性、反身性,但无传递性;
        \item 有反身性、传递性,但无对称性;
    \end{enumerate}
    \begin{proof}
        \begin{enumerate}
            \item $ S=\R $, $ a\sim b\leftrightarrow ab>0 $
            \item $ S=\R $, $ a\sim b\leftrightarrow \left\lvert a-b\right\rvert <1 $
            \item $ S=\R $, $ a\sim b\leftrightarrow ab>0 $
        \end{enumerate}
    \end{proof}
\subsubsection*{Problem 3}
    设 $ a,b,m $ 是正整数, 如果 $ \gcd(a,m)=1 $, 试证明: $ \gcd(ab,m)=\gcd(b,m) $.
\subsubsection*{Problem 4}
    给出乘法群$ \Z_{13}^* $中每个元素的逆元素.
\subsubsection*{Problem 5}
    令 $ G $是实数对 $ (a,b),a\neq 0 $的集合, 在 $ G $上定义 
    $ (a,b)\star (c,d)=(ac,ad+b) $. 证明: $G$关于$ \star $构成群.
\subsubsection*{Problem 6}
    设 $ G=\{x\in \R\mid 0\leq x<1\} $且 $ x\star y $是 $ x+y $的小数部分.
    证明: $ G $关于 $ \star $构成阿贝尔群.
\subsubsection*{Problem 7}
    设 $ G $是一个群, 证明: $ (gh)^{-1}=h^{-1}g^{-1},\forall h,g\in G $.
\subsubsection*{Problem 8}
    证明: 一个有限群是阿贝尔群当且仅当它的群表是对称矩阵.
\end{document}