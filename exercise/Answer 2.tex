\documentclass[a4paper,12pt]{ctexart}
\usepackage{fullpage,enumitem,amsmath,amssymb,graphicx}
\newcommand{\Z}{\mathbf{Z}}
\newcommand{\F}{\mathbf{F}}
\newcommand{\Com}{\mathbf{C}}
\newcommand{\ord}{\operatorname{ord}}
\newcommand{\Q}{\mathbf{Q}}
\newcommand{\R}{\mathbf{R}}


% \title{NIS2312-2 Spring 2022 Homework~1}
% \author{唐灯}



\begin{document}
%   \maketitle
  \begin{center}

  \vspace{-0.3in}
  \begin{tabular}{c}
    \textbf{\Large NIS2312-1 Spring 2021-2022} \\
    \textbf{\Large  } \\
    \textbf{\Large  信息安全的数学基础(1)} \\
    \textbf{\Large  } \\
    \textbf{\Large  Answer 2 } \\
    \textbf{\Large  } \\
    \textbf{\Large 2022年3月7日} \\
  \end{tabular}
  \end{center}
  \noindent
  \rule{\linewidth}{0.4pt}

      % \begin{enumerate}
        %   \item 一般而言, 大写字母代表集合(比如$S$一般用于代表集合(set), $G$一般用于表示群(group), $ R $一般用于表示环(ring)), 小写字母代表集合中的元素(比如 $ p $一般用于表示素数(prime), $ a,g $一般用于表示群的元素等等)
        %   \item 群中的运算如果不作特别说明不一定满足交换律(commutative), 运算满足交换律的群叫交换群或者阿贝尔群
        %   \item 今后, 如果不作特别说明, 总假定群的运算是``乘法''. cf P8 例 4 上面那一段
        %   \item 群中的乘法不一定是我们熟知的乘法, 比如 $ SL_n(\R) $的乘法就是矩阵乘法, 四面体群(dihedral group)的乘法是函数的复合, 甚至可以把整数加法群中的整数加法运算叫做``乘法''
        %   \item 群的运算如果是``加法''的话, 加法单位元(或者叫零元)常用 $0$表示, 元素 $a$的加法逆元记作 $ -a $
        %   \item 群的运算如果是``乘法''的话, 乘法单位元常用 $e$表示, 元素 $a$的乘法逆元记作 $ a^{-1} $
        %   \item $ \left\langle a\right\rangle=\{a^r\mid r\in\Z\}  $由元素 $ a $生成的循环群
        % %   \item 如果 $ S=\{a_1,a_2,\dots,a_r\} $, 则记 $ \langle S\rangle=\langle a_1,a_2,\dots,a_r\rangle=\{a_1^{l_1}a_2^{l_2}\cdots a_k^{l_k}\mid a_i\in S,\} $
        %   \item 对于群 $ G $中的集合 $ S,T $和元素 $ x $, $ Sx=\{ax\mid a\in S\} $, $ xS=\{xa\mid a\in S\} $(陪集,cf P66), $ xSx^{-1}=\{xax^{-1}\mid a\in S\} $, $ ST=\{ab\mid a\in S,b\in T\} $
        %   \item lcm 最小公倍数, mod 求模运算, 
      %     \item 群的外直积(cf P90)可以类比笛卡尔坐标系: $ \Z_n\times\Z_m\times\Z_k=\{(a,b,c)\mid a\in\Z_n,b\in\Z_m,c\in\Z_k\} $, 运算是按位运算(bitwise). 当 $ m=n=k $时, 
      %     将$ \Z_n\times\Z_n\times\Z_n $简写为 $ \Z_n^3 $
      %     \item 为了简便起见,从现在开始, 在不致误解的情况下, 我们将把 $ \Z_n $中的元素 $ \overline{a} $简记为 $ a $. 
      %     在运算过程中, 读者必须首先分清, $ a $所表示的究竟是数 $ a $还是 剩余类 $ \overline{a} $.
      %     \item 函数的复合(composition)是从右向左的: $ f\circ g(x)=f(g(x)) $
      % \end{enumerate}

      % The product of commutators maybe are not a commutator:

      

\subsubsection*{Problem 2}
      \begin{enumerate}
        \item[2] 实际上是一个复数集合, 因为复数集合在乘法运算下去掉$0$才构成群, 所以此题不构成群.
        \item[4] 注意到 $ xS=Sx $仅仅能得到集合之间的相等关系, 也就是
        \[xS=Sx\Rightarrow xa=bx,where~a,b\in S\] 
        证明:如果集合 $ S $属于群 $ G $, 那么 $ N(S)\leq G $.
        首先, 因为 $ eS=Se $, 所以$ N(S) $不是空集.
        然后假设 $ a,b,c\in S $, 所以我们有
        \[abS=a(bS)=a(Sb)=(aS)b=S(ab),\]
        意味着 $ ab\in S $.
        并且, 还有 \[ c^{-1}Sc=c^{-1}cS=S\Rightarrow c^{-1}S=Sc^{-1}, \]
        所以 $ N(S)\leq G $.
        \item[5] 不构成群, 因为运算不满足封闭性: 大家关于封闭性的证明都不严谨,
        此证明需要的背景知识大家在此书中不学习, 在此仅给出反例 (需要线代的部分知识). 
        
        举例:

        Let $ f $ be a polynomial in $ x $, $ g $ be a polynomial in $ y $ and $ h $ be a polynomial in $ x,y $, 
      all with rational coefficients. 
      Denote $ m(f,g,h)=\begin{pmatrix}
        1 &f(x) & h(x,y)\\
        0 &1    & g(y)\\
        0 &0    & 1
      \end{pmatrix} $, and let $ G $ be the set of all such matrices. It's easy to verify $ G $ is a group 
      under the matrix multiplication with the inverse of $ m(f,g,h) $ is $ m(-f,-g,-h+fg) $.
      Then commutator $ \left[m(f_1,g_1,h_1),m(f_2,g_2,h_2)\right] $ is $ \left[m(0,0,f_1g_2-f_2g_1)\right] $. 
      Thus, we observe that $ m(0,0,\sum a_{ij}x^iy^j)=\prod\left[m(a_{ij}x^i,0,0),m(0,y^j,0)\right] $ and 
      $ m(0,0,h_1)m(0,0,h_2)=m(0,0,h_1+h_2) $.

      So let $ n $ be a positive integer and $ h(x,y)=\sum_{i=0}^{2n+1}x^iy^i $.
      We confirm that $ m(0,0,h) $ cannot be the product of $ n $ commutators. Otherwise, suppose 
      \[h(x,y)=\sum_{j=1}^n f_j(x)g_j(y)-h_j(x)k_j(y).\]
      Write $ f_j=\sum_{i}a_{ij}x^i $ and $ h_j(x)=\sum_i b_{ij}x^i $ and from the coefficients of 
      $ 1,x,\dots,x^{2n+1} $ we have 
      \[\sum_{j=1}^na_{ij}g_j(y)-b_{ij}k_j(y)=y^i\quad i=1,\dots,2n+1.\]
      Therefore we have $ 2n $ polynomials in $ y $ to generate $ 2n+1 $ linear independent $ y^i $, 
      which is a contradiction. 
        \item[6] 注意使用书上的结论: 群$G$的任何子群的交集是子群.  
      \end{enumerate}
\subsubsection*{Problem 4}
      
      因为 $ e\in\bigcup_{i=1}^{\infty}H_i $, 此集合不是空集.
      假设 $ a\in H_j\subseteq\bigcup_{i=1}^{\infty}H_i $和
       $ b\in H_k\subseteq\bigcup_{i=1}^{\infty}H_i $, 
      不失一般性的可以假定 $ j\leq k $, 
      因此有 $ a\in H_j\leq H_k $, 又注意到 $ b^{-1}\in H_k $, 所以有
      $ ab^{-1}\in H_k\subseteq\bigcup_{i=1}^{\infty}H_i $, 故证得是子群.

\subsubsection*{Problem 5}

      假设 $ H\neq\{0\} $, 所以存在 $ x=a/b\in H $, 其中 $ a,b $为整数, 
      因此有 $ a = bx = \underbrace{x+x+\dots +x}_{\text{b times}} \in H $.
      得到 $ 1/a\in H $, 再使用同样的方法可以确定 $ 1\in H $.

      那么 $ \Z\subseteq H $. 取 $ \Q $中任意数 $ p/q $其中 $ p,q $是整数, 
      所以 $ q\in\Z\subseteq H $, 同时也有 $ 1/q\in H $. 因此
      \[\frac p q = \underbrace{\frac 1 q + \dots + \frac 1 q}_{\text{p times}} \in H.\] 
      所以证得 $ \Q\subseteq H $, 故 $ H=\Q $.
\end{document}