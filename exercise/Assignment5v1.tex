\documentclass[a4paper,12pt]{ctexart}
\usepackage{fullpage,enumitem,amsmath,amssymb,graphicx}
\usepackage{tikz-cd}
\newcommand{\Z}{\mathbf{Z}}
\newcommand{\F}{\mathbf{F}}
\newcommand{\Com}{\mathbf{C}}
\newcommand{\ord}{\operatorname{ord}}
\newcommand{\Q}{\mathbf{Q}}
\newcommand{\R}{\mathbf{R}}


% \title{NIS2312-2 Spring 2022 Homework~1}
% \author{唐灯}



\begin{document}
%   \maketitle
  \begin{center}

  \vspace{-0.3in}
  \begin{tabular}{c}
    \textbf{\Large NIS2312-1 Spring 2021-2022} \\
    \textbf{\Large  } \\
    \textbf{\Large  信息安全的数学基础(1)} \\
    \textbf{\Large  } \\
    \textbf{\Large  Assignment~5} \\
    \textbf{\Large  } \\
    \textbf{\Large 2022年3月23日} \\
  \end{tabular}
  \end{center}
  \noindent
  \rule{\linewidth}{0.4pt}

  在这个练习中, 默认 $G$ 是一个群, $H$ 和 $K$ 是 $ G $的子群.
\subsubsection*{Problem 1}
    对任意 $ n\in\Z^+ $, 证明 $ \Z/n\Z\cong\Z_n $.
\subsubsection*{Problem 2}
    第二同构定理: 给定群 $ G $和其正规子群 $ N $, 子群 $ H $, 证明:
    \begin{enumerate}
      \item $ HN $是一个群且 $ N\triangleleft HN $
      \item $ N\cap H $是 $ H $的正规子群;
      \item 有群同构: $ H/(N\cap H)\cong HN/N  $. 
    \end{enumerate}
    \begin{tikzcd}
      &HN \arrow[dl,dash] \arrow[dr,dash]&\\
      N\arrow[dr,dash] & & H \arrow[dl,dash] \\
      &N\cap H &
    \end{tikzcd}

    注意:书上习题2-1的14题, 看起来形式和第二同构定理非常相似, 但是证明不能使用此定理, 至少用此定理无法覆盖全部的情况, 因为如果 $ N $不是正规子群的话 (或者其他条件), $ HN $不一定是个群.

\subsubsection*{Problem 3}
    假设 $ C\triangleleft A $并且 $ D\triangleleft B $, 证明: $ C\times D\triangleleft A\times B $和 $ (A\times B)/(C\times D)\cong(A/C)\times(B/D) $ (直积保持正规性).
    
% \subsubsection*{Problem 5}
%     假设 $ H\unlhd G,K\unlhd G $, 并且 $ G=HK $, 证明:
%     $ G/(H\cap K)\cong (G/H)\times(G/K) $.

% \subsubsection*{Problem 6}
%     $H$ 和 $K$ 是 $ G $的有限子群, 证明 $ |H|\cdot|K|=|HK|\cdot|H\cap K| $.
% \subsubsection*{Problem 4}
%     在 $ \Z_{40}\oplus\Z_{30} $中, 给出两个阶等于 $ 12 $的子群. 
\subsubsection*{Problem 4}
    假设 $ G=H_1H_2\cdots H_n $, 且对任意 $ i\in\{1,2,\dots,n\} $都有 $ H_i\triangleleft G $. 证明下面的条件是等价的:
    \begin{enumerate}
      \item $ G $是 $ H_i $的内直积, 其中 $ i=1,2,\dots,n $;
      \item $ H_1H_2\cdots H_{i-1}\cap H_i=\{e\},~\forall i=2,3,\dots,n $;
      \item $ H_1\cdots H_{i-1}H_{i+1}\cdots H_n\cap H_i=\{e\},~\forall i=1,2,\dots,n $.
    \end{enumerate}
% \subsubsection*{*}
%     如果有群 $ G $和正规子群 $ H_1,H_2,\dots,H_n $且 $ G=H_1H_2\cdots H_n $, 此外如果 $ i\neq j $, 则$ H_i\cap H_j=\{e\} $. 证明或给出反例
%      $ G\cong H_1\times H_2\times \cdots \times H_n $.

    
 
\end{document}