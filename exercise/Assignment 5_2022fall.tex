\documentclass[a4paper,12pt]{ctexart}
\usepackage{fullpage,enumitem,amsmath,amssymb,graphicx}
\newcommand{\Z}{\mathbb{Z}}
\newcommand{\F}{\mathbb{F}}
\newcommand{\Com}{\mathbf{C}}
\newcommand{\ord}{\operatorname{ord}}
\newcommand{\Q}{\mathbb{Q}}
\newcommand{\R}{\mathbb{R}}


\title{NIS2312-1 2022-2023 Fall Homework~1}
\author{唐灯}



\begin{document}
%   \maketitle
  \begin{center}

  \vspace{-0.3in}
  \begin{tabular}{c}
    \textbf{\Large NIS2312-1 2022-2023 Fall} \\
    \textbf{\Large  } \\
    \textbf{\Large  信息安全的数学基础(1)} \\
    \textbf{\Large  } \\
    \textbf{\Large  Assignment~5} \\
    \textbf{\Large  } \\
    \textbf{\Large 2022年9月28日} \\
  \end{tabular}
  \end{center}
  \noindent
  \rule{\linewidth}{0.4pt}
  
%   可以使用计算机求模的运算.
$ \R $是实数域, $ \Q $是有理数域, $ \Z $是整数集合.

\subsubsection*{Problem 1}
  在 $ \Z_{10} $中, 令 $ H=\{\overline{2},\overline{4},\overline{6},\overline{8}\} $. 证明: $ H $关于剩余类的乘法构成群. $ H $是 $ (\Z_{10},\cdot) $的子群吗? 为什么?
      
  解: \begin{enumerate}[label=(\arabic{*})]
    \item 直接计算可以发现 $ H $关于剩余类的乘法是封闭的;
    \item 剩余类的乘法满足结合律, 所以 $ H $的乘法也满足结合律;
    \item 可以验证 
    \begin{align*}
      \overline{2}\cdot\overline{6}&=\overline{12}=\overline{2}\\
      \overline{4}\cdot\overline{6}&=\overline{24}=\overline{4}\\ 
      \overline{6}\cdot\overline{6}&=\overline{36}=\overline{6}\\ 
      \overline{8}\cdot\overline{6}&=\overline{48}=\overline{8} 
    \end{align*}
    故 $ \overline{6} $是 $ H $的单位元;
    \item 从上述等式可以发现 $ H $中的每个元素都是可逆的.
  \end{enumerate}
    综上 $ H $是一个群. 但$ H $不是 $ (\Z_{10},\cdot) $的子群: 因为 $ (\Z_{10},\cdot) $不构成群.

\subsubsection*{Problem 2} 
  设 $ G=\operatorname{GL}_2(\R),~H=\{A\in G\mid\det(A)\text{是3的整数幂次}\} $. 证明: $ H $是 $ G $的子群.
  
  解: $ H $是非空集合. 假设 $ A,B\in H $, 那么存在 $ m,n\in\Z $ s.t. $ \det(A)=3^m,\det(B)=3^n $. 因此
  \[\det(AB^{-1})=\det(A)\det(B^{-1})=\det(A)\det(B)^{-1}=3^m3^{-n}=3^{m-n}.\]
  故 $ AB^{-1}\in H $, 因此 $ H $为 $ G $的子群.

\subsubsection*{Problem 3}
  设 $ G $是交换群, $ m $是固定的整数. 令 $ H=\{a\in G\mid a^m=e\} $. 证明: $ H $是 $ G $的子群.    

  解: 因为 $ e^m=e $ 故 $ e\in H $, 即 $ H $不是空集. 设 $ a,b\in H $, 则 $ a^m=b^m=e $. 因此
  \[(ab^{-1})^m=a^m(b^{-1})^m=a^m(b^m)^{-1}=e.\]
  故 $ ab^{-1}\in H $, 因此 $ H $为 $ G $的子群.

\subsubsection*{Problem 4}
  设 $ H $是 $ G $的子群. 证明: 对任意的 $ g\in G $, 集合 $ gHg^{-1}=\{ghg^{-1}\mid h\in H\} $是 $ G $的子群.

  解: 设 $ h_1,h_2\in H $, 故 $ x=gh_1g^{-1}\in gHg^{-1},y=gh_2g^{-1}\in gHg^{-1} $. 因此
  \[xy^{-1}=gh_1g^{-1}(gh_2g^{-1})^{-1}=gh_1g^{-1}gh_2^{-1}g^{-1}=gh_1h_2^{-1}g^{-1}\in gHg^{-1}.\]
  其中 $ h_1h_2^{-1}\in H $是因为$ H $为 $ G $的子群. 故集合 $ gHg^{-1}=\{ghg^{-1}\mid h\in H\} $是 $ G $的子群.

\subsubsection*{Problem 5}
  设 $ a $是群 $ G $的元素. 定义 $ a $在 $ G $中的中心化子 (centralizer) 为 
  \[C(a)=\{g\in G\mid ga=ag\}.\]
  证明: $ C(a) $ 是 $ G $的子群.

  解: 显然 $ e\in C(a) $, 故 $ C(a) $不是空集. 假设 $ g_1,g_2\in C(a) $, 则 $ g_1a=ag_1,g_2a=ag_2 $, 整理得到
  $ g_1g_2a=g_1ag_2=ag_1g_2 $, 故 $ g_1g_2\in C(a) $; 同时假设 $ g\in C(a) $, 则  $ ga=ag $, 整理得到 
  $ g^{-1}ga=g^{-1}ag=a\Rightarrow g^{-1}agg^{-1}=ag^{-1} $, 即 $ g^{-1}a=ag^{-1} $, 故 $ g^{-1}\in C(a) $.
  因此, $ C(a) $ 是 $ G $的子群.

\subsubsection*{Problem 6}
  设 $ G $的群. 证明: $ C(G)=\bigcap_{a\in G}C(a) $ (即 $ G $的中心是所有形如 $ C(a) $的子群的交).

  解: \begin{enumerate}
    \item[$ \subseteq $] 对任意$ x\in G $, $ g\in C(G) $, 都有 $ gx=xg $, 因此 $ g\in C(x) $, 故 $ C(G)\subseteq C(x) $, 由于 $ x $任意, 则 $ C(G)\subseteq\bigcap_{x\in G}C(x) $;
    \item[$ \supseteq $] 设 $ g\in\bigcap_{a\in G}C(a) $, 则 $ g\in C(a) $, 
    其中 $ a\in G $. 因此 $ ag=ga $, 由于 $ a $任意, 则 $ g\in C(G) $, 故 $ \bigcap_{a\in G}C(a)\subseteq C(G) $.
  \end{enumerate}
  综上,  $ C(G)=\bigcap_{a\in G}C(a) $.

\subsubsection*{Problem 7}
  设 $ G $的群, $ a\in G $. 证明: $ C(a)=C(a^{-1}) $.

  解: $ \forall a,g\in G $, 有 $ ag=ga\Leftrightarrow ga^{-1}=a^{-1}g $, 故 
  \[C(a)=\{g\mid ga=ag\}=\{g\mid ga^{-1}=a^{-1}g\}=C(a^{-1}).\]
  
\subsubsection*{Problem 8}
  设 $ H,K $是 $ G $的两个子群. 证明: 当且仅当 $ H\subseteq K $或 $ K\subseteq H $时, $ H\cup K $是 $ G $的子群. 
  利用此结论证明, 群 $ G $不能被它的两个真子群所覆盖. $ G $能被它的三个真子群所覆盖吗?

  解: 充分性: 假设 $ H\subseteq K $, 则 $ H\cup K=K $显然是 $ G $的子群;

  必要性: 假设 $ H\cup K<G $. 如果 $ H\subseteq K $, 则结论成立. 
  因此假设 $ H\nsubseteq K $, 则假设 $ h\in H\setminus K $, 故 $ hkk^{-1}=h\in H $.
  由 $ H\cup K<G $, 有 $ h\in H,k\in K $且 $ hk\in H $ 或 $ hk\in K $. 如果 $ hk\in K $, 则 $ h=hkk^{-1}\in K $与假设矛盾, 故 $ hk\in H $, 故 $ k=h^{-1}hk\in H $, 
  即 $ K\subseteq H $.

  如果 $ G $能被它的两个真子群覆盖, 那么假设为 $ G=H\cup K $, 则 $ G=H $或 $ G=K $, 与真子群的性质矛盾, 故群 $ G $不能被它的两个真子群所覆盖.

  $ G $能被它的三个真子群所覆盖: 
  举例克莱因加法群 $ \Z_2\times\Z_2=\{(0,0),(0,1),(1,0),(1,1)\} $, 三个真子群分别是
  $ \{(0,0),(0,1)\},\{(0,0),(1,0)\},\{(0,0),(1,1)\} $.

\subsubsection*{Problem 9}
  设群 $ K $ 由元素 $ a,b $和关系 $ a^2=b^2=e,ab=ba $所定义. 试给出群 $ K $的乘法表 (乘法表定义见 Page 11).

  解:

  \[\begin{array}{c|cccc}
    \hline 
       & e  & a  & b  & ab \\
    \hline    
    e  & e  & a  & b  & ab \\
    a  & a  & e  & ab & b  \\
    b  & b  & ab & e  & a  \\
    ab & ab & b  & a  & e  \\
    \hline
    \end{array}\]
\end{document}