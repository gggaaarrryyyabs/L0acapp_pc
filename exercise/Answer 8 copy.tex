\documentclass[a4paper,12pt]{ctexart}
\usepackage{fullpage,enumitem,amsmath,amssymb,graphicx}
\usepackage{tikz-cd}
\newcommand{\Z}{\mathbf{Z}}
\newcommand{\F}{\mathbf{F}}
\newcommand{\Com}{\mathbf{C}}
\newcommand{\ord}{\operatorname{ord}}
\newcommand{\Q}{\mathbf{Q}}
\newcommand{\R}{\mathbf{R}}


% \title{NIS2312-2 Spring 2022 Homework~1}
% \author{唐灯}



\begin{document}
%   \maketitle
  \begin{center}

  \vspace{-0.3in}
  \begin{tabular}{c}
    \textbf{\Large NIS2312-1 Spring 2021-2022} \\
    \textbf{\Large  } \\
    \textbf{\Large  信息安全的数学基础(1)} \\
    \textbf{\Large  } \\
    \textbf{\Large  Answer~8} \\
    \textbf{\Large  } \\
    \textbf{\Large 2022年5月23日} \\
  \end{tabular}
  \end{center}
  \noindent
  \rule{\linewidth}{0.4pt}

  % 在这个练习中, 带 * 的题目属于数论方面, 可以不做的 (考试不会单独考数论的题).  
\subsubsection*{Problem 1}
    % 判断下列代数结构在给定运算下是否构成环
    \begin{enumerate}
      \item 证明 $ \langle x\rangle $是 $ \Z[x] $ 的素理想, 此外该素理想是极大理想吗?
      \item 证明任意非零有限含幺交换环的素理想都是极大理想;
      \item 如果非零含幺交换环 $ R $ 的素理想 $ I $ 不包含零因子 (零因子是相对于环 $ R $来说的), 证明这个交换环是整环;
    \end{enumerate}

    解:\begin{enumerate}
      \item 因为 $ \Z[x]/\langle x\rangle\cong\Z $ 是一个整环, 所以可以确定 $ \langle x\rangle $是 $ \Z[x] $的素理想, 但由于 $ \Z $不是域, 则 这个素理想不是极大理想;
      \item 设任意非零有限含幺交换环为 $ R $且素理想为 $ I $, 那么 $ R/I $是一个有限整环, 所以 $ R/I $是一个域, 所以 $ I $是 $ R $的极大理想;
      \item 假设 $ R $不是整环, 设 $ a\in R $是零因子, 也就是存在非零 $ b\in R $满足 $ ab=0 $, 所以 $ ab\in I $, 则 $ a\in I $或者  $ b\in I $. 假设$ b\in I $. 
      因为 $ a\neq 0 $, $ b\in I $, 所以 $ ab=0 $可以确定 $ b=0 $. 所以 $ a $不是零因子, 与题设矛盾. 
    \end{enumerate}
\subsubsection*{Problem 2}
   给定多项式环 $ R=\F_2[x]=\{a_0+a_1x+a_2x^2+\cdots+a_nx^n+\cdots\mid a_i\in\F_2=\{0,1\},i=0,1,\dots\} $以及环中的元素 $ x^2+x+1 $. 
   定义商环 $ F=\F_2[x]/\langle x^2+x+1\rangle $以及多项式环到商环的映射 $ -:R\rightarrow F $形如
   $ x\mapsto \overline{x} $, 其中 $ x\in R $且$ \overline{x}=x+\langle x^2+x+1\rangle\in F $.
   证明:
   \begin{enumerate}
     \item $ \overline{R} $中有四个元素: $ \overline{0},\overline{1},\overline{x} $和 $ \overline{1+x} $;
     \item 直接写出 $ \overline{R} $的元素的加法表, 直接给出其同构的群;
     \item 直接写出 $ (\overline{R})^* $的元素的乘法表;
   \end{enumerate}

   解:
   \begin{enumerate}
     \item $ \overline{R} $的元素是陪集的代表元, 且代表元是多项式次数小于等于 $ 1 $的, 因此只有 $ 4 $种情况, 如题;
     \item 同构的群是克莱因群, 可以写为  $ \Z_2\times\Z_2 $, 其元素的加法表如下:
     \item 
   \end{enumerate}
   
   \vspace{60 pt}
\subsubsection*{Problem 3}
    假设 $ R $是非零含幺交换环, 对每一个 $ a\in R $都存在大于$1$的 $ n\in\Z^+ $使得 $ a^n=a $, 证明 $ R $的
    每个素理想都是极大理想. (hint: 证明对任意素理想 $ I $, 整环$ R/I $的每个非零元素都是单位). 

    解: 假设 $ I $是一个素理想且 $ a\in I $, 那么 $ a^n-a=0\rightarrow a(a^{n-1}-1)=0\in I $. 由于 $ I $是素理想, 则 $ a\in I $或者 $ a^{n-1}-1\in I $.
    假设 $ a\notin I $, 即 $ \overline{a}\in R/I $不是零元, 则 $ a^{n-1}-1\in I\rightarrow\overbrace{a^{n-1}}=\overline{1} $, 注意到 $ R/I $是整环, 所以
    $ \overline{a}\in R/I $是单位, 即整环的非零元均是单位, 则 $ R/I $是域, 也就是说 $ I $是极大理想.
\subsubsection*{Problem 4}
    假设 $ R,S $ 是含幺交换环, $ \phi:R\rightarrow S $ 是一个环的同态, 那么 $ Char(R) $是否等于 $ Char(S) $? 

    解: 不一定相等, 最简单的例子 $ \phi(\Z)\rightarrow\Z_2 $, $ Char(\Z)=0\neq Char(\Z_2)=2 $.
    \subsubsection*{Problem 5}
    假设 $ F $是特征为 $ p>0 $的域, 域的元素数量有限. 映射 $ \phi:F\rightarrow F $定义为 $ \phi(x)=x^p,x\in F $:
    那么\begin{enumerate}
      \item 映射 $ \phi $是否为环同态;
      \item 映射 $ \phi $是否为环同构;
    \end{enumerate} 

    解:\begin{enumerate}
      \item 是环同态: 显然是个映射; 因为交换, 所以满足乘法 $ \phi(x\cdot y)=(xy)^p=x^p\cdot y^p=\phi(x)\phi(y) $;直接使用书P158 例3, 满足加法 $ \phi(x+y)=(x+y)^p=x^p+y^p=\phi(x)+\phi(y) $;
      \item 是环同构: $ p=2 $时, $ x+y=x-y $故 $ (x-y)^p=x^p-y^p $; $ p>2 $为素数时, $ (x+(-y))^p=x^p+(-y)^p=x^p-y^p $; 借此证明单射: $ \forall a,b\in F $, 若
      $ \phi(a)=\phi(b) $, 则 $ a^p-b^p=(a-b)^p=0\rightarrow a=b $. 由因为 $ F $这个集合元素数量有限, 那么 $ \phi $是单射, 也必定是满射, 故 $ \phi $ 是环同构.
    \end{enumerate}
\subsubsection*{Problem 6}
    假设 $ I $是含幺交换环 $ R $的一个理想, 令 $ I[x] $是 $ R[x] $的理想 (多项式集合, 多项式的系数属于 $ I $), 证明: $ R[x]/I[x]\cong (R/I)[x] $, 且如果
    $ I $是 $ R $的素理想, 那么 $ I[x] $是 $ R[x] $的素理想. 

    解: 构造映射 $ \phi:R[x]\rightarrow R/I[x] $形如 $ \phi(\sum_{i=0}^{n}a_ix^i)=\sum_{i=0}^{n}\overline{a_i}x^i $. 显然 $ \phi $是一个满同态, $ ker(\phi)=I[x] $.
    则利用换同台基本定理得到 $ R[x]/I[x]\cong R/I[x] $, 又$ I $是含幺交换环 $ R $的一个素理想, 那么 $ R/I $是整环, $ R/I[x] $也是整环, 注意到 $ R[x] $同样是整环, 可以确定 
    $ I[x] $是素理想.
\subsubsection*{Problem 7}
    假设 $ p(x,y,z)=2x^2y-3xy^3z+4y^2z^5\in\Z[x,y,z] $.
    \begin{enumerate}
      \item 给出 $ p $的次数;
      \item 给出 $ p $关于未定元$ x $和 $ z $ 的次数;
      \item 当 $ q(x,y,z)=7x^2+5x^2y^3z^4 $, 写出 $ pq $在 $ z $是未定元, 系数属于 $ \Z[x,y] $时的多项式;
    \end{enumerate}

    解:\begin{enumerate}
      \item $ 2+5=7 $次
      \item $ x:2,z:5 $
      \item $ pq=20x^2y^5z^9+(28x^2y^2-15x^3y^6)z^5+10x^4y^4z^4-21x^3y^3z+14x^4y $.
    \end{enumerate}
\end{document}