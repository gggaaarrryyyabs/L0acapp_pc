\documentclass[runningheads,a4paper]{article}
\usepackage{graphicx}
\usepackage{indentfirst,mathrsfs}
\usepackage{amsfonts}
\usepackage{amsmath,amssymb}
\usepackage{color}
\usepackage{multirow}
\usepackage[citecolor=blue]{hyperref}
\usepackage{fullpage,enumitem,graphicx,empheq}
\usepackage{physics}


\newcommand{\Z}{\mathbf{Z}}
\newcommand{\Com}{\mathbf{C}}
\newcommand{\ord}{\operatorname{ord}}
\newcommand{\Q}{\mathbf{Q}}
\newcommand{\R}{\mathbf{R}}
%\renewcommand{\Tr}{\operatorname{Tr}_1^k}
\newtheorem{proof}{Proof}
%\documentclass[11 pt]{article}
%\usepackage{indentfirst,mathrsfs}
%\usepackage{amsfonts}
%\usepackage{amsmath,amsthm,amssymb}
%\usepackage[colorlinks,linkcolor=black,
%            pdftitle={title},
%              pdfauthor={author},
%              pdfkeywords={}]{hyperref}
%\renewcommand{\baselinestretch}{1.1}
%\setlength{\oddsidemargin}{0.1in} \setlength{\textwidth}{6.0in}
%\setlength{\topmargin}{-0.25in} \setlength{\textheight}{8.7in}
%\newtheorem{theorem}{Theorem}
%\newtheorem{corollary}{Corollary}
%\newtheorem{definition}{Definition}
%\newtheorem{proposition}{Proposition}
%\newtheorem{example}{Example}
\newtheorem{lemma}{Lemma}
\newtheorem{remark}{Remark}


\newtheorem{construction}{Construction}
\newcommand{\F}{\mathbb{F}}
\newcommand{\E}{\mathbb{E}}
\newcommand{\0}{\textbf{0}}
\newcommand{\1}{\textbf{1}}
\newcommand{\B}{\mathcal{B}}
\newcommand{\nl}{\mathrm{nl}}
 \renewcommand{\Tr}{\mathrm{Tr}_1^n}
 \renewcommand{\tr}{\mathrm{Tr}_1^k}
%
%\newcommand{\C}{{\mathcal C}}
%\newcommand{\E}{\mathcal{E}}
%\newcommand{\Z}{\mathbb{Z}}
\begin{document}

\begin{center}
\Large{Some Results on the Inverse Function}
\end{center}

\begin{center}
%Deng Tang
\end{center}

\section{Introduction}

\section{Preliminaries}
The Walsh transform of $f$ at point $\alpha \in \F_{2^n}$ is defined as
\begin{equation*}
\widehat{f}(\alpha)=\sum_{x \in \F_{2^n}}(-1)^{f(x)+\Tr(\alpha x)}.
\end{equation*}


% \begin{lemma}\label{L:Parsevalrelation}
% For any $n$-variable Boolean function $f$, we have  $$\sum_{a\in\F_{2^n}}\widehat{f}\;^2(a)=2^{2n}.$$
% \end{lemma}

% \begin{lemma}\label{L:autoshift}
% Let $f$ be an arbitrary $n$-variable Boolean function. For any $\alpha,\beta\in\F_{2^n}*$, we have
% $$\sum_{x\in\F_{2^n}}(-1)^{f(\alpha x)+f(\beta x)}=2^{-n}\sum_{u\in\F_{2^n}}\widehat{f}(\alpha^{-1}u)\widehat{f}(\beta^{-1}u).$$
% \end{lemma}

\section{The Walsh spectra of the derivatives of the inverse function}

For any integer $n>0$, let us define $I_\nu(x)=\Tr(\nu x^{-1})$ over $\B_n$.
The Kloosterman sums over $\F_{2^n}$ are defined as
$\mathcal{K}(a)=\widehat{I_1}(\alpha)=\sum_{x\in\F_{2^n}}(-1)^{\Tr(x^{-1}+\alpha x)}$, where $\alpha\in\F_{2^n}$.
In fact, the Kloosterman sums are generally defined on the multiplicative
group $\F_{2^n}^*$. We extend them to $0$ by assuming $(-1)^0=1$.
%\begin{lemma}[\cite{CarlitzKloo1969}]\label{L:Kloostermansumsone}
%For any integer $m>0$, $\mathcal{K}(1)=1-\sum_{t=0}^{\lfloor m/2\rfloor}(-1)^{m-t}\frac{m}{m-t}{{m-t}\choose {t}}2^t$.
%\end{lemma}
% \begin{lemma}[\cite{LW90}] \label{inverse-nl}
% For any positive integer $n$ and arbitrary $a\in\F_{2^n}^*$,  the
% Walsh spectrum of $I_1(x)$ defined on $\F_{2^n}$ can take
% any value divisible by $4$ in the range
% $[-2^{{n/2}+1}+1,2^{{n/2}+1}+1]$.
% \end{lemma}








% \begin{lemma}\label{inverse-df}
%      Let $n\ge 3$ be an integer. Define
%      $C_{{\mu,\nu}}(\tau)=\sum_{x\in\F_{2^n}}(-1)^{\Tr(\mu x^{-1}+\nu(x+\tau)^{-1})}$,
%      where $\mu,\nu,\tau\in\F_{2^n}^*$. Then the value of
%      $C_{{\mu,\nu}}(\tau)$ belongs to $[-2^{{n/2}+1}-3, 2^{{n/2}+1}+1]$ and is divisible by $4$.
%       More preciously,
% $$C_{{\mu,\nu}}(\tau)=(-1)^{\Tr(\frac{\mu}{\tau}+\frac{\nu}{\tau})}W_{I_1}\left(\frac{\mu \nu}{\tau^2}\right)-2\Big((-1)^{\Tr(\frac{\mu}{\tau})\Tr(\frac{\nu}{\tau})}-1\Big).$$
% \end{lemma}
    \begin{proof} For any $\mu,\nu,\tau\in\F_{2^n}^*$,
     we have (still using the convention $\frac 10=0$)
    \begin{eqnarray*}
    &&C_{{\mu,\nu}}(\tau)\\
    &=&\sum_{x\in\F_{2^n}}(-1)^{tr_1^n(\frac{\mu}{x}+\frac{\nu}{x+\tau})}\\
    &=&\sum_{x\in\F_{2^n}\setminus\{0,\tau\}}(-1)^{tr_1^n(\frac{\mu}{x}+\frac{\nu}{x+\tau})}+(-1)^{tr_1^n(\frac{\mu}{\tau})}+(-1)^{tr_1^n(\frac{\nu}{\tau})}\\
    &=&\sum_{x\in\F_{2^n}\setminus\{0,\tau^{-1}\}}(-1)^{tr_1^n(\mu x+\frac{\nu x}{1+\tau x})}+(-1)^{tr_1^n(\frac{\mu}{\tau})}+(-1)^{tr_1^n(\frac{\nu}{\tau})}\\
    &=&\sum_{x\in\F_{2^n}\setminus\{0,\tau^{-1}\}}(-1)^{tr_1^n(\mu x+\frac{1}{1+\tau x}\cdot\frac{\nu}{\tau}+\frac{\nu}{\tau})}+(-1)^{tr_1^n(\frac{\mu}{\tau})}+(-1)^{tr_1^n(\frac{\nu}{\tau})}\\
    &=&\sum_{x\in\F_{2^n}\setminus\{0,1\}}(-1)^{tr_1^n(\frac{\mu x}{\tau}+\frac{\nu}{\tau x}+\frac{\mu}{\tau}+\frac{\nu}{\tau})}+(-1)^{tr_1^n(\frac{\mu}{\tau})}+(-1)^{tr_1^n(\frac{\nu}{\tau})}\\
    &=&\sum_{x\in\F_{2^n}\setminus\{0,\frac{\tau}{\nu}\}}(-1)^{tr_1^n(\frac{1}{x}+\frac{\mu \nu}{\tau^2}x)+tr_1^n(\frac{\mu}{\tau}+\frac{\nu}{\tau})}+(-1)^{tr_1^n(\frac{\mu}{\tau})}+(-1)^{tr_1^n(\frac{\nu}{\tau})}\\
    &=&\sum_{x\in\F_{2^n}}(-1)^{tr_1^n(\frac{1}{x}+\frac{\mu \nu}{\tau^2}x)+tr_1^n(\frac{\mu}{\tau}+\frac{\nu}{\tau})}-
    (-1)^{tr_1^n(\frac{\mu}{\tau}+\frac{\nu}{\tau})}-(-1)^{tr_1^n(0)}+(-1)^{tr_1^n(\frac{\mu}{\tau})}+(-1)^{tr_1^n(\frac{\nu}{\tau})}
   \end{eqnarray*}
   where the third, fifth, and sixth identities hold by changing $x$ to ${1\over x}$, ${x+1\over \tau}$, and ${\nu x\over \tau}$ respectively.
   Note that $-(-1)^{tr_1^n(\frac{\mu}{\tau}+\frac{\nu}{\tau})}-(-1)^{tr_1^n(0)}+(-1)^{tr_1^n(\frac{\mu}{\tau})}+(-1)^{tr_1^n(\frac{\nu}{\tau})}$
   equals $0$ or $-4$. According to Lemma~\ref{inverse-nl}, we can see that $C_{{\mu,\nu}}(\tau)$ belongs to $[-2^{{n/2}+1}-3, 2^{{n/2}+1}+1]$ and is divisible by $4$.
   This finishes the proof.
   \end{proof}



%\begin{lemma}\label{L:solution}
%(Nyberg-Inverse)
%the solutions of $$\frac1x+\frac{1}{x+\alpha}=\beta$$
%\end{lemma}

\section{Lemmas}
\subsection{The multiplicative inverse function}
For any finite field $\F_{2^n}$, the multiplicative inverse function of $\F_{2^n}$, denoted by $I$, is defined as $I(x)=x^{2^n-2}$. In the sequel, we will use $x^{-1}$ or $\frac{1}{x}$ to
denote $x^{2^n-2}$ with the convention that $x^{-1}=\frac{1}{x}=0$ when $x=0$. We recall that, for any $v \neq 0$, $I_v(x) = \mathrm{Tr}_1^n(vx^{-1})$ is a component function of $I$.
The Walsh--Hadamard transform of $I_1$ at any point $\alpha$ is commonly known as Kloosterman sum over $\F_{2^n}$ at $\alpha$, which is usually denoted by $\mathcal{K}(\alpha)$,
i.e., $\mathcal{K}(\alpha)=\widehat{I_1}(\alpha)=\sum_{x\in\F_{2^n}}(-1)^{\mathrm{Tr}_1^n(x^{-1}+\alpha x)}$.
The original Kloosterman sums are generally defined on the multiplicative group $\F_{2^n}^*$. We extend them to $0$ by assuming $(-1)^0=1$. Regarding the Kloosterman sums,
the following results are well known and we will use them in the sequel.
% \begin{lemma}\cite{CarlitzKloo1969}
% \label{L:Kloostermansumsone}
% For any integer $n>0$, $\widehat{I_1}(1)=1-\sum_{t=0}^{\lfloor n/2\rfloor}(-1)^{n-t}\frac{n}{n-t}{{n-t}\choose {t}}2^t$.
% \end{lemma}
% \begin{lemma}\cite{LW90}
% \label{inverse-nl}
% For any positive integer $n$ and arbitrary $a\in\F_{2^n}^*$, the Walsh--Hadamard spectrum of $I_1(x)$ defined on $\F_{2^n}$ can take any value divisible by $4$ in the range
% $[-2^{{n/2}+1}+1,2^{{n/2}+1}+1]$.
% \end{lemma}
% Let $n=2t+1$ be an odd integer and $P$ be the largest positive integer such that $P \equiv 0 \pmod 4$ and $P\leq 2^{t+1}\sqrt{2}+1$.
% \begin{remark}
% \label{rem-max-min}
% The possible maximum absolute value of Walsh--Hadamard spectrum of $I_1$ over $\mathbb F_{2^n}$ is
% \begin{eqnarray*}
% \max_{\alpha\in\mathbb F_{2^n}^*}|\widehat{I_1}(\alpha)|=\left\{
% \begin{array}{llll}
% 2^{\frac{n}{2}+1}, &\mbox{ if } n \mbox{ is even}\\
% P, &\mbox{ if } n \mbox{ is odd}
% \end{array}
% \right.,
% \end{eqnarray*}
% where $P$ is as defined above.
% \end{remark}
% \subsection{Basic Lemmas}
% \begin{lemma}[\cite{MS1977}]\label{L:solutiondegree2}
% For any $(\alpha,\beta)\in\F_{2^n}^*\times\F_{2^n}$, we define a polynomial
% $\mu(x)=\alpha x^2+\beta x+\gamma\in\F_{2^n}[x]$.
% Then the equation $\mu(x)=0$ has 2 solutions if and only if $\mathrm{Tr}_1^n(\beta^{-2}\alpha\gamma)=0$.
% \end{lemma}
% \begin{lemma}\label{L:tr0sum}
% Let $n$ be a positive integer and $T_0=\{\upsilon^2+\upsilon : \upsilon\in\F_{2^n}\}$.
% Then we have
% \begin{eqnarray*}
% \sum_{x\in T_0}(-1)^{\Tr\left(\frac{1}{x+1}\right)}=\frac12(-1)^{n}\widehat{I_1}(1).
% \end{eqnarray*}
% \end{lemma}


% \begin{lemma}\label{L:rootssum}
% Let $n$ be a positive integer. We have
% \begin{eqnarray*}
% \sum_{\upsilon\in\F_{2^n}\setminus\F_2}(-1)^{\mathrm{Tr}_1^n
% \left(\frac{{\upsilon}^2}{\upsilon^2+\upsilon+1}\right)}=\sum_{\upsilon\in\F_{2^n}\setminus\F_2}(-1)^{\mathrm{Tr}_1^n\left(\frac{{\upsilon}^2+1}{\upsilon^2+\upsilon+1}\right)}=(-1)^n\left(\widehat{I}_1(1)-2\right).
% \end{eqnarray*}
% \end{lemma}

\begin{lemma}\label{L:SumInv00}
Let $n\geq 3$ be an arbitrary integer. We define
$$L=\#\left\{c\in\F_{2^n} : \mathrm{Tr}_1^n\left(\frac{1}{c^2+c+1}\right)=\mathrm{Tr}_1^n\left(\frac{c^2}{c^2+c+1}\right)=0\right\}.$$
Then we have $L=2^{n-2}+\frac{3}{4}(-1)^n\widehat{I_1}(1)+\frac{1}{2}\left(1-(-1)^n\right)$, where $ \widehat{I_1}(1)=1-\sum_{t=0}^{\lfloor n/2\rfloor}(-1)^{n-t}\frac{n}{n-t}{{n-t}\choose {t}}2^t $.
% where $\widehat{I_1}(1)$ can be computed using Lemma~\ref{L:Kloostermansumsone}.
\end{lemma}

Let $F$ be an $(n,m)$-function. For any $\gamma,\eta\in\F_{2^n}$ and
$\omega\in\F_{2^m}$, let us define
\begin{equation}
\label{2nddrivative}
\mathcal{N}_F(\gamma,\eta,\omega)=\#\left\{x\in\F_{2^n} : F(x)+F(x+\gamma)+F(x+\eta)+F(x+\eta+\gamma)=\omega\right\}.
\end{equation}
It is clear that for $\gamma=0$ or $\eta=0$ or $\gamma=\eta$, we have $\mathcal{N}_F(\gamma,\eta,0)=2^n$, and when $\omega\neq 0$, $\mathcal{N}_F(\gamma,\eta,\omega)=0$. If $F$ is the multiplicative inverse function over $\mathbb F_{2^n}$, we denote $\mathcal{N}_I(\gamma,\eta,\omega)$ by $\mathcal{N}(\gamma,\eta,\omega)$.

\begin{lemma}\label{Secondderivativesolution}
Let $n\geq 3$ be a positive integer and $\mathcal{N}(\gamma,\eta,\omega)$ be defined as in \eqref{2nddrivative}.
Let $\gamma,\eta$ be two elements of $\F_{2^n}^*$ such that $\gamma\neq \eta$. Then for any $\omega\in\F_{2^n}$,  we have $\mathcal{N}(\gamma,\eta,\omega)\in \{0,4,8\}$.
Moreover, the number of $(\gamma,\eta,\omega)\in\F_{2^n}^3$ such that $\mathcal{N}(\gamma,\eta,\omega)=8$ is
$$\left(2^{n-2}+\frac{3}{4}(-1)^{n}\widehat{I_1}(1)-\frac{5}{2}(-1)^{n}-\frac{3}{2}\right)\left(2^n-1\right).$$
\end{lemma}

\begin{lemma}\label{lemma:N_ij_tracefunction}
    Assume  $ k\ge 3 $, let $ N_{i,j} =\left\lvert\left\{x\in\F_{2^k}\middle|\Tr\left(\theta_1x+\gamma_1\right)=i,\Tr\left(\theta_2x+\gamma_2\right)=j\right\}\right\rvert $ 
    where  $ \gamma_1,\gamma_2\in\F_{2^k} $ and $ \theta_1,\theta_2\in\F_{2^k}^* $ are distinct, then $ N_{0,0} =2^{k-2} $.
% 	 N_{0,1}=N_{1,0}=N_{1,1}= 
\end{lemma}   
  
  \begin{proof}
      We have $ N_{0,0}+N_{0,1}+N_{1,0}+N_{1,1}=2^k $ and $ N_{0,0}+N_{0,1}=2^{k-1} $, $ N_{1,1}+N_{0,1}=2^{k-1}  $, then we get 
      $ N_{0,0} = N_{1,1} $. 
      Besides, $ N_{0,0}+N_{1,1} = \left\lvert\left\{x\in\F_{2^k}\middle|\Tr\left((\theta_1+\theta_2)x+(\gamma_1+\gamma_2)\right)=0\right\}\right\rvert=2^{k-1} $  since $ \theta_1\ne \theta_2 $. 
       Therefore $ N_{0,0}=2^{k-2} $.
%    	 and $ N_{0,1}=N_{1,0}=2^{k-2} $.
  \end{proof}
  
 \begin{lemma}\label{lemma:N_ijk_tracefunction}
    Assume  $ k\ge 3 $, 
  % Assume $ V_i=\left\{x\in\F_{2^k}\middle| \Tr\left(\theta_ix+c_i\right)=0 \right\} $ for $ i=1,2,3 $. 
  % It's obvious that $ \operatorname{dim}_{\F_2}(V_i)=k-1 $ for $ i=1,2,3 $ 
  % and $ \operatorname{dim}_{\F_2}(V_i\cap V_j)=k-2 $ for $ i\ne j $ 
  % and $ \operatorname{dim}_{\F_2}(V_1+V_2+V_3)=k $, then we have 
  % \begin{align*}
  %     \operatorname{dim}_{\F_2}(V_1\cap V_2\cap V_3)&=\operatorname{dim}_{\F_2}(V_1+V_2+V_3)-\sum_{i=1,2,3}\operatorname{dim}_{\F_2}(V_i)+\sum_{1\le i<j\le 3}\operatorname{dim}_{\F_2}(V_i\cap V_j)+\\
  %     &=k-3*(k-1)+3*(k-2)\\
  %     &=k-3.
  % \end{align*}
  let $ N_{i_1,i_2,i_3}=\left\lvert\left\{x\in\F_{2^k}\middle| \Tr\left(\theta_1x+\gamma_1\right)=i_1,\Tr\left(\theta_2x+\gamma_2\right)=i_2,\Tr\left(\theta_3x+\gamma_3\right)=i_3 \right\} \right\rvert$, 
  where  $ \gamma_1,\gamma_2,\gamma_3\in\F_{2^k} $ and $ \theta_1,\theta_2,\theta_3\in\F_{2^k}^* $ are distinct and satisfy 
  $ \theta_3\ne\theta_1+\theta_2 $. Then $ N_{0,0,0}= 2^{k-3} $.
\end{lemma}

\begin{proof}
  From equations 
  \begin{equation}\label{eq:from_lemma_1}\left\{\begin{alignedat}{3}
      &N_{0,0,0}+N_{0,0,1}=\left\lvert\left\{x\in\F_{2^k}\middle|\Tr\left(\theta_1x+\gamma_1\right)=0, \Tr\left(\theta_2x+\gamma_2\right)=0\right\}\right\rvert=2^{k-2}\\
      &N_{0,0,0}+N_{0,1,0}=\left\lvert\left\{x\in\F_{2^k}\middle|\Tr\left(\theta_1x+\gamma_1\right)=0, \Tr\left(\theta_3x+\gamma_3\right)=0\right\}\right\rvert=2^{k-2}\\
      &N_{0,0,0}+N_{1,0,0}=\left\lvert\left\{x\in\F_{2^k}\middle|\Tr\left(\theta_2x+\gamma_2\right)=0, \Tr\left(\theta_3x+\gamma_3\right)=0\right\}\right\rvert=2^{k-2}.\\
\end{alignedat}\right.\end{equation}
    we get $ N_{0,0,1}=N_{0,1,0}=N_{1,0,0} $. With the same reason we can also obtain  $ N_{0,1,1}=N_{1,0,1}=N_{1,1,0} $. 
    Because $ \theta_1+\theta_2+\theta_3\ne 0 $, we can get equations: 
   \begin{equation}\label{eq:sum_three_trace} \left\{\begin{alignedat}{2}
        &N_{0,0,1}+N_{0,1,0}+N_{1,0,0}+N_{1,1,1}=\left\lvert\left\{x\in\F_{2^k}\middle|\Tr\left(\left(\theta_1+\theta_2+\theta_3\right)x+\left(\gamma_1+\gamma_2+\gamma_3\right)\right)=1\right\}\right\rvert=2^{k-1}\\
        &N_{0,1,1}+N_{1,0,1}+N_{1,1,0}+N_{0,0,0}=\left\lvert\left\{x\in\F_{2^k}\middle|\Tr\left(\left(\theta_1+\theta_2+\theta_3\right)x+\left(\gamma_1+\gamma_2+\gamma_3\right)\right)=0\right\}\right\rvert=2^{k-1}.
     \end{alignedat}\right.\end{equation}
  Combine $ N_{0,0,1}=N_{0,1,0}=N_{1,0,0} $,  $ N_{0,1,1}=N_{1,0,1}=N_{1,1,0} $, equations \eqref{eq:sum_three_trace} with equations 
\begin{equation}\label{eq:sum_N_0jk}\left\{\begin{alignedat}{2}
    N_{0,0,0}+N_{0,0,1}+N_{0,1,0}+N_{0,1,1}=\left\lvert\left\{x\in\F_{2^k}\middle|\Tr\left(\theta_1x+\gamma_1\right)=0\right\}\right\rvert=2^{k-1}\\
    N_{1,0,0}+N_{1,0,1}+N_{1,1,0}+N_{1,1,1}=\left\lvert\left\{x\in\F_{2^k}\middle|\Tr\left(\theta_1x+\gamma_1\right)=1\right\}\right\rvert=2^{k-1}.\\
\end{alignedat}\right.\end{equation}
  we obtain the result $ N_{0,0,1}=N_{0,1,1} $. 
  Therefore from equations \eqref{eq:from_lemma_1} and equations \eqref{eq:sum_N_0jk} we have 
  \begin{equation}\left\{\begin{alignedat}{2}
          &N_{0,0,0}+N_{0,0,1}=2^{k-2}\\
          &N_{0,0,0}+3N_{0,0,1}=2^{k-1}.
  \end{alignedat}\right.\end{equation}
  and the solution is $ N_{0,0,0}=N_{0,0,1}=2^{k-3} $.
\end{proof}
  



\section{Profile of Dillon bent functions}



Dillon presented a $\mathcal{PS}$ bent function class $f(x,y)$ from $\F_{2^n}=\F_{2^k}^2$ to $\F_2$ as
\begin{eqnarray*}\label{Eqn_PS_bent}
\mathcal{D}(x,y)=g\left({x\over y}\right)
\end{eqnarray*}
where $g$ is a balanced Boolean function on $\F_{2^{k}}$ with $g(0)=0$, and ${x\over y}$ is defined to be $0$ if $y=0$ (we shall
always assume this kind of convention in the sequel).

In this paper, our goal is to give a lower bound on the third-order nonlinearity of the simplest $\mathcal{PS}$ bent function, \emph{i.e.}
\begin{eqnarray}\label{sub-bent}
f(x,y)=\tr(\frac{\lambda x}{y})
\end{eqnarray}
where $(x,y)\in\F_{2^k}^2$, $\lambda\in\F_{2^k}^{*}$ and $\tr(x)=\sum\limits_{i=0}^{n-1}x^{2^i}$ is the trace function from
$\F_{2^k}$ to $\F_2$.

Let us consider the Walsh transform of the second-order derivative of $f$ at points
$\alpha=(\alpha_1,\alpha_2),\beta=(\beta_1,\beta_2)\in\F_{2^k}^2$. We have
%\begin{eqnarray*}
%W_{D_{\beta}D_{\alpha}f}(\mu,\nu)&=&\sum\limits_{x\in\F_{2^k}}\sum\limits_{y\in\F_{2^k}}
%(-1)^{\tr\big(\frac{\lambda x}{y}+\frac{\lambda (x+\alpha_1)}{y+\alpha_2}+\frac{\lambda (x+\beta_1)}{y+\beta_2}
%+\frac{\lambda (x+\alpha_1+\beta_1)}{y+\alpha_2+\beta_2}+\mu x+\nu y\big)}\\
%&=& \sum\limits_{y\in\F_{2^k}}(-1)^{\tr\big(\frac{\lambda\alpha_1}{y+\alpha_2}+\frac{\lambda\beta_1}{y+\beta_2}
%+\frac{\lambda(\alpha_1+\beta_1)}{y+\alpha_2+\beta_2}+\nu y\big)}\\
%&&\times\sum\limits_{x\in\F_{2^k}}(-1)^{\tr\big((\frac{\lambda}{y}+\frac{\lambda}{y+\alpha_2}+\frac{\lambda}{y+\beta_2}+\frac{\lambda}{y+\alpha_2+\beta_2}+\mu)x\big)}.
%\end{eqnarray*}
    \begin{align}\label{eq:secondordersum}
       W_{D_{\beta}D_{\alpha}f}(\mu,\nu)=&\sum_{x\in\F_{2^k}}\sum_{y\in\F_{2^k}}(-1)^{\Tr\left(\frac{\lambda x}{y}+\frac{\lambda (x+\alpha_1)}{y+\alpha_2}+\frac{\lambda (x+\beta_1)}{y+\beta_2}+\frac{\lambda (x+\alpha_1+\beta_1)}{y+\alpha_2+\beta_2}+\mu x+\nu y\right)}\nonumber\\
        =&\sum_{y\in\F_{2^k}}(-1)^{\Tr\left(\frac{\lambda\alpha_1}{y+\alpha_2}+\frac{\lambda\beta_1}{y+\beta_2}+\frac{\lambda(\alpha_1+\beta_1)}{y+\alpha_2+\beta_2}+\nu y\right)}\nonumber\\
        &\times \sum_{x\in\F_{2^k}}(-1)^{\Tr\left(\left(\frac{\lambda}{y}+\frac{\lambda}{y+\alpha_2}+\frac{\lambda}{y+\beta_2}+\frac{\lambda}{y+\alpha_2+\beta_2}+\mu\right)x\right)}\nonumber\\
        =&\begin{cases}
            2^k\sum_{y\in S}(-1)^{\Tr\left(\frac{\lambda\alpha_1}{y+\alpha_2}+\frac{\lambda\beta_1}{y+\beta_2}+\frac{\lambda(\alpha_1+\beta_1)}{y+\alpha_2+\beta_2}+\nu y\right)},&~\text{if}~\frac{\lambda}{y}+\frac{\lambda}{y+\alpha_2}+\frac{\lambda}{y+\beta_2}+\frac{\lambda}{y+\alpha_2+\beta_2}=\mu~\text{has solutions};\\
            0, &~\text{otherwise}.
        \end{cases}
    \end{align}
    where $ S $ is the set of solutions $ \frac{\lambda}{y}+\frac{\lambda}{y+\alpha_2}+\frac{\lambda}{y+\beta_2}+\frac{\lambda}{y+\alpha_2+\beta_2}=\mu $.

    Thus, we consider the solutions of the equation 
    \begin{equation}\label{eq:coefficient}
        \frac{\lambda}{y}+\frac{\lambda}{y+\alpha_2}+\frac{\lambda}{y+\beta_2}+\frac{\lambda}{y+\alpha_2+\beta_2}=\mu,
    \end{equation}  
    we have 
    \begin{enumerate}[label=(\arabic{*})]
        \item If $ \alpha_2=\beta_2 $ or $ \alpha_2=0 $ or $ \beta_2=0 $, then \eqref{eq:coefficient} has $ 0 $ solution when
        $ \mu\ne 0 $ and has $ 2^k $ solutions otherwise;
        \item If $ \alpha_2\ne\beta_2 $ and $ \alpha_2,\beta_2\in\F_{2^k}^* $, then 
        \begin{enumerate}
            \item when $ \lambda(\alpha_2^2+\beta_2^2+\alpha_2\beta_2)+\mu(\alpha_2^2\beta_2+\alpha_2\beta_2^2)=0 $, 
            we confirm $ \{0,\alpha_2,\beta_2,\alpha_2+\beta_2\} $ are $ 4 $ solutions of \eqref{eq:coefficient};
            \item when $ \mu=0 $, we have \eqref{eq:coefficient} in the form 
            \[\frac{\lambda\alpha_2}{y(y+\alpha_2)}+\frac{\lambda\alpha_2}{y(y+\alpha_2)+\alpha_2\beta_2+\beta_2^2}=0.\]
            It has solutions only when $ \alpha_2=0 $ or $ \beta_2=0 $ or $ \alpha_2=\beta_2 $, contradiction, so it has 
            $ 0 $ solution;
            \item When $ \mu\ne 0 $, when $ \Tr\left(\frac{\lambda\alpha_2}{\mu\beta_2(\alpha_2+\beta_2)}\right)=0 $ and 
            $ \Tr\left(u\left(\left(\frac{\beta_2}{\alpha_2}\right)^2+\frac{\beta_2}{\alpha_2}\right)\right)=0\Leftrightarrow\Tr\left(\frac{\lambda\beta_2}{\mu\alpha_2(\alpha_2+\beta_2)}\right)=0 $, 
            we confirm $ \{y_0,y_0+\alpha_2,y_0+\beta_2,y_0+\alpha_2+\beta_2\} $ are $ 4 $ solutions of \eqref{eq:coefficient}, 
            where $ y_0 $ is a solution of \eqref{eq:coefficient} and 
            $ u $ is the solution of $ t^2+t+\frac{\lambda\alpha_2}{\mu\beta_2(\alpha_2+\beta_2)}=0 $ 
            with $ t=\frac{y(y+\alpha_2)}{\alpha_2\beta_2+\beta_2^2} $.
        \end{enumerate}
    \end{enumerate} 
    Thus, we conclude that \eqref{eq:coefficient} has $ 0 $ solution, 
    $ 4 $ solutions (which are $ \{0,\alpha_2,\beta_2,\alpha_2+\beta_2\} $ or $\{y_0,y_0+\alpha_2,y_0+\beta_2,y_0+\alpha_2+\beta_2\} $), 
    $ 8 $ solutions (which are $ \{0,\alpha_2,\beta_2,\alpha_2+\beta_2,y_0,y_0+\alpha_2,y_0+\beta_2,y_0+\alpha_2+\beta_2\} $).

    So we have the following cases:

    \textbf{CASE.1} (trivial) If $ \alpha_2=\beta_2 $ or $ \alpha_2=0 $ or $ \beta_2=0 $ with $ \mu\ne 0 $, 
    then equation \eqref{eq:coefficient} has $ 0 $ solution, then
    \[W_{D_{\beta}D_{\alpha}f}(\mu,\nu)=0.\]

    \textbf{CASE.2} ?(nontrivial) If $ \alpha_2=\beta_2 $ or $ \alpha_2=0 $ or $ \beta_2=0 $ with $ \mu=0 $, 
    then equation \eqref{eq:coefficient} has $ 2^k $ solutions, we confirm that 
    \begin{equation}\label{eq:case2ksolutions}
        W_{D_{\beta}D_{\alpha}f}(\mu,\nu)=2^k\sum_{y\in\F_{2^k}}(-1)^{\Tr\left(\frac{\lambda\alpha_1}{y+\alpha_2}+\frac{\lambda\beta_1}{y+\beta_2}+\frac{\lambda(\alpha_1+\beta_1)}{y+\alpha_2+\beta_2}+\nu y\right)}.
    \end{equation}
    
    Furthermore, if $ \alpha_1=\beta_1 $ or $ \alpha_1=0 $ or $ \beta_1=0 $, equation \eqref{eq:case2ksolutions} can be transformed into a simple? form:
    \begin{enumerate}[label=(\arabic{*})]
        \item   If $ \alpha_1=\beta_1 $, then 
        \begin{equation}
            W_{D_{\beta}D_{\alpha}f}(\mu,\nu)=2^k\sum_{y\in\F_{2^k}}(-1)^{\Tr\left(\frac{\lambda\alpha_1}{y+\alpha_2}+\frac{\lambda\alpha_1}{y+\beta_2}+\nu y\right)}.
        \end{equation}
        \item   If $ \alpha_1=0 $, then 
        \begin{equation}
            W_{D_{\beta}D_{\alpha}f}(\mu,\nu)=2^k\sum_{y\in\F_{2^k}}(-1)^{\Tr\left(\frac{\lambda\beta_1}{y+\beta_2}+\frac{\lambda\beta_1}{y+\alpha_2+\beta_2}+\nu y\right)}.
        \end{equation}
        \item   If $ \beta_1=0 $, then 
        \begin{equation}
            W_{D_{\beta}D_{\alpha}f}(\mu,\nu)=2^k\sum_{y\in\F_{2^k}}(-1)^{\Tr\left(\frac{\lambda\alpha_1}{y+\alpha_2}+\frac{\lambda\alpha_1}{y+\alpha_2+\beta_2}+\nu y\right)}.
        \end{equation}
    \end{enumerate}

    \textbf{CASE.3} (trivial) If $ \alpha_2\ne\beta_2 $ and $ \alpha_2,\beta_2\in\F_{2^k}^* $, when $ \mu=0 $, 
    we confirm \eqref{eq:coefficient} has $ 0 $ solution, then
    \[W_{D_{\beta}D_{\alpha}f}(\mu,\nu)=0.\]

    \textbf{CASE.4} (trivial) If $ \alpha_2\ne\beta_2 $ and $ \alpha_2,\beta_2\in\F_{2^k}^* $, when $ \mu\ne 0 $, 
    $ \lambda(\alpha_2^2+\beta_2^2+\alpha_2\beta_2)+\mu(\alpha_2^2\beta_2+\alpha_2\beta_2^2)\ne 0 $ with $ \Tr\left(\frac{\lambda\alpha_2}{\mu\beta_2(\alpha_2+\beta_2)}\right)=1 $ or $ \Tr\left(\frac{\lambda\beta_2}{\mu\alpha_2(\alpha_2+\beta_2)}\right)=1 $, 
    then \eqref{eq:coefficient} has $ 0 $ solution, we get 
    \[ W_{D_{\beta}D_{\alpha}f}(\mu,\nu)=0. \]

    \textbf{CASE.5} (nontrivial) If $ \alpha_2\ne\beta_2 $ and $ \alpha_2,\beta_2\in\F_{2^k}^* $, 
    when $ \mu\ne 0 $ and only one of the below two conditions holds:  
    \begin{enumerate}[label=\arabic{*})]
        \item $ \lambda(\alpha_2^2+\beta_2^2+\alpha_2\beta_2)+\mu(\alpha_2^2\beta_2+\alpha_2\beta_2^2)= 0 \Leftrightarrow \{0,\alpha_2,\beta_2,\alpha_2+\beta_2\}$ are solutions; 
        \item $ \Tr\left(\frac{\lambda\alpha_2}{\mu\beta_2(\alpha_2+\beta_2)}\right)=0 $ and $ \Tr\left(\frac{\lambda\beta_2}{\mu\alpha_2(\alpha_2+\beta_2)}\right)=0 \Leftrightarrow\{y_0,y_0+\alpha_2,y_0+\beta_2,y_0+\alpha_2+\beta_2\} $ are solutions.
    \end{enumerate}
    we confirm that \eqref{eq:coefficient} has $ 4 $ solution, assume $ 4 $ solutions are 
    $ S_4=\{y^{\prime},y^{\prime}+\alpha_2,y^{\prime}+\beta_2,y^{\prime}+\alpha_2+\beta_2\} $ 
    where $ y^{\prime}=0 $ or  $ y^{\prime}=y_0 $, then we have 
    % \begin{align}\label{eq:foursolutionsum}
    %     W_{D_{\beta}D_{\alpha}f}(\mu,\nu)=2^k&\sum_{y\in S_4}(-1)^{\Tr\left(\frac{\lambda\alpha_1}{y+\alpha_2}+\frac{\lambda\beta_1}{y+\beta_2}+\frac{\lambda(\alpha_1+\beta_1)}{y+\alpha_2+\beta_2}+\nu y\right)}\nonumber\\
    %     =2^k&\left((-1)^{\Tr\left(\frac{\lambda\alpha_1}{y^{\prime}+\alpha_2}+\frac{\lambda\beta_1}{y^{\prime}+\beta_2}+\frac{\lambda(\alpha_1+\beta_1)}{y^{\prime}+\alpha_2+\beta_2}+\nu y^{\prime}\right)}+(-1)^{\Tr\left(\frac{\lambda\alpha_1}{y^{\prime}}+\frac{\lambda\beta_1}{y^{\prime}+\alpha_2+\beta_2}+\frac{\lambda(\alpha_1+\beta_1)}{y^{\prime}+\beta_2}+\nu (y^{\prime}+\alpha_2)\right)}\right.\nonumber\\
    %     &\left.+(-1)^{\Tr\left(\frac{\lambda\alpha_1}{y^{\prime}+\alpha_2+\beta_2}+\frac{\lambda\beta_1}{y^{\prime}}+\frac{\lambda(\alpha_1+\beta_1)}{y^{\prime}+\alpha_2}+\nu (y^{\prime}+\beta_2)\right)}
    %     +(-1)^{\Tr\left(\frac{\lambda\alpha_1}{y^{\prime}+\beta_2}+\frac{\lambda\beta_1}{y^{\prime}+\alpha_2}+\frac{\lambda(\alpha_1+\beta_1)}{y^{\prime}}+\nu (y^{\prime}+\alpha_2+\beta_2)\right)}\right).
    % \end{align}
    % We can sum the first and last part of equation \eqref{eq:foursolutionsum} to get 
    % \begin{align*}
    %     &(-1)^{\Tr\left(\frac{\lambda\alpha_1}{y^{\prime}+\alpha_2}+\frac{\lambda\beta_1}{y^{\prime}+\beta_2}+\frac{\lambda(\alpha_1+\beta_1)}{y^{\prime}+\alpha_2+\beta_2}+\nu y^{\prime}\right)}
    %     +(-1)^{\Tr\left(\frac{\lambda\alpha_1}{y^{\prime}+\beta_2}+\frac{\lambda\beta_1}{y^{\prime}+\alpha_2}+\frac{\lambda(\alpha_1+\beta_1)}{y^{\prime}}+\nu (y^{\prime}+\alpha_2+\beta_2)\right)}\\
    %     =&(-1)^{\Tr\left(\frac{\lambda\alpha_1}{y^{\prime}+\alpha_2}+\frac{\lambda\beta_1}{y^{\prime}+\beta_2}+\frac{\lambda(\alpha_1+\beta_1)}{y^{\prime}+\alpha_2+\beta_2}+\nu y^{\prime}\right)}\cdot
    %     \left[1+(-1)^{\Tr\left(\frac{\lambda(\alpha_1+\beta_1)}{y^{\prime}} \frac{\lambda(\alpha_1+\beta_1)}{y^{\prime}+\alpha_2}+\frac{\lambda(\alpha_1+\beta_1)}{y^{\prime}+\beta_2}+\frac{\lambda(\alpha_1+\beta_1)}{y^{\prime}+\alpha_2+\beta_2}+\nu (\alpha_2+\beta_2)\right)}\right]\\
    %     =&(-1)^{\Tr\left(\frac{\lambda\alpha_1}{y^{\prime}+\alpha_2}+\frac{\lambda\beta_1}{y^{\prime}+\beta_2}+\frac{\lambda(\alpha_1+\beta_1)}{y^{\prime}+\alpha_2+\beta_2}+\nu y^{\prime}\right)}\cdot
    %     \left[1+(-1)^{\Tr\left(\mu(\alpha_1+\beta_1)+\nu (\alpha_2+\beta_2)\right)}\right].
    % \end{align*}
    % We can also sum the second and third parts of equation \eqref{eq:foursolutionsum} to get
    % \begin{align*}
    %     &(-1)^{\Tr\left(\frac{\lambda\alpha_1}{y^{\prime}}+\frac{\lambda\beta_1}{y^{\prime}+\alpha_2+\beta_2}+\frac{\lambda(\alpha_1+\beta_1)}{y^{\prime}+\beta_2}+\nu (y^{\prime}+\alpha_2)\right)}
    %     +(-1)^{\Tr\left(\frac{\lambda\alpha_1}{y^{\prime}+\alpha_2+\beta_2}+\frac{\lambda\beta_1}{y^{\prime}}+\frac{\lambda(\alpha_1+\beta_1)}{y^{\prime}+\alpha_2}+\nu (y^{\prime}+\beta_2)\right)}\\
    %     =&(-1)^{\Tr\left(\frac{\lambda\alpha_1}{y^{\prime}}+\frac{\lambda\beta_1}{y^{\prime}+\alpha_2+\beta_2}+\frac{\lambda(\alpha_1+\beta_1)}{y^{\prime}+\beta_2}+\nu (y^{\prime}+\alpha_2)\right)}\cdot
    %     \left[1+(-1)^{\Tr\left(\mu(\alpha_1+\beta_1)+\nu (\alpha_2+\beta_2)\right)}\right].
    % \end{align*}
    Then we have 
    \begin{align}\label{eq:simpleforms_4}
        &W_{D_{\beta}D_{\alpha}f}(\mu,\nu)\nonumber\\
        =&2^k\left[1+(-1)^{\Tr\left(\mu(\alpha_1+\beta_1)+\nu (\alpha_2+\beta_2)\right)}\right]\cdot
        \left[(-1)^{\Tr\left(\frac{\lambda\alpha_1}{y^{\prime}+\alpha_2}+\frac{\lambda\beta_1}{y^{\prime}+\beta_2}+\frac{\lambda(\alpha_1+\beta_1)}{y^{\prime}+\alpha_2+\beta_2}+\nu y^{\prime}\right)}+
        (-1)^{\Tr\left(\frac{\lambda\alpha_1}{y^{\prime}}+\frac{\lambda\beta_1}{y^{\prime}+\alpha_2+\beta_2}+\frac{\lambda(\alpha_1+\beta_1)}{y^{\prime}+\beta_2}+\nu (y^{\prime}+\alpha_2)\right)}\right]\nonumber\\
        =&2^k\left[1+(-1)^{\Tr\left(\mu(\alpha_1+\beta_1)+\nu (\alpha_2+\beta_2)\right)}\right]\cdot
        (-1)^{\Tr\left(\frac{\lambda\alpha_1}{y^{\prime}+\alpha_2}+\frac{\lambda\beta_1}{y^{\prime}+\beta_2}+\frac{\lambda(\alpha_1+\beta_1)}{y^{\prime}+\alpha_2+\beta_2}+\nu y^{\prime}\right)}\cdot
        \left[1+(-1)^{\Tr\left(\frac{\lambda\alpha_1}{y^{\prime}}+\frac{\lambda\alpha_1}{y^{\prime}+\alpha_2}+\frac{\lambda\alpha_1}{y^{\prime}+\beta_2}+\frac{\lambda\alpha_1}{y^{\prime}+\alpha_2+\beta_2}+\nu\alpha_2\right)}\right]\nonumber\\
        =&2^k\left[1+(-1)^{\Tr\left(\mu(\alpha_1+\beta_1)+\nu (\alpha_2+\beta_2)\right)}\right]\cdot
        \left[1+(-1)^{\Tr\left(\mu\alpha_1+\nu\alpha_2\right)}\right]\cdot
        (-1)^{\Tr\left(\frac{\lambda\alpha_1}{y^{\prime}+\alpha_2}+\frac{\lambda\beta_1}{y^{\prime}+\beta_2}+\frac{\lambda(\alpha_1+\beta_1)}{y^{\prime}+\alpha_2+\beta_2}+\nu y^{\prime}\right)}\nonumber\\
        =&\begin{cases}
            2^{k+2}\cdot(-1)^{\Tr\left(\frac{\lambda\alpha_1}{y^{\prime}+\alpha_2}+\frac{\lambda\beta_1}{y^{\prime}+\beta_2}+\frac{\lambda(\alpha_1+\beta_1)}{y^{\prime}+\alpha_2+\beta_2}+\nu y^{\prime}\right)},\text{if}~\Tr\left(\mu\alpha_1+\nu\alpha_2\right)=0 ~
            \text{and}~\Tr\left(\mu\beta_1+\nu\beta_2\right)=0 \\
            0,~\text{otherwise}
        \end{cases}
    \end{align}
    Observing \eqref{eq:simpleforms_4} we can easily find it only has values $ \{0,\pm 2^{k+2}\} $. 
    Besides, when it arrives at the values $ \pm 2^{k+2} $, we conclude that $ \Tr\left(\mu\alpha_1+\nu\alpha_2\right)=0 $
    and $ \Tr\left(\mu\beta_1+\nu\beta_2\right)=0 $. 

    
    \textbf{CASE.6} (nontrivial)
    If $ \alpha_2\ne\beta_2 $ and $ \alpha_2,\beta_2\in\F_{2^k}^* $, when $ \mu\ne 0 $ and both two conditions holds: 
    \begin{enumerate}[label=\arabic{*})]
        \item $ \lambda(\alpha_2^2+\beta_2^2+\alpha_2\beta_2)+\mu(\alpha_2^2\beta_2+\alpha_2\beta_2^2)= 0 $,
        % which means that $  \{0,\alpha_2,\beta_2,\alpha_2+\beta_2\}$ are solutions of equation \eqref{eq:coefficient}.
        \item $ \Tr\left(\frac{\lambda\alpha_2}{\mu\beta_2(\alpha_2+\beta_2)}\right)=0 $ and $ \Tr\left(\frac{\lambda\beta_2}{\mu\alpha_2(\alpha_2+\beta_2)}\right)=0  $. 
        % which means that $  \{y_0,y_0+\alpha_2,y_0+\beta_2,y_0+\alpha_2+\beta_2\} $ are solutions  of equation \eqref{eq:coefficient} where $ y_0\ne \alpha_2,\beta_2 $ or $ \alpha_2+\beta_2 $.
    \end{enumerate}
    then equation \eqref{eq:coefficient} has $ 8 $ distinct solutions 
    $ \{0,\alpha_2,\beta_2,\alpha_2+\beta_2,y_0,y_0+\alpha_2,y_0+\beta_2,y_0+\alpha_2+\beta_2\} $.
    
    Note that conditions $ \alpha_2,\beta_2\in\F_{2^k}^* $, $ \alpha_2\ne\beta_2 $ and $ \mu\ne 0 $ 
    can tell us $ \mu(\alpha_2^2\beta_2+\alpha_2\beta_2^2)\ne 0 $, 
    hence $ \lambda(\alpha_2^2+\beta_2^2+\alpha_2\beta_2)\ne 0 $, 
    implies $ \frac{\beta_2}{\alpha_2}\notin\F_8 $. 

    So take $ \mu=\lambda(\alpha_2^2+\beta_2^2+\alpha_2\beta_2)/(\alpha_2^2\beta_2+\alpha_2\beta_2^2) $ 
    into $ \Tr\left(\frac{\lambda\alpha_2}{\mu\beta_2(\alpha_2+\beta_2)}\right)=0 $ 
    and $ \Tr\left(\frac{\lambda\beta_2}{\mu\alpha_2(\alpha_2+\beta_2)}\right)=0  $ respectively, we can get 
    $ \Tr\left(\frac{c}{c^2+c+1}\right)=0 $ and $ \Tr\left(\frac{c^2}{c^2+c+1}\right)=0 $ 
    where $ c=\frac{\beta_2}{\alpha_2} $. Furthermore, according to Lemma \ref{L:SumInv00},  there always exist 

    And take the $ 8 $ solutions into equation \eqref{eq:secondordersum}, we get the summation
    \begin{align}\label{eq:case2kplus3}
        &W_{D_{\beta}D_{\alpha}f}(\mu,\nu)\nonumber\\
        =&2^k\left[1+(-1)^{\Tr\left(\mu(\alpha_1+\beta_1)+\nu (\alpha_2+\beta_2)\right)}\right]\cdot
        \left[1+(-1)^{\Tr\left(\mu\alpha_1+\nu\alpha_2\right)}\right]\nonumber\\
        &\cdot
        \left[(-1)^{\Tr\left(\frac{\lambda\alpha_1}{\alpha_2}+\frac{\lambda\beta_1}{\beta_2}+\frac{\lambda(\alpha_1+\beta_1)}{\alpha_2+\beta_2}\right)}+(-1)^{\Tr\left(\frac{\lambda\alpha_1}{y_0+\alpha_2}+\frac{\lambda\beta_1}{y_0+\beta_2}+\frac{\lambda(\alpha_1+\beta_1)}{y_0+\alpha_2+\beta_2}+\nu y_0\right)}\right]\nonumber\\
        =&(-1)^{c_0}2^k\cdot\left[1+(-1)^{\Tr\left(\mu(\alpha_1+\beta_1)+\nu (\alpha_2+\beta_2)\right)}\right]\cdot
        \left[1+(-1)^{\Tr\left(\mu\alpha_1+\nu\alpha_2\right)}\right]\cdot\left[1+(-1)^{c_0+c_1}\right]\nonumber\\
        =&\begin{cases}
            \pm 2^{k+3},~\text{if}~ \Tr\left(\mu\alpha_1+\nu\alpha_2\right)=0 , \Tr\left(\mu\beta_1+\nu\beta_2\right)=0 ~\text{and}~ c_0+c_1=0;\\
            0,~\text{otherwise}.
        \end{cases}
    \end{align}
    where $ c_0=\Tr\left(\frac{\lambda\alpha_1}{\alpha_2}+\frac{\lambda\beta_1}{\beta_2}+\frac{\lambda(\alpha_1+\beta_1)}{\alpha_2+\beta_2}\right) $ and 
    $ c_1= \Tr\left(\frac{\lambda\alpha_1}{y_0+\alpha_2}+\frac{\lambda\beta_1}{y_0+\beta_2}+\frac{\lambda(\alpha_1+\beta_1)}{y_0+\alpha_2+\beta_2}+\nu y_0\right)$.

    % If we fix the ratio of $ \beta_2 $ and $ \alpha_2 $: set $ \gamma=\frac{\beta_2}{\alpha_2}\in\F_{2^k}\setminus\F_{4} $ which holds $ \Tr\left(\frac{\gamma^2}{\gamma^2+\gamma+1}\right)=\Tr\left(\frac{\gamma^2}{\gamma^2+\gamma+1}\right)=0 $, then
    % we find $ \forall \alpha_1\in\F_{2^k}^* $, 
    % there always exists $ \beta_1\in\F_{2^k}^* $ s.t. $ W_{D_{\beta}D_{\alpha}f}(\mu,\nu)=\pm 2^{k+3} $, 
    % besides, we find that only $ \alpha_2,\beta_2,\alpha_1,\beta_1 $ can influence positive or negetive, and for every points 
    % $ \alpha,\beta $, there are $ 2^{k-3} $ $ \nu $'s leading to $ \pm 2^{k+3} $.

    Note that $ c_0+c_1=\Tr\left(\frac{\lambda\alpha_1}{\alpha_2}+\frac{\lambda\beta_1}{\beta_2}+\frac{\lambda(\alpha_1+\beta_1)}{\alpha_2+\beta_2}+\frac{\lambda\alpha_1}{y_0+\alpha_2}+\frac{\lambda\beta_1}{y_0+\beta_2}+\frac{\lambda(\alpha_1+\beta_1)}{y_0+\alpha_2+\beta_2}+\nu y_0\right) $.

    Therefore we need to determine for every points $ \alpha=(\alpha_1,\alpha_2) $ and $ \beta=(\beta_1,\beta_2) $ 
    with $ \frac{\beta_2}{\alpha_2}\in\F_{2^k}\setminus\F_{2^2} $ and $ y_0\notin\{0,\alpha_2,\beta_2,\alpha_2+\beta_2\} $, 
    whether or not there always exists $ \nu\in\F_{2^k} $ s.t. 
    \begin{equation}\label{eq:3trace0}\left\{\begin{alignedat}{3}
        &\Tr\left(\mu\alpha_1+\nu\alpha_2\right)=0\\ 
        &\Tr\left(\mu\beta_1+\nu\beta_2\right)=0\\
        &\Tr\left(\frac{\lambda\alpha_1}{\alpha_2}+\frac{\lambda\beta_1}{\beta_2}+\frac{\lambda(\alpha_1+\beta_1)}{\alpha_2+\beta_2}+\frac{\lambda\alpha_1}{y_0+\alpha_2}+\frac{\lambda\beta_1}{y_0+\beta_2}+\frac{\lambda(\alpha_1+\beta_1)}{y_0+\alpha_2+\beta_2}+\nu y_0\right)=0.
    \end{alignedat}\right.\end{equation}

    In fact  all of three equations are linear functions  since $ \mu $ are fixed once $ \alpha_2,\beta_2 $ are fixed, and 
    $ y_0 $ is also fixed since it's one of eight solutions of equation \eqref{eq:coefficient} and equation \eqref{eq:coefficient} is 
    determined by $ \lambda,\alpha_2,\beta_2 $ and $ \mu $. 
    
    Thus, using Lemma \ref{lemma:N_ijk_tracefunction}, we confirm that equations \eqref{eq:3trace0} have   solutions $ \nu\in\F_{2^k} $
    for every points $ \alpha=(\alpha_1,\alpha_2)\in\F_{2^k}\times\F_{2^k}^* $ and $ \beta=(\beta_1,\beta_2)\in\F_{2^k}\times\F_{2^k}^* $ 
    with $ \gamma=\frac{\beta_2}{\alpha_2}\in\F_{2^k}\setminus\F_{2^2} $ satisfying 
    $ \Tr\left(\frac{1}{\gamma^2+\gamma+1}\right)=\Tr\left(\frac{\gamma^2}{\gamma^2+\gamma+1}\right)=0 $ 
    and $ \mu=\frac{\lambda(\alpha_2^2+\beta_2^2+\alpha_2\beta_2)}{\alpha_2^2\beta_2+\alpha_2\beta_2^2} $.
    
 


    So equation \eqref{eq:case2kplus3} will always have points $ (\mu,\nu) $ leading to values $ \pm 2^{k+3} $ for every points 
    $ \alpha=(\alpha_1,\alpha_2)\in\F_{2^k}\times\F_{2^k}^* $ and $ \beta=(\beta_1,\beta_2)\in\F_{2^k}\times\F_{2^k}^* $ 
    with $ \gamma=\frac{\beta_2}{\alpha_2}\in\F_{2^k}\setminus\F_{2^2} $ satisfying 
    $ \Tr\left(\frac{1}{\gamma^2+\gamma+1}\right)=\Tr\left(\frac{\gamma^2}{\gamma^2+\gamma+1}\right)=0 $.

 




    % \textbf{Nothing}:

    % We find in that case, $ c_1 $ don't change the value when $ y_0 $ is one of the four solutions such that
    % \begin{align*}
    %     c_1=&\Tr\left(\frac{\lambda\alpha_1}{y_0+\alpha_2}+\frac{\lambda\beta_1}{y_0+\beta_2}+\frac{\lambda(\alpha_1+\beta_1)}{y_0+\alpha_2+\beta_2}+\nu y_0\right)\\
    %     =&\Tr\left(\frac{\lambda(\alpha_1+\beta_1)}{y_0+\alpha_2}+\frac{\lambda(\alpha_1+\beta_1)}{y_0+\beta_2}+\frac{\lambda(\alpha_1+\beta_1)}{y_0+\alpha_2+\beta_2}+\frac{\lambda(\alpha_1+\beta_1)}{y_0}+\frac{\lambda(\alpha_1+\beta_1)}{y_0}+\frac{\lambda\alpha_1}{y_0+\beta_2}+\frac{\lambda\beta_1}{y_0+\alpha_2}+\nu y_0\right)\\
    %     =&\Tr\left(\mu(\alpha_1+\beta_1)+\frac{\lambda(\alpha_1+\beta_1)}{y_0}+\frac{\lambda\alpha_1}{y_0+\beta_2}+\frac{\lambda\beta_1}{y_0+\alpha_2}+\nu y_0\right)\\
    %     =&\Tr\left(\frac{\lambda(\alpha_1+\beta_1)}{y_0}+\frac{\lambda\alpha_1}{y_0+\beta_2}+\frac{\lambda\beta_1}{y_0+\alpha_2}+\nu (y_0+\alpha_2+\beta_2)\right).
    % \end{align*}
    % The last equation holds iff both two equations $ \Tr\left(\mu\alpha_1+\nu\alpha_2\right)=0 $ 
    % and $ \Tr\left(\mu\beta_1+\nu\beta_2\right)=0 $ holds.



    %         % \begin{empheq}[left=\empheqlbrace]{align}
    %         %     &F_1(\vb*{x})+F_1(\vb*{x}+\vb*{a})=\vb*{b}\label{eq:3-0-1}\\
    %         %     &f_1(\vb*{x})+f_1(\vb*{x}+\vb*{a})=b_n\label{eq:3-0-2}.
    %         % \end{empheq}
    % From the first condition we have $ \mu=\frac{\lambda(\alpha_2^2+\beta_2^2+\alpha_2\beta_2)}{\alpha_2^2\beta_2+\alpha_2\beta_2^2} $
    % and we can subsititute $ \mu $ into the second condition then we have 
    % \[\Tr\left(\frac{\lambda\alpha_2}{\frac{\lambda(\alpha_2^2+\beta_2^2+\alpha_2\beta_2)}{\alpha_2^2\beta_2+\alpha_2\beta_2^2}\cdot\beta_2(\alpha_2+\beta_2)}\right)=\Tr\left(\frac{\alpha_2^2}{\alpha_2^2+\beta_2^2+\alpha_2\beta_2}\right)=\Tr\left(\frac{1}{\gamma^2+\gamma+1}\right)=0.\]
    % and 
    % \[\Tr\left(\frac{\lambda\beta_2}{\frac{\lambda(\alpha_2^2+\beta_2^2+\alpha_2\beta_2)}{\alpha_2^2\beta_2+\alpha_2\beta_2^2}\cdot\alpha_2(\alpha_2+\beta_2)}\right)=\Tr\left(\frac{\beta_2^2}{\alpha_2^2+\beta_2^2+\alpha_2\beta_2}\right)=\Tr\left(\frac{\gamma^2}{\gamma^2+\gamma+1}\right)=0.\]
    % where $ \gamma=\frac{\beta_2}{\alpha_2}\in\F_{2^k}\setminus\F_{2^2} $ since $ \mu\ne 0 $ and $ \alpha_2\ne\beta_2 $ in condition 1 
    % means $ \alpha_2^2+\beta_2^2+\alpha_2\beta_2\ne 0 $.

    

\end{document}



% for i in [1..2^5] do
%     inputvector:=Intseq(i-1,2,n);
%     eltvector:=&+[inputvector[j]*v^(j-1):j in [1..n]];
%     Append(~input,(eltvector));
% end for;

% for i in [1..2^n] do
%     inputvector:=Intseq(sbox[i],2,n);
%     eltvector:=&+[inputvector[j]*v^(j-1):j in [1..n]];
%     Append(~output,(eltvector));
% end for;

% function newsbox(x)
%     if x eq 0 then 
%         return 0; 
%     end if;
%     for i in [1..1024] do
%         if x eq input[i] then
%             return output[i];
%         end if;
%     end for;
% end function;



% G:=OrthoTest(newsbox,n);
% ddtG:=DDTexe(G,n);






\bibliographystyle{unsrt}
\bibliography{BF20160315}
\end{document}
