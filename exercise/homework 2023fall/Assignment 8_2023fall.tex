\documentclass[a4paper,12pt]{ctexart}
\usepackage{fullpage,enumitem,amsmath,amssymb,graphicx}
\newcommand{\Z}{\mathbf{Z}}
\newcommand{\F}{\mathbf{F}}
\newcommand{\Com}{\mathbf{C}}
\newcommand{\ord}{\operatorname{ord}}
\newcommand{\Q}{\mathbf{Q}}
\newcommand{\R}{\mathbf{R}}


\title{NIS2312-01 Fall 2023 Homework~8}
\author{唐灯}



\begin{document}
%   \maketitle
  \begin{center}

  \vspace{-0.3in}
  \begin{tabular}{c}
    \textbf{\Large NIS2312-01 Fall 2023-2024} \\
    \textbf{\Large  } \\
    \textbf{\Large  信息安全的数学基础(1)} \\
    \textbf{\Large  } \\
    \textbf{\Large  Assignment~8} \\
    \textbf{\Large  } \\
    \textbf{\Large 2023年10月14日} \\
  \end{tabular}
  \end{center}
  \noindent
  \rule{\linewidth}{0.4pt}
  
\subsubsection*{Problem 1}
  判断下列映射是否为同态映射:
  \begin{enumerate}[label=(\arabic{*})]
    \item 定义映射$\phi:\R^*\rightarrow \left\{ \pm 1 \right\}$, 其中$\phi(x)=\frac{x}{|x|}$, $x\in \R^*$, $|x|$代表$x$的绝对值.
    \item 定义映射$\pi:\Com^*\rightarrow\R^*$, 其中$\pi(a+b\sqrt{-1})=a^2+b^2$.
    \item 定义映射$\varphi:\R^2\rightarrow\R$, 其中$\varphi((x,y))=x+y$.
  \end{enumerate}

\subsubsection*{Problem 2} 
  [hint: 考虑单位元的象]能否找到一个非平凡映射$\phi$, 此映射将群$(\Z_4,+)$映射到群$(\Z_5,+)$. 
  (平凡映射指将任意元素映射为单位元, 见书82页例1).
\subsubsection*{Problem 3} 
  假设$G_1$和$G_2$是两个有限群且满足条件$(|G_1|,|G_2|)=1$, 同时假设$\phi:G_1\rightarrow G_2$是一个群同态. 证明: 
  \begin{enumerate}[label=(\arabic{*})]
    \item $\forall y\in G_2$, $\ord(y)\mid |G_1|$;
    \item $ker(\phi)=G$.
  \end{enumerate}
\end{document}