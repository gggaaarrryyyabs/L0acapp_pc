\documentclass[a4paper,12pt]{ctexart}
\usepackage{fullpage,enumitem,amsmath,amssymb,graphicx}
\newcommand{\Z}{\mathbf{Z}}
\newcommand{\F}{\mathbf{F}}
\newcommand{\Com}{\mathbf{C}}
\newcommand{\ord}{\operatorname{ord}}
\newcommand{\Q}{\mathbf{Q}}
\newcommand{\R}{\mathbf{R}}


\newtheorem{theorem}{Theorem}
\newtheorem{corollary}{Corollary}[theorem]
% \newtheorem{example}{例}
\newtheorem{innercustomthm}{Example}
\newtheorem{example}{Example}[subsection]
\newtheorem{lemma}{Lemma}
\newtheorem{definition}{Definition}
\newtheorem{remark}{Remark}
\newtheorem{proof}{Proof}


\begin{document}
\section*{$t+2$阶的TI结果}
    对于$n$比特次数为$t$的置换S盒$F(x)$, 对输入做$t+2$的share, 有$x=x_1+x_2+\cdots+x_{t+2}$, 则有$F(x)$的$t+2$阶TI$\mathcal{F}$如下:
    \begin{align}
        \mathcal{F}_1 &= x_1\\
        \mathcal{F}_2 &= x_2+G_1(x,x_1)\\
        \mathcal{F}_3 &= x_3+G_2(x,x_1,x_2)\\
        &\vdots\\
        \mathcal{F}_j &= x_j+G_{j-1}(x,x_1,x_2,\dots,x_{j-1})\\
        &\vdots\\
        \mathcal{F}_{t+2} &= \sum_{i=1}^{t+1}x_i + G_{t+1}(x_1,x_2,\dots,x_t),
    \end{align}
    其中 $j=2,3,...,t+1$ 和$\sum_{i=1}^{t+1}G_i=F(x)$. 
    该TI正确性和不完备性容易验证.

    下证明该TI的一致性, 即证明对于任意固定的输入$x\in\F_2^n$, 其share为$x=x_1+x_2+\cdots+x_{t+2}$, 当share遍历所有可能结果时, $\mathcal{F}$是个置换:

    当$x_1$遍历$\F_2^n$时, $\mathcal{F}_1$遍历$\F_2^n$;

    固定$x_1$, 当$x_2$遍历$\F_2^n$时, $\mathcal{F}_2=x_2+G_1(x,x_1)$是$x_2$的线性函数, 故遍历$\F_2^n$;

    固定$x_1,x_2$, 当$x_3$遍历$\F_2^n$时, $\mathcal{F}_3=x_3+G_2(x,x_1,x_2)$是$x_3$的线性函数, 故遍历$\F_2^n$;

    以此类推, 固定$x_1,x_2,\dots,x_{j-1}$, 当$x_j$遍历$\F_2^n$时, $\mathcal{F}_j=x_j+G_{j-1}(x,x_1,x_2,\dots,x_{j-1})$是$x_j$的线性函数, 故遍历$\F_2^n$, 其中$j=2,3,\dots,t+1$;
    
    固定$x_1,x_2,\dots,x_t$时, 当$x_{t+1}$遍历$\F_2^n$时, $\mathcal{F}_{t+2}=x_{t+1}+x_1+x_2+\cdots+x_t+G_t(x,x_1,x_2,\dots,x_t)$遍历$\F_2^n$.

    因此我们证明了, 当$(x_1,x_2,\dots,x_{t+1})$遍历$(\F_2^n)^{t+1}$时, $(\mathcal{F}_1,\mathcal{F}_2,\dots,\mathcal{F}_{t+1},\mathcal{F}_{t+2})$是一个置换.

    \section*{特殊形式的函数的$t+1$阶TI}
    设$G:\F_2^n\rightarrow\F_2^m$且次数为$t$, 且$x\in\F_2^n,y\in\F_2^m$, 则形如$F(x,y)=(x,G(x)+y)$的函数的$t+1$阶TI如下:
    $x$的$(t+1)$-share为$x=x_1+x_2+\cdots+x_{t+1}$, 同理$y$的$(t+1)$-share为$y=y_1+y_2+\cdots+y_{t+1}$, 则$(x,y)$的$(t+1)$-share为$(x_1,y_1),(x_2,y_2),\dots,(x_{t+1},y_{t+1})$, 而$G(x)$的$t+1$阶正确且不完备的TI为$G(x)=\sum_{i=1}^{t+1}\mathcal{G}_i(x_1,x_2,\dots,x_{i-1},x_{i+1},\dots,x_{t+1})$. 则$F(x,y)$的$t+1$阶TI结果$\mathcal{F}$的$t+1$个坐标函数为
    \begin{align}
        \mathcal{F}_1 &= (x_1,\mathcal{G}_2(x_1,x_3,\dots,x_{t+1})+y_1)&\text{缺少}(x_2,y_2)\\
        \mathcal{F}_2 &= (x_2,\mathcal{G}_3(x_1,x_2,x_4,\dots,x_{t+1})+y_2)&\text{缺少}(x_3,y_3)\\
        &\vdots&\\
        \mathcal{F}_i &= (x_i,\mathcal{G}_{i+1}(x_1,x_2,\dots,x_{i},x_{i+2},\dots,x_{t+1})+y_i)&\text{缺少}(x_{i+1},y_{i+1})\\
        &\vdots&\\
        \mathcal{F}_{t+1} &= (x_{t+1},\mathcal{G}_1(x_2,x_3,\dots,x_{t+1})+y_{t+1})&\text{缺少}(x_1,y_1),
    \end{align}
    其中$i=1,2,\dots,t$.

    此TI的正确性与不完备性易得.

    该TI的一致性证明转换为证明 $(\mathcal{F}_1,\mathcal{F}_2,\dots,\mathcal{F}_{t+1})$置换$(\F_2^n,\F_2^m)^{t+1}$即可. 固定$x_1,x_2,\dots,x_{t+1}$, 当$y_i$遍历$\F_2^m$时, $\mathcal{F}_i$遍历$(x_i,\F_2^m)$. 故$\mathcal{F}$是一个置换. 

    举例: $G(x)=x^3$且$m=n$ 故$t=\deg(G)=2$, 首先给出$G(x)$的$3$阶正确不完备的TI结果:
    对输入$x$进行3-share可得$x=x_1+x_2+x_3$, 故
    \begin{align}
        \mathcal{G}_1(x_2,x_3) &= x_2^3 + x_2^2x_3+x_2x_3^2&\text{缺少}x_1\\
        \mathcal{G}_2(x_1,x_3) &= x_3^3 + x_1^2x_3+x_1x_3^2&\text{缺少}x_2\\
        \mathcal{G}_3(x_1,x_2) &= x_1^3 + x_1^2x_2+x_1x_2^2&\text{缺少}x_3,
    \end{align}
    可以验证$\mathcal{G}$的TI是正确且不完备的. 
    故给出一个$F(x,y)=(x,G(x)+y)$的$3$阶TI结果: 
    \begin{align}
        \mathcal{F}_1 &= (x_1,G_2(x_1,x_3)+y_1)&\text{缺少}(x_2,y_2)\\
        \mathcal{F}_2 &= (x_2,G_3(x_1,x_2)+y_2)&\text{缺少}(x_3,y_3)\\
        \mathcal{F}_3 &= (x_3,G_1(x_2,x_3)+y_3)&\text{缺少}(x_1,y_1).
    \end{align}
\section*{Generalization of Lemma 1\footnote{Efficient and First-Order DPA Resistant Implementations of Keccak}?}
\begin{example}
    $F(x,y)=(x,G(x)+y)$ can be interpreted by this Lemma, where $x\in\F_2^t$, $y\in\F_2^s$ and $G:\F_2^t\rightarrow\F_2^s$. 
    The share of $x$ and $y$ are $\underline{x}=(x_1,x_2,\dots,x_n)$ and $\underline{y}=(y_1,y_2,\dots,y_n)$ with $x=x_1+x_2+\cdots+x_n$ and $y=y_1+y_2+\cdots+y_n$, respectively, where $n$ is the least number of share, that is, $n=\deg(G)+1$. 
    The correct and non-completeness share of $G(x)$ is $G(x)=\mathcal{G}_1(\underline{x})+\mathcal{G}_2(\underline{x})+\cdots+\mathcal{G}_n(\underline{x})$. 
    The uniformity comes from the independence between $G(x)$ and $y$. 
\end{example}
\begin{lemma}
    Let $(Y_1,Y_2,\dots,Y_n)$ be $m$-bit $n$-share (not necessarily uniform) of a fixed native value and $(X_1,X_2,\dots,X_n)$ be uniform $m$-bit $n$-share statistically independent of $(Y_1,Y_2,\dots,Y_{n-1})$. 
    Then, $(X_1 +Y_1, X_2 +Y_2,\dots, X_n + Y_n)$ are uniform $n$-share.
\end{lemma}    
\begin{proof}
    First, since $\sum_{i=1}^{n}X_i$ and $\sum_{i=1}^{n}Y_n$ take a fixed value with probability one, so does $\sum_{i=1}^{n}X_i+Y_i$. 
    Then, it suffices to verify that for each fixed value $x_1 + y_1, x_2+y_2,\dots,x_{n-1}+ y_{n-1}$: 
    \begin{align*}
        &\operatorname{Pr}\left[ X_1+Y_1=x_1+y_1,X_2+Y_2=x_2+y_2,\dots,X_{n-1}+ Y_{n-1}=x_{n-1}+ y_{n-1} \right]\\
        =&\sum_{x_i\in\F_2^n,i=1,2,\dots,n-1}\operatorname{Pr}\left[ X_1=x_1,X_2=x_2,\dots,X_{n-1}=x_{n-1} \right]\times\\
        &\qquad\qquad\qquad\operatorname{Pr}\left[ Y_1=(x_1+y_1)+x_1,Y_2=(x_2+y_2)+x_2,\dots,Y_{n-1}=(x_{n-1}+ y_{n-1})+x_{n-1} \right]\\
        ={}&2^{-m(n-1)}\sum_{x_i\in\F_2^n,i=1,2,\dots,n-1}\operatorname{Pr}\left[ Y_1=(x_1+y_1)+x_1,Y_2=(x_2+y_2)+x_2,\dots,Y_{n-1}=(x_{n-1}+y_{n-1})+x_{n-1} \right]\\
        ={}&2^{-m(n-1)},
    \end{align*}
    where the last equation comes from the statistically independent between $X_i$ and $Y_i$ for $i=1,2,\dots,n-1$. 
\end{proof}
此引理可以先保证uniform, 可能是TI一个方向. 
该引理保证了添加随机数来保持TI的uniform的正确性, 想法是添加一些函数, 这些函数既是随机数又和输入的share有关, 能否找到这样的函数呢?
是不是可以将函数的输入分为两部分, 且函数的输出同样可以分为这两部分的函数值的和?
\begin{example}
    找到了一个$F(x)=x^3$的一阶TI实现, 其中输入的share是$x=x_0+x_1+x_2+x_3$, 输出share是$y=y_0+y_1+y_2=x^3=(x_0+x_1+x_2+x_3)^3$.
    \begin{align*}
        y_0 &=  x_0 + x_0^3 + x_0^2x_2 + x_0^2x_3 + x_0x_2^2 + x_0x_3^2 + x_2^3 &\text{缺少}x_1\\
        y_1 &= x_1+x_1^2x_2 + x_1^2x_3 + x_1x_2^2 + x_1x_3^2  + x_2^2x_3 + x_2x_3^2 + x_3^3&\text{缺少}x_0\\
        y_2 &=  x_0 + x_1+x_0^2x_1 + x_0x_1^2 + x_1^3&\text{缺少}x_2,x_3,
    \end{align*}
    按照引理的结果, $X_0=x_0,X_1=x_1,X_2=x_0+x_1$和$Y_0=(x_0+x_2)^3+(x_0+x_3)^3+x_0^3+x_3^3$ (直观上看$Y_0$和$x_0,x_1,x_0+x_1$无关), $Y_1=(x_1+x_2)^3+(x_1+x_3)^3+(x_2+x_3)^3+x_3^3$(直观上看$Y_1$同样和$x_0,x_1,x_0+x_1$无关)和$Y_2=x_0+x_1+(x_0+x_1)^3+x_0^3$. 

    其他类似结果有 (在此只列出$(x_2+x_3)^3$和$x_0^3+x_1^3$的位置, 这里$x_0$和$x_1$可以互换, $x_2$和$x_3$可以互换):
    $\left[ (x_2+x_3)^3,x_1^3+x_2^3,x_0^3+x_2^3 \right]$, $\left[ (x_2+x_3)^3+x_0^3,x_1^3+x_2^3,x_2^3 \right]$和
    $\left[ (x_2^3+x_3^3)^3,x_3^3,x_0^3+x_1^3+x_3^3 \right]$

\end{example}
    但还是有说不清的地方, 尝试寻找其他的一阶TI实现: 
\begin{example}
    这里取$F(x)=x^7$, $s_{in}=t+2$且$s_{out}=t+1$, 其中$t=\deg(F)$, 其中输入的share是$x=x_0+x_1+x_2+x_3+x_4$, 输出share是$y=y_0+y_1+y_2+y_3=x^7=(x_0+x_1+x_2+x_3+x_4)^7$.

    首先有一个基础的结果, 下列share有$y_0+y_1+y_2+y_3+x_0^7+x_1^7+x_2^7+x_3^7+x_4^7+M(x_3,x_4)=x^7$, 且share是不完备的.
    \begin{align*}
        y_0 &= x_0+M(x_0,x_2,x_3)+M(x_0,x_2,x_4)+M(x_0,x_3,x_4)+M(x_0,x_3)+M(x_0,x_4)&\text{缺少}x_1\\
        y_1 &= x_1+M(x_0,x_1,x_3)+M(x_0,x_1,x_4)+M(x_1,x_3,x_4)+M(x_1,x_3)+M(x_1,x_4)&\text{缺少}x_2\\
        y_2 &= x_2+M(x_1,x_2,x_3)+M(x_1,x_2,x_4)+M(x_2,x_3,x_4)+M(x_2,x_3)+M(x_2,x_4)&\text{缺少}x_0\\
        y_3 &= x_0+x_1+x_2+M(x_0,x_1,x_2)+M(x_0,x_1)+M(x_0,x_2)+M(x_1,x_2)&\text{缺少}x_3,x_4,
    \end{align*}

\end{example}

\begin{example}[False]
    If we set 
    \begin{align}
        X_1&=x_1,&Y_1=0\\
        X_2&=x_2,&Y_2=G_1(x,x_1)\\
        X_3&=x_3,&Y_3=G_2(x,x_1,x_2)\\
        \dots& &\dots\\
        X_{t+1}&=x_{t+1},&Y_{t+1}=G_t(x,x_1,x_2,\dots,x_t)\\
        X_{t+2}&=x_1+x_2+\cdots+x_{t+1},&Y_{t+2}=G_{t+1}(x_1,x_2,\dots,x_t),
    \end{align}
    we can obtain that both $\sum_{i=1}^{t+2}X_i=0$ and $\sum_{i=1}^{t+2}Y_i=F(x)$ are fixed values, and $(X_1,X_2,\dots,X_{t+2})$ are statistically independent??? of $(Y_1,Y_2,\dots,Y_{t+2})$ and uniform.
\end{example}



    \section*{More generalized Lemma 1, False}
    \begin{lemma}
        Let $(Y_1,Y_2,\dots,Y_n)$ be $m$-bit $n$-share (not necessarily uniform) of a fixed native value and $(X_1,X_2,\dots,X_n)$ be uniform $m$-bit $n$-share, where $(X_1,X_2,\dots,X_{n-1})$ is statistically independent of $(Y_1,Y_2,\dots,Y_{n-1})$. 
        Then, $(X_1 +Y_1, X_2 +Y_2,\dots, X_n + Y_n)$ are uniform $n$-share. 
    \end{lemma}
    \begin{proof}[$f(x)+g(y)$ is balanced over $(\F_2^n)^2$ if $f(x)$ is balanced over $\F_2^n$]
        For any fixed value $X=X_1+X_2+\cdots+X_n$, $(X_1,X_2,\dots,X_{n-1})$ is balanced over $\F_2^n$ since $(X_1,X_2,\dots,X_n)$ is uniform. 
        Note that $(X_1,X_2,\dots,X_{n-1})$ is statistically independent of $(Y_1,Y_2,\dots,Y_{n-1})$, which means $(X_1+Y_1,X_2+Y_2,\dots,X_{n-1}+Y_{n-1})$ is also balanced over $\F_2^n$, that is, $(X_1 +Y_1, X_2 +Y_2,\dots, X_n + Y_n)$ are uniform $n$-share.
        Note that $X$ is a fixed known value, which means $Y_i$ is independent with $(X_1,X_2,\dots,X_{n-1},X_n)\cap$  
    \end{proof}

\end{document}







F<v>:=GF(2,4);
flag := 0;
for x in F do 
    if flag ne 0 then break; end if;
    y:={* *};
    for x0,x1,x2,x3 in F do
        x4 := x + x0 + x1 + x2 + x3;
        y0 :=  x0^6*x2 + x0^6*x3 + x0^6*x4 + x0^5*x2^2 + x0^5*x3^2 + x0^5*x4^2 + x0^4*x2^3 + x0^4*x2^2*x3 + x0^4*x2^2*x4 + x0^4*x2*x3^2 + x0^4*x2*x4^2 + x0^4*x3^3 + x0^4*x3^2*x4 + x0^4*x3*x4^2 + x0^4*x4^3 + x0^3*x2^4 + x0^3*x3^4 + x0^3*x4^4 + x0^2*x2^5 + x0^2*x2^4*x3 + x0^2*x2^4*x4 + x0^2*x2*x3^4 + x0^2*x2*x4^4 + x0^2*x3^5 + x0^2*x3^4*x4 + x0^2*x3*x4^4 + x0^2*x4^5 + x0*x2^6 + x0*x2^4*x3^2 + x0*x2^4*x4^2 + x0*x2^2*x3^4 + x0*x2^2*x4^4 + x0*x3^6 + x0*x3^4*x4^2 + x0*x3^2*x4^4 + x0*x4^6 + x0 + x3^5*x4^2 + x3^4*x4^3 + x3^3*x4^4 + x3^2*x4^5 +x3^7;
        y1 := x0^6*x1 + x0^5*x1^2 + x0^4*x1^3 + x0^4*x1^2*x3 + x0^4*x1^2*x4 + x0^4*x1*x3^2 + x0^4*x1*x4^2 + x0^3*x1^4 + x0^2*x1^5 + x0^2*x1^4*x3 + x0^2*x1^4*x4 + x0^2*x1*x3^4 + x0^2*x1*x4^4 + x0*x1^6 + x0*x1^4*x3^2 + x0*x1^4*x4^2 + x0*x1^2*x3^4 + x0*x1^2*x4^4 + x1^6*x3 + x1^6*x4 + x1^5*x3^2 + x1^5*x4^2 + x1^4*x3^3 + x1^4*x3^2*x4 + x1^4*x3*x4^2 + x1^4*x4^3 + x1^3*x3^4 + x1^3*x4^4 + x1^2*x3^5 + x1^2*x3^4*x4 + x1^2*x3*x4^4 + x1^2*x4^5 + x1*x3^6 + x1*x3^4*x4^2 + x1*x3^2*x4^4 + x1*x4^6 + x1;
        y2 := x1^6*x2 + x1^5*x2^2 + x1^4*x2^3 + x1^4*x2^2*x3 + x1^4*x2^2*x4 + x1^4*x2*x3^2 + x1^4*x2*x4^2 + x1^3*x2^4 + x1^2*x2^5 + x1^2*x2^4*x3 + x1^2*x2^4*x4 + x1^2*x2*x3^4 + x1^2*x2*x4^4 + x1*x2^6 + x1*x2^4*x3^2 + x1*x2^4*x4^2 + x1*x2^2*x3^4 + x1*x2^2*x4^4  + x2^6*x3 + x2^6*x4 + x2^5*x3^2 + x2^5*x4^2 + x2^4*x3^3 + x2^4*x3^2*x4 + x2^4*x3*x4^2 + x2^4*x4^3 + x2^3*x3^4 + x2^3*x4^4 + x2^2*x3^5 + x2^2*x3^4*x4 + x2^2*x3*x4^4 + x2^2*x4^5 + x2*x3^6 + x2*x3^4*x4^2 + x2*x3^2*x4^4 + x2*x4^6 + x2 +  x3^6*x4 + x3*x4^6 + x3^7;
        y3 := x0^4*x1^2*x2 + x0^4*x1*x2^2 + x0^2*x1^4*x2 + x0^2*x1*x2^4 + x0*x1^4*x2^2 + x0*x1^2*x2^4 + x0 + x1 + x2 + x0^7+ x2^7 + x1^7 ;
        Include(~y,[y0,y1,y2,y3]);
    end for;
    share_out_freq := Multiplicities(y);
    for i in share_out_freq do 
        if i ne 2^4 then 
            flag := 1;
            print "this share is not uniform";
            break;
        end if;
    end for;
    if x eq 0 then 
        print "this share is uniform";
    end if;
end for;






y1:=x1+x1^3+x4^3+x1x4^2+x1^2x4;
y2:=x2+x2^3+x1^2x2+x1x2^2+x2^2x4+x2x4^2;
y3:=x3+x3^3+x2^2x3+x2x3^2+x3^2x4+x3x4^2;
y4:=x1+x2+x3+x1^2x3+x1x3^2;










{ x1+x1x4^2+x1^2x4:x1,x2,x3,x4 in F };

{ x1+x1^3+x4^3+x1x4^2+x1^2x4:x1,x2,x3,x4 in F };

{ x2+x2^3+x1^2x2+x1x2^2+x2^2x4+x2x4^2:x1,x2,x3,x4 in F };
{ x3+x3^3+x2^2x3+x2x3^2+x3^2x4+x3x4^2:x1,x2,x3,x4 in F };
{ x1+x2+x3+x1^2x3+x1x3^2:x1,x2,x3,x4 in F };