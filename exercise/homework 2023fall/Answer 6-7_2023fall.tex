\documentclass[a4paper,12pt]{ctexart}
\usepackage{fullpage,enumitem,amsmath,amssymb,graphicx,xcolor}
\newcommand{\Z}{\mathbb{Z}}
\newcommand{\F}{\mathbb{F}}
\newcommand{\Com}{\mathbf{C}}
\newcommand{\ord}{\operatorname{ord}}
\newcommand{\Q}{\mathbb{Q}}
\newcommand{\R}{\mathbb{R}}
\newtheorem{theorem}{Theorem}
\newtheorem{corollary}{Corollary}[theorem]
% \newtheorem{example}{例}
\newtheorem{innercustomthm}{Example}
\newtheorem{example}{Example}[subsection]
\newtheorem{lemma}{Lemma}
\newtheorem{definition}{Definition}
\newtheorem{remark}{Remark}
\newtheorem{proof}{Proof}

\title{NIS2312-01 2023-2024 Fall Answer 6-7}
\author{唐灯}



\begin{document}
%   \maketitle
  \begin{center}

  \vspace{-0.3in}
  \begin{tabular}{c}
    \textbf{\Large NIS2312-01 2023-2024 Fall} \\
    \textbf{\Large  } \\
    \textbf{\Large  信息安全的数学基础(1)} \\
    \textbf{\Large  } \\
    \textbf{\Large  Answer~6-7} \\
    \textbf{\Large  } \\
    \textbf{\Large 2023年10月13日} \\
  \end{tabular}
  \end{center}
  \noindent
  \rule{\linewidth}{0.4pt}
  
%   可以使用计算机求模的运算.
$ \R $是实数域, $ \Q $是有理数域, $ \Z $是整数集合.
\section*{Assignment 6}
\subsubsection*{Problem 1}
  在 $ (\Z_{12},+) $中, 求子群 $ H=\langle \overline{4}\rangle $的所有左陪集.
      
  解: 可知 $ H=\{\overline{0},\overline{4},\overline{8}\} $, 因此左陪集有
  \begin{align*}
    &\overline{0}+H=H=\{\overline{0},\overline{4},\overline{8}\};\\
    &\overline{1}+H=\{\overline{1},\overline{5},\overline{9}\};\\
    &\overline{2}+H=\{\overline{2},\overline{6},\overline{10}\};\\
    &\overline{3}+H=\{\overline{3},\overline{7},\overline{11}\}.
  \end{align*}

\subsubsection*{Problem 2} 
  设 $ H=\{0,\pm 3,\pm 6,\pm 9,\dots\} $. 求子群 $ H $在 $ \Z $中的所有左陪集.
  
  解: 子群 $ H $在 $ \Z $中的所有左陪集为
  \begin{align*}
    0+H=\{0,\pm 3,\pm 6,\pm 9,\dots\};\\
    1+H=\{1,1\pm 3,1\pm 6,1\pm 9,\dots\};\\
    2+H=\{2,2\pm 3,2\pm 6,2\pm 9,\dots\}.
  \end{align*}

\subsubsection*{Problem 3}
  设 $ \operatorname{ord} a = 30 $. 问 $ \langle a^4\rangle $ 在  $ \langle a\rangle $中有多少个左陪集? 试将它们列出.

  解: $ \operatorname{ord}a^4=30/(30,4)=15 $, 故 $ \langle a^4\rangle $ 在  $ \langle a\rangle $中有 $ 30/15=2 $ 个左陪集. 
  其中之一为 $ e\langle a^4\rangle $, 因为 $ a\notin \langle a^4\rangle $, 故 $ a $属于另一个左陪集, 则另一个为 $ a\langle a^4\rangle $.

\subsubsection*{Problem 4}
  设 $ H_1,H_2 $是 $ G $的子群. 证明: $ a\left(H_1\cap H_2\right)=aH_1\cap aH_2 $.

  解: $ \subseteq $: $ \forall x\in H_1\cap H_2 $, 都有 $ x\in H_1,H_2 $, 故 $ ax\in aH_1,aH_2 $, 即 $ ax\in aH_1\cap aH_2 $, 
  因此 $ a\left(H_1\cap H_2\right)\subseteq aH_1\cap aH_2 $;

  $ \supseteq $: $ \forall x\in aH_1\cap aH_2 $, 都有 $ x=ah_1=ah_2 $, 其中 $ h_1\in H_1,h_2\in H_2 $. 则 $ a^{-1}x\in H_1 $, 
  $ a^{-1}x\in H_2 $, 即 $ a^{-1}x\in H_1\cap H_2 $, 那么 $ x\in a\left( H_1\cap H_2\right) $, 因此 $ a\left(H_1\cap H_2\right) \supseteq aH_1\cap aH_2 $.

  综上, $ a\left(H_1\cap H_2\right)=aH_1\cap aH_2 $.

\subsubsection*{Problem 5}
  设 $ H $是有限群 $ G $的子群, $ K $是 $ H $的子群. 证明: $ \left[G:K\right]=\left[G:H\right]\left[H:K\right] $.

  解: 根据拉格朗日定理可知, $ \left[G:K\right]= \frac{|G|}{|K|}=\frac{|G|}{|H|}\times\frac{|H|}{|K|}=\left[G:H\right]\left[H:K\right] $.

\subsubsection*{Problem 6}
  证明: $ 15 $ 阶群至多含有一个 $ 5 $ 阶子群.

  解: 第一种方法: 假设$5$阶子群至少有$2$个, 分别设为 $H,K$. 
  
  因为 $ 5 $是素数, 故 $ 5 $阶群是循环群, 即$H=<h>=\left\{ e,h,h^2,h^3,h^4 \right\}$和$K=<k>=\left\{ e,k,k^2,k^3,k^4 \right\}$, 且元素的阶只有$1,5$, 即$5$阶群的元素只有单位元和生成元. 
  那么任意两个不同的 $ 5 $ 阶群交集为单位元, 否则, $H$和$K$有相同的生成元, 则 $H=K$. 

  故构造集合 $ HK=\{hk: h\in H,k\in K\} $, 显然 $ HK\subseteq G $, 即$|HK|\le|G|=15$. 
  现讨论$|HK|$的大小. 假设存在 $ h_1,h_2\in H,k_1,k_2\in K $, 
  s.t. $ h_1k_1=h_2k_2 $, 则 $ h_2^{-1}h_1=k_2k_1^{-1}\in H\cap K=\{e\} $, 即 $ h_2=h_1,k_2=k_1 $. 因此集合  $ HK $元素数量为 $ 5\times 5=25>15=|G| $, 矛盾, 故至多含有一个 $ 5 $ 阶子群.

  第二种方法:   不妨假设$15$阶群$G$存在两个$5$阶子群$H$和$K$. 
  对$\forall h\in H,k \in K,hk \in G$
  \begin{align*}
  |HK| &= |\bigcup_{h \in H} hK | \\
  &= |\{hK|h\in H\}|\cdot |K| 
  \end{align*}
  
  不妨设存在对应关系$\phi :h(H\cap K) \longmapsto hK,\forall h \in H$
  若$h_1(H\cap K) = h_2(H \cap K)$,则 $h_1^{-1}h_2 \in H \cap K$
  所以$h_1^{-1}h_2 \in K$,即$h_1K = h_2K$,所以$\phi$ 是一个映射.
  若$h_1K = h_2K$,则$h_1^{-1}h_2 \in K$,而$h_1,h_2 \in H$,所以$h_1^{-1}h_2 \in H$,即$h_1^{-1}h_2 \in H \cap K$,故 $h_1(H\cap K) = h_2(H \cap K)$,所以$\phi$ 是一个单射
  $\forall h \in H,hK = \phi (h(H\cap K))$,所以$\phi$ 是一个满射$
  $综上所述,$\phi$ 是从$\{h(H \cap K)| h \in H\}$到$\{hK|h \in H\}$的一个一一映射.
  所以$|\{hK|h\in H\}| = |\{h(H \cap K)| h \in H\}|$
  \begin{align*}
  |HK| &= |\{hK|h\in H\}|\cdot |K|\\
  &= |\{h(H \cap K)| h \in H\}| \cdot |K|\\
  &= [H:H \cap K] \cdot |K| \\ 
  &= |H||K|/|H \cap K|
  \end{align*}
  $\forall x \in HK, \exists h \in H,k \in K,s.t. x = hk \in G$
  所以$|G| \geq |HK| = |H||K|/|H \cap K|$
\section*{Assignment 7}
  \subsubsection*{Problem 1}
  证明: 群 $ G $的中心 $ C(G) $是 $ G $的正规子群.      
  
  解:  $C(G)$是群$G$的子群结论在之前的作业中已经证明. 
  因为 $ C(G)=\{a\in G\mid \forall g\in G,ag=ga\} $, 
  故 $ \forall g\in G,\forall a\in C(G) $, 都有 
  $ gag^{-1}=agg^{-1}=a\in C(G) $, 即 $ gC(G)g^{-1}\subseteq C(G) $, 因此 $ C(G)\triangleleft G $.

\subsubsection*{Problem 2} 
    证明: 群的两个正规子群的交或者积都是正规子群.
  
    解: 
    设群 $ G $的两个正规子群为 $ H,K $. 
    
    \textcolor{red}{两个正规子群的积仍然是子群: 显然$HK=\left\{ hk:h\in H,k\in K \right\}$是非空的. 对于任意$h_1k_1,h_2k_2\in HK$, 都有$h_1,h_2\in H$和$k_1,k_2\in K$, 因此有$(h_1k_1)(h_2k_2)^{-1}=h_1k_1k_{2}^{-1}h_2^{-1}$.
    由于$H\triangleleft G$, 故存在$h_0\in H$, s.t. $k_1k_2^{-1}h_2^{-1}=h_0k_1k_2^{-1}$, 则$(h_1k_1)(h_2k_2)^{-1}=h_1h_0k_1k_2^{-1}\in HK$, 故$HK$是$G$的子群.}
    则对任意 $ g\in G $, $ \forall hk\in HK $, 
    都有 $ ghkg^{-1}=ghg^{-1}gkg^{-1}\in HK $成立, 故 $ HK $为 $ G $的正规子群;

    显然, 子群的交仍然是子群, 故$H\cap K<G$成立. 
    对任意 $ g\in G $, $ \forall x\in H\cap K $, 都有 $ gxg^{-1}\in gHg^{-1}=H $, $ gxg^{-1}\in gKg^{-1}=K $, 故
    $ gxg^{-1}\in H\cap K $, 故 $ H\cap K $为 $ G $的正规子群.
  
\subsubsection*{Problem 3}
  设 $ G $为群, $ H $是 $ G $的子群. 定义 $ H $的正规化子 (normalizer) 为
  \[N(H)=\{g\in G\mid gHg^{-1}=H\}.\]
  证明: $ N(H) $是 $ G $的子群, $ H $ 是 $ N(H) $的正规子群.

  解: 由于$e\in N(H)$, 故$N(H)$是非空的. 由$ N(H)=\{g\in G\mid gHg^{-1}=H\}=\{g\in G\mid H=g^{-1}Hg\}=\{g^{-1}\in G\mid gHg^{-1}=H\}$可知 对任意$g\in N(H)$都有$g^{-1}\in N(H)$. 
  且 $ \forall x,y\in N(H) $ 都有 $ xyH(xy)^{-1}=xyHy^{-1}x^{-1}=xHx^{-1}=H $, 即 $ xy\in N(H) $. 故$ N(H) $是 $ G $的子群;

  $ \forall g\in N(H) $ 都有 $ gHg^{-1}=H $, 故 $ H $ 是 $ N(H) $的正规子群.
\subsubsection*{Problem 4}
  设 $ G $为群, $ H\triangleleft G $ 且 $ [G:H]=m $. 证明: 对每个 $ x\in G $ 都有 $ x^m\in H $.

    解: $ H\triangleleft G $ 且 $ [G:H]=m $可以得到商群 $ G/H $, 且 $ |G/H|=m $. 因此商群的任意元素 $ gH $ 的阶均整除 $ m $, 即 $ \operatorname{ord} gH\mid m $, 故 $ (gH)^m=g^mH=H $, 故 $ g^m\in H $成立. 

\subsubsection*{Problem 5}
  设 $ H $是循环群 $ G $的子群. 证明: $ G/H $也是循环群.

  解: 设 $ G=\langle g\rangle $, 则 $ G $是交换群, 故 $ H\triangleleft G $, $ G/H $是一个群; 
  那么 $ \forall xH\in G/H $, 都有 $ x=g^k $, 其中 $ k\in\Z $, 则 $ xH=g^kH=(gH)^k $, 因此 $ gH $是群 $ G/H $的生成元, 
  即 $ G/H $也是循环群.

\subsubsection*{Problem 6}
  设 $ |G|=15 $. 证明: 如果 $ G $有唯一的 $ 3 $阶子群和唯一的 $ 5 $阶子群, 则 $ G $ 是循环群. 
  将此结果推广到 $ |G|=pq $的情况, 其中  $ p,q $为不同的素数.
  
  解: (找到阶为$pq$的元素即可证明循环群)
  假设$ N $是唯一的$ 3 $阶子群, $ H $是唯一的$ 5 $阶子群, 故 $N\cap H=\left\{ e \right\}$, 否则, 存在$x\in N\cap H$, s.t. $\ord(x)\mid |N|=3$和$\ord(x)\mid |H|=5$, 显然$\ord(x)=1$, 即$x=e$. 

  再证明$nh=hn$, 其中$n\in N$和$h\in H$: 由于$(hn^{-1}h^{-1})^3=e$且$N$是唯一的$3$阶子群, 故$hn^{-1}h^{-1}\in N$, 那么$nhn^{-1}h^{-1}\in N$. 同理$nhn^{-1}\in H$,  故$nhn^{-1}h^{-1}\in H$. 所以$nhn^{-1}h^{-1}\in N\cap H$, 即$nhn^{-1}h^{-1}=e$, $nh=hn$. 

  故$\ord(nh)=15$, 其中$n\in N$且$h\in H$均非单位元.

  \textcolor{red}{给出  $ |G|=pq  $的情况:} 
  假设 $ N $是唯一的$ p $阶子群, $ H $是唯一的$ q $阶子群, 则 $ N\cup H $中的元素的阶有三种: $ 1,p,q $且元素数量为 $ p+q-1<pq $. 
  因此存在 $ x\in G\setminus(N\cup H) $, 显然 $ \operatorname{ord}x\ne p,q,1 $, 根据拉格朗日定理,  $ G $的元素的阶有 
  $ pq,p,q,1 $这四种情况, 因此 $ \operatorname{ord}x $只能为 $ pq $, 即 $ G=\langle x\rangle $.

\subsubsection*{Problem 7* (选做)}
  设 $ G $为交换群, $ |G|=n $, $ m $是一个正整数. 证明: 如果 $ m\mid n $, 则 $ G $有 $ m $阶子群.

  解: $ n=2 $时结论成立; 
  假设结论对阶小于 $ n $的交换群成立, 则
  由柯西定理可知, 当 $ m $为素数时, $ G $有 $ m $阶子群, 结论成立;
  % 当 $ n=pq $其中 $ p,q $为不同的素数时同样成立;
  当 $ m $不是素数时, 假设 $ m=m'p $, 其中 $ p $是一个素数, 根据柯西定理可知存在 $ a\in G $ s.t. $ \operatorname{ord} a=p $, 
  则令 $ H=\langle a\rangle $为循环群, 则 $ G/H $为交换群且 $ |G/H|=n/p<n $, 
  那么根据归纳法可知商群 $ G/H $有阶为 $ m/p $的子群, 设为 $ N/H $, 有 $ N=\{n\in G\mid nH\in N/H\} $为所求, 其中 $ |N|=m/p\cdot p=m $.

  \begin{theorem}[Cauchy theorem]
    假设$G$是一个有限群, $p$是一个素数. 如果$p\mid|G|$, 那么$G$有阶为$p$的元素.
  \end{theorem}
  \begin{proof}
    设集合
    \[X=\left\{ (x_1,x_2,\dots,x_p):x_1x_2\cdots x_p=e, x_i\in G, i=1,2,\dots,p \right\}.\] 
    显然$|X|=|G|^{p-1}$. 
    根据$X$的元素是否是循环移位得到的, 可以发现, 这$|G|^{p-1}$个元素可以划分成$2$种情况, 循环移位$1$次就是自身的$(x,x,\dots,x)$和
    循环移位$p$次才能回到自身的 (注, 没有循环移位$1<t<p$次回到自身的元素, 否则, 会出现$(x_1,x_2,\dots,x_{t-1},x_1,x_2,\dots,x_{t-1})$的情况, 此时元素坐标有$2(t-1)\ne p$): 比如$(x_1,x_2,\dots,x_p)\rightarrow(x_2,x_3,\dots,x_p,x_1)\rightarrow(x_3,x_4,\dots,x_p,x_1,x_2)\rightarrow\cdots\rightarrow(x_p,x_1,x_2,\dots,x_{p-1})$.
    符合第一种情况的元素必然有$(e,e,\dots,e)$.  
    此外, 符合后一种情况的元素数量必然是$p$的倍数, 故不需要循环移位的元素数量是$|G|^{p-1}-mp$, 同样是$p$的倍数, 
    即符合$x^p=e$的元素数量大于$1$. 
    又因为$e^p=e$成立, 故存在非单位元$x_0\in G$, s.t. $x_0^p=e$. 
  \end{proof}
\end{document}