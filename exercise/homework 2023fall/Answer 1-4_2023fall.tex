\documentclass[a4paper,12pt]{ctexart}
\usepackage{fullpage,enumitem,amsmath,amssymb,graphicx}
\newcommand{\Z}{\mathbf{Z}}
\newcommand{\F}{\mathbf{F}}
\newcommand{\Com}{\mathbf{C}}
\newcommand{\ord}{\operatorname{ord}}
\newcommand{\Q}{\mathbf{Q}}
\newcommand{\R}{\mathbf{R}}


\title{NIS2312-01 Fall 2023 Homework~4}
\author{唐灯}



\begin{document}
%   \maketitle
  \begin{center}

  \vspace{-0.3in}
  \begin{tabular}{c}
    \textbf{\Large NIS2312-01 Fall 2023-2024} \\
    \textbf{\Large  } \\
    \textbf{\Large  信息安全的数学基础(1)} \\
    \textbf{\Large  } \\
    \textbf{\Large  Answer~1-4} \\
    \textbf{\Large  } \\
    \textbf{\Large 2023年9月27日} \\
  \end{tabular}
  \end{center}
  \noindent
  \rule{\linewidth}{0.4pt}

  \section*{Assignment 1}
  \subsubsection*{Problem 1}
  设$a,b$是任意两个正整数, 则: 
  \begin{enumerate}[label=(\arabic{*})]
      \item $a,b$的所有公倍数就是$\left[ a,b \right]$的所有倍数; 
      \item $\left[ a,b \right]=\frac{ab}{(a,b)}$.
  \end{enumerate}
  解: 
  \begin{enumerate}[label=(\arabic{*})]
      \item 设正整数$c=\left[ a,b\right]$, 正整数$d$为$a,b$的任意公倍数, 即$a\mid d,b\mid d$. 故对于两个正整数$c,d$, 由定理3(带余除法)可知一定存在整数$q,r$使得$d=qc+r$, 其中$0\le r\le c-1$. 由于$a\mid c$, $a\mid d$和$r=d-qc$, 故有$a\mid r$, 同理有$b\mid r$, 故$r$是$a,b$的公倍数. 又因为$r\le c-1$, 故$r=0$, 即$d=qc$, 证毕.
      \item  设正整数$d=\gcd(a,b)$和正整数$l=\left[ a,b \right]$. 由$d\mid a$且$d\mid b$可假设 $a=da_0,b=db_0$, 其中$a_0$和$b_0$互素, 故$\frac{ab}{d}=a_0b=b_0a$是$a,b$的公倍数, 则由本题(1)结论可知$l\mid \frac{ab}{d}$. 假设$m$是$a,b$的非零公倍数, 故有$m=ka=ka_0d$和 $b=b_0d\mid m$, 因此有$b_0\mid ka_0$, 但$a_0$和$b_0$互素, 故有推论11(2)得到$b_0\mid k$, 因此有$\frac{ab}{\gcd(a,b)}=a_0b_0d\mid m$. 因为$m$是$a,b$的公倍数而$l$是$a,b$的最小公倍数, 故$\frac{ab}{\gcd(a,b)}=l=\left[ a,b \right]$.
  \end{enumerate}
\subsubsection*{Problem 2} 
  \begin{enumerate}[label=(\arabic{*})]
      \item 利用辗转相除法计算$13$和$31$的最大公因数;
      \item 设$a,b$是任意两个互素的正整数, 证明$\gcd(a,a+b^2)=1$且$\gcd(ab,a^2+b^2)=1$.
  \end{enumerate}
  解: 
       \begin{enumerate}[label=(\arabic{*})]
          \item 我们有 \begin{align*}
              31&=2\times 31&+5\\
              13&=2\times 5 &+3\\
               5&=1\times 3 &+2\\
               3&=1\times 2 &+1\\
               2&=2\times 1&
          \end{align*}
          因此$13$和$31$的最大公因数为$1$.
          \item 由定理10可知, 根据$\gcd(a,b)=1$, 有$\gcd(a,b^2)=1$, 由定理12的注12.1可知, $\gcd(a,a+b^2)=\gcd(a,b^2)=1$; 同样根据定理12的注12.1可知$\gcd(a,a^2+b^2)=\gcd(a,b^2)=1$, 同理$\gcd(b,a^2+b^2)=1$, 故由推论11的(1)得到结论$\gcd(ab,a^2+b^2)=1$.
      \end{enumerate}  


\section*{Assignment 2}
  \subsubsection*{Problem 1}
  若 $a\equiv b\pmod{m_i}$, $i =1,2,\dots,k$, 证明
  \[a\equiv b\pmod{\left[ m_1,m_2,\dots,m_k \right]}.\]
  证明: 因为$a\equiv b\pmod{m_i}$, 故$m_i\mid a-b$, 其中$i=1,2,\dots,k$, 则$a-b$是$m_1,m_2,\dots,m_k$的公倍数. 又因为$\left[ m_1,m_2,\dots,m_k \right]$整除$m_1,m_2,\dots,m_k$的公倍数, 即$\left[ m_1,m_2,\dots,m_k \right]\mid a-b$, 故$a\equiv b\pmod{\left[ m_1,m_2,\dots,m_k \right]}$.
\subsubsection*{Problem 2} 
  设$m,n$是两个互素的正整数. $x_1, x_2,\dots , x_{\phi(m)}$是模$m$的一个简化剩余系, $y_1, y_2, \dots , y_{\phi(n)}$是模$n$的一个化剩余系, 证明$my_1 + nx_1, my_1 + nx_2, \dots , my_1 + nx_{\phi(m)}, \dots , my_{\phi(n)} + nx_1, my_{\phi(n)} + nx_2, \dots , my_{\phi(n)} + nx_{\phi(m)}$ 是模$mn$的一个简化剩余系.
  
  证明: 首先证明这 $\varphi(m) \varphi(n)$ 个数都与 $m n$ 互素. 只需证明对任 意 $1 \leq i \leq \varphi(n), 1 \leq j \leq \varphi(m)$ 都有 $\left(m y_i+n x_j, m n\right)=1$ 即可. 利用反证法. 假设存在 $1 \leq i^{\prime} \leq \varphi(n), 1 \leq j^{\prime} \leq \varphi(m)$ 使得 $m y_{i^{\prime}}+n x_{j^{\prime}}$ 与 $m n$ 不互素, 则必定存在素数 $p$ 使得 $p \mid m y_{i^{\prime}}+n x_{j^{\prime}}$ 且 $p \mid m n$. 现证明 $p \mid m n$ 不成立, 从而可推出假 设不成立. 若 $p \mid m n$ 成立, 则由推论 11 第 (3) 条知必有 $p \mid m$ 或 $p \mid n$. 

  情形 1. 若 $p \mid m$, 则 $p \mid n x_{j^{\prime}}$. 因 $p \nmid x_{j^{\prime}}$ (否则 $p$ 为 $m, x_{j^{\prime}}$ 的因 子, 这与 $\left(m, x_{j^{\prime}}\right)=1$ 矛盾), 故 $p \mid n$, 这与 $(m, n)=1$ 矛盾;

  情形 2. 若 $p \mid n$, 则 $p \mid m y_{i^{\prime}}$. 因 $p \nmid y_{i^{\prime}}$, 故 $p \mid m$, 这与 $(m, n)=1$ 矛盾.

  因此, $p \nmid m$ 且 $p \nmid n$, 从而 $p \mid m n$ 不成立, 从而假设不成立. 于是, 对任意 $1 \leq i \leq \varphi(n), 1 \leq j \leq \varphi(m)$ 都有 $\left(m y_i+n x_j, m n\right)=1$. 

  其次证明对于模$mn$的一个简化剩余系, 其元素均有$my_i+nx_j$的形式, 其中$1\le i\le \varphi(n),1\le j\le\varphi(m)$.
  进一步, 由注 22.1 第(1)条()以及定理28(这$\varphi(mn)$个数对模$mn$两两不同余)可知, 若能证明对任意 $x, y \in \mathbb{Z}$ 使得 $(m y+n x, m n)=1$ 时必有 $(x, m)=(y, n)=1$ 则 定理成立. 由 $(m y+n x, m n)=1$ 可得$(m y+n x, m)=(m y+n x, n)=1$ (否则若$(m y+n x, m)=d>1$则显然 $(m y+n x, m n) \geq d)$. 于是由定理 3 或注 12.1 可得 $(n x, m)=(m y, n)=1$.


\section*{Assignment 3}

    RSA和Rabin密码算法的结果分别参考课件slice的第50页和第51页.

\subsubsection*{引理54}
    映射$f:A\rightarrow B$ 是一一映射的充分必要条件是$f$是可逆映射.

    证明:  
    
    必要性: 假设$f$是可逆映射, 则$f$存在逆映射$f^{-1}:B\rightarrow A$. 对于$\forall b\in B$, 我们都有$f^{-1}(b)=a\in A$, 则$f(a)=f(f^{-1}(b))=b$, 因此$f$是满射; 假设$a_1,a_2\in A$且满足$f(a_1)=f(a_2)$, 则有$a_1=f^{-1}(f(a_1))=f^{-1}(f(a_2))=a_2$, 故$f$是单射. 因此$f$是一一映射.

    充分性: 若$f$是一一映射, 故根据$f$的满射性质有, 对于任意给定的$b\in B$我们都能找到$a\in A$使得 $f(a)=b$, 同时因为$f$是单射, 则$a$是唯一确定的; 因此定义映射$g:B\rightarrow A$, 其中$g(b)=a$, 则$f(g(b))=f(a)=b$和$g(f(a))=g(b)=a$, 因此$f$是可逆映射.

\subsubsection*{引理55}
    设$f:A\rightarrow B$和$g:B\rightarrow C$都是一一映射, 则$g\circ f:A\rightarrow C$也是一一映射, 并且$(g\circ f)^{-1} = f^{-1}\circ g^{-1}$.

    证明: 因为$f,g$都是一一映射, 根据引理54可知$f,g$都是可逆映射, 分别记为$f^{-1},g^{-1}$. 因此直接检验
    \begin{align*}
        (g\circ f)\circ(f^{-1}\circ g^{-1}) &=g\circ f\circ f^{-1}\circ g^{-1}\\
                                            &=g\circ (f\circ f^{-1})\circ g^{-1}\\
                                            &=g\circ \mathsf{id}_B\circ g^{-1}\\
                                            &=g\circ g^{-1}\\
                                            &=\mathsf{id}_C 
    \end{align*}
    同理可以得到 $(f^{-1}\circ g^{-1})\circ(g\circ f)=\mathsf{id}_A$, 其中$\mathsf{id}_A,\mathsf{id}_B$和$\mathsf{id}_C$分别是集合$A,B$和$C$的恒等映射.
    故$g\circ f:A\rightarrow C$是可逆映射, 因此是一一映射. 


    \section*{Assignment 4}
\subsubsection*{Problem}
    验证下列集合$G$与给定的运算能否构成群, 即判断是否满足群定义中的四个条件.
    \begin{enumerate}[label=(\arabic{*})]
        \item $G=\mathbb{C}$为复数集, 运算为复数域上的乘法.
        \item $G=\mathbb{Z}$为整数集, 运算为整数上的减法.
        \item $G=\mathbb{R}$为实数集, 运算为$\star$, 其定义如下: \[a\star b=(a+1)(b+1)-1,\forall a,b\in \mathbb{R}.\]
    \end{enumerate}

    证明: 
    \begin{enumerate}[label=(\arabic{*})]
        \item 对于代数结构$(\mathbb{C},*)$, 我们有:
        \begin{enumerate}
            \item 由于复数乘复数的结果仍然是复数, 故代数结构满足封闭性.
            \item 对于$\forall a,b,c\in\mathbb{C}$, 都有$a*(b*c)=(a*b)*c$, 故结合律成立.
            \item 对于$\forall a\in\mathbb{C}$, 我们有$a*1=1*a=a$, 则$1$是单位元.
            \item 但除了$0$以外的元素才有逆元: $a\in\mathbb{C}\setminus\left\{ 0 \right\}$的逆元$1/a$.
        \end{enumerate} 
        因此该代数结构不构成群.
        \item 对于代数结构$(\mathbb{Z},-)$, 我们有:
        \begin{enumerate}
            \item 由于整数相减的结果仍然是整数, 故代数结构满足封闭性.
            \item 对于$\forall a,b,c\in\mathbb{Z}$, 等式$a-(b-c)=a-b+c=(a-b)-c$成立当且仅当$c=0$, 故结合律不成立.
            \item 设 $\mathsf{id}\in\mathbb{Z}$为单位元, 对于$\forall a\in\mathbb{Z}$, 我们有$a-\mathsf{id}=\mathsf{id}-a=a$, 即$\mathsf{id}=0$且$\mathsf{id}=2a$, 故不存在单位元.
        \end{enumerate} 
        因此该代数结构不构成群.
        \item 对于代数结构$(\mathbb{R},\star)$, 我们有:
        \begin{enumerate}
            \item 由于实数之间的加减乘的结果仍然是实数, 故代数结构满足封闭性.
            \item 对于$\forall a,b,c\in\mathbb{R}$, 都有$a\star(b\star c)=a\star\left( (b+1)(c+1)-1 \right)=(a+1)\left( (b+1)(c+1)-1 +1\right)-1=(a+1)(b+1)(c+1)-1$, 同时也有$(a\star b)\star c=(a+1)(b+1)(c+1)-1$, 则$a\star(b\star c)=(a\star b)\star c$, 故结合律成立.
            \item 假设$\mathsf{id}\in\mathbb{R}$是单位元, 对于$\forall a\in\mathbb{R}$, 我们有$a\star \mathsf{id}=\mathsf{id}\star a=a$, 即$(a+1)\mathsf{id}=0$, 则$0$是单位元.
            \item $\forall a\in\mathbb{R}$, 其逆元为$a^{-1}$满足$a\star a^{-1}=0$, 即$(a+1)(a^{-1}+1)-1=0\Rightarrow (a+1)(a^{-1}+1)=1$, 但$a=-1$时上式不成立, 故$-1$没有逆元.
        \end{enumerate} 
        因此该代数结构不构成群.
    \end{enumerate}
\end{document}