\documentclass[a4paper,12pt]{ctexart}
\usepackage{fullpage,enumitem,amsmath,amssymb,graphicx}
\usepackage{tikz-cd}
\newcommand{\Z}{\mathbf{Z}}
\newcommand{\F}{\mathbf{F}}
\newcommand{\Com}{\mathbf{C}}
\newcommand{\ord}{\operatorname{ord}}
\newcommand{\Q}{\mathbf{Q}}
\newcommand{\R}{\mathbf{R}}


% \title{NIS2312-2 Spring 2022 Homework~1}
% \author{唐灯}



\begin{document}
%   \maketitle
  \begin{center}

  \vspace{-0.3in}
  \begin{tabular}{c}
    \textbf{\Large NIS2312-1 Spring 2021-2022} \\
    \textbf{\Large  } \\
    \textbf{\Large  信息安全的数学基础(1)} \\
    \textbf{\Large  } \\
    \textbf{\Large  Answer~6 (第一题我后面补)} \\
    \textbf{\Large  } \\
    \textbf{\Large 2022年4月7日} \\
  \end{tabular}
  \end{center}
  \noindent
  \rule{\linewidth}{0.4pt}

  在这个练习中, 带 * 的题目属于数论方面, 可以不做的 (考试不会单独考数论的题).  
\subsubsection*{Problem *}
  \begin{enumerate}
    \item AKS素数检测算法, 第一个被发表的多项式时间, 确定性, 无依赖性且对任意整数成立的素数检测算法.
    最基本的原理: 一个整数 $ n $和整数 $ a $使得 $ \gcd(a,n)=1 $, 
    
    那么$ n $是素数当且仅当 
    \[ (x+a)^n=x^n+\sum_{i=1}^{n-1}a_ix^i +a^n\equiv x^n+a \mod{n},\text{其中所有的}~a_i~\text{均被}~n~\text{整除}. \]
    
    请证明其充分性, 即 如果 $ n $是素数, 那么 $ n\mid \binom{n}{k} $, 其中 $ k=1,2,\dots,n-1 $, 且 $ a^n\equiv a\mod{n} $.

    思路: 两部分, 前者是证明: $ n\mid \binom{n}{k}=\frac{n!}{k!(n-k)!}=\frac{n\cdot (n-1)!}{k!(n-k)!} $, 注意到 $ n $是素数, 所以 $ k!(n-k)! $的因子不整除 $ n $, 所以 $ \binom{n}{k} $可以
    写为两个整数相乘的结果 $ n $和 $ \frac{(n-1)!}{k!(n-k)!} $, 所以 $ n\mid \binom{n}{k} $;

    后者直接使用费马小定理即可.
    \item 欧拉函数: 证明 (叙述一下证明思路即可), \[\sum_{d\mid n}\phi(d)=n\]
    
    思路: 将 $ \{1/n,2/n,...,n-1/n\} $这 $ n-1 $个分式化简, 得到的分式的分母均为 $ n $的因子, 
    且分母等于 $ d $的分式的数量就是 和 $ d $互素的整数的数量, 即 $ \phi(d) $, 因此求和结果为 $ n $.
  \end{enumerate}    
\subsubsection*{Problem 1}
    判断下列代数结构在给定运算下是否构成环
    \begin{enumerate}
      \item 假设 $ A,B $是环, 那么 $ A\times B $在 $ A,B $ 默认的运算下是否构成环?
      \item 假设 $ D $是一个有理数且不是完全平方数, 定义集合
      \[\Q(\sqrt{D})=\{a+b\sqrt{D}\mid a,b\in\Q\}.\]
      那么集合 $ \Q(\sqrt{D}) $在复数加法乘法的运算下是否构成环, 如果构成环是否构成 $ \Com $的子环?
      \item 集合 $ C(a)=\{r\in R\mid ar=ra\} $, 其中 $ a\in R $. 那么集合 $ C(a) $在环$ R $ 的运算下是否构成环, 如果构成环是否构成 $ R $的子环?
    \end{enumerate}

    解: 这题答案有点长, 等我过几天再写出答案来. (-.-;)...
\subsubsection*{Problem 2}
   直接写出下面哪个代数结构是 $ \Q $的子环
   \begin{enumerate}
     \item 最简形式的分式中分母是奇数的有理数集合, 是子环
     \item 最简形式的分式中分母是偶数的有理数集合, 不是子环, 反例 $ 1/2+1/2=1/1 $, 分母不是偶数
     \item 最简形式的分式中分子是奇数的有理数集合, 不是子环, 反例 $ 1/5+3/5=4/5 $, 分子不是奇数
     \item 最简形式的分式中分子是偶数的有理数集合, 是子环
   \end{enumerate}
\subsubsection*{Problem 3}
    证明在任意无零因子有单位元的有限交换环中, 非零元素均是单位 (实际就是证明有限整环是域).

    解: 先证明对任意非零 $ c\in R $, 有$ cR=R $: 显然 $ cR\subseteq R $, 假设 $ |cR|<|R| $, 则有
    $ cr_i=cr_j $, 其中 $ r_i,r_j\in R $, 即 $ r_i=r_j $, 因此 $ |cR|=|R| $, 故 $ cR=R $.
    又因为 $ R $中有乘法单位元, 则 $ \exists d\in R $, 有 $ cd=e $, 即 $ c $是单位, 所以非零元素均是单位
\subsubsection*{Problem 4}
    如果一个元素 $ x\in R $满足等式 $ x^n=0,n\in\Z^+ $, 那么称其幂零元 (简单的例子是线性代数中的幂零矩阵). 证明: 在交换环 $ R $中,
    有 $ x\in R $是幂零元, 那么:
    \begin{enumerate}
      \item $ x $ 不是零元素就是零因子;
      \item $ rx $仍然是幂零元, 其中 $ r\in R $;
      \item $ 1+x $是单位; 
    \end{enumerate}

    解:\begin{enumerate}
      \item 因为 $ x^n=x\dot x^{n-1}=0 $, 显然 $ x=0 $满足等式. 假设 $ x\neq 0 $, 则 $ x^{n-1}=0 $或者
      $ x $是零因子, 以此类推, 得到 $ x=0 $或者 $ x $是零因子.
      \item 显然 $ (rx)^n=r^nx^n=0 $
      \item 显然 $ (1+x)(1-x)(1+x^2)(1+x^4)\cdots(1+x^{2^m})=1-x^{2^{m+1}}=1 $, 其中 $ 2^{m+1}>n $, 
      且 $ (1-x)(1+x^2)(1+x^4)\cdots(1+x^{2^m})\in R $, 因此 $ 1+x $是单位.
      \item 也可 $ (1+x)(1-x+x^2-x^3+\cdots+(-1)^{n-1}x^{n-1})=1+(-1)^nx^n=1 $
    \end{enumerate}
\subsubsection*{Problem 5}
    如果环 $ R $的元素满足 $ r^2=r $, 其中 $ r\in R $, 那么 $ R $是一个交换环 (hint: $ r=x+y $ where $ x,y,r\in R $ or $ 2xy=0 $ for all $ x,y\in R $).

    解: 令 $ r=x+y $其中 $ x,y,r\in R $ , 那么 $ r^2=(x+y)^2=x^2+xy+yx+y^2=x+xy+yx+y $, 即 $ xy+yx=0 $.
    同时将 $ y=x $带入上式, 得到 $ 2x^2=2x=x+x=0 $, 因为 $ x $的任意性, 可以得到 $ xy+xy=0 $.
    综上, $ xy=yx $, 那么 $ R $是一个交换环.

    注意:\begin{enumerate}
      \item 在群里说了, 环中乘法不一定有交换性, 此外 $ 2x=0\nRightarrow x=0 $. 
      \item 还有我看到部分同学直接带入 $ -1 $, 这个也是不严谨的, 因为环不一定有乘法单位元. 
      
      \item 还有, 我当时批作业批错了: 我直接说, 环中除了零, 单位就是零因子. 这个是错误的.
      
      直接给个结论: 有限环中只有零, 单位和零因子.
      
      无限环直接给出一个简单的反例: $ \Z $. $ 2\in\Z $, 显然 $ 2 $不是零元, 也不是单位, 更不是零因子.
    \end{enumerate} 
% \subsubsection*{*}
%     如果有群 $ G $和正规子群 $ H_1,H_2,\dots,H_n $且 $ G=H_1H_2\cdots H_n $, 此外如果 $ i\neq j $, 则$ H_i\cap H_j=\{e\} $. 证明或给出反例
%      $ G\cong H_1\times H_2\times \cdots \times H_n $.

    
 
\end{document}