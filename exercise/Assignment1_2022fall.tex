\documentclass[a4paper,12pt]{ctexart}
\usepackage{fullpage,enumitem,amsmath,amssymb,graphicx}
\newcommand{\Z}{\mathbf{Z}}
\newcommand{\F}{\mathbf{F}}
\newcommand{\Com}{\mathbf{C}}
\newcommand{\ord}{\operatorname{ord}}
\newcommand{\Q}{\mathbf{Q}}
\newcommand{\R}{\mathbf{R}}


\title{NIS2312-1 2022-2023 Fall Homework~1}
\author{唐灯}



\begin{document}
%   \maketitle
  \begin{center}

  \vspace{-0.3in}
  \begin{tabular}{c}
    \textbf{\Large NIS2312-1 2022-2023 Fall} \\
    \textbf{\Large  } \\
    \textbf{\Large  信息安全的数学基础(1)} \\
    \textbf{\Large  } \\
    \textbf{\Large  Assignment~1} \\
    \textbf{\Large  } \\
    \textbf{\Large 2022年9月14日} \\
  \end{tabular}
  \end{center}
  \noindent
  \rule{\linewidth}{0.4pt}
  
\subsubsection*{Problem 1}
    证明: 设 $ a,b $ 是两个不全为零的整数, 令 $ S=\{xa+yb>0\mid x,y\in\mathbb{Z}\} $, 则 $ (a,b)=\min S $. 

    解:设 $ d=\min S $, 则 $ d\mid a,d\mid b $:  
    假设 $ d\nmid a $, 则 $ a=qd+r $, 其中 $ q\ge 0,0<r<d $. 因此有
    $ qd=q(ax+by)=a-r $, 即 $ r=(1-qx)a-qyb $. 注意到 $ 1-qx,-qy $均为整数, 因此确定 $ r\in S $, 但 $ r<d $, 
    与 $ d=\min S $矛盾, 所以 $ d\mid a $. 同样的方法可以证明 $ d\mid b $. 因此 $ d=\min S $是 $ a,b $的公因子.  
    
    下证 $ a,b $的任意公因子均整除 $ d $: 假设 $ a=cu,b=cv $, 那么 $ d=ax+by=c(au+bv) $, 即 $ c\mid d $. 
    Q.E.D.
\subsubsection*{Problem 2} 
    证明: 若 $ a,b,c $ 是三个整数, 则: 
    \begin{enumerate}[label=(\arabic*)]
        \item 若 $ (a,c)=1 $且 $ (b,c)=1 $, 则 $ (ab,c)=1 $;
        \item 若 $ (a,c)=1 $且 $ c\mid ab $, 则 $ c\mid b $;
        \item 若 $ c $为素数且 $ c\mid ab $, 则 $ c\mid a $或者 $ c\mid b $.
    \end{enumerate}
    解: \begin{enumerate}[label=(\arabic*)]
        \item  $ (a,c)=1\Rightarrow\exists m,n\in\mathbb{Z} $, s.t. $ ma+nc=1 $, 同理得到 $ m'b+n'c=1 $.
        因此 $ (ma+nc)(m'b+n'c)=1 $, 整理得到 $ mm'ab+(man'+m'bn+ncn')c=1 $, 再根据第一题, 可确定 $ (ab,c)=1 $.
        \item 因为 $ c\mid ab $, 故 $ ab=xc $, 那么由 $ (a,c)=1 $可知 $ \exists m,n\in\mathbb{Z} $ s.t. 
        $ ma+nc=1\Rightarrow mab+ncb=b\Rightarrow mxc+ncb=b $, 即 $ c(mx+nb)=b $, Q.E.D.
        \item 反证: 假设$ c $为素数且 $ c\mid ab $, 则$ c\nmid a $且 $ c\nmid b $, 因此 $ (a,c)=1,(b,c)=1 $, 同时根据第二题第一小题, 可知 $ (ab,c)=1 $, 与 $ c\mid ab $矛盾. Q.E.D.
    \end{enumerate}
\subsubsection*{Problem 3}
    设 $ a,b $是任意两个正整数, 证明: 
    \begin{enumerate}[label=(\arabic*)]
        \item $ a,b $的所有公倍数就是 $ [a,b] $的所有倍数;
        \item $ [a,b]=\frac{ab}{(a,b)} $.
    \end{enumerate}
    解:\begin{enumerate}[label=(\arabic*)]
        \item 设 $ m $为 $ a,b $的公倍数, 则 $ a\mid m,b\mid m $. 反证: 假设 $ m $ 不是 $ [a,b] $的公倍数, 则有
        $ m=q[a,b]+r $, 其中 $ q\ge 0,0<r<[a,b] $, 注意到 $ a\mid m,a\mid [a,b] $, 则 $ a\mid r $, 
        同理 $ b\mid r $, 故 $ r $也是 $ a,b $的公倍数, 与 $ [a,b] $是最小公倍数矛盾. Q.E.D.
        \item 设 $ d=(a,b) $, 则 $ a=md,b=nd $且 $ (n,m)=1 $. 显然 $ mnd $是 $ a,b $的公倍数, 
        下面证明 $ mnd=[a,b] $: 设 $ c $是 $ a,b $的公倍数, 则有 $ c=ka=kmd $, 同时 $ b=nd\mid c=kmd $, 得到
        $ n\mid km $, 利用第二题第二小问可确定 $ n\mid k $, 因此 $ mnd\mid c $, 故 $ [a,b](a,b)=ab $. Q.E.D.
    \end{enumerate}
\end{document}