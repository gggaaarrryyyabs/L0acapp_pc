\documentclass[a4paper,12pt]{ctexart}
\usepackage{fullpage,enumitem,amsmath,amssymb,graphicx}
\newcommand{\Z}{\mathbf{Z}}
\newcommand{\F}{\mathbf{F}}
\newcommand{\Com}{\mathbf{C}}
\newcommand{\ord}{\operatorname{ord}}
\newcommand{\Q}{\mathbf{Q}}
\newcommand{\R}{\mathbf{R}}


% \title{NIS2312-2 Spring 2022 Homework~1}
% \author{唐灯}



\begin{document}
%   \maketitle
  \begin{center}

  \vspace{-0.3in}
  \begin{tabular}{c}
    \textbf{\Large NIS2312-1 Spring 2021-2022} \\
    \textbf{\Large  } \\
    \textbf{\Large  信息安全的数学基础(1)} \\
    \textbf{\Large  } \\
    \textbf{\Large  Answer 3 } \\
    \textbf{\Large  } \\
    \textbf{\Large 2022年3月17日} \\
  \end{tabular}
  \end{center}
  \noindent
  \rule{\linewidth}{0.4pt}

      % \begin{enumerate}
        %   \item 一般而言, 大写字母代表集合(比如$S$一般用于代表集合(set), $G$一般用于表示群(group), $ R $一般用于表示环(ring)), 小写字母代表集合中的元素(比如 $ p $一般用于表示素数(prime), $ a,g $一般用于表示群的元素等等)
        %   \item 群中的运算如果不作特别说明不一定满足交换律(commutative), 运算满足交换律的群叫交换群或者阿贝尔群
        %   \item 今后, 如果不作特别说明, 总假定群的运算是``乘法''. cf P8 例 4 上面那一段
        %   \item 群中的乘法不一定是我们熟知的乘法, 比如 $ SL_n(\R) $的乘法就是矩阵乘法, 四面体群(dihedral group)的乘法是函数的复合, 甚至可以把整数加法群中的整数加法运算叫做``乘法''
        %   \item 群的运算如果是``加法''的话, 加法单位元(或者叫零元)常用 $0$表示, 元素 $a$的加法逆元记作 $ -a $
        %   \item 群的运算如果是``乘法''的话, 乘法单位元常用 $e$表示, 元素 $a$的乘法逆元记作 $ a^{-1} $
        %   \item $ \left\langle a\right\rangle=\{a^r\mid r\in\Z\}  $由元素 $ a $生成的循环群
        % %   \item 如果 $ S=\{a_1,a_2,\dots,a_r\} $, 则记 $ \langle S\rangle=\langle a_1,a_2,\dots,a_r\rangle=\{a_1^{l_1}a_2^{l_2}\cdots a_k^{l_k}\mid a_i\in S,\} $
        %   \item 对于群 $ G $中的集合 $ S,T $和元素 $ x $, $ Sx=\{ax\mid a\in S\} $, $ xS=\{xa\mid a\in S\} $(陪集,cf P66), $ xSx^{-1}=\{xax^{-1}\mid a\in S\} $, $ ST=\{ab\mid a\in S,b\in T\} $
        %   \item lcm 最小公倍数, mod 求模运算, 
      %     \item 群的外直积(cf P90)可以类比笛卡尔坐标系: $ \Z_n\times\Z_m\times\Z_k=\{(a,b,c)\mid a\in\Z_n,b\in\Z_m,c\in\Z_k\} $, 运算是按位运算(bitwise). 当 $ m=n=k $时, 
      %     将$ \Z_n\times\Z_n\times\Z_n $简写为 $ \Z_n^3 $
      %     \item 为了简便起见,从现在开始, 在不致误解的情况下, 我们将把 $ \Z_n $中的元素 $ \overline{a} $简记为 $ a $. 
      %     在运算过程中, 读者必须首先分清, $ a $所表示的究竟是数 $ a $还是 剩余类 $ \overline{a} $.
      %     \item 函数的复合(composition)是从右向左的: $ f\circ g(x)=f(g(x)) $
      % \end{enumerate}

      % The product of commutators maybe are not a commutator:

      

\subsubsection*{Problem 1}


        \[\Z_3[x]=\left\{ a_0+a_1x+a_2x^2+\cdots \mid a_i\in\Z_3,i=0,1,2,\dots \right\}\]

        所以 
        \[\Z_3[x]/\langle x^2+1\rangle = \left\{ a_0+a_1x+\langle x^2+1\rangle\mid a_i\in\Z_3,i=0,1 \right\}\]
        这里 $ a_i $中的 $ i $只能取 $ 0,1 $是因为 理想 $ \langle x^2+1\rangle $ 会让二次及以上的单项式降次 (处于同一个陪集中),比如
        \[x^2+x+1+\langle x^2+1\rangle=x+(x^2+1)+\langle x^2+1\rangle=x+\langle x^2+1\rangle\]


      \begin{enumerate}
        \item[1.2.] 不是 (写这两道小题是因为有人做错了...)
        \item[3]  该映射不是 $ \Z_8 $到 $ \Z_2\times\Z_2\times\Z_2 $的同构映射, 
        因为不保持运算, 举例: $ dec2bin(7+1)=dec2bin(0)=(0,0,0)\neq dec2bin(7)+dec2bin(1)=(1,1,1)+(0,0,1)=(1,1,0) $.
        \item[4] 对称群作用的集合元素数量相等的话, 对称群是同构的.
        \item[7] 注意到 $ a=1 $为单位元的时候, 是一个恒等映射显然是同构映射.    
      \end{enumerate}
\subsubsection*{Problem 2}
      证明群之间不同构, 可以有几种: 一个是循环群一个不是; 两个群的阶不相等; 两个群中元素阶相等的元素数量不同; 
    \begin{enumerate}
      \item 乘法群 $ \R^* $和乘法群 $ \Com^* $不同构: 假设同构, 那么存在一个同构映射
      $ f:\Com^*\rightarrow\R^* $使得 $ f(i)^4=f(i^4)=f(1)=1 $, 那么 $ \ord(f(i))=4 $, 但是在 $ \R^* $中没有四阶的元素.
      \item 加法群 $ \Z $和 $ \Q $不同构: 假设同构, 
      那么有同构映射 $ f:\Q\rightarrow\Z $使得 $ f(1)\in\Z $, 所以对任意 $ n $ 有 $ f(1/n)=f(1)/n\in\Z $, 显然 $ f(1)=0 $, 矛盾, 因为将非单位元映射到单位元了.

      也可以这样证明: $ \exists n\in\Q $使得 $ f(n)=1 $, 所以 $ f(n/2)=1/2\notin\Z $矛盾.

      或者证明 $ \Z $是循环群 (显然的), 但$ \Q $不是循环群: 如果 $ \Q $是循环群, $ a $是生成元, 那么有整数 $ n $使得 $ na=a/2\rightarrow n=1/2 $不是整数, 矛盾.
      \item 这里证明只需写出 $ \Z_2\times\Z_2 $不是循环群即可, 比如所有的非单位元元素都 $ 2 $阶的.
      \item 构造映射: $ f:G\times H\rightarrow H\times G $使得 $ f((g,h))=(h,g) $, 
      其中 $ g\in G,h\in H $.
      \begin{enumerate}
        \item[(1)]是一个映射, 因为 $ f((g,h))=(h,g)\in H\times G $
        \item[(2)]单射: 若 $ (g_1,h_1)\neq (g_2,h_2) $, 显然 $ f((g_1,h_1))=(h_1,g_1)\neq f((g_2,h_2))=(h_2,g_2) $
        \item[(3)]满射: $ \forall (h,g)\in H\times G $, 都有 $ (g,h)\in G\times H $ 使得 $ f((g,h))=(h,g) $.
        \item[(4)]保持运算: $ \forall (g_1,h_1),(g_2,h_2)\in G\times H $, 都有 $ f((g_1,h_1)(g_2,h_2))=f((g_1g_2,h_1h_2))=(h_1h_2,g_1g_2)=(h_1,g_1)(h_2,g_2)=f((g_1,h_1))f((g_2,h_2)) $ 
      \end{enumerate}
      \item[5.] 证明 $ \Z_n\times\Z_m $是循环群: 部分同学仅仅写出 $ mn(1,1)=(0,0) $就说这是生成元, 这是不对的!
      生成元的定义是按照元素的阶, 元素的阶为 $ n $ 还要证明小于 $ n $的整数无法使得元素的幂为单位元.
      
      显然 $ mn(1,1)=(0,0) $, 所以如果 $ \ord((1,1))=k $, 那么 $ k\mid mn $; 同时假设 $ k=mn/d $, 其中 $ d\in\Z $是 $ mn $的因子, 因此 $ m\mid mn/d $和 $ n\mid mn/d $, 所以 $ n/d,m/d\in\Z $, 显然只有 $ d=1 $满足条件. 因此 $ k=mn $. 故 $ \ord((1,1))=mn $, 又因为 $ |\Z_n\times\Z_m|=mn $, 就能得到结论 群是个循环群. 
    \end{enumerate}

\subsubsection*{Problem 3}
群中的每一个元素的阶均不为 $ 0 $, 且单位元是其中惟一的阶为 $1$ 的元素。
因为任一阶大于 $2$ 的元素和它的逆元的阶相等。且当一个元素的阶大于 $2$ 时,
其逆元和它本身不相等。故阶大于 $2$ 的元素是成对的。
从而阶为$1$的元素与阶大于$2$的元素个数之和是奇数。
因为该群的阶是偶数,从而它一定有阶为 $2$ 的元素。
\subsubsection*{Problem 4}
    凯莱定理的推广: 仅证明 $ GL_n(\F_2) $同构于置换矩阵即可. 置换矩阵是 $ GL_n(F) $的子群, 其中 $ F $是任意域 (题目中我给出的是 $ \F_2 $, 实际上通常用的是 $ \R $, 域的概念暂时不需要掌握). 证明是显然的...

    考虑映射 $ \phi:S_n\rightarrow P_{n\times n} $, 后者是 $ n $阶置换矩阵, 
    映射的方式为: 给定 $ \tau\in S_n $, $ \phi(\tau) $的第 $ i $列第 $ \tau(i) $行为 $ 1 $, 此外 $ 1 $所在的行和列其他位置全部为 $ 0 $.
    \begin{enumerate}
      \item 显然这个是一个映射: $ (\phi(i),i) $处为 $ 1 $, 同时矩阵各行各列都只有一个 $ 1 $, 因此 $ \phi(\tau) $是一个置换矩阵.
      \item 映射是一个单映射: 显然的.
      \item 满映射同样显然的.
      \item 保持运算: 对于任意 $ \tau,\sigma\in S_n $, 
      有 $ \phi(\tau\circ\sigma)_{ij}=1 $当且仅当 $ i=\tau\circ\sigma(j) $, 其他情况是 $ 0 $. 对于置换矩阵的乘法运算, $ (\phi(\tau)\phi(\sigma))_{ij}=\sum_{k=1}^n\phi(\tau)_{ik}\phi(\sigma)_{kj} $. 而 $ \phi(\tau)_{ik}=1 $当且仅当 $ i=\tau(k) $, $ \phi(\sigma)_{kj}=1 $当且仅当 $ k=\sigma(j) $. 所以 $ \phi(\tau)_{ik}\phi(\sigma)_{kj}=1$当且仅当 $ i=\tau(k),k=\sigma(j) $, 也就是 $ i=\tau\circ\sigma(j) $. 并且$ (\phi(\tau)\phi(\sigma))_{ij}\leq 1 $因为不存在多个 $ k $使得 $ \tau(k)=i $.
    \end{enumerate}
    因此映射 $ \phi $是一个同构映射.    
\subsubsection*{Problem 5}
    参考 欧拉公式, 书上11页

    $ \Z_233 $的生成元数量是 $ \phi(233)=232 $
    $ \Z_4900 $的生成元数量是 $ \phi(4900)=1680 $
    
    推论1: 因为 $ (n,r)=d $, 所以 $ d\mid r $, 故 $ \langle a^r\rangle\subseteq\langle a^d\rangle $. 同时因为 $ (n,r)=d $, 则存在 $ u,v\in\Z $ 使得 $ d=un+vr $. 于是 $ a^d=a^{un+vr}=a^{vr}\in\langle a^r\rangle $. 证毕.

    推论2: (1)  如果 $|G|=\infty$, 因为对任意的 $r_{1}>r_{2}>0$, 有 $r_{1} \nmid r_{2}$, 所以 $a^{r_{2}} \notin\left\langle a^{r_{1}}\right\rangle$, 于是 $\left\langle a^{r_{1}}\right\rangle \neq\left\langle a^{r_{2}}\right\rangle .$
    另一方面, 对任意的 $r>0$, 显然 $a^{r} \notin\left\langle a^{0}\right\rangle=\langle e\rangle$, 所以又有
    \[\left\langle a^{r}\right\rangle \neq\langle e\rangle .\]
    由此得 $G$ 的全部子群为
    \[\left\{\left\langle a^{d}\right\rangle \mid d=0,1,2, \cdots\right\} .\]

    (2) 如果 $ |G|=n $, 从上题知道对 $ d=(n,r) $, 有$ \left\langle a^{r}\right\rangle=\left\langle a^{d}\right\rangle $
    
    又如果 $d_{1}>d_{2}$ 为 $n$ 的两个不同的正因子, 则 $d_{1} \nmid d_{2}$, 于是 $a^{d_{2}} \notin\left\langle a^{d_{1}}\right\rangle$, 从而
    $$
    \left\langle a^{d_{1}}\right\rangle \neq\left\langle a^{d_{2}}\right\rangle .
    $$
    另一方面, 对 $n$ 的任一正因子 $d<n$, 显然 $a^{d} \neq e$, 所以又有
    $$
    \left\langle a^{d}\right\rangle \neq\langle e\rangle .
    $$
    而
    $$
    \langle e\rangle=\left\langle a^{n}\right\rangle
    $$
    由此得 $G$ 的全部子群为
    $$
    \left\{\left\langle a^{d}\right\rangle \mid d \text { 为 } n \text { 的正因子 }\right\} .
    $$
\subsubsection*{Problem 6}
    假设 $ \sigma=(1 2 3\cdots m) $, 于是 $ \sigma^i(k)=k+i $, 且 $ \sigma^i(k+i)=k+2i $, 因此 $ \sigma^i:k\mapsto k+i\mapsto k+2i\mapsto\cdots k+(m-1)i\mapsto k $.
    显然此时 $ \sigma^i $是一个 $ m $-轮换, 当且仅当上式元素均不相同, i.e. $ k+xi\neq k+yi\mod{m} $, 其中 $ x,y\in\{0,1,\dots,m-1\} $, 也就是 $ (x-y)i\neq 0\mod{m} $, 即 $ x=y $. 

    其实利用probblem 5的推论1就好
\subsubsection*{Problem 7}
    使用书上的公式 $ (F1) $, 书上46页, 
    结果分别是 $ (1 ~3),(1~4~2~7),(1~2)(4~7) $
\subsubsection*{Problem 8}
    (F2)(F3)的证明完全可以硬算, $ S_n $子群更是简单 (偶置换和偶置换复合肯定还是偶置换, 逆也是偶置换) ...略过
    
    % 或者 (F2): $ (kl)(ka\cdots b)(lc\cdots d)=(kl)(ka)(lc)(a\cdots b)(c\cdots d)=(kal)(lc)(a\cdots b)(c\cdots d)=(lcka)(a\cdots b)(c\cdots d)=(ka\cdots blc\cdots d) $,

\end{document}