\documentclass[a4paper,12pt]{article}
\usepackage{fullpage,enumitem,amsmath,amssymb,graphicx}
\newcommand{\F}{\mathbf{F}}
\newtheorem{theorem}{Theorem}
\newtheorem{corollary}{Corollary}


\begin{document}
    We can define $ \gamma_F(a,b) $ as below:
    $ \forall a,b\in\F_2^n,\gamma_F(a,b)= $
    \[\left\{\begin{aligned}        
        &1 ~\text{if} ~a\neq 0_n~\text{and}~ F(x)+F(x+a)=b ~\text{has~solutions}\\
        &0 ~\text{otherwise}
    \end{aligned}\right.\]

    Thus, for every APN $ (n,n) $-function $ F $, we view it as a Boolean function
    $ \frac{|(D_aF)^{-1}(b)|}{2}-2^{n-1}\delta_0(a,b) $, then we have 
    \[\widehat{\gamma_F(u,v)}=\frac{1}{2}W_F^2(u,v)-2^{n-1}.\]
    So we confirm that for every $ u,v $:
    \[W_{\gamma_F}(u,v)=\left\{\begin{aligned}
        &2^n ~\text{if}~ v=0_n\\
        &2^n-W_F^2(u,v) ~\text{if}~ v\neq 0_n.
    \end{aligned}\right.\]
    
    The fourth moment of the Walsh transform of an APN function $ F $:
    \[\sum_{u,v\in\F_2^n}^{}W_F^4(u,v)=3\cdot 2^{4n}-2^{3n+1}.\]
    When apply the Titsworth relation on the $ \gamma_F $, we have for all $ (u_0,v_0)\neq(0_n,0_n) $,
    \[\sum_{u,v\in\F_2^n}^{}W_{\gamma_F}(u,v)W_{\gamma_F}(u+u_0,v+v_0)=0.\]
    Then we have:
    \begin{theorem}
        Any APN $ (n,n) $-function $ F $ satisfies that $ \forall (u_0,v_0) $,
        \[\sum_{\substack{u,v\in\F_2^n\\v\neq 0_n,v\neq v_0}}W_F^2(u,v)W_F^2(u+u_0,v+v_0)=2^{4n}-2^{3n+1}+2^{4n}\delta_0(u_0,v_0).\] 
    \end{theorem}

    \begin{corollary}
        If there exists $ (u_0,v_0)\neq(0_n,0_n) $ such that $ |W_F(u,v)| $ and $ |W_F(u+u_0,v+v_0)| $ 
        both achieve the maximun value of $ \{|W_F(u,v)|\mid u,v\in\F_2^n;v\neq 0_n\} $, 
        then we have 
        \[nl(F)\geq 2^{n-1}-\frac{1}{2}\sqrt[4]{2^{4n-1}-2^{3n}}.\]
    \end{corollary}
\end{document}