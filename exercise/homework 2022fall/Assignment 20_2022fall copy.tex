\documentclass[a4paper,12pt]{ctexart}
\usepackage{fullpage,enumitem,amsmath,amssymb,graphicx}
\newcommand{\Z}{\mathbb{Z}}
\newcommand{\F}{\mathbb{F}}
\newcommand{\Com}{\mathbb{C}}
\newcommand{\ord}{\operatorname{ord}}
\newcommand{\Q}{\mathbb{Q}}
\newcommand{\R}{\mathbb{R}}


\title{NIS2312-1 2022-2023 Fall Homework~1}
\author{唐灯}



\begin{document}
%   \maketitle
  \begin{center}

  \vspace{-0.3in}
  \begin{tabular}{c}
    \textbf{\Large NIS2312-1 2022-2023 Fall} \\
    \textbf{\Large  } \\
    \textbf{\Large  信息安全的数学基础(1)} \\
    \textbf{\Large  } \\
    \textbf{\Large  Answer} \\
    \textbf{\Large  } \\
    \textbf{\Large 2022年12月19日} \\
  \end{tabular}
  \end{center}
  \noindent
  \rule{\linewidth}{0.4pt}
  
%   可以使用计算机求模的运算.

\section{}
    本次考试共有$ 6 $ 道证明题, 共 $ 15\times 4+20\times 2=100 $分, 其中 $ 20 $分的题会有 $ 2/3 $道小题, $ 15 $分的题是相对基础的题.

    考试知识点:
    \begin{enumerate}[label=(\arabic{*})]
      \item 集合、等价关系、映射、
      群的定义、
      子群的判定、
      循环群、置换群、
      群的阶、群元素的阶、
      陪集、拉格朗日定理、
      正规子群、商群、
      同态、同构、同态基本定理、
      直积、直和;
      \item 环的定义、
      子环的判定、
      理想、商环、
      极大理想、素理想、
      单位、零因子、
      整环、多项式环、多项式的根;
      \item 域的定义、
      素域、
      扩域、
      有限域的结构、
      有限域的阶、
      有限域的乘法群、
      有限域的存在、唯一性
      
    \end{enumerate}
      
      \subsubsection*{Problem 1}
    设 $\alpha$ 是 $\mathbb{F}_{16}$ 的一个本原元, 其中 $\alpha$ 是 $x^4+x+1\in \mathbb{F}_2$ 在 $\mathbb{F}_{16}$ 上的一个根.
    计算 $\mathbb{F}_{16}^*$ 中全部元素的极小多项式, 并把 $x^{15}-1\in \mathbb{F}_2$ 分解成 $\mathbb{F}_2$ 上的不可约多项式的乘积.

    解: 如果 $\alpha$ 是 $x^4+x+1\in \mathbb{F}_2$ 在 $\mathbb{F}_{16}$ 上的一个根, 那么 $ \alpha^2 $也是根, 同理 $ \alpha^4,\alpha^8 $, 故 如果 $ f $是 $ \alpha $的极小多项式, 那么 $ f $ 也是 $ \alpha^2,\alpha^4,\alpha^8,... $的极小多项式. 因此有划分
    \begin{align*}
      \alpha^0\\
      \alpha,\alpha^2,\alpha^4,\alpha^8\\
      \alpha^3,\alpha^6,\alpha^{12},\alpha^9\\
      \alpha^5,\alpha^{10}\\
      \alpha^7,\alpha^{14},\alpha^{13},\alpha^{11}
    \end{align*}
      显然 $ \alpha^0=1 $的极小多项式是 $ x+1 $;

      因为 $ x^4+x+1 $ 在 $ \F_2 $上没有根, 在 $ \F_4 $上也没有根 (因为 $ x^2+x+1 $不整除 $ x^4+x+1 $), 故 $ x^4+x+1 $ 是不可约多项式. 注意到 $ f $ 在 $ \F_{16} $上的一个根是 $ \alpha $, 则 $ \alpha,\alpha^2,\alpha^4,\alpha^8 $ 对应
      的极小多项式为 $ x^4+x+1 $, 又因为 $ \alpha^7,\alpha^{14},\alpha^{13},\alpha^{11} $ 是$ \alpha,\alpha^2,\alpha^4,\alpha^8 $ 的逆, 故 其极小多项式是互反的, 即 $ x^4+x^3+1 $; 
      
      $ \alpha^5,\alpha^{10} $的极小多项式次数为 $ 2 $, 所以只能是 $ x^2+x+1 $ (因为只有 $ \F_2 $上的 $ 2 $次不可约多项式只有一个);

      $ \alpha^3,\alpha^6,\alpha^{12},\alpha^9\ $ 本身根是互反的 ($ \alpha^3*\alpha^{12}=1,\alpha^6*\alpha^9=1 $), 故其极小多项式是自反的, $ \F_2 $上 $ 4 $次自反多项式只有 $ x^4+x^3+x^2+x+1 $ 和 $ x^4+x^2+1=(x^2+x+1)^2 $, 故极小多项式为 $ x^4+x^3+x^2+x+1 $ .

      上述极小多项式的根的集合恰好是 $ \F_{2^4}^* $, 故 $ x^{15}-1 $的分解就是上述极小多项式的乘积
      \[x^{15}-1=(x+1)(x^2+x+1)(x^4+x+1)(x^4+x^3+1)(x^4+x^3+x^2+x+1).\]

      \subsubsection*{Problem 2}
      有限域的例题 (仅给出部分例题, 考试绝对不会出现原题):
      分解 $ x^{p^n-1}-1 $, 解题方法如上;

      写出乘法表, 习题已经做过了;

      构造一个 $ p^n $阶有限域, 并找出乘法群的生成元: 构造方法 $ \F_p[x]/<f> $, 其中 $ f $是次数为 $ n $的首一不可约多项式, 乘法群的生成元就是寻找阶为 $ p^n-1 $的元素;

      证明多项式为 $ \F_p $上的 不可约多项式: 一般是反证;

      有限域上的多项式的根: $ x^3+2x+1 $在 $ \F_3 $上如果有根 $ \alpha $, 那么 $ \alpha+1 $也是根: $ (x+1)^3+2(x+1)+1=x^3+1+2x+2+1=0 $, 同理 $ \alpha+1+1 $也是根.
\end{document}



 



