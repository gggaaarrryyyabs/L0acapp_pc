\documentclass[a4paper,12pt]{ctexart}
\usepackage{fullpage,enumitem,amsmath,amssymb,graphicx}
\newcommand{\Z}{\mathbb{Z}}
\newcommand{\F}{\mathbb{F}}
\newcommand{\Com}{\mathbf{C}}
\newcommand{\ord}{\operatorname{ord}}
\newcommand{\Q}{\mathbb{Q}}
\newcommand{\R}{\mathbb{R}}


\title{NIS2312-1 2022-2023 Fall Homework~1}
\author{唐灯}



\begin{document}
%   \maketitle
  \begin{center}

  \vspace{-0.3in}
  \begin{tabular}{c}
    \textbf{\Large NIS2312-1 2022-2023 Fall} \\
    \textbf{\Large  } \\
    \textbf{\Large  信息安全的数学基础(1)} \\
    \textbf{\Large  } \\
    \textbf{\Large  Assignment~8} \\
    \textbf{\Large  } \\
    \textbf{\Large 2022年10月17日} \\
  \end{tabular}
  \end{center}
  \noindent
  \rule{\linewidth}{0.4pt}
  
%   可以使用计算机求模的运算.

\subsubsection*{Problem 1}
  设 $ \phi $ 是群 $ G $到 群$ G' $的同态映射, $ e $是 $ G $的单位元. 证明:
  \[\phi\text{~是单同态}\Longleftrightarrow\operatorname{Ker}\phi=\{e\}.\]
  
  解: $ \Rightarrow $: 因为 $ \phi $ 是单同态, 则 $ \forall a,b\in G $, $ \phi(a)=\phi(b) $ 当且仅当 $ a=b $. 故如果有非单位元 
  $ x\in G' $ s.t. $ \phi(x)=e'=\phi(e) $, 那么 $ x=e $, 矛盾, 故 $ \operatorname{Ker}\phi=\{e\} $.

  $ \Leftarrow $: 假设有 $ x,y\in G $ s.t. $ \phi(x)=\phi(y) $, 故 $ \phi(xy^{-1})=\phi(x)\phi(y)^{-1}=e' $, 
  即 $ xy^{-1}\in \operatorname{Ker}\phi=\{e\} $, 则 $ x=y $, 因此 $ \phi $是单同态.

\subsubsection*{Problem 2} 
    设 $ C $为 $ \R $上全体连续实函数关于函数的加法所构成的群. 对 $ f(x)\in C $, 令
    \[\phi(f(x))=\int_{0}^{x}f(t)\mathrm{d} t.\]
    \begin{enumerate}[label=(\arabic{*})]
      \item 证明: $ \phi $是 $ C $到它自身的同态映射;
      \item 求 $ \phi $的象与核.
    \end{enumerate}
  
    解: \begin{enumerate}[label=(\arabic{*})]
      \item 连续函数 $ f(x) $的不定积分 $ \int_{0}^{x}f(t)\mathrm{d} t $也是连续函数, 故 $ \phi $是一个映射;
      $ \forall f(x),g(x)\in C $, 都有 
      \begin{align*}
        \phi(f(x)+g(x))&=\int_{0}^{x}f(t)+g(t)\mathrm{d} t=\int_{0}^{x}f(t)\mathrm{d} t+\int_{0}^{x}g(t)\mathrm{d} t\\
        &=\phi(f(x))+\phi(g(x)),
      \end{align*}
      所以 $ \phi $是 $ C $的自同态映射.
      \item $ \operatorname{Ker}\phi=\{0\} $且 $ \phi(C)=\left\{f(x) \middle| f'(x)\in C,f(0)=0\right\} $.
    \end{enumerate}
  
\subsubsection*{Problem 3}
  求 $ \Z_{20} $到 $ \Z_8 $的所有同态映射.

  解: $ \Z_{20} $为循环群, 考虑生成元 $ 1 $即可. $ \phi(1) $的阶有 $ 1,2,4,8 $四种情况, 
  但因为同态, 可知$ \operatorname{ord}\phi(1)\mid 20 $, 
  故 $ \operatorname{ord}\phi(1)=\{1,2,4\} $, 此时对应的 $\Z_8 $的元素分别是 $ \{0,4,2\text{or}6\} $. 
  因此可以写出所有的同态映射 $ \phi(x)=\overline{ax} $, 其中 $ a\in\{0,4,2,6\} $

\subsubsection*{Problem 4}
    设 $ \phi $是 $ \Z_{30} $到 $ U_5 $的满同态, 求 $ \operatorname{Ker}\phi $.
    
    解: 根据同态基本定理可知 $ \Z_{30}/\operatorname{Ker}\phi\cong U_5 $. 因此 $ \left\lvert \operatorname{Ker}\phi\right\rvert=\left\lvert \Z_{30}\right\rvert /\left\lvert U_5\right\rvert =6 $. 
    故 $  \operatorname{Ker}\phi=\langle 5\rangle $.

\subsubsection*{Problem 5}
  设 $ \phi $是群 $ G $到 群$ G' $的同态映射, $ a,b\in G $. 证明: $ \phi(a)=\phi(b) $当且仅当 $ a\operatorname{Ker}\phi=b\operatorname{Ker}\phi $.

  解: $ \Rightarrow $: 因为 $ \phi(a)=\phi(b) $, 则 $ \phi(ab^{-1})=e' $, 故 $ ab^{-1}\in\operatorname{Ker}\phi $, 因此
  $ a\operatorname{Ker}\phi=b\operatorname{Ker}\phi $.

  $ \Leftarrow $: $ a\operatorname{Ker}\phi=b\operatorname{Ker}\phi\Rightarrow ab^{-1}\in\operatorname{Ker}\phi $, 故
  $ \phi(a)=\phi(ab^{-1}b)=\phi(ab^{-1})\phi(b)=\phi(b) $.

\subsubsection*{Problem 6}
  [上一届期末就考了同态基本定理, 得分点是下面的4条+同态基本定理的公式, 15分的送分题] 
  
  设 $ k $是 $ m $的正因子. 证明 $ \Z_m/\langle\overline{k}\rangle\cong\Z_k $.
  
  解: (这道题主要是 $ n\overline{1}_{\Z_m} $和 $ n\overline{1}_{\Z_k} $不好写, 没有歧义的话用 $ \overline{n} $即可, 我这么写主要是为了区分两个群中的元素, 也可以用 $ \overline{n}\in\Z_m $和 $ [n]\in\Z_k $来表示).

  设 
  \[\begin{array}{cccc}
    \phi:&~\Z_m&\rightarrow&\Z_k\\
    &~n\overline{1}_{\Z_m}&\mapsto& n\overline{1}_{\Z_k}
  \end{array}.\]
  \begin{enumerate}[label=(\arabic{*})]
    \item 显然是映射;
    \item 对 $ \forall n\overline{1}_{\Z_k}\in\Z_{k} $, 都 $\exists n\overline{1}_{\Z_m}\in\Z_m $ s.t.
    $ \phi(n\overline{1}_{\Z_m})=n\overline{1}_{\Z_k} $, 故为满射;
    \item $ \forall x\overline{1}_{\Z_m},y\overline{1}_{\Z_m}\in\Z_m $, 
    都有 $ \phi(x\overline{1}_{\Z_m}+y\overline{1}_{\Z_m})=(x+y)\overline{1}_{\Z_k}=x\overline{1}_{\Z_k}+y\overline{1}_{\Z_k}=\phi(x\overline{1}_{\Z_k})+\phi(y\overline{1}_{\Z_k}) $, 因此是同态映射;
    \item 且
    \begin{align*}
      \operatorname{Ker}\phi&=\{n\overline{1}_{\Z_m}\in\Z_m\mid n\pmod{k}=0\}=k\Z_m=\langle\overline{k}\rangle.
    \end{align*}
  \end{enumerate}
  因此根据同态基本定理可知$ \Z_m/\langle\overline{k}\rangle\cong\Z_k $.

\subsubsection*{Problem 7}
设群 $U_{4}=\{1,-1, \mathrm{i},-\mathrm{i}\}$ 是 $4$ 次单位根群, $K=\{e, a, b, ab\}$  
(Klein 四元群, 它是最小的非循环群) 是由元素
    $a, b$ 及关系 $a^{2}=b^{2}=e$ 和 $a b=b a$ 所定义的群. 问 $U_{4}$ 与 $K$ 是否同构, 为什么?

  解: 如果 $U_4$ 与 $K$ 同构, 设 $\phi$ 是 $U_4$ 到 $K$ 的同构映射. 令
  $\phi(i) = x$. 易知, $x^2 = e$. 从而
  \[\phi(−1) = \phi(i^2) = (\phi(i))^2 = x^2 = e\].
  另一方面, $\phi(1) = e$. 由于 $\phi$ 是单映射, 所以 $−1 = 1$. 这是一个矛盾. 从而知 $U_4$ 与 $K$ 不同构.
  
  \subsubsection*{Problem 8}
  证明: $4$ 阶群必同构于 $U_4$ 或 Klein 四元群 $K=\{e, a, b, a b\}$.

  解: 设 $H$ 为一个 $4$ 阶群.
  \begin{enumerate}[label=(\arabic{*})]
    \item 如果 $H$ 有 $4$ 阶元, 则 $H$ 为 $4$ 阶循环群, 从而 $H$ 与 $U_4$ 同构.
    \item 如果 $H$ 不含有 $4$ 阶元, 则除单位元 $e$ 外, $H$ 的其余三个元素的阶都是 $2$, 不妨设这三个元素为 $a,b,ab = ba$. 
    显然 $H$ 是交换群. 从而 $H$ 的元素与 $K$ 的元素一一对应, 且有完全一致的运算关系. 所以 $H$ 与 $K$ 同构.
  \end{enumerate}
\end{document}