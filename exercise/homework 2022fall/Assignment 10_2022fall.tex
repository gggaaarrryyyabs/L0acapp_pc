\documentclass[a4paper,12pt]{ctexart}
\usepackage{fullpage,enumitem,amsmath,amssymb,graphicx}
\newcommand{\Z}{\mathbb{Z}}
\newcommand{\F}{\mathbb{F}}
\newcommand{\Com}{\mathbf{C}}
\newcommand{\ord}{\operatorname{ord}}
\newcommand{\Q}{\mathbb{Q}}
\newcommand{\R}{\mathbb{R}}


\title{NIS2312-1 2022-2023 Fall Homework~1}
\author{唐灯}



\begin{document}
%   \maketitle
  \begin{center}

  \vspace{-0.3in}
  \begin{tabular}{c}
    \textbf{\Large NIS2312-1 2022-2023 Fall} \\
    \textbf{\Large  } \\
    \textbf{\Large  信息安全的数学基础(1)} \\
    \textbf{\Large  } \\
    \textbf{\Large  Answer~10} \\
    \textbf{\Large  } \\
    \textbf{\Large 2022年10月24日} \\
  \end{tabular}
  \end{center}
  \noindent
  \rule{\linewidth}{0.4pt}
  
%   可以使用计算机求模的运算.

\subsubsection*{Problem 1}
  把下列置换写成不相交轮换的乘积, 并计算置换的奇偶性:
  \begin{enumerate}[label=(\arabic{*})]
    \item  $\begin{pmatrix}
      1 &2 &3 &4 &5 \\
      1 &3 &4 &5 &2 \\
    \end{pmatrix}\begin{pmatrix}
      1 &2 &3 &4 &5 \\
      3 &2 &4 &1 &5 \\
    \end{pmatrix};$
    \item $\begin{pmatrix}
      1 &2 &3 &4 &5 &6\\
      1 &3 &6 &5 &2 &4\\
    \end{pmatrix}\begin{pmatrix}
      1 &2 &3 &4 &5 &6\\
      6 &2 &4 &1 &5 &3\\
    \end{pmatrix}.$
  \end{enumerate}
  
  解: \begin{enumerate}[label=(\arabic{*})]
    \item  $\begin{pmatrix}
      1 &2 &3 &4 &5 \\
      1 &3 &4 &5 &2 \\
    \end{pmatrix}\begin{pmatrix}
      1 &2 &3 &4 &5 \\
      3 &2 &4 &1 &5 \\
    \end{pmatrix}=(1~4)(2~3~5)$ 奇置换;
    \item $\begin{pmatrix}
      1 &2 &3 &4 &5 &6\\
      1 &3 &6 &5 &2 &4\\
    \end{pmatrix}\begin{pmatrix}
      1 &2 &3 &4 &5 &6\\
      6 &2 &4 &1 &5 &3\\
    \end{pmatrix}=(1~4)(2~3~5)$ 奇置换.
  \end{enumerate}
\subsubsection*{Problem 2} 
    计算置换的乘积, 把结果写成不相交轮换的乘积, 并计算置换的奇偶性:
    \[\left(4~9~6~7~8\right)\left(2~6~4\right)\left(1~8~7\right)\left(3~5\right).\]
  
    解: $ (3~5) $不与其他的相交, 故不考虑. 
    
    其他的可以写为 $ 4\rightarrow 2,2\rightarrow 6\rightarrow 7,7\rightarrow 1, 8\rightarrow 7\rightarrow 8,1\rightarrow 8\rightarrow 4,9\rightarrow 6,6\rightarrow 4\rightarrow 9 $.
    因此有 $ (3~5)(4~2~7~1)(6~9) $.

\subsubsection*{Problem 3}
  设 $ \sigma = \left(1~2~3~4~5~6\right)\in S_6 $, 求 $ \langle\sigma\rangle $.

    解: $ \langle \sigma\rangle=\{(1),(1~2~3~4~5~6),(1~3~5)(2~4~6),(1~4)(2~5)(3~6),(1~5~3)(2~6~4),(1~6~5~4~3~2)\} $.

\subsubsection*{Problem 4}
    证明: $ \left(i_1~i_2~\cdots~i_r\right)^{-1}=\left(i_r~i_{r-1}~\cdots~i_1\right) $.

    解: 当 $ n\in\Z\setminus\{i_1,i_2,\dots,i_r\} $时, 
    $ \left(i_1~i_2~\cdots~i_r\right)\left(i_r~i_{r-1}~\cdots~i_1\right)(n)=n $是显然的, 
    因此考虑 $ n\in\{i_1,i_2,\dots,i_r\} $的情况: 

    不失一般性的假设 $ n=i_m $, 其中 $ 1\le m\le r $, 因此 $ \left(i_r~i_{r-1}~\cdots~i_1\right)(n)=i_{m-1} $, 
    故 $ \left(i_1~i_2~\cdots~i_r\right)\left(i_r~i_{r-1}~\cdots~i_1\right)(n)=\left(i_1~i_2~\cdots~i_r\right)(i_{m-1})=i_m=n $. 
    显然对所有的 $ n\in\{i_1,i_2,...,i_r\} $成立, 故 $ \left(i_1~i_2~\cdots~i_r\right)\left(i_r~i_{r-1}~\cdots~i_1\right)=(1) $, 即
    \[\left(i_1~i_2~\cdots~i_r\right)^{-1}=\left(i_r~i_{r-1}~\cdots~i_1\right).\]

\subsubsection*{Problem 5}
  设 $ \sigma\in S_n $. 证明: 
  \[\sigma\left(i_1~i_2~\cdots~i_r\right)\sigma^{-1}=\left(\sigma(i_1)~\sigma(i_2)~\cdots~\sigma(i_r)\right).\]

    解: 首先考虑 $ m=\sigma(i)\notin\{\sigma(i_n)|n=1,2,\dots,r\} $的情况: 此时 $ \sigma^{-1}(m)\notin\{i_n\mid n=1,2,...,r\} $, 
    即轮换 $ \left(i_1~i_2~\cdots~i_r\right)(\sigma^{-1}(m))=\sigma^{-1}(m) $, 因此等式左边对 $ m $作用的结果仍是 $ m $, 
    此时等式右边也是 $ m $. 
    
    其次考虑等式两边对$ \sigma(i_1) $的作用情况: 
    等式左边为 $ \sigma\left(i_1~i_2~\cdots~i_r\right)\sigma^{-1}(\sigma(i_1))=\sigma(i_2) $, 等式右边为 $ \sigma(i_2) $.
    等式两边对其他的 $ \sigma(i_n) $ 其中 $ n=2,3,...,r $的作用情况是一样的.
  % 解: 设 $ G=\langle g\rangle $, 则 $ G $是交换群, 故 $ H\triangleleft G $, $ G/H $是一个群; 
  % 那么 $ \forall xH\in G/H $, 都有 $ x=g^k $, 其中 $ k\in\Z $, 则 $ xH=g^kH=(gH)^k $, 因此 $ gH $是群 $ G/H $的生成元, 
  % 即 $ G/H $也是循环群.

\subsubsection*{Problem 6}
  设 $ \sigma $为一个 $ n $ 阶置换, 集合 $ X=\{1,2,...,n\} $. 在 $ X $中, 规定关系 ``$\sim$'': 
  \[k\sim l~\Longleftrightarrow~\text{存在}~r\in\Z, ~\text{使} ~\sigma^r(k)=l.\]
  \begin{enumerate}[label=(\arabic{*})]
    \item 证明: $ \sim $是 $ X $的一个等价关系;
    \item 证明: $ k\sim l $的充分必要条件是 $ k $与 $ l $属于 $ \sigma  $的同一个轮换;
    \item 对于置换
    \[\sigma=\begin{pmatrix}
      1 &2 &3 &4 &5 &6 &7 &8  &9 &10 \\
      3 &2 &6 &8 &9 &1 &7 &10 &4 &5 \\
    \end{pmatrix},\]
    试确定集合 $ X=\{1,2,...,10\} $的所有等价类.
  \end{enumerate}
  
  解: \begin{enumerate}[label=(\arabic{*})]
    \item 因为 $ \sigma^0(k)=k $, 故反身性成立;
    
    如果 $ k\sim l $, 那么 $ \sigma^r(k)=l $, 故 $ \sigma^{-r}(l)=k $, 因此 $ l\sim k $, 故对称性成立;
    
    如果 $ j\sim k,k\sim l $, 那么有 $ \r_1,r_2\in\Z $使得 $ \sigma^{r_1}(j)=k $和 $ \sigma^{r_2}(k)=l $, 
    故 $ \sigma^{r_1+r_2}=l $, 即 $ j\sim l  $, 因此传递性成立.
    
    所以 $ \sim $是 $ X $上的一个等价关系;
    \item 将 $ \sigma $表示为不相交轮换的乘积 $ \sigma_1\sigma_2\cdots\sigma_n $:

    $ \Rightarrow $: 因为 $ k\sim l  $, 那么 $ \sigma^r(k)=\sigma_1^r\sigma_2^r\cdots\sigma_n^r(k)=l $, 因此 $ k $和
    $ l $必然等于 $ \sigma_i^r(k) $, 其中 $ 1\le i\le n $;

    $ \Leftarrow $: 假如 $ k $和 $ l $属于同一个 $ \sigma_i $, 那么有 $ \sigma_i^r(k)=l $, 
    因此有 $ \sigma_i^r(k)=(\sigma_1\sigma_2\cdots\sigma_i\cdots\sigma_n)^r(k)=\sigma_r(k)=l $, 即 $ k\sim l $.
    \item 有 $ \sigma=(1~3~6)(2)(4~8~10~5~9)(7) $, 故等价类是 $ \{1,3,6\},\{2\},\{4,8,10,5,9\},\{7\} $.
  \end{enumerate}

\end{document}