\documentclass[a4paper,12pt]{ctexart}
\usepackage{fullpage,enumitem,amsmath,amssymb,graphicx}
\newcommand{\Z}{\mathbb{Z}}
\newcommand{\F}{\mathbb{F}}
\newcommand{\Com}{\mathbb{C}}
\newcommand{\ord}{\operatorname{ord}}
\newcommand{\Q}{\mathbb{Q}}
\newcommand{\R}{\mathbb{R}}


\title{NIS2312-1 2022-2023 Fall Homework~1}
\author{唐灯}



\begin{document}
%   \maketitle
  \begin{center}

  \vspace{-0.3in}
  \begin{tabular}{c}
    \textbf{\Large NIS2312-1 2022-2023 Fall} \\
    \textbf{\Large  } \\
    \textbf{\Large  信息安全的数学基础(1)} \\
    \textbf{\Large  } \\
    \textbf{\Large  Answer~19} \\
    \textbf{\Large  } \\
    \textbf{\Large 2022年12月12日} \\
  \end{tabular}
  \end{center}
  \noindent
  \rule{\linewidth}{0.4pt}
  
%   可以使用计算机求模的运算.

\subsubsection*{Problem 1}
    设 $ a,b\in\R $, $ b\ne 0 $. 证明: $ \R(a+b\operatorname{i})=\Com $.

    解: 因为 $ \left[ \Com:\R \right]=\left[ \Com:\R(a+b\operatorname{i}) \right]\left[ \R(a+b\operatorname{i}):\R \right]=2 $, 
    则 $ \left[ \R(a+b\operatorname{i}):\R \right]\le 2 $. 
    同时 $ \operatorname{i}\notin\R $, 有 $ \left[ \R(a+b\operatorname{i}):\R \right]>1 $, 
    故$ \left[ \R(a+b\operatorname{i}):\R \right]= 2 $, 即 $ \R(a+b\operatorname{i})=\Com $.

\subsubsection*{Problem 2}
    设 $ F $是个域, $ a,b\in F,a\ne 0 $. 如果$ c $属于 $ F $的某个扩域, 证明: $ F(c)=F(ac+b) $(即 $ F $ ``吸收''它自己的元素).
    
    解: 因为 $ F(ac+b) $是包含 $ F $和 $ ac+b $的最小域, 且 $ a,b\in F $, $ ac+b\in F(c) $, 则 $ F(c)\supseteq F(ac+b) $;
    又因为 $ c=a^-1\cdot((ac+b)-b) $, 同理可得 $ F(c)\subseteq F(ac+b) $. 
    故 $ F(c)=F(ac+b) $.

\subsubsection*{Problem 3}
    证明: 商环 $ \R[x]/\langle x^2+1\rangle $与复数域 $ \Com $同构. 

    解: 因为 $ \Com=\R(i) $, 即 $ \Com $是 $ \R $的扩域. 
    所以利用定理5.3.1.
    \begin{enumerate}[label=(\arabic{*})]
      \item 显然 $ i $ 是 $ x^2+1=0 $的根, 即$ i $ 是 $ \R $上的代数元.
      \item 注意到 $ x^2+1 $ 是 $ \R[x] $中的以 $ i $为根的不可约多项式, 
    \end{enumerate}
    故 $ \R[x]/\langle x^2+1\rangle $同构于 $ \R(i) $, 即商环 $ \R[x]/\langle x^2+1\rangle $与复数域 $ \Com $同构. 
\end{document}



F<v>:=GF(2,3);
F_l<w>:=GF(2,9);
function likangquan(i)
  q:=2^i;
  tt:=[];
  for x,y,z in F do
    Append(~tt,(x^(q+1)+x*z^q+y^q*z)+(x^q*z+y^(q+1))*w+(x^q*y+y*z^q+z^(q+1))*w^2);
  end for;
  return tt;
end function;
tt:=likangquan(1);
index_y:=[Eltseq(tt[i]):i in [1..2^9]];
truthtable_y:=[&+[Integers()!(index_y[j][i])*2^(9-i):i in [1..9]]:j in [1..2^9]];

function isAPN(F,n)
    Field:=FiniteField(2,n);
	for a in Field do
		if a eq 0 then
			continue;
		end if;
		derivative_image := { F(x) + F(x+a) : x in Field };
		if #derivative_image ne 2^(n-1) then
			return false;
		end if;
	end for;
	return true;
end function;



index_x:=[Eltseq(x+y*w+z*w^2):x,y,z in F];
truthtable_x:=[&+[Integers()!(index_x[j][i])*2^(9-i):i in [1..9]]:j in [1..2^9]];



