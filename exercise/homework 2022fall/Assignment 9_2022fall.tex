\documentclass[a4paper,12pt]{ctexart}
\usepackage{fullpage,enumitem,amsmath,amssymb,graphicx}
\newcommand{\Z}{\mathbb{Z}}
\newcommand{\F}{\mathbb{F}}
\newcommand{\Com}{\mathbf{C}}
\newcommand{\ord}{\operatorname{ord}}
\newcommand{\Q}{\mathbb{Q}}
\newcommand{\R}{\mathbb{R}}


\title{NIS2312-1 2022-2023 Fall Homework~1}
\author{唐灯}



\begin{document}
%   \maketitle
  \begin{center}

  \vspace{-0.3in}
  \begin{tabular}{c}
    \textbf{\Large NIS2312-1 2022-2023 Fall} \\
    \textbf{\Large  } \\
    \textbf{\Large  信息安全的数学基础(1)} \\
    \textbf{\Large  } \\
    \textbf{\Large  Answer~9} \\
    \textbf{\Large  } \\
    \textbf{\Large 2022年10月19日} \\
  \end{tabular}
  \end{center}
  \noindent
  \rule{\linewidth}{0.4pt}
  
%   可以使用计算机求模的运算.

\subsubsection*{Problem 1}
  求群 $ U(20) $的所有循环子群.
  
  解: 因为 $ 20 $ 不是素数, 故 $ U(20) $不是循环群. 因此群中的元素的阶可能有 $ 1,2,4 $. 
  \begin{enumerate}[label=(\arabic{*})]
    \item 阶为 $ 1 $的显然是单位元构成的循环群;
    \item 阶为 $ 2 $的元素只能构成 $ 2 $阶循环群, 不同的二阶元对应二阶循环群必然不同, 因此可以列出 $ \langle 9\rangle,\langle 11\rangle,\langle 19\rangle $三种;
    \item 阶为 $ 4 $的元素能直接找到 $ 3 $: 因此对应的四阶循环群是 $ \langle 3\rangle=\{1,3,9,7\} $, 注意到还剩下两个四阶元素, 任取其一作为生成元即可, $ \langle 13\rangle $.  
  \end{enumerate}
\subsubsection*{Problem 2} 
    在群 $ G=GL_2(\R) $中确定元素
    \[A=\begin{pmatrix}
      2 &3\\
      -1 &-1
    \end{pmatrix}\]
    所生成的循环群 $ H $的所有元素.
    
    解: 有 $A=\begin{pmatrix}
      2 &3\\
      -1 &-1
    \end{pmatrix},A^2= \begin{pmatrix}
      1 &3\\
      -1 &-2
    \end{pmatrix},A^3=\begin{pmatrix}
      -1 &0\\
      0 &-1
    \end{pmatrix},A^4=\begin{pmatrix}
      -2 &-3\\
      1 &1
    \end{pmatrix},A^5=\begin{pmatrix}
      -1 &-3\\
       1 &2
    \end{pmatrix},E=\begin{pmatrix}
      1 &0\\
      0 &1
    \end{pmatrix}$.

\subsubsection*{Problem 3}
  设 $ \phi $是群 $ G $到 $ G' $的同构映射, $ a\in G $. 证明:
  \[\operatorname{ord}a=\operatorname{ord}\phi(a).\]

    解: $ (\phi(a))^{\operatorname{ord}a}=\phi(a^{\operatorname{ord}a})=\phi(e)=e' $, 
    故 $ \operatorname{ord}\phi(a)\mid\operatorname{ord}a $;

    又因为 $ \phi(a^{\operatorname{ord}\phi(a)})=(\phi(a))^{\operatorname{ord}\phi(a)}=e'=\phi(e) $且 $ \phi $是同构映射,
    故 $ a^{\operatorname{ord}\phi(a)}=e $, 即 $ \operatorname{ord}a\mid\operatorname{ord}\phi(a) $.

    综上 $ \operatorname{ord}a=\operatorname{ord}\phi(a) $.

\subsubsection*{Problem 4}
    设 $ G $ 是群, $ a,b\in G,\operatorname{ord}a=m,\operatorname{ord}b =n $. 证明: 如果 $ ab=ba $且 $ \langle a\rangle\cap\langle b\rangle=\{e\} $, 则 $ \operatorname{ord}ab=[m,n] $.

    解: 因为 $ (ab)^{[n,m]}=a^{[n,m]}b^{[n,m]}=e $, 故 $ \ord ab\mid[n,m] $;

    假设 $ \ord ab = r $, 那么 $ (ab)^r=a^rb^r=e $, 则 $ a^r=(b^{-1})^r\in\langle b\rangle $, 
    故 $ a^r\in\langle a\rangle\cap\langle b\rangle=\{e\}  $, 因此 $ m\mid r $; 同理 $ n\mid r $; 即 $ [m,n]\mid \ord ab $.

    因此 $ \ord ab=[m,n] $.

\subsubsection*{Problem 5}
  设正整数 $ n $的标准分解式为 
  \[n=p_1^{r_1}p_2^{r_2}\cdots p_s^{r_s},\]
  其中 $ p_1,p_2,\dots,p_s  $是 $ n $不同的素因子. 证明: $ n $阶循环群 $ G $的子群的个数为 
  \[r=(r_1+1)(r_2+1)\cdots (r_s+1).\]

  解: 根据定理1.5.5推论1, 子群个数为因子的个数, 恰好有  $ r=(r_1+1)(r_2+1)\cdots (r_s+1) $ 个.

\subsubsection*{Problem 6}
    证明: 群 $ G $仅有平凡子群的充分必要条件是 $ G=\{e\} $或者 $ G $是素数阶循环群.

    解: $ \Rightarrow $: 如果 $ G=\{e\} $, 结论成立. 假设 $ G\ne\{e\} $, 则对非单位元 $ x\in G $, 
    都有$ \langle x\rangle=G $. 
    需要一个结论, $ G $ 有有限个子群, 那么$ G $是有限群: 
    
    利用反证法, 
    假设 $ G $是无限群, 构造集合 $ S= \{\langle x\rangle\mid x\in G\} $, 该集合为有限集合. 
    显然 $ \cup_{s\in S}s = G $, 因此必然有 $ s=\langle x\rangle\in S $是无限循环群. 
    再根据定理1.5.5推论1可知有无限个 $ \langle x^i\rangle $循环群, 其中 $ i=1,2,... $, 矛盾. 故 $ G $是有限群.

    因此设 $ |G|=n=mp $, 其中 $ p $ 是素数且 $ m\ge 2 $, 选取非单位元 $ a\in G $且 $ a^n=e $:
    \begin{enumerate}
      \item 如果 $ a^p\ne e $, 则 $ \langle a^p\rangle $ 是一个阶整除 $ m $的真子群, 矛盾;
      \item 如果 $ a^p=e $, 那么 $ \langle a\rangle $是一个 $ p $阶真子群, 矛盾.
    \end{enumerate} 
    故 $ G $的阶必然是素数.
    
    $ \Leftarrow $: $ G=\{e\} $是显然的. 如果 $ G $是素数阶循环群, 那么 $ |G| $的因子仅有 $ 1 $和 $ |G| $, 
    分别对应 $ \{e\} $和  $ G $, 因此 $ G $仅有平凡子群. 


\end{document}