\documentclass[a4paper,12pt]{ctexart}
\usepackage{fullpage,enumitem,amsmath,amssymb,graphicx}
\newcommand{\Z}{\mathbb{Z}}
\newcommand{\F}{\mathbb{F}}
\newcommand{\Com}{\mathbf{C}}
\newcommand{\ord}{\operatorname{ord}}
\newcommand{\Q}{\mathbb{Q}}
\newcommand{\R}{\mathbb{R}}


\title{NIS2312-1 2022-2023 Fall Homework~1}
\author{唐灯}



\begin{document}
%   \maketitle
  \begin{center}

  \vspace{-0.3in}
  \begin{tabular}{c}
    \textbf{\Large NIS2312-1 2022-2023 Fall} \\
    \textbf{\Large  } \\
    \textbf{\Large  信息安全的数学基础(1)} \\
    \textbf{\Large  } \\
    \textbf{\Large  Answer~18} \\
    \textbf{\Large  } \\
    \textbf{\Large 2022年12月5日} \\
  \end{tabular}
  \end{center}
  \noindent
  \rule{\linewidth}{0.4pt}
  
%   可以使用计算机求模的运算.

\subsubsection*{Problem 1}
    设 $ f(x) $是 $ \R[x]_n $内任意多项式, 其中 $ \R[x]_n $是次数小于 $ n $的系数在 $ \R $上的多项式集合. 设 $ a\in\R $且非零. 
    \begin{enumerate}[label=(\arabic{*})]
      \item 证明: $ \left\{ 1,x-a,(x-a)^2,(x-a)^3,...,(x-a)^{n-1} \right\} $是 $ \R[x]_n $的一组基底.
      \item 设 $ f(x)=\sum_{i=0}^{n-1}c_ix^i $, 其中 $ c_i\in\R,i=0,1,...,n-1 $, 则 $ f(x) $可以用新的基底表示为: 
      \[ f(x)=a_0+a_1(x-a)+a_2(x-a)^2+\cdots+a_{n-1}(x-a)^{n-1}. \] 给出 $ a_i $的表达式 (用 $ c_i $和 $ a $ 表示).
    \end{enumerate}

    解: \begin{enumerate}[label=(\arabic{*})]
      \item 假设 $ S=\left\{ 1,x-a,(x-a)^2,(x-a)^3,...,(x-a)^{n-1} \right\} $. 
      注意到 $ \dim_{\R}(\R[x]_n)=n $且 $ |S|=n $, 故仅需证明 $ S $中的元素是线性独立即可. 
      又因为 
      \[ (x-a)^i=\begin{bmatrix}
        0&\cdots&0&1&\binom{i}{1}(-a)&\binom{i}{2}(-a)^2&\cdots&(-a)^i
      \end{bmatrix}\begin{bmatrix}
        x^{n-1}\\\vdots\\x^{i+1}\\x^{i}\\x^{i-1}\\\vdots\\1
      \end{bmatrix} \]
      使用线性变换可以得到结果: 
      \[ \begin{bmatrix}
        (x-a)^{n-1}\\\vdots\\(x-a)^{i-1}\\(x-a)^i\\(x-a)^{i-1}\\\vdots\\1  
      \end{bmatrix}=
      P\begin{bmatrix}
        x^{n-1}\\\vdots\\x^{i+1}\\x^{i}\\x^{i-1}\\\vdots\\1
      \end{bmatrix} \]
      其中 
      \[P=\begin{bmatrix}
        1&\binom{n-1}{1}&\cdots&\binom{n-1}{i+1}(-a)^{i+1}&\binom{n-1}{i}(-a)^i&\cdots&(-a)^{n-1}\\
        0&1&\cdots&\binom{n-2}{i}(-a)^i&\binom{n-2}{i-1}(-a)^{i-1}&\cdots&(-a)^{n-2}\\
        \vdots&\vdots&\vdots&\vdots&\vdots&\vdots&\vdots\\
        0&0&\cdots&1&\binom{i}{1}(-a)&\cdots&(-a)^i\\
        \vdots&\vdots&\vdots&\vdots&\vdots&\vdots&\vdots\\
        0&0&\cdots&0&0&\cdots&1\\
      \end{bmatrix}\]是一个上三角矩阵, 因此是一个可逆线性变换, 故
      因为 $ \left\{ 1,x,x^2,...,x^{n-1} \right\} $是一组基底, 所以 $ \left\{ 1,x-a,(x-a)^2,(x-a)^3,...,(x-a)^{n-1} \right\} $是 $ \R[x]_n $的一组基底.

      
      \item 利用泰勒公式可得到答案:
      \begin{align*}
        a_0&=f(a)=\sum_{i=0}^{n-1}c_ia^i\\
        a_1&=f'(a)=\sum_{i=1}^{n-1}ic_ia^{i-1}\\
        &\vdots\\
        a_j&=f^{(j)}(a)=\frac{1}{j!}\sum_{i=j}^{n-1}i(i-1)(i-2)\cdots(i-j+1)c_ia^{i-j},j=2,3,...,n-1\\
      \end{align*}
    \end{enumerate} 
    
\end{document}

