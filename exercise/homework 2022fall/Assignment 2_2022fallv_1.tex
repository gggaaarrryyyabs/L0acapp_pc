\documentclass[a4paper,12pt]{ctexart}
\usepackage{fullpage,enumitem,amsmath,amssymb,graphicx}
\newcommand{\Z}{\mathbf{Z}}
\newcommand{\F}{\mathbf{F}}
\newcommand{\Com}{\mathbf{C}}
\newcommand{\ord}{\operatorname{ord}}
\newcommand{\Q}{\mathbf{Q}}
\newcommand{\R}{\mathbf{R}}


\title{NIS2312-1 2022-2023 Fall Homework~1}
\author{唐灯}



\begin{document}
%   \maketitle
  \begin{center}

  \vspace{-0.3in}
  \begin{tabular}{c}
    \textbf{\Large NIS2312-1 2022-2023 Fall} \\
    \textbf{\Large  } \\
    \textbf{\Large  信息安全的数学基础(1)} \\
    \textbf{\Large  } \\
    \textbf{\Large  Assignment~2} \\
    \textbf{\Large  } \\
    \textbf{\Large 2022年9月20日} \\
  \end{tabular}
  \end{center}
  \noindent
  \rule{\linewidth}{0.4pt}
  
%   可以使用计算机求模的运算.
  
  \subsubsection*{Problem 1} 
      RSA公钥密码方案: 
      \begin{enumerate}[label=(\arabic*)]
          \item 密钥生成:
          
          随机选取两个大素数 $ p,q $, 计算 $ n=pq,\varphi(n)=(p-1)(q-1) $; 任意选取一个大整数 $ e $满足
          $ 1\le e\le \varphi(n) $且满足 $ (e,\varphi(n))=1 $; 计算 $ d $, 满足 $ de\equiv 1\pmod{\varphi(n)} $. 
          以 $ \{e, n\} $ 为公钥, $ \{d, n\} $ 为私钥. 
          
          \item 加密运算:
          
          对明文 $m < n$ 进行加密: $ c = E(m) \equiv m^e \pmod{n} $.
          \item 解密运算:
          
          接收方对 $ c $进行解密: $ m = D(c) \equiv c^d \pmod{n} $.
        \end{enumerate}
        那么: 
        \begin{enumerate}[label=(\arabic*)]
            \item 试证明解密运算的正确性;
            \item 取素数 $ p=3,q=11 $, 则 $ n=33,\varphi(n)=20 $. 取 $ e=7 $. 尝试计算密文 $ c=29 $对应的明文.
        \end{enumerate}

      解:\begin{enumerate}
          \item $ c^d=(m^e)^d=m^{de}=m^{k\varphi(n)+1}\pmod{n} $. 当 $ (m,n)=1 $时, 欧拉定理可得到 
          $ c^d=(m^{\varphi(n)})^k\cdot m\equiv m\pmod{n} $; 当 $ (m,n)\ne 1 $时, 由于 $ m<n=pq $, 则 $ p\mid m $ 或 $ q\mid m $. 不妨假设 $ p\mid m $, 则 $ m=tp $, 其中 $ 0<t<q $. 此时 $ (m,q)=1 $, 因此由欧拉定理得
          $ m^{k\varphi(n)}=m^{k(p-1)(q-1)}=(m^{q-1})^{k(p-1)}\equiv 1\pmod{q} $, 故假设 $ m^{k\varphi(n)}=k'q+1 $, 则 $ m^{k\varphi(n)+1}=k'qm+m=k'qtp+m\equiv m\pmod{n} $. 综上 $ c^d\equiv m\pmod{n} $ 成立. Q.E.D.
          \item $ e=7 $, 则由 $ de\equiv 1\pmod{\varphi(n)} $可得到 $ d=3 $, 因此 $ m=D(c)\equiv c^d\pmod{n}\equiv 29^3\pmod{33}\equiv 2 $.
      \end{enumerate}
\subsubsection*{Problem 2}
    Rabin数字签名方案: 
    
    随机选取两个大素数 $ p,q $ 且 $ p\equiv q\equiv 3\pmod{4} $, 令 $ n=pq $. 以 $\{n\}$为公钥, $ \{p,q\} $为私钥. 加密运算是将明文 $ m $加密为 $ c\equiv m^2\pmod{n} $.  那么:
    \begin{enumerate}[label=(\arabic*)]
        \item 尝试设计一种Rabin密码算法的解密运算;
        \item 证明你所设计的解密运算的正确性;
        \item 取素数 $ p=7,q=11 $, 则 $ n=77 $, 对明文 $ m=20 $进行加密得到 $ c\equiv m^2\pmod{n}=15 $, 尝试计算密文 $ c=15 $对应的明文.
    \end{enumerate}
    解:\begin{enumerate}[label=(\arabic*)]
        \item 利用数论公式直接给出密文 $ c $的模 $ p $和模 $ q $ 平方根: $ m_p=c^{(p+1)/4}\pmod{p},m_q=c^{(q+1)/4}\pmod{q} $; 用欧几里得算法给出两个整数 $ y_p,y_q $使得$ y_p\cdot p+y_q\cdot q=1 $; 利用中国剩余定理给出 $ c $模 $ n $的四个平方根:
            \begin{align*}
                r_1&=y_p\cdot p\cdot m_q+y_q\cdot q\cdot m_p\pmod{n}\\
                r_2&=n-r_1\\
                r_3&=y_p\cdot p\cdot m_q-y_q\cdot q\cdot m_p\pmod{n}\\
                r_4&=n-r_3, 
            \end{align*}
        注意到四个解中有一个是明文, 所以可以确定该明文是拥有 $ (p,q) $的用户发送的; 

        \item 可以得到 $ m_p^2\equiv c\pmod{p}=k_1p+c $, 同理 $ m_q^2=k_2q+c $, 那么
        \begin{align*}
            r_1^2&\equiv(y_p\cdot p\cdot m_q+y_q\cdot q\cdot m_p)^2\\
            &\equiv(y_p\cdot p\cdot m_q)^2+(y_q\cdot q\cdot m_p)^2\\
            &\equiv y_p^2\cdot p^2\cdot m_q^2+y_q^2\cdot q^2\cdot m_p^2\\
            &\equiv (y_p^2\cdot p^2+y_q^2\cdot q^2)c\\
            &\equiv ((y_p\cdot p+y_q\cdot q)^2-2y_p\cdot p\cdot y_q\cdot q)c\\
            &\equiv c\pmod{n}
        \end{align*}
        同理, $ r_2,~r_3,~r_4 $可以用同样的方法证明. 
        \item 计算得到 $ m_p\equiv 15^2\pmod{7}=1,m_q=9 $; 同时得到 $ y_p=-3,y_q=2 $; 最终计算得到 $ r_1=64,r_2=13,r_3=20,r_4=57 $;
    \end{enumerate}
\subsubsection*{Problem 3}
    本原根: 当 $ a $是满足 $(a,n)=1 $且$ a^{\varphi(n)}\equiv 1 \pmod{n} $ 的最小正整数时, 称 $ a $是 $ n $的本原根. 

    ElGamal公钥密码方案: 
    \begin{enumerate}[label=(\arabic*)]
        \item 密钥生成:
        
        随机选择一个大素数 $p$, 且要求 $ p-1 $ 有大素数因子. 再选择一个模 $ p$的本原元 $ g $. 
        随机取整数 $ x $满足 $ 2\le x\le p-2 $作为私钥, 计算出 $ h\equiv g^x\pmod{p} $, 则公钥为 $ \{p,g,h\} $.

        \item 加密运算:
        
        随机选取 $ 1\le y\le p-2 $, 然后计算 $ s\equiv h^y\pmod{p} $, 计算 $ c_1\equiv g^y\pmod{p} $, 同时明文 $ m<p $进行加密计算得到 $ c_2\equiv m\cdot s\pmod{p} $, 发送的密文为 $ (c_1,c_2) $.

        \item 解密运算:
        
        接收方接收到密文 $ (c_1,c_2) $后, 计算 $ s\equiv c_1^x\pmod{p} $, 然后计算出 $ s^{-1}\pmod{p} $的值, 其中$ s\cdot s^{-1}\equiv 1\pmod{p} $, 则明文 $ m\equiv c_2\cdot s^{-1}\pmod{p} $;
    \end{enumerate}
    那么: 
    \begin{enumerate}[label=(\arabic*)]
        \item 证明上述解密运算的正确性. 
        \item 当 $ p=2539, ~g=2, ~x=51,~y=15 $时, 给出明文 $ m=804 $ 对应的密文 $ (c_1,c_2) $和 $ (c_1=2300,c_2=224) $对应的明文 $ m $.
    \end{enumerate}
    解:\begin{enumerate}[label=(\arabic*)]
        \item 由 $ c_1\equiv g^y\pmod{p},~ s\equiv c_1^x\pmod{p} $可知 $ s\equiv g^{xy}\pmod{p} $, 所以 $ c_2\cdot s^{-1}\equiv \frac{m\cdot h^y}{s}\equiv\frac{m\cdot g^{xy}}{g^{xy}}\equiv m\pmod{p}  $.
        \item 当 $ y=15 $时, $ c_1=2300,~c_2=224 $, 密文为 $ (2300,224) $; $ s=1794 $, 则 $ s^{-1}=593 $, $ m=804 $.
    \end{enumerate}
\end{document}