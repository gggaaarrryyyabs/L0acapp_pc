\documentclass[a4paper,12pt]{ctexart}
\usepackage{fullpage,enumitem,amsmath,amssymb,graphicx}
\newcommand{\Z}{\mathbb{Z}}
\newcommand{\F}{\mathbb{F}}
\newcommand{\Com}{\mathbf{C}}
\newcommand{\ord}{\operatorname{ord}}
\newcommand{\Q}{\mathbb{Q}}
\newcommand{\R}{\mathbb{R}}


\title{NIS2312-1 2022-2023 Fall Homework~1}
\author{唐灯}



\begin{document}
%   \maketitle
  \begin{center}

  \vspace{-0.3in}
  \begin{tabular}{c}
    \textbf{\Large NIS2312-1 2022-2023 Fall} \\
    \textbf{\Large  } \\
    \textbf{\Large  信息安全的数学基础(1)} \\
    \textbf{\Large  } \\
    \textbf{\Large  Answer~14} \\
    \textbf{\Large  } \\
    \textbf{\Large 2022年11月7日} \\
  \end{tabular}
  \end{center}
  \noindent
  \rule{\linewidth}{0.4pt}
  
%   可以使用计算机求模的运算.

\subsubsection*{Problem 1}
    设 $ R=\Z $为整数集. 对任意的 $ x,y\in R $, 规定 
    \[x\oplus y=x+y+1,\quad x\odot y=xy+x+y.\]
    \begin{enumerate}[label=(\arabic{*})]
      \item 证明: $ (R,\oplus,\odot) $构成一个环;
      \item 证明: $ R $与整数环 $ \Z $同构.
    \end{enumerate}

    解: \begin{enumerate}[label=(\arabic{*})]
      \item (a) 显然是代数运算;

      (b) 对任意的 $x, y \in R$, 有
      $$
      \begin{aligned}
      &x \oplus y=x+y+1=y+x+1=y \oplus x, \\
      &x \odot y=x y+x+y=y x+y+x=y \odot x,
      \end{aligned}
      $$
      所以 $R$ 的两个运算都满足交换律;

      (c) 对任意的 $x, y, z \in R$, 有
      $$
      \begin{aligned}
      (x \oplus y) \oplus z &=(x+y+1) \oplus z=(x+y+1)+z+1=x+y+z+2, \\
      x \oplus(y \oplus z) &=x \oplus(y+z+1)=x+(y+z+1)+1=x+y+z+2, \\
      (x \odot y) \odot z &=(x y+x+y) \odot z=(x y+x+y) z+(x y+x+y)+z \\
      &=x y z+x y+x z+y z+x+y+z, \\
      x \odot(y \odot z) &=x \odot(y z+y+z)=x(y z+y+z)+x+(y z+y+z) \\
      &=x y z+x y+x z+y z+x+y+z,
      \end{aligned}
      $$
      所以 $ R $ 的两个运算都满足结合律;

      (d) 对任意的 $x \in R$, 有
      $$
      x \oplus(-1)=x+(-1)+1=x,
      $$
      所以 $-1$ 为 $R$ 的零元.

      (e) 对任意的 $x \in R$, 有
      $$
      x \oplus(-2-x)=x+(-2-x)+1=-1,
      $$
      所以 $-2-x$ 为 $x$ 的负元.

      (f) 对任意的 $x \in R$, 有
      $$
      x \odot 0=x \cdot 0+x+0=x,
      $$
      所以 0 为 $R$ 的单位元.
      
      (g) 对任意的 $x, y, z \in R$, 有
      $$
      \begin{aligned}
      (x \oplus y) \odot z &=(x+y+1) \odot z=(x+y+1) z+(x+y+1)+z \\
      &=x z+y z+x+y+2 z+1, \\
      (x \odot z) \oplus(y \odot z) &=(x z+x+z) \oplus(y z+y+z) \\
      &=x z+y z+x+y+2 z+1,
      \end{aligned}
      $$
      所以 $\odot$ 对 $\oplus$ 满足分配律.
      因此 $(R, \oplus, \odot)$ 构成一个有单位元的交换环.
      \item 令
      $$
      \begin{aligned}
      \phi: & R \longrightarrow \Z \\
      & x \longmapsto x+1 .
      \end{aligned}
      $$
      (a) 显然 $\phi$ 为 $R$ 到 $\Z$ 的单且满映射.

      (b) 对任意的 $x, y \in R$, 有
      $$
      \begin{aligned}
      \phi(x \oplus y) &=\phi(x+y+1)=x+y+2=(x+1)+(y+1) \\
      &=\phi(x)+\phi(y) \\
      \phi(x \odot y) &=\phi(x y+x+y)=x y+x+y+1=(x+1)(y+1) \\
      &=\phi(x) \phi(y)
      \end{aligned}
      $$
      所以 $ \phi $ 为 $ R $到 $ \Z $的同构映射, 即 $ R\cong \Z $.
    \end{enumerate}
    \subsubsection*{Problem 2} 
    设 $ m $与 $ n $是互素的正整数. 证明: 存在环同构 $ \Z_{mn}\cong \Z_m\oplus\Z_n $.
    
    解: 
    $$
    \begin{array}{crcl}
    \phi: & \Z_{m n} & \longrightarrow &\Z_m \oplus \Z_n \\
    &\overline{x} & \longmapsto&([x]_m, [x]_n) .
    \end{array}
    $$
    \begin{enumerate}[label=(\arabic{*})]
      \item 如果 $\overline{x}=\overline{y}$, 则 $m n \mid x-y$, 于是 $m|(x-y), n|(x-y)$. 所以 $([x]_m, [x]_n)=([y]_m, [y]_n)$, 因此 $\phi$ 为 $\Z_{m n}$ 到 $\Z_m \oplus \Z_n$ 的映射.
      \item 设 $\overline{x}, \overline{y} \in \Z_{m n}$, 如果 $\phi(\overline{x})=\phi(\overline{y})$, 即 $([x]_m, [x]_n)=([y]_m, [y]_n)$, 则 $m|(x-y), n|(x-y)$. 由于 $(m, n)=1$, 因此 $m n \mid(x-y)$, 从而 $\overline{x}=\overline{y} \in \Z_{m n}$. 这说明 $\phi$ 是 $\Z_{m n}$ 到 $\Z_m \oplus \Z_n$ 的单映射. 又因为 $\left|\Z_{m n}\right|=m n=\left|\Z_m \oplus \Z_n\right|$, 所以 $\phi\left(\Z_{m n}\right)=\Z_m \oplus \Z_n$, 因此 $\phi$ 也是 $\Z_{m n}$ 到 $\Z_m \oplus \Z_n$ 的满映射.
      \item 对任意的 $\overline{x}, \overline{y} \in \Z_{m n}$, 有
      $$
      \begin{aligned}
        \phi(\overline{x}+\overline{y}) &=\phi(\overline{x+\overline{y}})=(\overline{x+y}, \overline{x+\overline{y}})=([x]_m, [x]_n)+([y]_m, [y]_n)=\phi(\overline{x})+\phi(\overline{y}), \\
        \phi(\overline{x} \cdot \overline{y}) &=\phi(\overline{x} \overline{y})=([xy]_m, [xy]_n)=([x]_m, [x]_n)([y]_m, [y]_n)=\phi(\overline{x}) \cdot \phi(\overline{y}),
      \end{aligned}
      $$
\end{enumerate}
    所以 $\phi$ 为 $\Z_{m n}$ 到 $\Z_m \oplus \Z_n$ 环同构. 即
    $$
    \quad \Z_{m n} \cong \Z_m \oplus \Z_n .
    $$

\subsubsection*{Problem 3}
    设 $ S $为 $ R $的子环, $ I $为 $ R $的理想, 则 $ S\cap I $是 $ S $的理想且 
    \[\frac{S}{S\cap I}\cong \frac{S+I}{I}.\]

    解: 
    显然 $I$ 为环 $S+I$ 的理想, 从而有自然同态.
    $$
    \eta: S+I \longrightarrow(S+I) / I .
    $$
    因而 $\eta$ 在 $S$ 上的限制
    $$
    \begin{aligned}
    \left.\eta\right|_S: S & \longrightarrow(S+I) / I, \\
    s & \longmapsto \eta(s)
    \end{aligned}
    $$
    是一个 $S$ 到 $(S+I) / I$ 的同态.
    又对任意的 $\overline{s+x} \in(S+I) / I(s \in S, x \in I)$, 有
    $$
    \left.\eta\right|_S(s)=\eta(s)=\bar{s}=\overline{s+x},
    $$
    所以 $\left.\eta\right|_S$ 为满同态. 而
    $$
    \text { Ker }\left.\eta\right|_S=\{s \in S \mid \eta(s)=\overline{0}\}=\{s \in S \mid s \in I\}=S \cap I .
    $$
    从而 $S \cap I$ 是 $S$ 的理想. 由环同态基本定理知, 有环同构
    $$
    S /(S \cap I) \cong(S+I) / I
    $$

\subsubsection*{Problem 4}
    设 $ R $是一个没有单位元的环, 则存在一个有单位元的环 $ R' $, 使得 $ R $为 $ R' $的子环.
    
    解: 
    \begin{enumerate}[label=(\arabic{*})]
      \item  令 $S^{\prime}=\{(n, x) \mid n \in \mathbf{Z}, x \in R\}$. 对任意的 $(n, x),(m, y) \in S^{\prime}$, 规定
$$
\begin{aligned}
(n, x)+(m, y) &=(n+m, x+y), \\
(n, x) \cdot(m, y) &=(n m, n y+m x+x y) .
\end{aligned}
$$
易知 $S^{\prime}$ 关于如此定义的加法与乘法构成一个环.
\item 对任意的 $(n, x) \in S^{\prime}$, 有
$$
\begin{aligned}
&(n, x)(1,0)=(n \cdot 1, n 0+1 x+x 0)=(n, x), \\
&(1,0)(n, x)=(1 \cdot n, 1 x+n 0+0 x)=(n, x),
\end{aligned}
$$
即 $(1,0)$ 是 $S^{\prime}$ 的单位元. 所以 $S^{\prime}$ 是有单位元的环.
\item 对任意的 $x \in R$, 令
$$
\begin{aligned}
\phi: \quad R & \longrightarrow S^{\prime}, \\
x & \longmapsto(0, x),
\end{aligned}
$$
则 $\phi$ 为 $R$ 到 $S^{\prime}$ 的单同态. 又显然 $R$ 与 $S^{\prime}$ 没有公共元素, 从而由定理 3.4.5, 存在 $R$ 的扩环 $R^{\prime}$ 使 $R^{\prime} \cong S^{\prime}$. 因 $S^{\prime}$ 是有单位元的环, 所以 $R^{\prime}$ 也是有单位元的环.
\end{enumerate}

\end{document}

