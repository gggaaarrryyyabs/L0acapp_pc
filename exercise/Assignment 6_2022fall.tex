\documentclass[a4paper,12pt]{ctexart}
\usepackage{fullpage,enumitem,amsmath,amssymb,graphicx}
\newcommand{\Z}{\mathbb{Z}}
\newcommand{\F}{\mathbb{F}}
\newcommand{\Com}{\mathbf{C}}
\newcommand{\ord}{\operatorname{ord}}
\newcommand{\Q}{\mathbb{Q}}
\newcommand{\R}{\mathbb{R}}


\title{NIS2312-1 2022-2023 Fall Homework~1}
\author{唐灯}



\begin{document}
%   \maketitle
  \begin{center}

  \vspace{-0.3in}
  \begin{tabular}{c}
    \textbf{\Large NIS2312-1 2022-2023 Fall} \\
    \textbf{\Large  } \\
    \textbf{\Large  信息安全的数学基础(1)} \\
    \textbf{\Large  } \\
    \textbf{\Large  Assignment~6} \\
    \textbf{\Large  } \\
    \textbf{\Large 2022年10月10日} \\
  \end{tabular}
  \end{center}
  \noindent
  \rule{\linewidth}{0.4pt}
  
%   可以使用计算机求模的运算.
$ \R $是实数域, $ \Q $是有理数域, $ \Z $是整数集合.

\subsubsection*{Problem 1}
  在 $ \Z_{12} $中, 求子群 $ H=\langle \overline{4}\rangle $的所有左陪集.
      
  解: 可知 $ H=\{\overline{0},\overline{4},\overline{8}\} $, 因此左陪集有
  \begin{align*}
    &\overline{0}+H=H=\{\overline{0},\overline{4},\overline{8}\};\\
    &\overline{1}+H=\{\overline{1},\overline{5},\overline{9}\};\\
    &\overline{2}+H=\{\overline{2},\overline{6},\overline{10}\};\\
    &\overline{3}+H=\{\overline{3},\overline{7},\overline{11}\}.
  \end{align*}

\subsubsection*{Problem 2} 
  设 $ H=\{0,\pm 3,\pm 6,\pm 9,\dots\} $. 求子群 $ H $在 $ \Z $中的所有左陪集.
  
  解: 子群 $ H $在 $ \Z $中的所有左陪集为
  \begin{align*}
    0+H=\{0,\pm 3,\pm 6,\pm 9,\dots\};\\
    1+H=\{1,1\pm 3,1\pm 6,1\pm 9,\dots\};\\
    2+H=\{2,2\pm 3,2\pm 6,2\pm 9,\dots\}.
  \end{align*}

\subsubsection*{Problem 3}
  设 $ \operatorname{ord} a = 30 $. 问 $ \langle a^4\rangle $ 在  $ \langle a\rangle $中有多少个左陪集? 试将它们列出.

  解: $ \operatorname{ord}a^4=30/(30,4)=15 $, 故 $ \langle a^4\rangle $ 在  $ \langle a\rangle $中有 $ 30/15=2 $ 个左陪集. 
  其中之一为 $ e\langle a^4\rangle $, 因为 $ a\notin \langle a^4\rangle $, 故 $ a $属于另一个左陪集, 则另一个为 $ a\langle a^4\rangle $.

\subsubsection*{Problem 4}
  设 $ H_1,H_2 $是 $ G $的子群. 证明: $ a\left(H_1\cap H_2\right)=aH_1\cap aH_2 $.

  解: $ \subseteq $: $ \forall x\in H_1\cap H_2 $, 都有 $ x\in H_1,H_2 $, 故 $ ax\in aH_1,aH_2 $, 即 $ ax\in aH_1\cap aH_2 $, 
  因此 $ a\left(H_1\cap H_2\right)\subseteq aH_1\cap aH_2 $;

  $ \supseteq $: $ \forall x\in aH_1\cap aH_2 $, 都有 $ x=ah_1=ah_2 $, 其中 $ h_1\in H_1,h_2\in H_2 $. 则 $ a^{-1}x\in H_1 $, 
  $ a^{-1}x\in H_2 $, 即 $ a^{-1}x\in H_1\cap H_2 $, 那么 $ x\in a\left( H_1\cap H_2\right) $, 因此 $ a\left(H_1\cap H_2\right) \supseteq aH_1\cap aH_2 $.

  综上, $ a\left(H_1\cap H_2\right)=aH_1\cap aH_2 $.

\subsubsection*{Problem 5}
  设 $ H $是有限群 $ G $的子群, $ K $是 $ H $的子群. 证明: $ \left[G:K\right]=\left[G:H\right]\left[H:K\right] $.

  解: $ \left[G:K\right]= \frac{|G|}{|K|}=\frac{|G|}{|H|}\times\frac{|H|}{|K|}=\left[G:H\right]\left[H:K\right] $.

\subsubsection*{Problem 6}
  证明: $ 15 $ 阶群至多含有一个 $ 5 $ 阶子群.

  解: 因为 $ 5 $是素数, 故 $ 5 $阶群是循环群, 且群只有单位元和生成元. 
  那么任意两个不同的 $ 5 $ 阶群交集为单位元, 否则, 两个群均可以用同一个生成元表示, 即两个群是相同的. 
  假设有两个不同的 $ 5 $阶子群 $ H,K $ , 则构造集合 $ HK=\{hk\mid h\in H,k\in K\} $, 显然 $ HK\subseteq G $. 
  假设存在 $ h_1,h_2\in H,k_1,k_2\in K $, 
  s.t. $ h_1k_1=h_2k_2 $, 则 $ h_2^{-1}h_1=k_2k_1^{-1}\in H\cap K=\{e\} $, 即 $ h_2=h_1,k_2=k_1 $. 因此集合  $ HK $元素数量为 $ 5\times 5=25>15=|G| $, 矛盾, 故至多含有一个 $ 5 $ 阶子群.







  就是这个DDT的计算,用并行的方法,看着是很简单的,不知道有没有坑:

  注意图片上的

  我的想法是:计算DDT,每一列的计算方法是一样的,比如 $ a=0 $ 这一列,

\end{document}