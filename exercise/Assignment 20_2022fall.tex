\documentclass[a4paper,12pt]{ctexart}
\usepackage{fullpage,enumitem,amsmath,amssymb,graphicx}
\newcommand{\Z}{\mathbb{Z}}
\newcommand{\F}{\mathbb{F}}
\newcommand{\Com}{\mathbb{C}}
\newcommand{\ord}{\operatorname{ord}}
\newcommand{\Q}{\mathbb{Q}}
\newcommand{\R}{\mathbb{R}}


\title{NIS2312-1 2022-2023 Fall Homework~1}
\author{唐灯}



\begin{document}
%   \maketitle
  \begin{center}

  \vspace{-0.3in}
  \begin{tabular}{c}
    \textbf{\Large NIS2312-1 2022-2023 Fall} \\
    \textbf{\Large  } \\
    \textbf{\Large  信息安全的数学基础(1)} \\
    \textbf{\Large  } \\
    \textbf{\Large  Asignment~20} \\
    \textbf{\Large  } \\
    \textbf{\Large 2022年12月19日} \\
  \end{tabular}
  \end{center}
  \noindent
  \rule{\linewidth}{0.4pt}
  
%   可以使用计算机求模的运算.

\subsubsection*{Problem 1}
    将 $ x^3+2x+1\in\Z_3[x] $写成 $ \Z_3 $的某个扩域中的一次因式的乘积.

\subsubsection*{Problem 2}
    求 $ x^4-x^2+1 $在 $ \Z_3 $上的分裂域.

\subsubsection*{Problem 3}
    求 $ f(x)=x^3+x+1 $在 $ \Z_2 $上的分裂域, 并将 $ f(x) $在该分裂域上分解为一次因式的乘积. 
    
\end{document}



 



