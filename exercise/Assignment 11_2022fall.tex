\documentclass[a4paper,12pt]{ctexart}
\usepackage{fullpage,enumitem,amsmath,amssymb,graphicx}
\newcommand{\Z}{\mathbb{Z}}
\newcommand{\F}{\mathbb{F}}
\newcommand{\Com}{\mathbf{C}}
\newcommand{\ord}{\operatorname{ord}}
\newcommand{\Q}{\mathbb{Q}}
\newcommand{\R}{\mathbb{R}}


\title{NIS2312-1 2022-2023 Fall Homework~1}
\author{唐灯}



\begin{document}
%   \maketitle
  \begin{center}

  \vspace{-0.3in}
  \begin{tabular}{c}
    \textbf{\Large NIS2312-1 2022-2023 Fall} \\
    \textbf{\Large  } \\
    \textbf{\Large  信息安全的数学基础(1)} \\
    \textbf{\Large  } \\
    \textbf{\Large  Answer~11} \\
    \textbf{\Large  } \\
    \textbf{\Large 2022年10月26日} \\
  \end{tabular}
  \end{center}
  \noindent
  \rule{\linewidth}{0.4pt}
  
%   可以使用计算机求模的运算.

\subsubsection*{Problem 1}
  证明或否定 $ \Z\oplus\Z $是循环群.
  
  解: 假设 $ \Z\oplus\Z $是循环群, 生成元为 $ (a,b) $, 故有 $ (m,-n)=k(a,b) $, 当 $ m=0,n\ne 0 $时, 有 $ a=0 $, 当 $ m\ne 0,n=0 $时, 有 $ b=0 $, 因此 生成元为 $ (0,0) $, 矛盾. 故 $ \Z\oplus\Z $不是循环群.

\subsubsection*{Problem 2} 
    假设 $ G_1\cong H_1,~G_2\cong H_2 $. 证明: $ G_1\times G_2\cong H_1\times H_2 $.
  
    解: 假设 $ \phi_1(G_1)=G_2 $且 $ \phi_2(H_1)=H_2 $. 因此构造
    \[\begin{array}{crcl}
      \phi: &\quad G_1 \times G_2 &\longrightarrow & H_1 \times H_2\\
      &(a, b) &\longmapsto&\left(\phi_1(a), \phi_2(b)\right), \quad \forall a \in G_1, b \in G_2 .
    \end{array}\]
    \begin{enumerate}[label=(\arabic{*})]
      \item 显然 $\phi$ 是 $G_1 \times G_2$ 到 $H_1 \times H_2$ 的映射;
      \item 假设有 $(a, b),\left(a^{\prime}, b^{\prime}\right) \in G_1 \times G_2$, 
      使得 $ \phi(a, b)=\phi\left(a^{\prime}, b^{\prime}\right)$, 
      即 $\left(\phi_1(a), \phi_2(b)\right)=\left(\phi_1\left(a^{\prime}\right)\right.$, $\left.\phi_2\left(b^{\prime}\right)\right)$. 
      因为 $\phi_1, \phi_2$ 都是单射, 故 $ a=a^{\prime}, b=b^{\prime}$, 于是 $(a, b)=\left(a^{\prime}, b^{\prime}\right)$, 则 $\phi$ 为单射;
      \item 对 $ \forall (h, k) \in H_1 \times H_2 $, 由于 $ \phi_1, \phi_2 $ 都是满射, 因此 $\exists a \in G_1, b \in G_2$, 
      使得 $ \phi_1(a)=h, \phi_2(b)=k $, 故 $ \phi(a, b)=\left(\phi_1(a), \phi_2(b)\right)=(h, k) $, 因此 $ \phi $是满射;
      \item $ \forall (a, b),\left(a^{\prime}, b^{\prime}\right) \in G_1 \times G_2 $ 都有
      \begin{align*}
        \phi((a,b)(a',b'))=&\phi(aa',bb')=(\phi_1(aa'),\phi_2(bb'))\\
                          =&(\phi_1(a)\phi_1(a'),\phi_2(b)\phi_2(b'))\\
                          =&(\phi_1(a),\phi_2(b))(\phi_1(a'),\phi_2(b'))\\
                          =&\phi(a,b)\phi(a',b').
      \end{align*}
    \end{enumerate}
    因此 $ \phi $是 $ G_1\times G_2 $到 $ H_1\times H_2 $的同构映射, 故 $ G_1\times G_2\cong H_1\times H_2 $.

    \subsubsection*{Problem 3}
    在 $ \Z $中, 设 $ H=\langle 3\rangle,~ K=\langle 5\rangle $. 证明: $ \Z=H+K $. 请问 $ \Z $ 与 $ H\oplus K $同构吗?

    解: $ \Z $的生成元是 $ 1 $, 且 $ (3,5)=1 $, 即 $ \exists a,b\in\Z $满足 $ 3a+5b=1 $, 因此 $ \forall z\in\Z $, 都有
    \[z=z\cdot 1=z\cdot(3a+5b)=3az+5bz\in H+K.\]
    同时 $ H+K\subseteq\Z $, 故 $ Z=H+K $.

    假设 $ Z\cong H\oplus K $, 那么 有 $ H\oplus K\cong H+K $, 即有同构映射 $ \phi: H\oplus K\rightarrow H+K $, 
    即 $ \phi(h,k)=h+k $. 注意到 $ 15\in H\cap K $, 因此有 $ \phi(0,15)=0+15=15+0=\phi(15,0) $, 不满足单射, 故不同构.

    第二部分也可以通过第一题和第二题的结论证明: 有 $ \Z\cong H $ 和 $ \Z\cong K $, 故
    $ \Z\oplus\Z\cong H\oplus K\cong\Z $, 与第一题结论矛盾, 故 两者不同构.

\subsubsection*{Problem 4}
    证明: $ U(15) $ 同构于 $ U(3)\times U(5) $.

    解: 构造
    \[\begin{array}{crcl}
      \phi&U(15)&\longrightarrow&U(3)\times U(5)\\
          & n   &\longmapsto    &(n\pmod{3},n\pmod{5}), where~n~is~coprime~to~15.
    \end{array}\]
    \begin{enumerate}[label=(\arabic{*})]
      \item 显然  $ \phi $ 是一个映射;
      \item 假设有 $ n_1,n_2\in U(15) $使得 $ \phi(n_1)=\phi(n_2) $, 那么 $ (n_1\pmod{3},n_1\pmod{5})=(n_2\pmod{3},n_2\pmod{5}) $, 故 $ n_1-n_2\equiv 0\pmod{3} $, $ n_1-n_2\equiv 0\pmod{5} $,  即 $ 3\mid n_1-n_2,5\mid n_1-n_2 $且
      $ 0\le n_1-n_2\le 14 $, 故 $ n_1=n_2 $, 即 $ \phi $是单射;
      % \item 对于 $ \forall (n\pmod{3},n\pmod{5})\in U(3)\times U(5) $, 都有 $ \exists u_a,v_a,u_b,v_b\in\Z $使得
      % \begin{align*}
      %   u_a\cdot n +v_a\cdot 3&=1\\
      %   u_b\cdot n +v_b\cdot 5&=1
      % \end{align*}
      % 因此有 $ (u_au_bn+5u_av_b+3u_bv_a)\cdot n+v_av_b\cdot 15=1 $, 即 $ (n,15)=1 $, 因此对于任意
      \item 因为 $ |U(15)|=8=2\times 4=|U(3)|\times|U(5)| $, 故 $ \phi $ 是满射;
      \item $ \forall n_1,n_2\in U(15) $, 都有 $ \phi(n_1n_2)=(n_1n_2\pmod{3},n_1n_2\pmod{5})=(n_1\pmod{3},n_1\pmod{5})(n_2\pmod{3},n_2\pmod{5})=\phi(n_1)\phi(n_2) $, 其中第二个等号成立是因为 $ (p,ab)=1\Rightarrow (p,a)=1 $且 $ (p,b)=1 $.
    \end{enumerate}
    综上, $ \phi $是从  $ U(15) $ 到 $ U(3)\times U(5) $的同构映射, 故 $ U(15) $ 同构于 $ U(3)\times U(5) $.

\subsubsection*{Problem 5}
  设 $ G=G_1\times G_2\times\cdots\times G_n $, 每个 $ a_i $是 $ G_i $中的有限阶元素. 证明: 
  \[\operatorname{ord}(a_1,a_2,\dots,a_n)=[\operatorname{ord}a_1,\operatorname{ord}a_2,\dots,\operatorname{ord}a_n].\]

  解: 设 $ d=[\operatorname{ord}a_1,\operatorname{ord}a_2,\dots,\operatorname{ord}a_n] $, 那么 $ (a_1,a_2,\dots,a_n)^d=(a_1^d,a_2^d,\dots,a_n^d)=(e_1,e_2,\dots,e_n) $, 故 $ \operatorname{ord}(a_1,a_2,\dots,a_n)\mid [\operatorname{ord}a_1,\operatorname{ord}a_2,\dots,\operatorname{ord}a_n] $;

  假设 $ d'\in \Z $ 使得 $ (a_1,a_2,\dots,a_n)^d=(e_1,e_2,\dots,e_n) $, 那么 $ a_i^{d'}=e_i $, 故 $ \ord a_i\mid d' $, 因此
  $ d\mid d' $.

  综上 $ \operatorname{ord}(a_1,a_2,\dots,a_n)=[\operatorname{ord}a_1,\operatorname{ord}a_2,\dots,\operatorname{ord}a_n] $.

\end{document}