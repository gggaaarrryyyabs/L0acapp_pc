\documentclass[a4paper,12pt]{ctexart}
\usepackage{fullpage,enumitem,amsmath,amssymb,graphicx}
\newcommand{\Z}{\mathbb{Z}}
\newcommand{\F}{\mathbb{F}}
\newcommand{\Com}{\mathbb{C}}
\newcommand{\ord}{\operatorname{ord}}
\newcommand{\Q}{\mathbb{Q}}
\newcommand{\R}{\mathbb{R}}


\title{NIS2312-1 2022-2023 Fall Homework~1}
\author{唐灯}



\begin{document}
%   \maketitle
  \begin{center}

  \vspace{-0.3in}
  \begin{tabular}{c}
    \textbf{\Large NIS2312-1 2022-2023 Fall} \\
    \textbf{\Large  } \\
    \textbf{\Large  信息安全的数学基础(1)} \\
    \textbf{\Large  } \\
    \textbf{\Large  Asignment~19} \\
    \textbf{\Large  } \\
    \textbf{\Large 2022年12月12日} \\
  \end{tabular}
  \end{center}
  \noindent
  \rule{\linewidth}{0.4pt}
  
%   可以使用计算机求模的运算.

\subsubsection*{Problem 1}
    设 $ a,b\in\R $, $ b\ne 0 $. 证明: $ \R(a+b\operatorname{i})=\Com $.

\subsubsection*{Problem 2}
    设 $ F $是个域, $ a,b\in F,a\ne 0 $. 如果$ c $属于 $ F $的某个扩域, 证明: $ F(c)=F(ac+b) $(即 $ F $ ``吸收''它自己的元素).
    
\subsubsection*{Problem 3}
    证明: 商环 $ \R[x]/\langle x^2+1\rangle $与复数域 $ \Com $同构. 

\end{document}

