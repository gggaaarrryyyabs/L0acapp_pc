\documentclass[a4paper,12pt]{ctexart}
\usepackage{fullpage,enumitem,amsmath,amssymb,graphicx}
\newcommand{\Z}{\mathbf{Z}}
\newcommand{\F}{\mathbf{F}}
\newcommand{\Com}{\mathbf{C}}
\newcommand{\ord}{\operatorname{ord}}
\newcommand{\Q}{\mathbf{Q}}
\newcommand{\R}{\mathbf{R}}


% \title{NIS2312-2 Spring 2022 Homework~1}
% \author{唐灯}



\begin{document}
%   \maketitle
  \begin{center}

  \vspace{-0.3in}
  \begin{tabular}{c}
    \textbf{\Large NIS2312-1 Spring 2021-2022} \\
    \textbf{\Large  } \\
    \textbf{\Large  信息安全的数学基础(1)} \\
    \textbf{\Large  } \\
    \textbf{\Large  Assignment~3} \\
    \textbf{\Large  } \\
    \textbf{\Large 2022年3月7日} \\
  \end{tabular}
  \end{center}
  \noindent
  \rule{\linewidth}{0.4pt}
  
\subsubsection*{Problem 1}
    判断下面的映射是否为同构映射, \CJKunderdot{\textbf{是的话直接写``是''无需证明; 如果不是, 给出反例或证明}}
    \begin{enumerate}
        \item $ f:\Z\rightarrow\Z $, 其中 $ f(x)=(-2)^x,x\in\Z $且$ \Z $是加法群
        \item $ exp:\Z\rightarrow\Z $, 其中 $ exp(x)=e^x,x\in\Z $且$ \Z $上的运算是乘法
        \item $ dec2bin:\Z_{8}\rightarrow\Z_{2}\times\Z_2\times\Z_2 $, 其中 $ dec2bin(x)=(x_0,x_1,x_2) $, $ x=x_0+2x_1+2^2x_2 $, 且 $ \Z_8 $是加法群, $ \Z_2\times\Z_2\times\Z_2 $上的运算是按位加法
        % \item $ f:\Z_4\rightarrow\Z_2\times\Z_2 $, 其中 $  $
        \item 有对称群之间的映射$ f:S_{\{1,2,\dots,26\}}\rightarrow S_{\{a,b,c,\dots,z\}}$, 其中 $ f $将数字 $ i $ 映射到26个字母中的第 $ i $个字母
        \item $ det:GL_n(\R)\rightarrow\R^* $, 其中 $ det $是计算矩阵行列式值, $GL_n(\R)$的运算是矩阵乘法
        \item $ det:SL_n(\R)\rightarrow\R^* $, 其中 $ det $是计算矩阵行列式值, $SL_n(\R)$的运算是矩阵乘法
        \item 设 $ G $是群, $ a\in G $, $ \phi:G\rightarrow G $, 且$ \phi(x)=ax $, $  x\in G $
        \item 设 $ G $是群, $ a\in G $, $ \phi:G\rightarrow G $, 且$ \phi(x)=axa^{-1} $, $  x\in G $
        \item $ T:M_n(\R)\rightarrow M_n(\R) $, 其中 $ T $是方阵的转置, $ M_n(\R) $上的运算是$n$阶方阵的加法
        \item $ f:\{-1,1\}\rightarrow\{0,1\} $, 其中 $ f(x)=(-1)^x,x\in\{-1,1\} $, 前一个集合运算是乘法, 后一个集合的运算是加法.
    \end{enumerate}
\subsubsection*{Problem 2} 
    证明: \begin{enumerate}
      \item 乘法群 $ \R^* $和乘法群 $ \mathbb{C}^* $不同构; 
      \item 加法群 $ \Z $和加法群 $ \Q $不同构; 
      % \item 加法群 $ \Z $和加法群 $ \Z\times\Z_2 $不同构, 其中 $ \Z\times\Z_2=\{(n,m)\mid n\in\Z,m\in\Z_2\} $, 加法是按位加法.
      \item 按位加法运算下的加法群 $ \Z_2\times\Z_2 $和加法群$ \Z_4 $不同构
      \item 假设 $ G,H $是两个群, 那么按位乘法的 $ G\times H $和 $ H\times G $同构
      \item 如果 $ \gcd(n,m)=1 $, 那么 $ \Z_n\times\Z_m $在按位加法运算下是循环群(思考是否和加法群 $ \Z_{nm} $同构)
    \end{enumerate}
\subsubsection*{Problem 3}
    设 $ G $为有限群, 定义取逆的映射 $ f:G\rightarrow G $, 其中 $ f(x)=x^{-1},x\in G $. 
    证明: $ f $同构映射. 如果$G$为偶数阶群, 证明: 该群必定含有阶为 $ 2 $的元素.
    % 假设 $ G $是一个群, $ Aut(G) $是群 $ G $到自身的所有的自同构映射的集合. 证明在映射复合的运算下, $ Aut(G) $是一个群. 
\subsubsection*{Problem 4}
    证明: 任何一个有限群同构于 $ GL_n(\F_2) $的一个子群, 其中 $ \F_2=\{0,1\} $是一个域(具有标准的模2加法且 $ 0\times 1=0,1\times 1=1 $和 $ 0\times 0=0 $, 两种运算均具有交换律, 且有分配律).
\subsubsection*{Problem 5}
    \begin{enumerate}
      \item 计算 $ \Z_{233} $和 $ \Z_{4900} $ 生成元的数量
      \item 计算 $ \Z_{54} $的元素 $ 30 $的阶, 同时给出$ \Z_{54} $中所有的阶和$ 30 $的阶相同的元素
      % \item 给出置换 
      %       \[\sigma=\begin{pmatrix}
      %         1 &2   &\dots& n\\
      %         n &n-1 &\dots& 1
      %       \end{pmatrix}\]
      %       的奇偶性
      \item 给出书上1.5循环群的两个留作习题的证明过程, 注意不是P36的习题 16也不是定理1.5.3
    \end{enumerate}
% \subsubsection*{Problem 6}
%     计算对称群 $ S_n $中长度为 $ m $的轮换的数量
\subsubsection*{Problem 6}
    假设 $ \sigma $是一个长度为 $ m $的轮换 $ (1~2~\cdots~m) $, 证明 $ \sigma^i $同样为长度 $ m $的轮换
    当且仅当 $ \gcd(m,i)=1 $
  \subsubsection*{Problem 7}
  计算下面各题中的置换 $ \delta $, 其中 $ \delta=\tau\sigma\tau^{-1} $
  \begin{enumerate}
    \item $ \sigma=(1~3) $, $ \tau=(2~5~4) $
    \item $ \sigma=(1~3~5~7) $, $ \tau=(2~3~4~5) $
    \item $ \sigma=(1~3)(5~7) $, $ \tau=(2~3)(4~5) $
  \end{enumerate}
\subsubsection*{Problem 8}
    证明: 
    \begin{enumerate}
      \item $ (k~l)(k~a~\cdots ~b)(l~c~\cdots ~d)=(k~a~\cdots ~b~l~c~\cdots ~d) $
      和 $ (k~l)(k~a~\cdots ~b~l~c~\cdots ~d)=(k~a~\cdots ~b)(l~c~\cdots ~d) $, 其中, 
      $ a,\cdots,b,c,\cdots,d,k,l $为互不相同的正整数.
      \item 在对称群 $ S_n $中, 全体偶置换构成 $ S_n $的子群.
    \end{enumerate}
% \subsubsection*{Problem 9}
%     假设 $ G $ 是一个有限群, $ a\in G $ 定义
%     \[Z(G)=\{a\in G\mid xax^{-1}=a,\forall x\in G\}\]
%     和
%     \[C(a)=\{xax^{-1}\mid x\in G\}\]
% \subsubsection*{Problem 9}
%     定理2.1.4(拉格朗日定理) 设 $ G $ 是一个有限群, $ H $是 $ G $的子群, 则 
\end{document}